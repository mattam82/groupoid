\coqlibrary{Groupoid.groupoid}{Library }{Groupoid.groupoid}

\begin{coqdoccode}
\end{coqdoccode}
  This section presents our formalization of groupoids in \Coq with
  universe polymorphism. 
  We first explain our overloaded management of
  equalities and introduce type classes for groupoids and their
  associated structures, i.e., functors, natural transformations and
  homotopy equivalences
  (\S\ref{sec:w2gpds}-\ref{sec:homequiv}).
  Natural transformations give access to a homotopic form of
  functional extensionality, while homotopy equivalences provide
  extensionality at the level of 0-types.  Polymorphic universes are
  needed to state that setoids and homotopy equivalences form a
  groupoid.  Homotopic equivalences directly provide access to a
  rewriting mechanism on types (\S\ref{sec:rew}). This rewriting is
  used to extend functors and products to dependent functors and
  dependent sums (\S\ref{sec:depprod}-\ref{sec:sigma}).
 \subsection{Notations} \coqlibrary{Groupoid.notations}{Library }{Groupoid.notations}

\begin{coqdoccode}
\end{coqdoccode}
We use the following notations throughout: Sigma type introduction
  is written (\coqdocvar{t} ; \coqdocvar{p}) when its predicate/fibration is inferrable from
  the context, and projections are denoted \coqdocabbreviation{$\pi_1$} and \coqdocabbreviation{$\pi_2$}. The bracket
  notation [\coqdocvar{\_}] is an alias for \coqdocabbreviation{$\pi_1$}. The following is directly
  extracted from Coq files using the \texttt{coqdoc} tool (source files
  are available at \url{http://mattam82.github.io/groupoid}). If
  you are reading the colored version of the paper, keywords are
  typeset in red, inductive types and classes in blue, inductive
  constructors in dark red, and defined constants and lemmas in green.
  \begin{coqdoccode}
\end{coqdoccode}



\subsection{Definition of groupoids \label{sec:w2gpds}}




We formalize groupoids using type classes.  Contrarily to what is done
in the setoid translation, the basic notion of a morphism is an
inhabitant of a relation in \coqdockw{Type} (i.e., a proof-relevant relation): \begin{coqdoccode}
\coqdocemptyline
\coqdocnoindent
\coqdockw{Definition} \coqdef{Groupoid.groupoid.HomT}{$\mathsf{HomSet}$}{\coqdocdefinition{$\mathsf{HomSet}$}} (\coqdocvar{T} : \coqdockw{Type}) := \coqdocvariable{T} \coqexternalref{:type scope:x '->' x}{http://coq.inria.fr/stdlib/Coq.Init.Logic}{\coqdocnotation{\ensuremath{\rightarrow}}} \coqdocvariable{T} \coqexternalref{:type scope:x '->' x}{http://coq.inria.fr/stdlib/Coq.Init.Logic}{\coqdocnotation{\ensuremath{\rightarrow}}} \coqdockw{Type}.\coqdoceol
\coqdocemptyline
\end{coqdoccode}
  \noindent
  Given \coqdocvariable{x} and \coqdocvariable{y} in \coqdocvariable{T}, \coqref{Groupoid.groupoid.HomT}{\coqdocdefinition{$\mathsf{HomSet}$}} \coqdocvariable{T} \coqdocvariable{x} \coqdocvariable{y} is the type of morphism from \coqdocvariable{x} to \coqdocvariable{y}. 
  To manipulate different \coqref{Groupoid.groupoid.HomT}{\coqdocdefinition{$\mathsf{HomSet}$}}'s at dimension 1 and 2 more abstractly, we use ad-hoc 
  polymorphism and introduce type classes \coqref{Groupoid.groupoid.HomT1}{\coqdocrecord{$\mathsf{HomSet}_1$}} and \coqref{Groupoid.groupoid.HomT2}{\coqdocrecord{$\mathsf{HomSet}_2$}} with according notations. 
\begin{coqdoccode}
\coqdocemptyline
\coqdocnoindent
\coqdockw{Class} \coqdef{Groupoid.groupoid.HomT1}{$\mathsf{HomSet}_1$}{\coqdocrecord{$\mathsf{HomSet}_1$}} \coqdocvar{T} := \{\coqdef{Groupoid.groupoid.eq1}{$\coqdoccst{eq}_1$}{\coqdocprojection{$\coqdoccst{eq}_1$}} : \coqref{Groupoid.groupoid.HomT}{\coqdocdefinition{$\mathsf{HomSet}$}} \coqdocvariable{T}\}.\coqdoceol
\coqdocnoindent
\coqdockw{Infix} \coqdef{Groupoid.groupoid.::x 'x7E1' x}{"}{"}$\sim_1$" := \coqref{Groupoid.groupoid.eq1}{\coqdocprojection{$\coqdoccst{eq}_1$}} (\coqdoctac{at} \coqdockw{level} 80).\coqdoceol
\coqdocemptyline
\coqdocnoindent
\coqdockw{Class} \coqdef{Groupoid.groupoid.HomT2}{$\mathsf{HomSet}_2$}{\coqdocrecord{$\mathsf{HomSet}_2$}} \{\coqdocvar{T}\} (\coqdocvar{Hom} : \coqref{Groupoid.groupoid.HomT}{\coqdocdefinition{$\mathsf{HomSet}$}} \coqdocvariable{T}) := \{\coqdef{Groupoid.groupoid.eq2}{$\coqdoccst{eq}_2$}{\coqdocprojection{$\coqdoccst{eq}_2$}} : \coqexternalref{:type scope:'xE2x88x80' x '..' x ',' x}{http://coq.inria.fr/stdlib/Coq.Unicode.Utf8\_core}{\coqdocnotation{∀}} \coqexternalref{:type scope:'xE2x88x80' x '..' x ',' x}{http://coq.inria.fr/stdlib/Coq.Unicode.Utf8\_core}{\coqdocnotation{\{}}\coqdocvar{x} \coqdocvar{y} : \coqdocvariable{T}\coqexternalref{:type scope:'xE2x88x80' x '..' x ',' x}{http://coq.inria.fr/stdlib/Coq.Unicode.Utf8\_core}{\coqdocnotation{\},}} \coqref{Groupoid.groupoid.HomT}{\coqdocdefinition{$\mathsf{HomSet}$}} (\coqdocvariable{Hom} \coqdocvariable{x} \coqdocvariable{y})\}.\coqdoceol
\coqdocnoindent
\coqdockw{Infix} \coqdef{Groupoid.groupoid.::x 'x7E2' x}{"}{"}$\sim_2$" := \coqref{Groupoid.groupoid.eq2}{\coqdocprojection{$\coqdoccst{eq}_2$}} (\coqdoctac{at} \coqdockw{level} 80).\coqdoceol
\coqdocemptyline
\coqdocemptyline
\end{coqdoccode}
Given a \coqref{Groupoid.groupoid.HomT}{\coqdocdefinition{$\mathsf{HomSet}$}}, we define type classes: \coqref{Groupoid.groupoid.Identity}{\coqdocrecord{Identity}} that gives the
  identity morphism, \coqref{Groupoid.groupoid.Inverse}{\coqdocrecord{Inverse}} which corresponds to the existence of an
  inverse morphism for every morphism (noted \coqdocvariable{f} $\hspace{-1ex}^{-1}$) and \coqref{Groupoid.groupoid.Composition}{\coqdocrecord{Composition}}
  which corresponds to morphism composition (noted \coqdocvariable{g} $\circ$ \coqdocvariable{f}). Those three
  properties are gathered by the type class \coqref{Groupoid.groupoid.Equivalence}{\coqdocrecord{Equivalence}}.\begin{coqdoccode}
\end{coqdoccode}
  A \coqref{Groupoid.groupoid.CategoryP}{\coqdocrecord{$\mathsf{PreCategory}$}} is defined as a category
  where coherences are given up-to an equivalence relation denoted by
  $\sim_2$.  Ordinary categories are derived with the additional requirement
  that higher equalities are trivial, which can be expressed using
  identity types (see the definition of \coqref{Groupoid.groupoid.Groupoid}{\coqdocrecord{$\mathsf{IsType_1}$}}).  


  We do not put this condition into the basic definition because
categories and functors form a pre-category but not a 1-category. Thus,
working with pre-categories and pre-groupoids allows to share more
structure and is closer to the ω-groupoid model which is itself enriched.
\begin{coqdoccode}
\coqdocemptyline
\coqdocnoindent
\coqdockw{Class} \coqdef{Groupoid.groupoid.CategoryP}{$\mathsf{PreCategory}$}{\coqdocrecord{$\mathsf{PreCategory}$}} \coqdocvar{T} := \{ \coqdef{Groupoid.groupoid.Hom1}{$\mathsf{Hom}_1$}{\coqdocprojection{$\mathsf{Hom}_1$}} :> \coqref{Groupoid.groupoid.HomT1}{\coqdocclass{$\mathsf{HomSet}_1$}} \coqdocvariable{T}; \coqdef{Groupoid.groupoid.Hom2}{$\mathsf{Hom}_2$}{\coqdocprojection{$\mathsf{Hom}_2$}} :> \coqref{Groupoid.groupoid.HomT2}{\coqdocclass{$\mathsf{HomSet}_2$}} \coqref{Groupoid.groupoid.eq1}{\coqdocprojection{$\coqdoccst{eq}_1$}};\coqdoceol
\coqdocindent{2.50em}
\coqdef{Groupoid.groupoid.Id}{Id}{\coqdocprojection{Id}} :> \coqref{Groupoid.groupoid.Identity}{\coqdocclass{Identity}} \coqref{Groupoid.groupoid.eq1}{\coqdocprojection{$\coqdoccst{eq}_1$}}; \coqdef{Groupoid.groupoid.Comp}{Comp}{\coqdocprojection{Comp}} :> \coqref{Groupoid.groupoid.Composition}{\coqdocclass{Composition}} \coqref{Groupoid.groupoid.eq1}{\coqdocprojection{$\coqdoccst{eq}_1$}};\coqdoceol
\coqdocindent{2.50em}
\coqdef{Groupoid.groupoid.Equivalence 2}{$\mathsf{Equivalence}_2$}{\coqdocprojection{$\mathsf{Equivalence}_2$}} :> \coqexternalref{:type scope:'xE2x88x80' x '..' x ',' x}{http://coq.inria.fr/stdlib/Coq.Unicode.Utf8\_core}{\coqdocnotation{∀}} \coqdocvar{x} \coqdocvar{y}\coqexternalref{:type scope:'xE2x88x80' x '..' x ',' x}{http://coq.inria.fr/stdlib/Coq.Unicode.Utf8\_core}{\coqdocnotation{,}} \coqexternalref{:type scope:'xE2x88x80' x '..' x ',' x}{http://coq.inria.fr/stdlib/Coq.Unicode.Utf8\_core}{\coqdocnotation{(}}\coqref{Groupoid.groupoid.Equivalence}{\coqdocclass{Equivalence}} (\coqref{Groupoid.groupoid.eq2}{\coqdocprojection{$\coqdoccst{eq}_2$}} (\coqdocvar{x}:=\coqdocvariable{x}) (\coqdocvar{y}:=\coqdocvariable{y}))\coqexternalref{:type scope:'xE2x88x80' x '..' x ',' x}{http://coq.inria.fr/stdlib/Coq.Unicode.Utf8\_core}{\coqdocnotation{)}};\coqdoceol
\coqdocindent{2.50em}
\coqdef{Groupoid.groupoid.id R}{$\coqdoccst{id}_R$}{\coqdocprojection{$\coqdoccst{id}_R$}} : \coqexternalref{:type scope:'xE2x88x80' x '..' x ',' x}{http://coq.inria.fr/stdlib/Coq.Unicode.Utf8\_core}{\coqdocnotation{∀}} \coqdocvar{x} \coqdocvar{y} (\coqdocvar{f} : \coqdocvariable{x} \coqref{Groupoid.groupoid.::x 'x7E1' x}{\coqdocnotation{$\sim_1$}} \coqdocvariable{y}\coqexternalref{:type scope:'xE2x88x80' x '..' x ',' x}{http://coq.inria.fr/stdlib/Coq.Unicode.Utf8\_core}{\coqdocnotation{),}} \coqdocvariable{f} \coqref{Groupoid.groupoid.::x 'xC2xB0' x}{\coqdocnotation{$\circ$}} \coqref{Groupoid.groupoid.identity}{\coqdocprojection{identity}} \coqdocvariable{x} \coqref{Groupoid.groupoid.::x 'x7E' x}{\coqdocnotation{$\sim_2$}} \coqdocvariable{f} ;\coqdoceol
\coqdocindent{2.50em}
\coqdef{Groupoid.groupoid.id L}{$\coqdoccst{id}_L$}{\coqdocprojection{$\coqdoccst{id}_L$}} : \coqexternalref{:type scope:'xE2x88x80' x '..' x ',' x}{http://coq.inria.fr/stdlib/Coq.Unicode.Utf8\_core}{\coqdocnotation{∀}} \coqdocvar{x} \coqdocvar{y} (\coqdocvar{f} : \coqdocvariable{x} \coqref{Groupoid.groupoid.::x 'x7E1' x}{\coqdocnotation{$\sim_1$}} \coqdocvariable{y}\coqexternalref{:type scope:'xE2x88x80' x '..' x ',' x}{http://coq.inria.fr/stdlib/Coq.Unicode.Utf8\_core}{\coqdocnotation{),}} \coqref{Groupoid.groupoid.identity}{\coqdocprojection{identity}} \coqdocvariable{y} \coqref{Groupoid.groupoid.::x 'xC2xB0' x}{\coqdocnotation{$\circ$}} \coqdocvariable{f} \coqref{Groupoid.groupoid.::x 'x7E' x}{\coqdocnotation{$\sim_2$}} \coqdocvariable{f} ;\coqdoceol
\coqdocindent{2.50em}
\coqdef{Groupoid.groupoid.assoc}{assoc}{\coqdocprojection{assoc}} : \coqexternalref{:type scope:'xE2x88x80' x '..' x ',' x}{http://coq.inria.fr/stdlib/Coq.Unicode.Utf8\_core}{\coqdocnotation{∀}} \coqdocvar{x} \coqdocvar{y} \coqdocvar{z} \coqdocvar{w} (\coqdocvar{f}: \coqdocvariable{x} \coqref{Groupoid.groupoid.::x 'x7E1' x}{\coqdocnotation{$\sim_1$}} \coqdocvariable{y}) (\coqdocvar{g}: \coqdocvariable{y} \coqref{Groupoid.groupoid.::x 'x7E1' x}{\coqdocnotation{$\sim_1$}} \coqdocvariable{z}) (\coqdocvar{h}: \coqdocvariable{z} \coqref{Groupoid.groupoid.::x 'x7E1' x}{\coqdocnotation{$\sim_1$}} \coqdocvariable{w}\coqexternalref{:type scope:'xE2x88x80' x '..' x ',' x}{http://coq.inria.fr/stdlib/Coq.Unicode.Utf8\_core}{\coqdocnotation{),}}\coqdoceol
\coqdocindent{7.00em}
\coqref{Groupoid.groupoid.::x 'xC2xB0' x}{\coqdocnotation{(}}\coqdocvariable{h} \coqref{Groupoid.groupoid.::x 'xC2xB0' x}{\coqdocnotation{$\circ$}} \coqdocvariable{g}\coqref{Groupoid.groupoid.::x 'xC2xB0' x}{\coqdocnotation{)}} \coqref{Groupoid.groupoid.::x 'xC2xB0' x}{\coqdocnotation{$\circ$}} \coqdocvariable{f} \coqref{Groupoid.groupoid.::x 'x7E' x}{\coqdocnotation{$\sim_2$}} \coqdocvariable{h} \coqref{Groupoid.groupoid.::x 'xC2xB0' x}{\coqdocnotation{$\circ$}} \coqref{Groupoid.groupoid.::x 'xC2xB0' x}{\coqdocnotation{(}}\coqdocvariable{g} \coqref{Groupoid.groupoid.::x 'xC2xB0' x}{\coqdocnotation{$\circ$}} \coqdocvariable{f}\coqref{Groupoid.groupoid.::x 'xC2xB0' x}{\coqdocnotation{)}};\coqdoceol
\coqdocindent{2.50em}
\coqdef{Groupoid.groupoid.comp}{comp}{\coqdocprojection{comp}} : \coqexternalref{:type scope:'xE2x88x80' x '..' x ',' x}{http://coq.inria.fr/stdlib/Coq.Unicode.Utf8\_core}{\coqdocnotation{∀}} \coqdocvar{x} \coqdocvar{y} \coqdocvar{z} (\coqdocvar{f} \coqdocvar{f'}: \coqdocvariable{x} \coqref{Groupoid.groupoid.::x 'x7E1' x}{\coqdocnotation{$\sim_1$}} \coqdocvariable{y}) (\coqdocvar{g} \coqdocvar{g'}: \coqdocvariable{y} \coqref{Groupoid.groupoid.::x 'x7E1' x}{\coqdocnotation{$\sim_1$}} \coqdocvariable{z}\coqexternalref{:type scope:'xE2x88x80' x '..' x ',' x}{http://coq.inria.fr/stdlib/Coq.Unicode.Utf8\_core}{\coqdocnotation{),}} \coqdoceol
\coqdocindent{7.00em}
\coqdocvariable{f} \coqref{Groupoid.groupoid.::x 'x7E' x}{\coqdocnotation{$\sim_2$}} \coqdocvariable{f'} \coqexternalref{:type scope:x '->' x}{http://coq.inria.fr/stdlib/Coq.Init.Logic}{\coqdocnotation{\ensuremath{\rightarrow}}} \coqdocvariable{g} \coqref{Groupoid.groupoid.::x 'x7E' x}{\coqdocnotation{$\sim_2$}} \coqdocvariable{g'} \coqexternalref{:type scope:x '->' x}{http://coq.inria.fr/stdlib/Coq.Init.Logic}{\coqdocnotation{\ensuremath{\rightarrow}}} \coqdocvariable{g} \coqref{Groupoid.groupoid.::x 'xC2xB0' x}{\coqdocnotation{$\circ$}} \coqdocvariable{f} \coqref{Groupoid.groupoid.::x 'x7E' x}{\coqdocnotation{$\sim_2$}} \coqdocvariable{g'} \coqref{Groupoid.groupoid.::x 'xC2xB0' x}{\coqdocnotation{$\circ$}} \coqdocvariable{f'} \}.\coqdoceol
\coqdocemptyline
\end{coqdoccode}
  In homotopy type theory, coherences are expressed using identity types, with a further requirement that the internal notion of equality induced by the category (isomorphism between two objects) coincides with its identity type.  
  We do not share this point of view because our goal is to restrict the use of identity types to the treatment of contractedness for higher cells. 
  Note that the \coqref{Groupoid.groupoid.comp}{\coqdocprojection{comp}} law is not present in traditional definition of categories 
  because it is automatically satisfied for the identity type.
\begin{coqdoccode}
\end{coqdoccode}


 A \coqref{Groupoid.groupoid.GroupoidP}{\coqdocrecord{$\mathsf{PreGroupoid}$}} is a \coqref{Groupoid.groupoid.CategoryP}{\coqdocrecord{$\mathsf{PreCategory}$}} where all 1-Homs are invertible
 and subject to additional compatibility laws for inverses.
\begin{coqdoccode}
\coqdocemptyline
\coqdocnoindent
\coqdockw{Class} \coqdef{Groupoid.groupoid.GroupoidP}{$\mathsf{PreGroupoid}$}{\coqdocrecord{$\mathsf{PreGroupoid}$}} \coqdocvar{T} := \{ \coqdef{Groupoid.groupoid.C}{C}{\coqdocprojection{C}} :> \coqref{Groupoid.groupoid.CategoryP}{\coqdocclass{$\mathsf{PreCategory}$}} \coqdocvariable{T} ;  \coqdef{Groupoid.groupoid.Inv}{Inv}{\coqdocprojection{Inv}} :> \coqref{Groupoid.groupoid.Inverse}{\coqdocclass{Inverse}} \coqref{Groupoid.groupoid.eq1}{\coqdocprojection{$\coqdoccst{eq}_1$}} ;\coqdoceol
\coqdocindent{2.50em}
\coqdef{Groupoid.groupoid.inv R}{$\coqdoccst{inv}_R$}{\coqdocprojection{$\coqdoccst{inv}_R$}} : \coqexternalref{:type scope:'xE2x88x80' x '..' x ',' x}{http://coq.inria.fr/stdlib/Coq.Unicode.Utf8\_core}{\coqdocnotation{∀}} \coqdocvar{x} \coqdocvar{y} (\coqdocvar{f}: \coqdocvariable{x} \coqref{Groupoid.groupoid.::x 'x7E1' x}{\coqdocnotation{$\sim_1$}} \coqdocvariable{y}\coqexternalref{:type scope:'xE2x88x80' x '..' x ',' x}{http://coq.inria.fr/stdlib/Coq.Unicode.Utf8\_core}{\coqdocnotation{),}} \coqdocvariable{f} \coqref{Groupoid.groupoid.::x 'xC2xB0' x}{\coqdocnotation{$\circ$}} \coqdocvariable{f} \coqref{Groupoid.groupoid.::x 'x5E-1'}{\coqdocnotation{$\hspace{-1ex}^{-1}$}} \coqref{Groupoid.groupoid.::x 'x7E' x}{\coqdocnotation{$\sim_2$}} \coqref{Groupoid.groupoid.identity}{\coqdocprojection{identity}} \coqdocvariable{y} ;\coqdoceol
\coqdocindent{2.50em}
\coqdef{Groupoid.groupoid.inv L}{$\coqdoccst{inv}_L$}{\coqdocprojection{$\coqdoccst{inv}_L$}} : \coqexternalref{:type scope:'xE2x88x80' x '..' x ',' x}{http://coq.inria.fr/stdlib/Coq.Unicode.Utf8\_core}{\coqdocnotation{∀}} \coqdocvar{x} \coqdocvar{y} (\coqdocvar{f}: \coqdocvariable{x} \coqref{Groupoid.groupoid.::x 'x7E1' x}{\coqdocnotation{$\sim_1$}} \coqdocvariable{y}\coqexternalref{:type scope:'xE2x88x80' x '..' x ',' x}{http://coq.inria.fr/stdlib/Coq.Unicode.Utf8\_core}{\coqdocnotation{),}} \coqdocvariable{f} \coqref{Groupoid.groupoid.::x 'x5E-1'}{\coqdocnotation{$\hspace{-1ex}^{-1}$}} \coqref{Groupoid.groupoid.::x 'xC2xB0' x}{\coqdocnotation{$\circ$}} \coqdocvariable{f} \coqref{Groupoid.groupoid.::x 'x7E' x}{\coqdocnotation{$\sim_2$}} \coqref{Groupoid.groupoid.identity}{\coqdocprojection{identity}} \coqdocvariable{x} ;\coqdoceol
\coqdocindent{2.50em}
\coqdef{Groupoid.groupoid.inv}{inv}{\coqdocprojection{inv}} :   \coqexternalref{:type scope:'xE2x88x80' x '..' x ',' x}{http://coq.inria.fr/stdlib/Coq.Unicode.Utf8\_core}{\coqdocnotation{∀}} \coqdocvar{x} \coqdocvar{y} (\coqdocvar{f} \coqdocvar{f'}: \coqdocvariable{x} \coqref{Groupoid.groupoid.::x 'x7E1' x}{\coqdocnotation{$\sim_1$}} \coqdocvariable{y}\coqexternalref{:type scope:'xE2x88x80' x '..' x ',' x}{http://coq.inria.fr/stdlib/Coq.Unicode.Utf8\_core}{\coqdocnotation{),}} \coqdocvariable{f} \coqref{Groupoid.groupoid.::x 'x7E' x}{\coqdocnotation{$\sim_2$}} \coqdocvariable{f'} \coqexternalref{:type scope:x '->' x}{http://coq.inria.fr/stdlib/Coq.Init.Logic}{\coqdocnotation{\ensuremath{\rightarrow}}} \coqdocvariable{f} \coqref{Groupoid.groupoid.::x 'x5E-1'}{\coqdocnotation{$\hspace{-1ex}^{-1}$}} \coqref{Groupoid.groupoid.::x 'x7E' x}{\coqdocnotation{$\sim_2$}} \coqdocvariable{f'} \coqref{Groupoid.groupoid.::x 'x5E-1'}{\coqdocnotation{$\hspace{-1ex}^{-1}$}}\}.\coqdoceol
\coqdocemptyline
\end{coqdoccode}
   Groupoids are then pre-groupoids where equality at
   dimension 2 is irrelevant. This irrelevance is defined using a
   notion of contractibility expressed with (relevant) identity types.  \begin{coqdoccode}
\coqdocemptyline
\end{coqdoccode}
This is a way to require that all higher-cells are trivial. In our setting, we do not have the possibility to say that all 2-cells are related by a 3-cell, and so on. The price to pay will be explicit reasoning on identity types when proving for instance contractedness for the function space. In that case, we need the axiom of functional extensionality.
By analogy to homotopy type theory, we note \coqref{Groupoid.groupoid.Groupoid}{\coqdocrecord{$\mathsf{IsType_1}$}} the property of being a groupoid. 
\begin{coqdoccode}
\coqdocemptyline
\coqdocnoindent
\coqdockw{Class} \coqdef{Groupoid.groupoid.Groupoid}{$\mathsf{IsType_1}$}{\coqdocrecord{$\mathsf{IsType_1}$}} \coqdocvar{T} := \{ \coqdef{Groupoid.groupoid.G}{G}{\coqdocprojection{G}} :> \coqref{Groupoid.groupoid.GroupoidP}{\coqdocclass{$\mathsf{PreGroupoid}$}} \coqdocvariable{T} ;\coqdoceol
\coqdocindent{1.00em}
\coqdef{Groupoid.groupoid.is Trunc 2}{is\_Trunc\_2}{\coqdocprojection{is\_Trunc\_2}} : \coqexternalref{:type scope:'xE2x88x80' x '..' x ',' x}{http://coq.inria.fr/stdlib/Coq.Unicode.Utf8\_core}{\coqdocnotation{∀}} \coqexternalref{:type scope:'xE2x88x80' x '..' x ',' x}{http://coq.inria.fr/stdlib/Coq.Unicode.Utf8\_core}{\coqdocnotation{(}}\coqdocvar{x} \coqdocvar{y} : \coqdocvariable{T}) (\coqdocvar{e} \coqdocvar{e'} : \coqdocvariable{x} \coqref{Groupoid.groupoid.::x 'x7E1' x}{\coqdocnotation{$\sim_1$}} \coqdocvariable{y}) (\coqdocvar{E} \coqdocvar{E'} : \coqdocvariable{e} \coqref{Groupoid.groupoid.::x 'x7E2' x}{\coqdocnotation{$\sim_2$}} \coqdocvariable{e'}\coqexternalref{:type scope:'xE2x88x80' x '..' x ',' x}{http://coq.inria.fr/stdlib/Coq.Unicode.Utf8\_core}{\coqdocnotation{),}} \coqdocclass{Contr} (\coqdocvariable{E} \coqdocnotation{=} \coqdocvariable{E'})\}.\coqdoceol
\coqdocemptyline
\end{coqdoccode}
\noindent 
    In the same way, we define \coqref{Groupoid.groupoid.Setoid}{\coqdocrecord{$\mathsf{IsType_0}$}} when equality is irrelevant at dimension 1.
\begin{coqdoccode}
\coqdocemptyline
\coqdocnoindent
\coqdockw{Class} \coqdef{Groupoid.groupoid.Setoid}{$\mathsf{IsType_0}$}{\coqdocrecord{$\mathsf{IsType_0}$}} \coqdocvar{T} := \{ \coqdef{Groupoid.groupoid.S}{S}{\coqdocprojection{S}} :> \coqref{Groupoid.groupoid.Groupoid}{\coqdocclass{$\mathsf{IsType_1}$}} \coqdocvariable{T} ; \coqdoceol
\coqdocindent{1.00em}
\coqdef{Groupoid.groupoid.is Trunc 1}{is\_Trunc\_1}{\coqdocprojection{is\_Trunc\_1}} : \coqexternalref{:type scope:'xE2x88x80' x '..' x ',' x}{http://coq.inria.fr/stdlib/Coq.Unicode.Utf8\_core}{\coqdocnotation{∀}} \coqexternalref{:type scope:'xE2x88x80' x '..' x ',' x}{http://coq.inria.fr/stdlib/Coq.Unicode.Utf8\_core}{\coqdocnotation{(}}\coqdocvar{x} \coqdocvar{y} : \coqdocvariable{T}) (\coqdocvar{e} \coqdocvar{e'} : \coqdocvariable{x} \coqref{Groupoid.groupoid.::x 'x7E1' x}{\coqdocnotation{$\sim_1$}} \coqdocvariable{y}\coqexternalref{:type scope:'xE2x88x80' x '..' x ',' x}{http://coq.inria.fr/stdlib/Coq.Unicode.Utf8\_core}{\coqdocnotation{)}} \coqexternalref{:type scope:'xE2x88x80' x '..' x ',' x}{http://coq.inria.fr/stdlib/Coq.Unicode.Utf8\_core}{\coqdocnotation{,}} \coqdocclass{Contr} (\coqdocvariable{e} \coqdocnotation{=} \coqdocvariable{e'})\}.\coqdoceol
\coqdocemptyline
\end{coqdoccode}
   We note \coqref{Groupoid.groupoid.GroupoidType}{\coqdocdefinition{$\mathsf{Type_1}$}} for the types that form a \coqref{Groupoid.groupoid.Groupoid}{\coqdocrecord{$\mathsf{IsType_1}$}}.
       The subscript $1$ comes from the fact that groupoids are 1-truncated types
       in homotopy type theory. In the same way,  
       we note \coqref{Groupoid.groupoid.SetoidType}{\coqdocdefinition{$\mathsf{Type_0}$}} 
       for the types that form a \coqref{Groupoid.groupoid.Setoid}{\coqdocrecord{$\mathsf{IsType_0}$}}.
       We define   \coqdocvariable{T}$_{\upharpoonright s}$  the lifting of setoids (inhabitants of \coqref{Groupoid.groupoid.SetoidType}{\coqdocdefinition{$\mathsf{Type_0}$}}) to groupoids.
\begin{coqdoccode}
\end{coqdoccode}
\subsection{Functors and natural transformations}


\label{sec:funextnat}
A morphism between two groupoids is a functor, i.e., a function
between objects of the groupoids that transports homs and
subject to compatibility laws. 
\begin{coqdoccode}
\coqdocemptyline
\coqdocnoindent
\coqdockw{Class} \coqdef{Groupoid.groupoid.Functor}{Functor}{\coqdocrecord{Functor}} \{\coqdocvar{T} \coqdocvar{U} : \coqref{Groupoid.groupoid.UGroupoidType}{\coqdocdefinition{$\mathsf{Type_{1}}$}}\} (\coqdocvar{f} : \coqref{Groupoid.groupoid.::'[' x ']'}{\coqdocnotation{[}}\coqdocvariable{T}\coqref{Groupoid.groupoid.::'[' x ']'}{\coqdocnotation{]}} \coqexternalref{:type scope:x '->' x}{http://coq.inria.fr/stdlib/Coq.Init.Logic}{\coqdocnotation{\ensuremath{\rightarrow}}} \coqref{Groupoid.groupoid.::'[' x ']'}{\coqdocnotation{[}}\coqdocvariable{U}\coqref{Groupoid.groupoid.::'[' x ']'}{\coqdocnotation{]}}) : \coqdockw{Type} :=\coqdoceol
\coqdocnoindent
\{ \coqdef{Groupoid.groupoid. map}{$\coqdoccst{map}$}{\coqdocprojection{$\coqdoccst{map}$}} : \coqexternalref{:type scope:'xE2x88x80' x '..' x ',' x}{http://coq.inria.fr/stdlib/Coq.Unicode.Utf8\_core}{\coqdocnotation{∀}} \{\coqdocvar{x} \coqdocvar{y}\}\coqexternalref{:type scope:'xE2x88x80' x '..' x ',' x}{http://coq.inria.fr/stdlib/Coq.Unicode.Utf8\_core}{\coqdocnotation{,}} \coqdocvariable{x} \coqref{Groupoid.groupoid.::x 'x7E1' x}{\coqdocnotation{$\sim_1$}} \coqdocvariable{y} \coqexternalref{:type scope:x '->' x}{http://coq.inria.fr/stdlib/Coq.Init.Logic}{\coqdocnotation{\ensuremath{\rightarrow}}} \coqdocvariable{f} \coqdocvariable{x} \coqref{Groupoid.groupoid.::x 'x7E1' x}{\coqdocnotation{$\sim_1$}} \coqdocvariable{f} \coqdocvariable{y} ;\coqdoceol
\coqdocindent{1.00em}
\coqdef{Groupoid.groupoid. map id}{$\coqdoccst{map}_\coqdoccst{id}$}{\coqdocprojection{$\coqdoccst{map}_\coqdoccst{id}$}} : \coqexternalref{:type scope:'xE2x88x80' x '..' x ',' x}{http://coq.inria.fr/stdlib/Coq.Unicode.Utf8\_core}{\coqdocnotation{∀}} \{\coqdocvar{x}\}\coqexternalref{:type scope:'xE2x88x80' x '..' x ',' x}{http://coq.inria.fr/stdlib/Coq.Unicode.Utf8\_core}{\coqdocnotation{,}} \coqref{Groupoid.groupoid. map}{\coqdocmethod{$\coqdoccst{map}$}} (\coqref{Groupoid.groupoid.identity}{\coqdocprojection{identity}} \coqdocvariable{x}) \coqref{Groupoid.groupoid.::x 'x7E' x}{\coqdocnotation{$\sim_2$}} \coqref{Groupoid.groupoid.identity}{\coqdocprojection{identity}} (\coqdocvariable{f} \coqdocvariable{x}) ;\coqdoceol
\coqdocindent{1.00em}
\coqdef{Groupoid.groupoid. map comp}{$\coqdoccst{map}_\coqdoccst{comp}$}{\coqdocprojection{$\coqdoccst{map}_\coqdoccst{comp}$}} : \coqexternalref{:type scope:'xE2x88x80' x '..' x ',' x}{http://coq.inria.fr/stdlib/Coq.Unicode.Utf8\_core}{\coqdocnotation{∀}} \{\coqdocvar{x} \coqdocvar{y} \coqdocvar{z}\} (\coqdocvar{e}:\coqdocvariable{x} \coqref{Groupoid.groupoid.::x 'x7E1' x}{\coqdocnotation{$\sim_1$}} \coqdocvariable{y}) (\coqdocvar{e'}:\coqdocvariable{y} \coqref{Groupoid.groupoid.::x 'x7E1' x}{\coqdocnotation{$\sim_1$}} \coqdocvariable{z}\coqexternalref{:type scope:'xE2x88x80' x '..' x ',' x}{http://coq.inria.fr/stdlib/Coq.Unicode.Utf8\_core}{\coqdocnotation{),}} \coqref{Groupoid.groupoid. map}{\coqdocmethod{$\coqdoccst{map}$}} (\coqdocvariable{e'} \coqref{Groupoid.groupoid.::x 'xC2xB0' x}{\coqdocnotation{$\circ$}} \coqdocvariable{e}) \coqref{Groupoid.groupoid.::x 'x7E2' x}{\coqdocnotation{$\sim_2$}} \coqref{Groupoid.groupoid. map}{\coqdocmethod{$\coqdoccst{map}$}} \coqdocvariable{e'} \coqref{Groupoid.groupoid.::x 'xC2xB0' x}{\coqdocnotation{$\circ$}} \coqref{Groupoid.groupoid. map}{\coqdocmethod{$\coqdoccst{map}$}} \coqdocvariable{e} ;\coqdoceol
\coqdocindent{1.00em}
\coqdef{Groupoid.groupoid. map2}{$\coqdoccst{map}_2$}{\coqdocprojection{$\coqdoccst{map}_2$}} : \coqexternalref{:type scope:'xE2x88x80' x '..' x ',' x}{http://coq.inria.fr/stdlib/Coq.Unicode.Utf8\_core}{\coqdocnotation{∀}} \coqexternalref{:type scope:'xE2x88x80' x '..' x ',' x}{http://coq.inria.fr/stdlib/Coq.Unicode.Utf8\_core}{\coqdocnotation{\{}}\coqdocvar{x} \coqdocvar{y}:\coqref{Groupoid.groupoid.::'[' x ']'}{\coqdocnotation{[}}\coqdocvariable{T}\coqref{Groupoid.groupoid.::'[' x ']'}{\coqdocnotation{]}}\} \{\coqdocvar{e} \coqdocvar{e'} : \coqdocvariable{x} \coqref{Groupoid.groupoid.::x 'x7E1' x}{\coqdocnotation{$\sim_1$}} \coqdocvariable{y}\coqexternalref{:type scope:'xE2x88x80' x '..' x ',' x}{http://coq.inria.fr/stdlib/Coq.Unicode.Utf8\_core}{\coqdocnotation{\},}} \coqexternalref{:type scope:x '->' x}{http://coq.inria.fr/stdlib/Coq.Init.Logic}{\coqdocnotation{(}}\coqdocvariable{e} \coqref{Groupoid.groupoid.::x 'x7E2' x}{\coqdocnotation{$\sim_2$}} \coqdocvariable{e'}\coqexternalref{:type scope:x '->' x}{http://coq.inria.fr/stdlib/Coq.Init.Logic}{\coqdocnotation{)}} \coqexternalref{:type scope:x '->' x}{http://coq.inria.fr/stdlib/Coq.Init.Logic}{\coqdocnotation{\ensuremath{\rightarrow}}} \coqref{Groupoid.groupoid. map}{\coqdocmethod{$\coqdoccst{map}$}}  \coqdocvariable{e} \coqref{Groupoid.groupoid.::x 'x7E2' x}{\coqdocnotation{$\sim_2$}} \coqref{Groupoid.groupoid. map}{\coqdocmethod{$\coqdoccst{map}$}} \coqdocvariable{e'} \}.\coqdoceol
\coqdocemptyline
\coqdocnoindent
\coqdockw{Definition} \coqdef{Groupoid.groupoid.Fun Type}{Fun\_Type}{\coqdocdefinition{Fun\_Type}} (\coqdocvar{T} \coqdocvar{U} : \coqref{Groupoid.groupoid.UGroupoidType}{\coqdocdefinition{$\mathsf{Type_{1}}$}}) := \coqdocnotation{\{}\coqdocvar{f} \coqdocnotation{:} \coqref{Groupoid.groupoid.::'[' x ']'}{\coqdocnotation{[}}\coqdocvariable{T}\coqref{Groupoid.groupoid.::'[' x ']'}{\coqdocnotation{]}} \coqexternalref{:type scope:x '->' x}{http://coq.inria.fr/stdlib/Coq.Init.Logic}{\coqdocnotation{\ensuremath{\rightarrow}}} \coqref{Groupoid.groupoid.::'[' x ']'}{\coqdocnotation{[}}\coqdocvariable{U}\coqref{Groupoid.groupoid.::'[' x ']'}{\coqdocnotation{]}} \coqdocnotation{\&} \coqref{Groupoid.groupoid.Functor}{\coqdocclass{Functor}} \coqdocvar{f}\coqdocnotation{\}}.\coqdoceol
\coqdocemptyline
\end{coqdoccode}
\noindent We note \coqdocvariable{T} $\longrightarrow$ \coqdocvariable{U} the type of functors from \coqdocvariable{T} to \coqdocvariable{U}.
Note that we only impose compatibility with the composition as
compatibilities with identities and inverse Homs can be deduced from
it. We note \coqdocvariable{M} $\star$ \coqdocvariable{N} the application of a function \coqdocvariable{M} in the first
component of a dependent pair. \begin{coqdoccode}
\end{coqdoccode}
Equivalence between functors is given by natural transformations.
  We insist here that this naturality condition in the definition of
  functor equality is crucial in a higher setting.  It is usually
  derivable in formalizations of homotopy theory in Coq because there they
  only consider the 1-groupoid case where the naturality comes for
  free from functional extensionality, see for instance~\cite{coq_unival_axiom}.  \begin{coqdoccode}
\coqdocemptyline
\coqdocnoindent
\coqdockw{Class} \coqdef{Groupoid.groupoid.NaturalTransformation}{$\mathsf{NaturalTrans}$}{\coqdocrecord{$\mathsf{NaturalTrans}$}} \coqdocvar{T} \coqdocvar{U} \{\coqdocvar{f} \coqdocvar{g} : \coqdocvariable{T} \coqref{Groupoid.groupoid.::x '--->' x}{\coqdocnotation{$\longrightarrow$}} \coqdocvariable{U}\} (\coqdocvar{α} : \coqexternalref{:type scope:'xE2x88x80' x '..' x ',' x}{http://coq.inria.fr/stdlib/Coq.Unicode.Utf8\_core}{\coqdocnotation{∀}} \coqdocvar{t} : \coqref{Groupoid.groupoid.::'[' x ']'}{\coqdocnotation{[}}\coqdocvariable{T}\coqref{Groupoid.groupoid.::'[' x ']'}{\coqdocnotation{]}}\coqexternalref{:type scope:'xE2x88x80' x '..' x ',' x}{http://coq.inria.fr/stdlib/Coq.Unicode.Utf8\_core}{\coqdocnotation{,}} \coqdocvariable{f} \coqref{Groupoid.groupoid.::x '@' x}{\coqdocnotation{$\star$}} \coqdocvariable{t} \coqref{Groupoid.groupoid.::x 'x7E1' x}{\coqdocnotation{$\sim_1$}} \coqdocvariable{g} \coqref{Groupoid.groupoid.::x '@' x}{\coqdocnotation{$\star$}} \coqdocvariable{t}) := \coqdoceol
\coqdocindent{1.00em}
\coqdef{Groupoid.groupoid. xCExB1 map}{$\coqdoccst{α}_\mathsf{map}$}{\coqdocprojection{$\coqdoccst{α}_\mathsf{map}$}} : \coqexternalref{:type scope:'xE2x88x80' x '..' x ',' x}{http://coq.inria.fr/stdlib/Coq.Unicode.Utf8\_core}{\coqdocnotation{∀}} \{\coqdocvar{t} \coqdocvar{t'}\} (\coqdocvar{e} : \coqdocvariable{t} \coqref{Groupoid.groupoid.::x 'x7E1' x}{\coqdocnotation{$\sim_1$}} \coqdocvariable{t'}\coqexternalref{:type scope:'xE2x88x80' x '..' x ',' x}{http://coq.inria.fr/stdlib/Coq.Unicode.Utf8\_core}{\coqdocnotation{),}} \coqdocvariable{α} \coqdocvariable{t'} \coqref{Groupoid.groupoid.::x 'xC2xB0' x}{\coqdocnotation{$\circ$}} \coqref{Groupoid.groupoid.map}{\coqdocabbreviation{map}} \coqdocvariable{f} \coqdocvariable{e} \coqref{Groupoid.groupoid.::x 'x7E' x}{\coqdocnotation{$\sim_2$}} \coqref{Groupoid.groupoid.map}{\coqdocabbreviation{map}} \coqdocvariable{g} \coqdocvariable{e} \coqref{Groupoid.groupoid.::x 'xC2xB0' x}{\coqdocnotation{$\circ$}} \coqdocvariable{α} \coqdocvariable{t}.\coqdoceol
\coqdocemptyline
\coqdocnoindent
\coqdockw{Definition} \coqdef{Groupoid.groupoid.nat trans}{nat\_trans}{\coqdocdefinition{nat\_trans}} \coqdocvar{T} \coqdocvar{U} : \coqref{Groupoid.groupoid.HomT}{\coqdocdefinition{$\mathsf{HomSet}$}} (\coqdocvariable{T} \coqref{Groupoid.groupoid.::x '--->' x}{\coqdocnotation{$\longrightarrow$}} \coqdocvariable{U}) \coqdoceol
\coqdocindent{0.50em}
:= \coqexternalref{::'xCExBB' x '..' x ',' x}{http://coq.inria.fr/stdlib/Coq.Unicode.Utf8\_core}{\coqdocnotation{\ensuremath{\lambda}}} \coqdocvar{f} \coqdocvar{g}\coqexternalref{::'xCExBB' x '..' x ',' x}{http://coq.inria.fr/stdlib/Coq.Unicode.Utf8\_core}{\coqdocnotation{,}} \coqdocnotation{\{}\coqdocvar{α} \coqdocnotation{:} \coqexternalref{:type scope:'xE2x88x80' x '..' x ',' x}{http://coq.inria.fr/stdlib/Coq.Unicode.Utf8\_core}{\coqdocnotation{∀}} \coqdocvar{t} : \coqref{Groupoid.groupoid.::'[' x ']'}{\coqdocnotation{[}}\coqdocvariable{T}\coqref{Groupoid.groupoid.::'[' x ']'}{\coqdocnotation{]}}\coqexternalref{:type scope:'xE2x88x80' x '..' x ',' x}{http://coq.inria.fr/stdlib/Coq.Unicode.Utf8\_core}{\coqdocnotation{,}} \coqdocvariable{f} \coqref{Groupoid.groupoid.::x '@' x}{\coqdocnotation{$\star$}} \coqdocvariable{t} \coqref{Groupoid.groupoid.::x 'x7E1' x}{\coqdocnotation{$\sim_1$}} \coqdocvariable{g} \coqref{Groupoid.groupoid.::x '@' x}{\coqdocnotation{$\star$}} \coqdocvariable{t} \coqdocnotation{\&} \coqref{Groupoid.groupoid.NaturalTransformation}{\coqdocclass{$\mathsf{NaturalTrans}$}} \coqdocvar{α}\coqdocnotation{\}}.\coqdoceol
\coqdocemptyline
\end{coqdoccode}
In our setting, equality between natural transformations is not expressed using identity types, but using the higher categorical notion of modification.
\begin{coqdoccode}
\coqdocemptyline
\coqdocnoindent
\coqdockw{Definition} \coqdef{Groupoid.groupoid.modification}{modification}{\coqdocdefinition{modification}} \coqdocvar{T} \coqdocvar{U} (\coqdocvar{f} \coqdocvar{g} : \coqdocvariable{T} \coqref{Groupoid.groupoid.::x '--->' x}{\coqdocnotation{$\longrightarrow$}} \coqdocvariable{U}) : \coqref{Groupoid.groupoid.HomT}{\coqdocdefinition{$\mathsf{HomSet}$}} (\coqdocvariable{f} \coqref{Groupoid.groupoid.::x 'x7E1' x}{\coqdocnotation{$\sim_1$}} \coqdocvariable{g}) \coqdoceol
\coqdocindent{1.00em}
:= \coqexternalref{::'xCExBB' x '..' x ',' x}{http://coq.inria.fr/stdlib/Coq.Unicode.Utf8\_core}{\coqdocnotation{\ensuremath{\lambda}}} \coqdocvar{α} \coqdocvar{$\beta$}\coqexternalref{::'xCExBB' x '..' x ',' x}{http://coq.inria.fr/stdlib/Coq.Unicode.Utf8\_core}{\coqdocnotation{,}} \coqexternalref{:type scope:'xE2x88x80' x '..' x ',' x}{http://coq.inria.fr/stdlib/Coq.Unicode.Utf8\_core}{\coqdocnotation{∀}} \coqdocvar{t} : \coqref{Groupoid.groupoid.::'[' x ']'}{\coqdocnotation{[}}\coqdocvariable{T}\coqref{Groupoid.groupoid.::'[' x ']'}{\coqdocnotation{]}}\coqexternalref{:type scope:'xE2x88x80' x '..' x ',' x}{http://coq.inria.fr/stdlib/Coq.Unicode.Utf8\_core}{\coqdocnotation{,}} \coqdocvariable{α} \coqref{Groupoid.groupoid.::x '@' x}{\coqdocnotation{$\star$}} \coqdocvariable{t} \coqref{Groupoid.groupoid.::x 'x7E' x}{\coqdocnotation{$\sim_2$}} \coqdocvariable{$\beta$} \coqref{Groupoid.groupoid.::x '@' x}{\coqdocnotation{$\star$}} \coqdocvariable{t}.\coqdoceol
\coqdocemptyline
\end{coqdoccode}
\noindent
    We can now equip the functor space with a groupoid structure. Note
    here that we (abusively) use the same notation for the functor type and 
    its corresponding groupoid. \begin{coqdoccode}
\coqdocemptyline
\coqdocnoindent
\coqdockw{Definition} \coqdef{Groupoid.groupoid. fun}{\_fun}{\coqdocdefinition{\_fun}} \coqdocvar{T} \coqdocvar{U} : \coqref{Groupoid.groupoid.UGroupoidType}{\coqdocdefinition{$\mathsf{Type_{1}}$}} := \coqdocnotation{(}\coqdocvariable{T} \coqref{Groupoid.groupoid.::x '--->' x}{\coqdocnotation{$\longrightarrow$}} \coqdocvariable{U} \coqdocnotation{;} \coqref{Groupoid.groupoid.nat trans grp}{\coqdocinstance{$\mathsf{fun_{grp}}$}} \coqdocvariable{T} \coqdocvariable{U}\coqdocnotation{)}.\coqdoceol
\coqdocemptyline
\end{coqdoccode}
 In the definition above, \coqref{Groupoid.groupoid.nat trans grp}{\coqdocinstance{$\mathsf{fun_{grp}}$}} is a proof that \coqref{Groupoid.groupoid.nat trans}{\coqdocdefinition{nat\_trans}} and \coqref{Groupoid.groupoid.modification}{\coqdocdefinition{modification}} form a groupoid on \coqdocvariable{T} $\longrightarrow$ \coqdocvariable{U}. In particular, it makes use of functional extensionality, which says that the canonical proof of \coqdocvariable{f} = \coqdocvariable{g} \ensuremath{\rightarrow} ∀ \coqdocvariable{x}, \coqdocvariable{f} \coqdocvariable{x} = \coqdocvariable{g} \coqdocvariable{x} is an equivalence (in the sense of homotopy type theory).   
 \begin{coqdoccode}
\coqdocemptyline
\end{coqdoccode}
\subsection{Homotopic equivalences}


 \label{sec:homequiv}   
    The standard notion of equivalence between groupoids is given by
    adjoint equivalences, that is a map with an \coqref{Groupoid.groupoid.adjoint}{\coqdocabbreviation{adjoint}} and two proofs
    that they form a \coqref{Groupoid.groupoid.section}{\coqdocabbreviation{section}} (or counit of the adjunction) and a
    \coqref{Groupoid.groupoid.retraction}{\coqdocabbreviation{retraction}} (or unit of the adjunction). \begin{coqdoccode}
\coqdocemptyline
\coqdocnoindent
\coqdockw{Class} \coqdef{Groupoid.groupoid.Iso struct}{Iso\_struct}{\coqdocrecord{Iso\_struct}} \coqdocvar{T} \coqdocvar{U} (\coqdocvar{f} : \coqref{Groupoid.groupoid.::'[' x ']'}{\coqdocnotation{[}}\coqdocvariable{T} \coqref{Groupoid.groupoid.::x '-->' x}{\coqdocnotation{$\longrightarrow$}} \coqdocvariable{U}\coqref{Groupoid.groupoid.::'[' x ']'}{\coqdocnotation{]}}) := \coqdoceol
\coqdocnoindent
\{ \coqdef{Groupoid.groupoid. adjoint}{$\coqdoccst{adjoint}$}{\coqdocprojection{$\coqdoccst{adjoint}$}} :    \coqref{Groupoid.groupoid.::'[' x ']'}{\coqdocnotation{[}}\coqdocvariable{U} \coqref{Groupoid.groupoid.::x '-->' x}{\coqdocnotation{$\longrightarrow$}} \coqdocvariable{T}\coqref{Groupoid.groupoid.::'[' x ']'}{\coqdocnotation{]}} ;\coqdoceol
\coqdocindent{1.00em}
\coqdef{Groupoid.groupoid. section}{$\coqdoccst{section}$}{\coqdocprojection{$\coqdoccst{section}$}} :    \coqdocvariable{f} \coqref{Groupoid.groupoid.::x 'xC2xB0' x}{\coqdocnotation{$\circ$}} \coqref{Groupoid.groupoid. adjoint}{\coqdocmethod{$\coqdoccst{adjoint}$}} \coqref{Groupoid.groupoid.::x 'x7E' x}{\coqdocnotation{$\sim_2$}} \coqref{Groupoid.groupoid.identity}{\coqdocprojection{identity}} \coqdocvariable{U} ;\coqdoceol
\coqdocindent{1.00em}
\coqdef{Groupoid.groupoid. retraction}{$\coqdoccst{retraction}$}{\coqdocprojection{$\coqdoccst{retraction}$}} : \coqref{Groupoid.groupoid. adjoint}{\coqdocmethod{$\coqdoccst{adjoint}$}} \coqref{Groupoid.groupoid.::x 'xC2xB0' x}{\coqdocnotation{$\circ$}} \coqdocvariable{f} \coqref{Groupoid.groupoid.::x 'x7E' x}{\coqdocnotation{$\sim_2$}} \coqref{Groupoid.groupoid.identity}{\coqdocprojection{identity}} \coqdocvariable{T}\}.\coqdoceol
\coqdocemptyline
\end{coqdoccode}
This type class defines usual equivalences. To get an adjoint
    equivalence, an additional triangle identity between sections and
    retractions is required. This allows to eliminate a section against
    a retraction in proofs. A corresponding triangle identity involving
    \coqref{Groupoid.groupoid.adjoint}{\coqdocabbreviation{adjoint}} \coqdocvariable{f} can also be expressed, but it can be shown that each
    condition implies the other.  \begin{coqdoccode}
\coqdocemptyline
\coqdocnoindent
\coqdockw{Class} \coqdef{Groupoid.groupoid.Equiv struct}{Equiv\_struct}{\coqdocrecord{Equiv\_struct}} \coqdocvar{T} \coqdocvar{U} (\coqdocvar{f} : \coqdocvariable{T} \coqref{Groupoid.groupoid.::x '--->' x}{\coqdocnotation{$\longrightarrow$}} \coqdocvariable{U}) := \coqdoceol
\coqdocnoindent
\{ \coqdef{Groupoid.groupoid.iso}{iso}{\coqdocprojection{iso}} :> \coqref{Groupoid.groupoid.Iso struct}{\coqdocclass{Iso\_struct}} \coqdocvariable{f};\coqdoceol
\coqdocindent{1.00em}
\coqdef{Groupoid.groupoid. triangle}{$\coqdoccst{triangle}$}{\coqdocprojection{$\coqdoccst{triangle}$}} : \coqexternalref{:type scope:'xE2x88x80' x '..' x ',' x}{http://coq.inria.fr/stdlib/Coq.Unicode.Utf8\_core}{\coqdocnotation{∀}} \coqdocvar{t}\coqexternalref{:type scope:'xE2x88x80' x '..' x ',' x}{http://coq.inria.fr/stdlib/Coq.Unicode.Utf8\_core}{\coqdocnotation{,}} \coqref{Groupoid.groupoid. section}{\coqdocprojection{$\coqdoccst{section}$}} \coqref{Groupoid.groupoid.::x '@' x}{\coqdocnotation{$\star$}} \coqref{Groupoid.groupoid.::x '@' x}{\coqdocnotation{(}}\coqdocvariable{f} \coqref{Groupoid.groupoid.::x '@' x}{\coqdocnotation{$\star$}} \coqdocvariable{t}\coqref{Groupoid.groupoid.::x '@' x}{\coqdocnotation{)}} \coqref{Groupoid.groupoid.::x 'x7E' x}{\coqdocnotation{$\sim_2$}} \coqref{Groupoid.groupoid.map}{\coqdocabbreviation{map}} \coqdocvariable{f} (\coqref{Groupoid.groupoid. retraction}{\coqdocprojection{$\coqdoccst{retraction}$}} \coqref{Groupoid.groupoid.::x '@' x}{\coqdocnotation{$\star$}} \coqdocvariable{t})\}.\coqdoceol
\coqdocemptyline
\coqdocnoindent
\coqdockw{Definition} \coqdef{Groupoid.groupoid.Equiv}{Equiv}{\coqdocdefinition{Equiv}} \coqdocvar{A} \coqdocvar{B} := \coqdocnotation{\{}\coqdocvar{f} \coqdocnotation{:} \coqdocvariable{A} \coqref{Groupoid.groupoid.::x '--->' x}{\coqdocnotation{$\longrightarrow$}} \coqdocvariable{B} \coqdocnotation{\&} \coqref{Groupoid.groupoid.Equiv struct}{\coqdocclass{Equiv\_struct}} \coqdocvar{f}\coqdocnotation{\}}.\coqdoceol
\coqdocemptyline
\end{coqdoccode}


   It is well known that any equivalence can be turned into an adjoint
   equivalence by slightly modifying the section. While available in
   our formalization, this result should be used with care as it
   opacifies the underlying notion of homotopy and can harden proofs.
\begin{coqdoccode}
\coqdocemptyline
\end{coqdoccode}
Equality of homotopy equivalences is given by equivalence of
  adjunctions. Two adjunctions are equivalent if their left adjoints are
  equivalent and they agree on their sections (up-to the isomorphism).
  Note that equivalence of the right adjoints and agreement on their
  retractions can be deduced so they are not part of the definition.  \begin{coqdoccode}
\coqdocemptyline
\coqdocnoindent
\coqdockw{Class} \coqdef{Groupoid.groupoid.EquivEq}{EquivEq}{\coqdocrecord{EquivEq}} \{\coqdocvar{T} \coqdocvar{U}\} \{\coqdocvar{f} \coqdocvar{g} : \coqref{Groupoid.groupoid.Equiv}{\coqdocdefinition{Equiv}} \coqdocvariable{T} \coqdocvariable{U}\} (\coqdocvar{α} : \coqref{Groupoid.groupoid.::'[' x ']'}{\coqdocnotation{[}}\coqdocvariable{f}\coqref{Groupoid.groupoid.::'[' x ']'}{\coqdocnotation{]}} \coqref{Groupoid.groupoid.::x 'x7E' x}{\coqdocnotation{$\sim_2$}} \coqref{Groupoid.groupoid.::'[' x ']'}{\coqdocnotation{[}}\coqdocvariable{g}\coqref{Groupoid.groupoid.::'[' x ']'}{\coqdocnotation{]}}) : \coqdockw{Type} :=  \coqdoceol
\coqdocindent{0.50em}
\coqdef{Groupoid.groupoid. eq section}{\_eq\_section}{\coqdocprojection{\_eq\_section}} : \coqref{Groupoid.groupoid.section}{\coqdocabbreviation{section}} \coqdocvariable{f} \coqref{Groupoid.groupoid.::x 'x7E' x}{\coqdocnotation{$\sim_2$}} \coqref{Groupoid.groupoid.section}{\coqdocabbreviation{section}} \coqdocvariable{g} \coqref{Groupoid.groupoid.::x 'xC2xB0' x}{\coqdocnotation{$\circ$}} \coqref{Groupoid.groupoid.::x 'xC2xB0' x}{\coqdocnotation{(}}\coqdocvariable{α} \coqref{Groupoid.groupoid.::x 'xC2xB0''' x}{\coqdocnotation{$\circ$}} \coqref{Groupoid.groupoid.::x 'xC2xB0''' x}{\coqdocnotation{(}}\coqref{Groupoid.groupoid.Equiv adjoint}{\coqdocdefinition{Equiv\_adjoint}} \coqdocvariable{α}\coqref{Groupoid.groupoid.::x 'xC2xB0''' x}{\coqdocnotation{)}}\coqref{Groupoid.groupoid.::x 'xC2xB0' x}{\coqdocnotation{)}}.\coqdoceol
\coqdocemptyline
\coqdocnoindent
\coqdockw{Definition} \coqdef{Groupoid.groupoid.Equiv eq}{Equiv\_eq}{\coqdocdefinition{Equiv\_eq}} \coqdocvar{T} \coqdocvar{U} (\coqdocvar{f} \coqdocvar{g} : \coqref{Groupoid.groupoid.Equiv}{\coqdocdefinition{Equiv}} \coqdocvariable{T} \coqdocvariable{U}) := \coqdocnotation{\{}\coqdocvar{α} \coqdocnotation{:} \coqref{Groupoid.groupoid.nat trans}{\coqdocdefinition{nat\_trans}} \coqref{Groupoid.groupoid.::'[' x ']'}{\coqdocnotation{[}}\coqdocvariable{f}\coqref{Groupoid.groupoid.::'[' x ']'}{\coqdocnotation{]}} \coqref{Groupoid.groupoid.::'[' x ']'}{\coqdocnotation{[}}\coqdocvariable{g}\coqref{Groupoid.groupoid.::'[' x ']'}{\coqdocnotation{]}} \coqdocnotation{\&} \coqref{Groupoid.groupoid.EquivEq}{\coqdocclass{EquivEq}} \coqdocvar{α}\coqdocnotation{\}}.\coqdoceol
\coqdocemptyline
\end{coqdoccode}
It is crucial here to be able to express the 2-dimensional equality
  between groupoids as a particular \coqdockw{Type} and not directly using the
  identity type. Indeed, whereas the functional extensionality principle
  makes the use of the identity type and modification equivalent to
  treat equality of natural transformations, the same is not possible
  for homotopy equivalences.  \begin{coqdoccode}
\coqdocemptyline
\end{coqdoccode}
We can define the pre-groupoid \coqref{Groupoid.groupoid. Type}{\coqdocabbreviation{$\mathsf{Type}_{1}^{1}$}} of groupoids and homotopy
 equivalences.  However, groupoids together with homotopy equivalences
 do not form a groupoid but rather a 2-groupoid. As we only have a
 formalization of groupoids, this can not be expressed in our
 setting. Nevertheless, we can state that setoids (inhabitants of
 \coqref{Groupoid.groupoid.SetoidType}{\coqdocdefinition{$\mathsf{Type_0}$}}) form a groupoid.  \begin{coqdoccode}
\coqdocemptyline
\coqdocnoindent
\coqdockw{Definition} \coqdef{Groupoid.groupoid.Type0}{$\mathsf{Type}_{0}^1$}{\coqdocdefinition{$\mathsf{Type}_{0}^1$}} : \coqref{Groupoid.groupoid.GroupoidType}{\coqdocdefinition{$\mathsf{Type_1}$}} := \coqdocnotation{(}\coqref{Groupoid.groupoid.SetoidType}{\coqdocdefinition{$\mathsf{Type_0}$}} \coqdocnotation{;} \coqref{Groupoid.groupoid.Equiv Groupoid}{\coqdocinstance{$\mathsf{Equiv_{Type_0}}$}}\coqdocnotation{)}.\coqdoceol
\coqdocemptyline
\end{coqdoccode}
\noindent In the definition above, \coqref{Groupoid.groupoid.Equiv Groupoid}{\coqdocinstance{$\mathsf{Equiv_{Type_0}}$}} is a proof
that \coqref{Groupoid.groupoid.Equiv}{\coqdocdefinition{Equiv}} and \coqref{Groupoid.groupoid.Equiv eq}{\coqdocdefinition{Equiv\_eq}} form a groupoid. It makes again use of
functional extensionality to prove contractibility of higher cells.  As
the type of pre-groupoids appears both in the term and the type, the use of
polymorphic universes is crucial here to avoid an inconsistency. \begin{coqdoccode}
\end{coqdoccode}
\subsection{Rewriting in homotopy type theory}


  \label{sec:rew}


  When considering a dependent family \coqdocvariable{F} of type [\coqdocvariable{A} $\longrightarrow$ \coqref{Groupoid.groupoid. Type}{\coqdocabbreviation{$\mathsf{Type}_{1}^{1}$}}], the \coqref{Groupoid.groupoid. map}{\coqdocprojection{$\coqdoccst{map}$}} function
  provides a homotopy equivalence between \coqdocvariable{F} $\star$ \coqdocvariable{x} and \coqdocvariable{F} $\star$ \coqdocvariable{y} for any \coqdocvariable{x}
  and \coqdocvariable{y} such that \coqdocvariable{x} $\sim_1$ \coqdocvariable{y}. The underlying map of homotopy equivalence
  can hence be used to cast any term of type [\coqdocvariable{F} $\star$ \coqdocvariable{x}] to [\coqdocvariable{F} $\star$ \coqdocvariable{y}].
\begin{coqdoccode}
\coqdocemptyline
\coqdocnoindent
\coqdockw{Definition} \coqdef{Groupoid.groupoid.transport}{transport}{\coqdocdefinition{transport}} \coqdocvar{A} (\coqdocvar{F}:\coqref{Groupoid.groupoid.::'[' x ']'}{\coqdocnotation{[}}\coqdocvariable{A} \coqref{Groupoid.groupoid.::x '-->' x}{\coqdocnotation{$\longrightarrow$}} \coqref{Groupoid.groupoid. Type}{\coqdocabbreviation{$\mathsf{Type}_{1}^{1}$}}\coqref{Groupoid.groupoid.::'[' x ']'}{\coqdocnotation{]}}) \{\coqdocvar{x} \coqdocvar{y}:\coqref{Groupoid.groupoid.::'[' x ']'}{\coqdocnotation{[}}\coqdocvariable{A}\coqref{Groupoid.groupoid.::'[' x ']'}{\coqdocnotation{]}}\} (\coqdocvar{e}:\coqdocvariable{x} \coqref{Groupoid.groupoid.::x 'x7E1' x}{\coqdocnotation{$\sim_1$}} \coqdocvariable{y}) \coqdoceol
\coqdocindent{1.00em}
: \coqref{Groupoid.groupoid.::x '--->' x}{\coqdocnotation{(}}\coqdocvariable{F} \coqref{Groupoid.groupoid.::x '@' x}{\coqdocnotation{$\star$}} \coqdocvariable{x}\coqref{Groupoid.groupoid.::x '--->' x}{\coqdocnotation{)}} \coqref{Groupoid.groupoid.::x '--->' x}{\coqdocnotation{$\longrightarrow$}} \coqref{Groupoid.groupoid.::x '--->' x}{\coqdocnotation{(}}\coqdocvariable{F} \coqref{Groupoid.groupoid.::x '@' x}{\coqdocnotation{$\star$}} \coqdocvariable{y}\coqref{Groupoid.groupoid.::x '--->' x}{\coqdocnotation{)}} := \coqref{Groupoid.groupoid.::'[' x ']'}{\coqdocnotation{[}}\coqref{Groupoid.groupoid.map}{\coqdocabbreviation{map}} \coqdocvariable{F} \coqdocvariable{e}\coqref{Groupoid.groupoid.::'[' x ']'}{\coqdocnotation{]}}.\coqdoceol
\coqdocemptyline
\end{coqdoccode}
Using compatibility on \coqref{Groupoid.groupoid. map}{\coqdocprojection{$\coqdoccst{map}$}}, we can reason on different transport paths.  
  Intuitively, any two transport maps with the same domain
  and codomain should be the same up to homotopy. As we only consider
  groupoids, there is only one relevant level of compatibilities,
  higher compatibilities are trivial. \coqref{Groupoid.groupoid.transport eq}{\coqdocdefinition{$\mathsf{transport_{eq}}$}} is an example of a
  derivable equality between two transport maps, when the proofs
  relating \coqdocvariable{x} and \coqdocvariable{y} are equal.  \begin{coqdoccode}
\coqdocemptyline
\coqdocnoindent
\coqdockw{Definition} \coqdef{Groupoid.groupoid.transport eq}{$\mathsf{transport_{eq}}$}{\coqdocdefinition{$\mathsf{transport_{eq}}$}} \coqdocvar{A} (\coqdocvar{F}:\coqref{Groupoid.groupoid.::'[' x ']'}{\coqdocnotation{[}}\coqdocvariable{A} \coqref{Groupoid.groupoid.::x '-->' x}{\coqdocnotation{$\longrightarrow$}} \coqref{Groupoid.groupoid. Type}{\coqdocabbreviation{$\mathsf{Type}_{1}^{1}$}}\coqref{Groupoid.groupoid.::'[' x ']'}{\coqdocnotation{]}}) \{\coqdocvar{x} \coqdocvar{y}:\coqref{Groupoid.groupoid.::'[' x ']'}{\coqdocnotation{[}}\coqdocvariable{A}\coqref{Groupoid.groupoid.::'[' x ']'}{\coqdocnotation{]}}\} \{\coqdocvar{e} \coqdocvar{e'}:\coqdocvariable{x} \coqref{Groupoid.groupoid.::x 'x7E1' x}{\coqdocnotation{$\sim_1$}} \coqdocvariable{y}\} (\coqdocvar{H}:\coqdocvariable{e} \coqref{Groupoid.groupoid.::x 'x7E' x}{\coqdocnotation{$\sim_2$}} \coqdocvariable{e'}) \coqdoceol
\coqdocindent{1.00em}
: \coqref{Groupoid.groupoid.transport}{\coqdocdefinition{transport}} \coqdocvariable{F} \coqdocvariable{e} \coqref{Groupoid.groupoid.::x 'x7E1' x}{\coqdocnotation{$\sim_1$}} \coqref{Groupoid.groupoid.transport}{\coqdocdefinition{transport}} \coqdocvariable{F} \coqdocvariable{e'} := \coqref{Groupoid.groupoid.::'[' x ']'}{\coqdocnotation{[}}\coqref{Groupoid.groupoid.map2}{\coqdocabbreviation{$\coqdoccst{map}_2$}} \coqdocvariable{F} \coqdocvariable{H}\coqref{Groupoid.groupoid.::'[' x ']'}{\coqdocnotation{]}}.\coqdoceol
\coqdocemptyline
\end{coqdoccode}
\noindent In the text, 
  we also use \coqref{Groupoid.groupoid.transport id}{\coqdocdefinition{$\mathsf{transport_{id}}$}}, \coqref{Groupoid.groupoid.transport comp}{\coqdocdefinition{$\mathsf{transport_{comp}}$}} and \coqref{Groupoid.groupoid.transport map}{\coqdocdefinition{$\mathsf{transport_{map}}$}} for compatibilities with identities, composition and for the functoriality of \coqref{Groupoid.groupoid.transport}{\coqdocdefinition{transport}}. \begin{coqdoccode}
\coqdocemptyline
\end{coqdoccode}
\subsection{Dependent Product}


  \label{sec:depprod}
  As for functions, dependent functions will be interpreted as functors. 
  But this time, the compatibilities with higher-order morphisms cannot
  be expressed as simple equalities, as some transport has to be done to 
  make those equalities typable. We call such a functor a 
  \emph{dependent functor}. Dependent functors are defined between a groupoid \coqdocvariable{T} and a functor \coqdocvariable{U} from \coqdocvariable{T} to \coqref{Groupoid.groupoid. Type}{\coqdocabbreviation{$\mathsf{Type}_{1}^{1}$}} (the pre-groupoid of groupoids). \coqdocvariable{U} must be seen as a type depending on \coqdocvariable{T}, or as a family of types indexed by \coqdocvariable{T}. 
\begin{coqdoccode}
\coqdocemptyline
\coqdocnoindent
\coqdockw{Class} \coqdef{Groupoid.groupoid.DependentFunctor}{$\mathsf{Functor}^\Pi$}{\coqdocrecord{$\mathsf{Functor}^\Pi$}} \coqdocvar{T} (\coqdocvar{U} : \coqref{Groupoid.groupoid.::'[' x ']'}{\coqdocnotation{[}}\coqdocvariable{T} \coqref{Groupoid.groupoid.::x '-->' x}{\coqdocnotation{$\longrightarrow$}} \coqref{Groupoid.groupoid. Type}{\coqdocabbreviation{$\mathsf{Type}_{1}^{1}$}}\coqref{Groupoid.groupoid.::'[' x ']'}{\coqdocnotation{]}}) (\coqdocvar{f} : \coqexternalref{:type scope:'xE2x88x80' x '..' x ',' x}{http://coq.inria.fr/stdlib/Coq.Unicode.Utf8\_core}{\coqdocnotation{∀}} \coqdocvar{t}\coqexternalref{:type scope:'xE2x88x80' x '..' x ',' x}{http://coq.inria.fr/stdlib/Coq.Unicode.Utf8\_core}{\coqdocnotation{,}} \coqref{Groupoid.groupoid.::'[' x ']'}{\coqdocnotation{[}}\coqdocvariable{U} \coqref{Groupoid.groupoid.::x '@' x}{\coqdocnotation{$\star$}} \coqdocvariable{t}\coqref{Groupoid.groupoid.::'[' x ']'}{\coqdocnotation{]}}) : \coqdockw{Type} := \{\coqdoceol
\coqdocindent{1.00em}
\coqdef{Groupoid.groupoid. Dmap}{$\coqdoccst{map}^{\cst{\Pi}}$}{\coqdocprojection{$\coqdoccst{map}^{\cst{\Pi}}$}}      : \coqexternalref{:type scope:'xE2x88x80' x '..' x ',' x}{http://coq.inria.fr/stdlib/Coq.Unicode.Utf8\_core}{\coqdocnotation{∀}} \{\coqdocvar{x} \coqdocvar{y}\} (\coqdocvar{e}: \coqdocvariable{x} \coqref{Groupoid.groupoid.::x 'x7E1' x}{\coqdocnotation{$\sim_1$}} \coqdocvariable{y}\coqexternalref{:type scope:'xE2x88x80' x '..' x ',' x}{http://coq.inria.fr/stdlib/Coq.Unicode.Utf8\_core}{\coqdocnotation{),}} \coqref{Groupoid.groupoid.transport}{\coqdocdefinition{transport}} \coqdocvariable{U} \coqdocvariable{e} \coqref{Groupoid.groupoid.::x '@' x}{\coqdocnotation{$\star$}} \coqref{Groupoid.groupoid.::x '@' x}{\coqdocnotation{(}}\coqdocvariable{f} \coqdocvariable{x}\coqref{Groupoid.groupoid.::x '@' x}{\coqdocnotation{)}} \coqref{Groupoid.groupoid.::x 'x7E1' x}{\coqdocnotation{$\sim_1$}} \coqdocvariable{f} \coqdocvariable{y} ;\coqdoceol
\coqdocindent{1.00em}
\coqdef{Groupoid.groupoid. Dmap id}{$\coqdoccst{map}^\Pi_\mathsf{id}$}{\coqdocprojection{$\coqdoccst{map}^\Pi_\mathsf{id}$}}   : \coqexternalref{:type scope:'xE2x88x80' x '..' x ',' x}{http://coq.inria.fr/stdlib/Coq.Unicode.Utf8\_core}{\coqdocnotation{∀}} \coqdocvar{x}\coqexternalref{:type scope:'xE2x88x80' x '..' x ',' x}{http://coq.inria.fr/stdlib/Coq.Unicode.Utf8\_core}{\coqdocnotation{,}} \coqref{Groupoid.groupoid. Dmap}{\coqdocmethod{$\coqdoccst{map}^{\cst{\Pi}}$}} (\coqref{Groupoid.groupoid.identity}{\coqdocprojection{identity}} \coqdocvariable{x}) \coqref{Groupoid.groupoid.::x 'x7E' x}{\coqdocnotation{$\sim_2$}} \coqref{Groupoid.groupoid.transport id}{\coqdocdefinition{$\mathsf{transport_{id}}$}} \coqdocvariable{U} \coqref{Groupoid.groupoid.::x '@' x}{\coqdocnotation{$\star$}} \coqref{Groupoid.groupoid.::x '@' x}{\coqdocnotation{(}}\coqdocvariable{f} \coqdocvariable{x}\coqref{Groupoid.groupoid.::x '@' x}{\coqdocnotation{)}};\coqdoceol
\coqdocindent{1.00em}
\coqdef{Groupoid.groupoid. Dmap comp}{$\coqdoccst{map}^\Pi_\mathsf{comp}$}{\coqdocprojection{$\coqdoccst{map}^\Pi_\mathsf{comp}$}} : \coqexternalref{:type scope:'xE2x88x80' x '..' x ',' x}{http://coq.inria.fr/stdlib/Coq.Unicode.Utf8\_core}{\coqdocnotation{∀}} \coqdocvar{x} \coqdocvar{y} \coqdocvar{z} (\coqdocvar{e} : \coqdocvariable{x} \coqref{Groupoid.groupoid.::x 'x7E1' x}{\coqdocnotation{$\sim_1$}} \coqdocvariable{y}) (\coqdocvar{e'} : \coqdocvariable{y} \coqref{Groupoid.groupoid.::x 'x7E1' x}{\coqdocnotation{$\sim_1$}} \coqdocvariable{z}\coqexternalref{:type scope:'xE2x88x80' x '..' x ',' x}{http://coq.inria.fr/stdlib/Coq.Unicode.Utf8\_core}{\coqdocnotation{),}}\coqdoceol
\coqdocindent{1.50em}
\coqref{Groupoid.groupoid. Dmap}{\coqdocmethod{$\coqdoccst{map}^{\cst{\Pi}}$}} (\coqdocvariable{e'} \coqref{Groupoid.groupoid.::x 'xC2xB0' x}{\coqdocnotation{$\circ$}} \coqdocvariable{e}) \coqref{Groupoid.groupoid.::x 'x7E2' x}{\coqdocnotation{$\sim_2$}} \coqref{Groupoid.groupoid. Dmap}{\coqdocmethod{$\coqdoccst{map}^{\cst{\Pi}}$}} \coqdocvariable{e'} \coqref{Groupoid.groupoid.::x 'xC2xB0' x}{\coqdocnotation{$\circ$}} \coqref{Groupoid.groupoid.transport map}{\coqdocdefinition{$\mathsf{transport_{map}}$}} \coqdocvariable{U} \coqdocvar{\_} (\coqref{Groupoid.groupoid. Dmap}{\coqdocmethod{$\coqdoccst{map}^{\cst{\Pi}}$}} \coqdocvariable{e}) \coqref{Groupoid.groupoid.::x 'xC2xB0' x}{\coqdocnotation{$\circ$}} \coqdoceol
\coqdocindent{10.50em}
\coqref{Groupoid.groupoid.::x 'xC2xB0' x}{\coqdocnotation{(}}\coqref{Groupoid.groupoid.transport comp}{\coqdocdefinition{$\mathsf{transport_{comp}}$}} \coqdocvariable{U} \coqdocvariable{e} \coqdocvariable{e'} \coqref{Groupoid.groupoid.::x '@' x}{\coqdocnotation{$\star$}} \coqdocvar{\_}\coqref{Groupoid.groupoid.::x 'xC2xB0' x}{\coqdocnotation{)}};\coqdoceol
\coqdocindent{1.00em}
\coqdef{Groupoid.groupoid. Dmap2}{$\coqdoccst{map}^\Pi_2$}{\coqdocprojection{$\coqdoccst{map}^\Pi_2$}}  : \coqexternalref{:type scope:'xE2x88x80' x '..' x ',' x}{http://coq.inria.fr/stdlib/Coq.Unicode.Utf8\_core}{\coqdocnotation{∀}} \coqdocvar{x} \coqdocvar{y} (\coqdocvar{e} \coqdocvar{e'}: \coqdocvariable{x} \coqref{Groupoid.groupoid.::x 'x7E1' x}{\coqdocnotation{$\sim_1$}} \coqdocvariable{y}) (\coqdocvar{H}: \coqdocvariable{e} \coqref{Groupoid.groupoid.::x 'x7E' x}{\coqdocnotation{$\sim_2$}} \coqdocvariable{e'}\coqexternalref{:type scope:'xE2x88x80' x '..' x ',' x}{http://coq.inria.fr/stdlib/Coq.Unicode.Utf8\_core}{\coqdocnotation{),}}\coqdoceol
\coqdocindent{2.00em}
\coqref{Groupoid.groupoid. Dmap}{\coqdocmethod{$\coqdoccst{map}^{\cst{\Pi}}$}} \coqdocvariable{e} \coqref{Groupoid.groupoid.::x 'x7E' x}{\coqdocnotation{$\sim_2$}} \coqref{Groupoid.groupoid. Dmap}{\coqdocmethod{$\coqdoccst{map}^{\cst{\Pi}}$}} \coqdocvariable{e'} \coqref{Groupoid.groupoid.::x 'xC2xB0' x}{\coqdocnotation{$\circ$}} \coqref{Groupoid.groupoid.::x 'xC2xB0' x}{\coqdocnotation{(}}\coqref{Groupoid.groupoid.transport eq}{\coqdocdefinition{$\mathsf{transport_{eq}}$}} \coqdocvariable{U} \coqdocvariable{H} \coqref{Groupoid.groupoid.::x '@' x}{\coqdocnotation{$\star$}} \coqref{Groupoid.groupoid.::x '@' x}{\coqdocnotation{(}}\coqdocvariable{f} \coqdocvariable{x}\coqref{Groupoid.groupoid.::x '@' x}{\coqdocnotation{)}}\coqref{Groupoid.groupoid.::x 'xC2xB0' x}{\coqdocnotation{)}}\}.\coqdoceol
\coqdocemptyline
\coqdocnoindent
\coqdockw{Definition} \coqdef{Groupoid.groupoid.Prod Type}{$\Pi_\coqdoccst{T}$}{\coqdocdefinition{$\Pi_\coqdoccst{T}$}} \coqdocvar{T} (\coqdocvar{U}:\coqref{Groupoid.groupoid.::'[' x ']'}{\coqdocnotation{[}}\coqdocvariable{T} \coqref{Groupoid.groupoid.::x '-->' x}{\coqdocnotation{$\longrightarrow$}} \coqref{Groupoid.groupoid. Type}{\coqdocabbreviation{$\mathsf{Type}_{1}^{1}$}}\coqref{Groupoid.groupoid.::'[' x ']'}{\coqdocnotation{]}}) := \coqdocnotation{\{}\coqdocvar{f} \coqdocnotation{:} \coqexternalref{:type scope:'xE2x88x80' x '..' x ',' x}{http://coq.inria.fr/stdlib/Coq.Unicode.Utf8\_core}{\coqdocnotation{∀}} \coqdocvar{t}\coqexternalref{:type scope:'xE2x88x80' x '..' x ',' x}{http://coq.inria.fr/stdlib/Coq.Unicode.Utf8\_core}{\coqdocnotation{,}} \coqref{Groupoid.groupoid.::'[' x ']'}{\coqdocnotation{[}}\coqdocvariable{U} \coqref{Groupoid.groupoid.::x '@' x}{\coqdocnotation{$\star$}} \coqdocvariable{t}\coqref{Groupoid.groupoid.::'[' x ']'}{\coqdocnotation{]}} \coqdocnotation{\&} \coqref{Groupoid.groupoid.DependentFunctor}{\coqdocclass{$\mathsf{Functor}^\Pi$}} \coqdocvariable{U} \coqdocvar{f}\coqdocnotation{\}}.\coqdoceol
\coqdocemptyline
\end{coqdoccode}


  Equality between dependent functors is given by dependent natural transformations
  and equality at level 2 is given by dependent modifications.
\begin{coqdoccode}
\coqdocemptyline
\coqdocemptyline
\end{coqdoccode}
\noindent We can now equip dependent functors with a groupoid structure
    as we have done for functors.
    We note \coqref{Groupoid.groupoid. Prod}{\coqdocdefinition{$\Pi$}} \coqdocvariable{U} the dependent product over a family of groupoids \coqdocvariable{U}.
\begin{coqdoccode}
\coqdocemptyline
\end{coqdoccode}
A family of setoids can be seen as a family of groupoids using a
lifting that we abusively note   \coqdocvariable{U}$_{\upharpoonright s}$ . We can prove that the
dependent product over a family of setoids is also a setoid. We note
\coqref{Groupoid.groupoid.Prod0}{\coqdocdefinition{$\Pi_0$}} the restriction of \coqref{Groupoid.groupoid. Prod}{\coqdocdefinition{$\Pi$}} to families of setoids.  \begin{coqdoccode}
\coqdocemptyline
\end{coqdoccode}
\subsection{Dependent sums}




\label{sec:sigma} In the interpretation of Σ types, we pay for the
fact that we are missing the 2-dimensional nature of \coqref{Groupoid.groupoid. Type}{\coqdocabbreviation{$\mathsf{Type}_{1}^{1}$}}. Indeed, as
we will need rewriting in the definition of equality on Σ types,
delivering the corresponding groupoid structure requires to reason on
compatibility between rewritings, which amount to the missing
2-dimensional laws. However, as \coqref{Groupoid.groupoid.Type0}{\coqdocdefinition{$\mathsf{Type}_{0}^1$}} is a groupoid, all 2-dimensional
equalities become trivial on a family of setoids, so we can define the
groupoid of Σ types over a groupoid \coqdocvariable{T} and a morphism of type  [\coqdocvariable{T} $\longrightarrow$ \coqref{Groupoid.groupoid.Type0}{\coqdocdefinition{$\mathsf{Type}_{0}^1$}}] .\begin{coqdoccode}
\coqdocemptyline
\coqdocemptyline
\coqdocnoindent
\coqdockw{Definition} \coqdef{Groupoid.groupoid.sum type}{$\Sigma_\coqdoccst{T}$}{\coqdocdefinition{$\Sigma_\coqdoccst{T}$}} \coqdocvar{T} (\coqdocvar{U} : \coqref{Groupoid.groupoid.::'[' x ']'}{\coqdocnotation{[}}\coqdocvariable{T} \coqref{Groupoid.groupoid.::x '-||->' x}{\coqdocnotation{$\longrightarrow$}} \coqref{Groupoid.groupoid.Type0}{\coqdocdefinition{$\mathsf{Type}_{0}^1$}}\coqref{Groupoid.groupoid.::'[' x ']'}{\coqdocnotation{]}}) := \coqdocnotation{\{}\coqdocvar{t} \coqdocnotation{:} \coqref{Groupoid.groupoid.::'[' x ']'}{\coqdocnotation{[}}\coqdocvariable{T}\coqref{Groupoid.groupoid.::'[' x ']'}{\coqdocnotation{]}} \coqdocnotation{\&} \coqref{Groupoid.groupoid.::'[' x ']'}{\coqdocnotation{[}}\coqdocvariable{U} \coqref{Groupoid.groupoid.::x '@' x}{\coqdocnotation{$\star$}} \coqdocvar{t}\coqref{Groupoid.groupoid.::'[' x ']'}{\coqdocnotation{]}}\coqdocnotation{\}}.\coqdoceol
\coqdocemptyline
\end{coqdoccode}
\noindent
  The 1-equality between dependent pairs is given by 1-equality on the
  first and second projections, with a transport on the second
  projection on the left.
\begin{coqdoccode}
\coqdocemptyline
\coqdocnoindent
\coqdockw{Definition} \coqdef{Groupoid.groupoid.sum eq}{$\Sigma_\coqdoccst{Eq}$}{\coqdocdefinition{$\Sigma_\coqdoccst{Eq}$}} \coqdocvar{T} (\coqdocvar{U} : \coqref{Groupoid.groupoid.::'[' x ']'}{\coqdocnotation{[}}\coqdocvariable{T} \coqref{Groupoid.groupoid.::x '-||->' x}{\coqdocnotation{$\longrightarrow$}} \coqref{Groupoid.groupoid.Type0}{\coqdocdefinition{$\mathsf{Type}_{0}^1$}}\coqref{Groupoid.groupoid.::'[' x ']'}{\coqdocnotation{]}}) : \coqref{Groupoid.groupoid.HomT}{\coqdocdefinition{$\mathsf{HomSet}$}} (\coqref{Groupoid.groupoid.sum type}{\coqdocdefinition{$\Sigma_\coqdoccst{T}$}} \coqdocvariable{U}) := \coqdoceol
\coqdocindent{1.00em}
\coqexternalref{::'xCExBB' x '..' x ',' x}{http://coq.inria.fr/stdlib/Coq.Unicode.Utf8\_core}{\coqdocnotation{\ensuremath{\lambda}}} \coqdocvar{m} \coqdocvar{n}\coqexternalref{::'xCExBB' x '..' x ',' x}{http://coq.inria.fr/stdlib/Coq.Unicode.Utf8\_core}{\coqdocnotation{,}} \coqdocnotation{\{}\coqdocvar{P} \coqdocnotation{:} \coqref{Groupoid.groupoid.::'[' x ']'}{\coqdocnotation{[}}\coqdocvariable{m}\coqref{Groupoid.groupoid.::'[' x ']'}{\coqdocnotation{]}} \coqref{Groupoid.groupoid.::x 'x7E1' x}{\coqdocnotation{$\sim_1$}} \coqref{Groupoid.groupoid.::'[' x ']'}{\coqdocnotation{[}}\coqdocvariable{n}\coqref{Groupoid.groupoid.::'[' x ']'}{\coqdocnotation{]}} \coqdocnotation{\&} \coqref{Groupoid.groupoid.transport}{\coqdocdefinition{transport}} (\coqref{Groupoid.groupoid.::'[[[' x ']]]'}{\coqdocnotation{ }}\coqdocvariable{U}\coqref{Groupoid.groupoid.::'[[[' x ']]]'}{\coqdocnotation{$_{\upharpoonright s}$}}) \coqdocvar{P} \coqref{Groupoid.groupoid.::x '@' x}{\coqdocnotation{$\star$}} \coqref{Groupoid.groupoid.::x '@' x}{\coqdocnotation{(}}\coqref{Groupoid.groupoid.xCExA02}{\coqdocabbreviation{$\pi_2$}} \coqdocvariable{m}\coqref{Groupoid.groupoid.::x '@' x}{\coqdocnotation{)}} \coqref{Groupoid.groupoid.::x 'x7E1' x}{\coqdocnotation{$\sim_1$}} \coqref{Groupoid.groupoid.xCExA02}{\coqdocabbreviation{$\pi_2$}} \coqdocvariable{n}\coqdocnotation{\}}.\coqdoceol
\coqdocemptyline
\end{coqdoccode}
\noindent
  In the same way, 2-equality between 1-equalities is given by projections
  and rewriting.
\begin{coqdoccode}
\coqdocemptyline
\coqdocnoindent
\coqdockw{Definition} \coqdef{Groupoid.groupoid.sum eq2}{$\Sigma_{\coqdoccst{Eq}_2}$}{\coqdocdefinition{$\Sigma_{\coqdoccst{Eq}_2}$}} \coqdocvar{T} (\coqdocvar{U} : \coqref{Groupoid.groupoid.::'[' x ']'}{\coqdocnotation{[}}\coqdocvariable{T} \coqref{Groupoid.groupoid.::x '-||->' x}{\coqdocnotation{$\longrightarrow$}} \coqref{Groupoid.groupoid.Type0}{\coqdocdefinition{$\mathsf{Type}_{0}^1$}}\coqref{Groupoid.groupoid.::'[' x ']'}{\coqdocnotation{]}}) (\coqdocvar{M} \coqdocvar{N} : \coqref{Groupoid.groupoid.sum type}{\coqdocdefinition{$\Sigma_\coqdoccst{T}$}} \coqdocvariable{U}) : \coqref{Groupoid.groupoid.HomT}{\coqdocdefinition{$\mathsf{HomSet}$}} (\coqdocvariable{M} \coqref{Groupoid.groupoid.::x 'x7E1' x}{\coqdocnotation{$\sim_1$}} \coqdocvariable{N}) \coqdoceol
\coqdocindent{1.00em}
:= \coqexternalref{::'xCExBB' x '..' x ',' x}{http://coq.inria.fr/stdlib/Coq.Unicode.Utf8\_core}{\coqdocnotation{\ensuremath{\lambda}}} \coqdocvar{e} \coqdocvar{e'} \coqexternalref{::'xCExBB' x '..' x ',' x}{http://coq.inria.fr/stdlib/Coq.Unicode.Utf8\_core}{\coqdocnotation{,}} \coqdocnotation{\{}\coqdocvar{P} \coqdocnotation{:} \coqref{Groupoid.groupoid.::'[' x ']'}{\coqdocnotation{[}}\coqdocvariable{e}\coqref{Groupoid.groupoid.::'[' x ']'}{\coqdocnotation{]}} \coqref{Groupoid.groupoid.::x 'x7E' x}{\coqdocnotation{$\sim_2$}} \coqref{Groupoid.groupoid.::'[' x ']'}{\coqdocnotation{[}}\coqdocvariable{e'}\coqref{Groupoid.groupoid.::'[' x ']'}{\coqdocnotation{]}} \coqdocnotation{\&} \coqref{Groupoid.groupoid.xCExA02}{\coqdocabbreviation{$\pi_2$}} \coqdocvariable{e} \coqref{Groupoid.groupoid.::x 'x7E' x}{\coqdocnotation{$\sim_2$}} \coqref{Groupoid.groupoid.xCExA02}{\coqdocabbreviation{$\pi_2$}} \coqdocvariable{e'} \coqref{Groupoid.groupoid.::x 'xC2xB0' x}{\coqdocnotation{$\circ$}} \coqref{Groupoid.groupoid.::x 'xC2xB0' x}{\coqdocnotation{(}}\coqref{Groupoid.groupoid.transport eq}{\coqdocdefinition{$\mathsf{transport_{eq}}$}} (\coqref{Groupoid.groupoid.::'[[[' x ']]]'}{\coqdocnotation{ }}\coqdocvariable{U}\coqref{Groupoid.groupoid.::'[[[' x ']]]'}{\coqdocnotation{$_{\upharpoonright s}$}}) \coqdocvar{P} \coqref{Groupoid.groupoid.::x '@' x}{\coqdocnotation{$\star$}} \coqref{Groupoid.groupoid.::x '@' x}{\coqdocnotation{(}}\coqref{Groupoid.groupoid.xCExA02}{\coqdocabbreviation{$\pi_2$}} \coqdocvariable{M}\coqref{Groupoid.groupoid.::x '@' x}{\coqdocnotation{)}}\coqref{Groupoid.groupoid.::x 'xC2xB0' x}{\coqdocnotation{)}}\coqdocnotation{\}}.\coqdoceol
\coqdocemptyline
\coqdocemptyline
\end{coqdoccode}
\noindent This way, we can define the groupoid \coqref{Groupoid.groupoid. Sum0}{\coqdocdefinition{$\Sigma$}} \coqdocvariable{U} of dependent sums for any family of setoids. When \coqdocvariable{T} is a setoid, \coqref{Groupoid.groupoid. Sum0}{\coqdocdefinition{$\Sigma$}} \coqdocvariable{U} is also a setoid.
\begin{coqdoccode}
\end{coqdoccode}
