\coqlibrary{Groupoid.groupoid interpretation def}{Library }{Groupoid.groupoid\_interpretation\_def}

\begin{coqdoccode}
\end{coqdoccode}


  We now organize our formalization of groupoids into a model of the dependent 
  type theory introduced in Section~\ref{sec:definitions}.
  The interpretation is based on the notion of categories with families 
  introduced by Dybjer~\cite{dybjer:internaltt} later used in \cite{groupoid-interp}.
  This interpretation can also be seen as an extension of the Takeuti-Gandy interpretation of simple type theory, recently generalized to dependent type theory by Coquand et al. using Kan semisimplicial sets or cubical sets~\cite{barras:_gener_takeut_gandy_inter}. 
  The main novelty of our interpretation is to
  take advantage of universe polymorphism to interpret dependent types
  directly as functors into \coqdocdefinition{$\mathsf{Type}_{0}^1$}. 
  We only present the computational part of the interpretation, the
  proofs of functoriality and naturality are available in the \Coq development.


\subsection{Dependent types}




  The judgment context $\Gamma \vdash$ of Section
  \ref{sec:definitions} is represented in \Coq as a setoid, noted
  \coqdockw{Context} := \coqdocdefinition{$\mathsf{Type_0}$}. The empty context (Rule \textsc{Empty})
  is interpreted as the setoid with exactly one element at each
  dimension.  Types in a context \coqdocvariable{Γ}, noted \coqref{Groupoid.groupoid interpretation def.Typ}{\coqdocdefinition{Typ}} \coqdocvariable{Γ}, are (context)
  functors from \coqdocvariable{Γ} to the groupoid of setoids \coqdocdefinition{$\mathsf{Type}_{0}^1$}.  Thus, a
  judgment $\Gamma \vdash A : \Type{}$ is represented as a term \coqdocvariable{A} of
  type \coqref{Groupoid.groupoid interpretation def.Typ}{\coqdocdefinition{Typ}} \coqdocvariable{Γ}. Context extension (Rule \textsc{Decl}) is given by
  dependent sums, i.e., the judgment $\Gamma, x:A \vdash$ is represented
  as \coqdocvar{$\Sigma$} \coqdocvariable{A}.


\begin{coqdoccode}
\end{coqdoccode}
Terms of \coqdocvariable{A} introduced by a sequent $\Gamma \vdash t:A$ are
  dependent (context) functors from \coqdocvariable{Γ} to \coqdocvariable{A} that return for each
  context valuation \coqdocvariable{$\gamma$}, an object of \coqdocvariable{A} $\star$ \coqdocvariable{$\gamma$} respecting equality of
  contexts.  The type of terms of \coqdocvariable{A} is noted \coqref{Groupoid.groupoid interpretation def.Elt}{\coqdocdefinition{$\mathsf{Tm}$}} \coqdocvariable{A} := [\coqdocdefinition{$\Pi$} \coqdocvariable{A}]
  (context is implicit).  

  A dependent type $\Gamma, x:A \vdash B$ is interpreted in two
  equivalent ways: simply as a type \coqref{Groupoid.groupoid interpretation def.TypDep}{\coqdocdefinition{TypDep}} \coqdocvariable{A} := \coqref{Groupoid.groupoid interpretation def.Typ}{\coqdocdefinition{Typ}} (\coqdocvar{$\Sigma$} \coqdocvariable{A}) over the
  dependent sum of \coqdocvariable{Γ} and \coqdocvariable{A} or as a type family \coqref{Groupoid.groupoid interpretation def.TypFam}{\coqdocdefinition{TypFam}} \coqdocvariable{A} over \coqdocvariable{A}
  (corresponding to a family of sets in constructive mathematics). A
  type family can be seen as a fibration (or bundle) from \coqdocvar{B} to \coqdocvariable{A}.
  In what follows, the indice $\mathsf{_{comp}}$ is given to proofs of 
  (dependent) functoriality.
\begin{coqdoccode}
\coqdocemptyline
\coqdocnoindent
\coqdockw{Definition} \coqdef{Groupoid.groupoid interpretation def.TypFam}{TypFam}{\coqdocdefinition{TypFam}} \{\coqdocvar{Γ} : \coqref{Groupoid.groupoid interpretation def.Context}{\coqdocdefinition{Context}}\} (\coqdocvar{A}: \coqref{Groupoid.groupoid interpretation def.Typ}{\coqdocdefinition{Typ}} \coqdocvariable{Γ}) := \coqdoceol
\coqdocindent{1.00em}
\coqdocnotation{[}\coqdocdefinition{$\Pi$} \coqdocnotation{(}\coqexternalref{::'xCExBB' x '..' x ',' x}{http://coq.inria.fr/stdlib/Coq.Unicode.Utf8\_core}{\coqdocnotation{\ensuremath{\lambda}}} \coqdocvar{$\gamma$}\coqexternalref{::'xCExBB' x '..' x ',' x}{http://coq.inria.fr/stdlib/Coq.Unicode.Utf8\_core}{\coqdocnotation{,}} \coqref{Groupoid.groupoid interpretation def.::'[[' x ']]'}{\coqdocnotation{ }} \coqref{Groupoid.groupoid interpretation def.::'[[' x ']]'}{\coqdocnotation{(}}\coqdocvariable{A} \coqdocnotation{$\star$} \coqdocvariable{$\gamma$}\coqref{Groupoid.groupoid interpretation def.::'[[' x ']]'}{\coqdocnotation{)}} \coqref{Groupoid.groupoid interpretation def.::'[[' x ']]'}{\coqdocnotation{$_{\upharpoonright s}$}} \coqdocnotation{$\longrightarrow$} \coqdocdefinition{$\mathsf{Type}_{0}^1$}\coqdocnotation{;} \coqref{Groupoid.groupoid interpretation def.TypFam 1}{\coqdocinstance{$\mathsf{TypFam_{comp}}$}} \coqdocvar{\_}\coqdocnotation{)}\coqdocnotation{]}.\coqdoceol
\end{coqdoccode}
