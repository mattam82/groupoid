\coqlibrary{Groupoid.cwf equations}{Library }{Groupoid.cwf\_equations}

\begin{coqdoccode}
\coqdocemptyline
\coqdocemptyline
\end{coqdoccode}
\section{Connection to internal categories with families}


   \label{section:cwf}


  We now turn to show that we have actually a model in the sense of
  internal categories with families~\cite{dybjer:internaltt}. More
  precisely, our work can be seen as a formalization of setoid-indexed
  families of setoids, where the notion of rewriting using
  notation \coqdocvariable{t} \coqdockw{$\coqdockw{with}$} \coqdocvariable{e} corresponds to the \emph{reindexing
  map} of families of setoids.
\begin{coqdoccode}
\coqdocemptyline
\coqdocemptyline
\end{coqdoccode}
  \paragraph{\lrule{Substitution Laws}.}


  In internal CwFs, substitution laws hold, but not definitionally. This
  means that substitution laws for terms need explicit rewriting using
  reindexing maps in their statements.  The situation is similar in our
  setting: a law that does not hold definitionally can only hold with
  respect to the notion of equality of the setoid/groupoid. Every
  substitution law holds using \coqdocprojection{identity} once a context has been applied, 
  which means that the only non-definitional coherences come from proofs of 
  naturality with respect to two equal contexts.


  We only present the substitution laws for dependent products. First,
  the rule at the level of types:
 \begin{coqdoccode}
\coqdocemptyline
\coqdocnoindent
\coqdockw{Definition} \coqdef{Groupoid.cwf equations.Prod sigma law}{$\mathsf{Prod_{\sigma law}}$}{\coqdocdefinition{$\mathsf{Prod_{\sigma law}}$}} \{\coqdocvar{Δ} \coqdocvar{Γ}\} \{σ : \coqdocnotation{[}\coqdocvariable{Δ} \coqdocnotation{$\longrightarrow$} \coqdocvariable{Γ}\coqdocnotation{]}\} \{\coqdocvar{A} : \coqdocdefinition{Typ} \coqdocvariable{Γ}\} \{\coqdocvar{F} : \coqdocdefinition{TypFam} \coqdocvariable{A}\}:\coqdoceol
\coqdocindent{1.00em}
\coqdocdefinition{Prod} \coqdocvariable{F} \coqdocnotation{$⋅$} \coqdocvariable{σ} \coqdocnotation{$\sim_1$} \coqdocdefinition{Prod} (\coqdocvariable{F} \coqdocnotation{$\circ$} \coqdocvariable{σ}) := \coqdocnotation{(}\coqexternalref{::'xCExBB' x '..' x ',' x}{http://coq.inria.fr/stdlib/Coq.Unicode.Utf8\_core}{\coqdocnotation{\ensuremath{\lambda}}} \coqdocvar{t}\coqexternalref{::'xCExBB' x '..' x ',' x}{http://coq.inria.fr/stdlib/Coq.Unicode.Utf8\_core}{\coqdocnotation{,}} \coqdocprojection{identity} \coqdocvar{\_}\coqdocnotation{;} \coqref{Groupoid.cwf equations. Prod sigma law}{\coqdocdefinition{\_Prod\_sigma\_law}} \coqdocvariable{σ} \coqdocvariable{F}\coqdocnotation{)}.\coqdoceol
\coqdocemptyline
\end{coqdoccode}
For the other substitution laws, we omit their definitions as they
  follow the very same pattern; the witness is always the identity plus
  a proof of naturality wrt context change. To express the
  substitution law of dependent functions, we first need to exhibit the
  law for type-level abstraction \coqdocdefinition{$\Lambda$}---where \coqref{Groupoid.cwf equations.SubSigma}{\coqdocdefinition{$\mathsf{Sub_\Sigma}$}} σ weakens the
  substitution σ.  \begin{coqdoccode}
\coqdocemptyline
\coqdocnoindent
\coqdockw{Definition} \coqdef{Groupoid.cwf equations.LamT sigma law}{$\mathsf{\Lambda_{\sigma law}}$}{\coqdocdefinition{$\mathsf{\Lambda_{\sigma law}}$}} \{\coqdocvar{Δ} \coqdocvar{Γ}\} \{\coqdocvar{A} : \coqdocdefinition{Typ} \coqdocvariable{Γ}\} \{\coqdocvar{B} : \coqdocdefinition{TypDep} \coqdocvariable{A}\} \{σ : \coqdocnotation{[}\coqdocvariable{Δ} \coqdocnotation{$\longrightarrow$} \coqdocvariable{Γ}\coqdocnotation{]}\}:\coqdoceol
\coqdocindent{1.00em}
\coqdocdefinition{$\Lambda$} \coqdocvariable{B} \coqdocnotation{$\circ$} \coqdocvariable{σ} \coqdocnotation{$\sim_1$} \coqdocdefinition{$\Lambda$} (\coqdocvariable{B} \coqdocnotation{$⋅$} \coqref{Groupoid.cwf equations.SubSigma}{\coqdocdefinition{$\mathsf{Sub_\Sigma}$}} \coqdocvariable{σ}).\coqdoceol
\coqdocemptyline
\end{coqdoccode}
  Finally, the law for term-level abstraction can be stated, using rewriting 
  provided by the \coqdocvariable{t} \coqdockw{$\coqdockw{with}$} \coqdocvariable{e} notation.
 \begin{coqdoccode}
\coqdocemptyline
\coqdocnoindent
\coqdockw{Definition} \coqdef{Groupoid.cwf equations.Lam sigma law}{$\mathsf{Lam_{\sigma law}}$}{\coqdocdefinition{$\mathsf{Lam_{\sigma law}}$}} \{\coqdocvar{Δ} \coqdocvar{Γ}\} (σ:\coqdocnotation{[}\coqdocvariable{Δ} \coqdocnotation{$\longrightarrow$} \coqdocvariable{Γ}\coqdocnotation{]}) \{\coqdocvar{A}:\coqdocdefinition{Typ} \coqdocvariable{Γ}\} \{\coqdocvar{B}:\coqdocdefinition{TypDep} \coqdocvariable{A}\} (\coqdocvar{b}:\coqdocdefinition{$\mathsf{Tm}$} \coqdocvariable{B}):\coqdoceol
\coqdocindent{1.00em}
\coqdocnotation{(}\coqdocdefinition{Lam} \coqdocvariable{b}\coqdocnotation{)} \coqdocnotation{$\circ$} \coqdocvariable{σ} \coqdocnotation{$\coqdockw{with}$} \coqref{Groupoid.cwf equations.Prod sigma law}{\coqdocdefinition{$\mathsf{Prod_{\sigma law}}$}} \coqdocnotation{$\coqdockw{with}$} \coqref{Groupoid.cwf equations.Prod eq}{\coqdocdefinition{Prod\_eq}} \coqref{Groupoid.cwf equations.LamT sigma law}{\coqdocdefinition{$\mathsf{\Lambda_{\sigma law}}$}} \coqdocnotation{$\sim_1$} \coqdocdefinition{Lam} (\coqdocvariable{b} \coqdocnotation{$\circ$} \coqdocnotation{(}\coqref{Groupoid.cwf equations.SubSigma}{\coqdocdefinition{$\mathsf{Sub_\Sigma}$}} \coqdocvariable{σ}\coqdocnotation{)}).\coqdoceol
\coqdocemptyline
\end{coqdoccode}
In the same way, to state the law for function application, we need
   a law \coqref{Groupoid.cwf equations.SubstT sigma law}{\coqdocdefinition{$\mathsf{Subst_{\sigma law}}$}} for application at the level of type families. 
\begin{coqdoccode}
\coqdocemptyline
\coqdocnoindent
\coqdockw{Definition} \coqdef{Groupoid.cwf equations.App sigma law}{$\mathsf{App_{\sigma law}}$}{\coqdocdefinition{$\mathsf{App_{\sigma law}}$}} \coqdocvar{Δ} \coqdocvar{Γ} (\coqdocvar{A}:\coqdocdefinition{Typ} \coqdocvariable{Γ}) (\coqdocvar{F}:\coqdocdefinition{TypFam} \coqdocvariable{A}) (σ:\coqdocnotation{[}\coqdocvariable{Δ} \coqdocnotation{$\longrightarrow$} \coqdocvariable{Γ}\coqdocnotation{]}) (\coqdocvar{c}:\coqdocdefinition{$\mathsf{Tm}$} (\coqdocdefinition{Prod} \coqdocvariable{F}))\coqdoceol
\coqdocindent{1.00em}
(\coqdocvar{a}:\coqdocdefinition{$\mathsf{Tm}$} \coqdocvariable{A}): \coqdocvariable{c} \coqdocnotation{$\star$} \coqdocvariable{a} \coqdocnotation{$\circ$} \coqdocvariable{σ} \coqdocnotation{$\coqdockw{with}$} \coqref{Groupoid.cwf equations.SubstT sigma law}{\coqdocdefinition{$\mathsf{Subst_{\sigma law}}$}} \coqdocnotation{$\sim_1$} \coqdocnotation{(}\coqdocvariable{c} \coqdocnotation{$\circ$} \coqdocvariable{σ} \coqdocnotation{$\coqdockw{with}$} \coqref{Groupoid.cwf equations.Prod sigma law}{\coqdocdefinition{$\mathsf{Prod_{\sigma law}}$}}\coqdocnotation{)} \coqdocnotation{$\star$} \coqdocnotation{(}\coqdocvariable{a} \coqdocnotation{$\circ$} \coqdocvariable{σ}\coqdocnotation{)}.\coqdoceol
\coqdocemptyline
\end{coqdoccode}
\paragraph{\lrule{Conv}.} The $\beta$ conversion rule for term-level
  abstractions is valid as a definitional equality (which is made explicit
  by the use of \coqdocconstructor{eq\_refl}), where \coqdocdefinition{SubExtId} is a specialization of
  \coqdocdefinition{SubExt} with the identity substitution.  \begin{coqdoccode}
\coqdocemptyline
\coqdocnoindent
\coqdockw{Definition} \coqdef{Groupoid.cwf equations.Beta}{Beta}{\coqdocdefinition{Beta}} \{\coqdocvar{Γ}\} \{\coqdocvar{A}:\coqdocdefinition{Typ} \coqdocvariable{Γ}\} \{\coqdocvar{F}:\coqdocdefinition{TypDep} \coqdocvariable{A}\} (\coqdocvar{b}:\coqdocdefinition{$\mathsf{Tm}$} \coqdocvariable{F}) (\coqdocvar{a}:\coqdocdefinition{$\mathsf{Tm}$} \coqdocvariable{A}):\coqdoceol
\coqdocindent{1.00em}
\coqdocnotation{[}\coqdocdefinition{Lam} \coqdocvariable{b} \coqdocnotation{$\star$} \coqdocvariable{a}\coqdocnotation{]} \coqdocnotation{=} \coqdocnotation{[}\coqdocvariable{b} \coqdocnotation{$\circ$} \coqdocdefinition{SubExtId} \coqdocvariable{a}\coqdocnotation{]} := \coqdocconstructor{eq\_refl} \coqdocvar{\_}.\coqdoceol
\coqdocemptyline
\end{coqdoccode}
 \noindent 
  However, $\eta$ conversion does not hold definitionally, and we need 
  $\eta$-conversion at the level of type abstractions (rule \coqref{Groupoid.cwf equations.EtaT}{\coqdocdefinition{EtaT}}) 
  to state it.
\begin{coqdoccode}
\coqdocemptyline
\coqdocemptyline
\coqdocnoindent
\coqdockw{Definition} \coqdef{Groupoid.cwf equations.Eta}{Eta}{\coqdocdefinition{Eta}} \{\coqdocvar{Γ}\} \{\coqdocvar{A}:\coqdocdefinition{Typ} \coqdocvariable{Γ}\} \{\coqdocvar{F}:\coqdocdefinition{TypFam} \coqdocvariable{A}\} (\coqdocvar{c}:\coqdocdefinition{$\mathsf{Tm}$} (\coqdocdefinition{Prod} \coqdocvariable{F})):\coqdoceol
\coqdocindent{1.00em}
\coqdocdefinition{Lam} (\coqref{Groupoid.cwf equations.::'xE2x86x91' x}{\coqdocnotation{$\shortuparrow$}} \coqdocvariable{c} \coqdocnotation{$\star$} \coqdocdefinition{Var} \coqdocvariable{A}) \coqdocnotation{$\coqdockw{with}$} \coqref{Groupoid.cwf equations.Prod eq}{\coqdocdefinition{Prod\_eq}} (\coqref{Groupoid.cwf equations.EtaT}{\coqdocdefinition{EtaT}} \coqdocvariable{F}) \coqdocnotation{$\sim_1$} \coqdocvariable{c}.\coqdoceol
\coqdocemptyline
\end{coqdoccode}
\paragraph{\lrule{Coherence of the interpretation}.}
  In~\cite{dybjer:internaltt}, the coherence of the interpretation is
  not entirely shown, and relies on Uniqueness of Identity Proofs through
  Alf's pattern-matching. In our setting the first level coherence can
  be directly expressed and proved by naturality of the interpretation,
  because we embed the setoid model inside groupoids (higher notions of coherence
  require to move to higher groupoids).  \begin{coqdoccode}
\coqdocemptyline
\coqdocnoindent
\coqdockw{Theorem} \coqdef{Groupoid.cwf equations.coherence of interpretation}{coherence\_of\_interpretation}{\coqdoclemma{coherence\_of\_interpretation}} \{\coqdocvar{Γ}\} \{\coqdocvar{A} \coqdocvar{B}:\coqdocdefinition{Typ} \coqdocvariable{Γ}\} (\coqdocvar{e} \coqdocvar{e'} : \coqdocvariable{A} \coqdocnotation{$\sim_1$} \coqdocvariable{B}) (\coqdocvar{a}:\coqdocdefinition{$\mathsf{Tm}$} \coqdocvariable{A}):\coqdoceol
\coqdocindent{1.00em}
\coqdocvariable{e} \coqdocnotation{$\sim_2$} \coqdocvariable{e'} \coqexternalref{:type scope:x '->' x}{http://coq.inria.fr/stdlib/Coq.Init.Logic}{\coqdocnotation{\ensuremath{\rightarrow}}} \coqdocvariable{a} \coqdocnotation{$\coqdockw{with}$} \coqdocvariable{e} \coqdocnotation{$\sim_1$} \coqdocvariable{a} \coqdocnotation{$\coqdockw{with}$} \coqdocvariable{e'}.\coqdoceol
\end{coqdoccode}
Thus, the coherence of our interpretation only requires to prove 
  that the interpretation of type equalities from the source type theory 
  only targets \textit{identity} isomorphisms (as then the \coqdocvariable{e} $\sim_2$ \coqdocvariable{e'} hypothesis would be given by 
  the identity modification). This proof is not difficult metatheoretically as every 
  conversion rule is interpreted by the identity isomorphism, but it can not be 
  internalized because it is not possible to reason on definitional equality inside 
  the theory.
 \begin{coqdoccode}
\end{coqdoccode}
