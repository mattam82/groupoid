\coqlibrary{Groupoid.groupoid interpretation}{Library }{Groupoid.groupoid\_interpretation}

\begin{coqdoccode}
\end{coqdoccode}


  Terms of \coqdocdefinition{TypDep} \coqdocvariable{A} and \coqdocdefinition{TypFam} \coqdocvariable{A} can be related using a dependent closure
  at the level of types. In the interpretation of typing judgments, this connection 
  will be used to switch between the fibration and the morphism points of view.
\begin{coqdoccode}
\coqdocemptyline
\coqdocnoindent
\coqdockw{Definition} \coqdef{Groupoid.groupoid interpretation.LamT}{$\Lambda$}{\coqdocdefinition{$\Lambda$}} \{\coqdocvar{Γ}: \coqdocdefinition{Context}\} \{\coqdocvar{A} : \coqdocdefinition{Typ} \coqdocvariable{Γ}\} (\coqdocvar{B}: \coqdocdefinition{TypDep} \coqdocvariable{A})\coqdoceol
\coqdocindent{1.00em}
: \coqdocdefinition{TypFam} \coqdocvariable{A} := \coqdocnotation{(}\coqexternalref{::'xCExBB' x '..' x ',' x}{http://coq.inria.fr/stdlib/Coq.Unicode.Utf8\_core}{\coqdocnotation{\ensuremath{\lambda}}} \coqdocvar{$\gamma$}\coqexternalref{::'xCExBB' x '..' x ',' x}{http://coq.inria.fr/stdlib/Coq.Unicode.Utf8\_core}{\coqdocnotation{,}} \coqdocnotation{(}\coqexternalref{::'xCExBB' x '..' x ',' x}{http://coq.inria.fr/stdlib/Coq.Unicode.Utf8\_core}{\coqdocnotation{\ensuremath{\lambda}}} \coqdocvar{t}\coqexternalref{::'xCExBB' x '..' x ',' x}{http://coq.inria.fr/stdlib/Coq.Unicode.Utf8\_core}{\coqdocnotation{,}} \coqdocvariable{B} \coqdocnotation{$\star$} \coqdocnotation{(}\coqdocvariable{$\gamma$}\coqdocnotation{;} \coqdocvariable{t}\coqdocnotation{)} \coqdocnotation{;} \coqdocvar{\_}\coqdocnotation{);} \coqref{Groupoid.groupoid interpretation.LamT 1}{\coqdocinstance{$\mathsf{\Lambda_{comp}}$}} \coqdocvariable{B}\coqdocnotation{)}.\coqdoceol
\coqdocemptyline
\end{coqdoccode}


\subsection{Substitutions}




  A substitution is represented by a context morphism [\coqdocvariable{Γ} $\longrightarrow$ \coqdocvariable{Δ}].  Note
  that although a substitution σ can be composed with a dependent type
  \coqdocvariable{A} by using composition of functors, we define a relaxed notion of
  composition, noted \coqdocvariable{A} ⋅ σ. It has the same computational content but
  a different relation on the universe indices: homogeneous functor
  composition otherwise forces the three categories and two functors to
  live at exactly the same levels, which is not necessary.


  A substitution σ can be extended by a term \coqdocvariable{a}: \coqdocdefinition{$\mathsf{Tm}$} (\coqdocvariable{A} ⋅ σ) 
  of \coqdocvariable{A} : \coqdocdefinition{Typ} \coqdocvariable{Δ}.


\begin{coqdoccode}
\coqdocemptyline
\coqdocnoindent
\coqdockw{Definition} \coqdef{Groupoid.groupoid interpretation.SubExt}{SubExt}{\coqdocdefinition{SubExt}} \{\coqdocvar{Γ} \coqdocvar{Δ} : \coqdocdefinition{Context}\} \{\coqdocvar{A} : \coqdocdefinition{Typ} \coqdocvariable{Δ}\} (σ: \coqdocnotation{[}\coqdocvariable{Γ} \coqdocnotation{$\longrightarrow$} \coqdocvariable{Δ}\coqdocnotation{]}) (\coqdocvar{a}: \coqdocdefinition{$\mathsf{Tm}$} (\coqdocvariable{A} \coqdocnotation{$⋅$} \coqdocvariable{σ})) \coqdoceol
\coqdocindent{1.00em}
: \coqdocnotation{[}\coqdocvariable{Γ} \coqdocnotation{$\longrightarrow$} \coqdocdefinition{$\Sigma$} \coqdocvariable{A} \coqdocnotation{]} := \coqdocnotation{(}\coqexternalref{::'xCExBB' x '..' x ',' x}{http://coq.inria.fr/stdlib/Coq.Unicode.Utf8\_core}{\coqdocnotation{\ensuremath{\lambda}}} \coqdocvar{$\gamma$}\coqexternalref{::'xCExBB' x '..' x ',' x}{http://coq.inria.fr/stdlib/Coq.Unicode.Utf8\_core}{\coqdocnotation{,}} \coqdocnotation{(}\coqdocvariable{σ} \coqdocnotation{$\star$} \coqdocvariable{$\gamma$}\coqdocnotation{;} \coqdocvariable{a} \coqdocnotation{$\star$} \coqdocvariable{$\gamma$}\coqdocnotation{)} \coqdocnotation{;} \coqref{Groupoid.groupoid interpretation.SubExt 1}{\coqdocinstance{$\mathsf{SubExt_{comp}}$}} \coqdocvar{\_} \coqdocvar{\_}\coqdocnotation{)}.\coqdoceol
\coqdocemptyline
\end{coqdoccode}
\noindent where \coqref{Groupoid.groupoid interpretation.SubExt 1}{\coqdocinstance{$\mathsf{SubExt_{comp}}$}} is a proof that it is functorial. 
  A substitution σ can be applied to a type family \coqdocvariable{F} using the
  composition of a functor with a dependent functor. 
\begin{coqdoccode}
\coqdocemptyline
\coqdocnoindent
\coqdockw{Definition} \coqdef{Groupoid.groupoid interpretation.substF}{substF}{\coqdocdefinition{substF}} \{\coqdocvar{T} \coqdocvar{Γ}\} \{\coqdocvar{A}:\coqdocdefinition{Typ} \coqdocvariable{Γ}\} (\coqdocvar{F}:\coqdocdefinition{TypFam} \coqdocvariable{A}) (σ:\coqdocnotation{[}\coqdocvariable{T} \coqdocnotation{$\longrightarrow$} \coqdocvariable{Γ}\coqdocnotation{]}) : \coqdocdefinition{TypFam} (\coqdocvariable{A} \coqdocnotation{$⋅$} \coqdocvariable{σ}) \coqdoceol
\coqdocindent{1.00em}
:= \coqdocnotation{(}\coqdocnotation{[}\coqdocvariable{F} \coqdocnotation{$\circ$} \coqdocvariable{σ}\coqdocnotation{]} : \coqexternalref{:type scope:'xE2x88x80' x '..' x ',' x}{http://coq.inria.fr/stdlib/Coq.Unicode.Utf8\_core}{\coqdocnotation{∀}} \coqdocvar{t} : \coqdocnotation{[}\coqdocvariable{T}\coqdocnotation{]}\coqexternalref{:type scope:'xE2x88x80' x '..' x ',' x}{http://coq.inria.fr/stdlib/Coq.Unicode.Utf8\_core}{\coqdocnotation{,}} \coqdocnotation{ }\coqdocvariable{A} \coqdocnotation{$⋅$} \coqdocvariable{σ}\coqdocnotation{$_{\upharpoonright s}$} \coqdocnotation{$\star$} \coqdocvariable{t} \coqref{Groupoid.groupoid interpretation.::x '--->' x}{\coqdocnotation{$\longrightarrow$}} \coqdocdefinition{$\mathsf{Type}_{0}^1$}\coqdocnotation{;} \coqref{Groupoid.groupoid interpretation.substF 1}{\coqdocinstance{$\mathsf{substF_{comp}}$}} \coqdocvariable{F} \coqdocvariable{σ}\coqdocnotation{)}.\coqdoceol
\coqdocemptyline
\end{coqdoccode}
  We abusively note all those different compositions with $\circ$ as it is done in
  mathematics, whereas they are distinct operators in the \Coq
  development.
  The weakening substitution of $\Gamma, x:A \vdash$ is given by the first
  projection. \begin{coqdoccode}
\coqdocemptyline
\end{coqdoccode}


  A type family \coqdocvariable{F} in \coqdocdefinition{TypFam} \coqdocvariable{A} can be partially substituted with an
  term \coqdocvariable{a} in \coqdocdefinition{$\mathsf{Tm}$} \coqdocvariable{A}, noted \coqdocvariable{F} \{\{\coqdocvariable{a}\}\}, to get its value (a type) at
  \coqdocvariable{a}. This process is defined as \coqdocvariable{F} \{\{\coqdocvariable{a}\}\} := (\coqdocvar{\ensuremath{\lambda}} \coqdocvariable{$\gamma$}, (\coqdocvariable{F} $\star$ \coqdocvariable{$\gamma$}) $\star$ (\coqdocvariable{a} $\star$ \coqdocvariable{$\gamma$}) ;
  \coqdocvar{\_}) (where \coqdocvar{\_} is a proof it is functorial). Note that this
  pattern of application \emph{up-to a context $\gamma$} will be used
  later to defined other notions of application. Although the
  computational definitions are the same, the compatibility conditions
  are always different.  This notion of partial substitution in a type
  family enables to state that \coqref{Groupoid.groupoid interpretation.LamT}{\coqdocdefinition{$\Lambda$}} defines a type level
  $\lambda$-abstraction.  \begin{coqdoccode}
\coqdocemptyline
\coqdocnoindent
\coqdockw{Definition} \coqdef{Groupoid.groupoid interpretation.BetaT}{BetaT}{\coqdocdefinition{BetaT}} \coqdocvar{Δ} \coqdocvar{Γ} (\coqdocvar{A}:\coqdocdefinition{Typ} \coqdocvariable{Γ}) (\coqdocvar{B}:\coqdocdefinition{TypDep} \coqdocvariable{A}) (σ:\coqdocnotation{[}\coqdocvariable{Δ} \coqdocnotation{$\longrightarrow$} \coqdocvariable{Γ}\coqdocnotation{]}) (\coqdocvar{a}:\coqdocdefinition{$\mathsf{Tm}$} (\coqdocvariable{A} \coqdocnotation{$⋅$} \coqdocvariable{σ})) \coqdoceol
\coqdocnoindent
: \coqref{Groupoid.groupoid interpretation.LamT}{\coqdocdefinition{$\Lambda$}} \coqdocvariable{B} \coqref{Groupoid.groupoid interpretation.::x 'xC2xB0xC2xB0xC2xB0' x}{\coqdocnotation{$\circ$}} \coqdocvariable{σ} \coqref{Groupoid.groupoid interpretation.::x 'x7Bx7B' x 'x7Dx7D'}{\coqdocnotation{\{\{}}\coqdocvariable{a}\coqref{Groupoid.groupoid interpretation.::x 'x7Bx7B' x 'x7Dx7D'}{\coqdocnotation{\}\}}} \coqdocnotation{$\sim_1$} \coqdocvariable{B} \coqdocnotation{$⋅$} \coqdocnotation{(}\coqref{Groupoid.groupoid interpretation.SubExt}{\coqdocdefinition{SubExt}} \coqdocvariable{σ} \coqdocvariable{a}\coqdocnotation{)} := \coqdocnotation{(}\coqexternalref{::'xCExBB' x '..' x ',' x}{http://coq.inria.fr/stdlib/Coq.Unicode.Utf8\_core}{\coqdocnotation{\ensuremath{\lambda}}} \coqdocvar{$\gamma$}\coqexternalref{::'xCExBB' x '..' x ',' x}{http://coq.inria.fr/stdlib/Coq.Unicode.Utf8\_core}{\coqdocnotation{,}} \coqdocprojection{identity} \coqdocvar{\_} \coqdocnotation{;} \coqref{Groupoid.groupoid interpretation.BetaT 1}{\coqdocinstance{$\mathsf{BetaT_{comp}}$}} \coqdocvariable{B} \coqdocvariable{σ} \coqdocvariable{a}\coqdocnotation{)}.\coqdoceol
\coqdocemptyline
\end{coqdoccode}


\subsection{Interpretation of the typing judgment}


  \label{sec:interp}


  The explicit substitution versions of the typing rules of Figure
  \ref{fig:emltt} are modelled as described below.


  \paragraph{\textsc{Var}.} 


  The rule \textsc{Var} is given by the second projection plus a proof
  that the projection is dependently functorial. Note the explicit
  weakening of \coqdocvariable{A} in the returned type. This is because we need to
  make explicit that the context used to type \coqdocvariable{A} is extended with an
  term of type \coqdocvariable{A}.


\begin{coqdoccode}
\coqdocemptyline
\coqdocnoindent
\coqdockw{Definition} \coqdef{Groupoid.groupoid interpretation.Var}{Var}{\coqdocdefinition{Var}} \{\coqdocvar{Γ}\} (\coqdocvar{A}:\coqdocdefinition{Typ} \coqdocvariable{Γ}) : \coqdocdefinition{$\mathsf{Tm}$} \coqref{Groupoid.groupoid interpretation.::'xE2x87x91' x}{\coqdocnotation{$\shortuparrow$}}\coqdocvariable{A} := \coqdocnotation{(}\coqexternalref{::'xCExBB' x '..' x ',' x}{http://coq.inria.fr/stdlib/Coq.Unicode.Utf8\_core}{\coqdocnotation{\ensuremath{\lambda}}} \coqdocvar{t}\coqexternalref{::'xCExBB' x '..' x ',' x}{http://coq.inria.fr/stdlib/Coq.Unicode.Utf8\_core}{\coqdocnotation{,}} \coqdocabbreviation{$\pi_2$} \coqdocvariable{t}\coqdocnotation{;} \coqref{Groupoid.groupoid interpretation.Var 1}{\coqdocinstance{$\mathsf{Var_{comp}}$}} \coqdocvariable{A}\coqdocnotation{)}.\coqdoceol
\coqdocemptyline
\coqdocemptyline
\end{coqdoccode}
\paragraph{\textsc{Prod}.} The rule \textsc{Prod} is interpreted
  using the dependent functor space, plus a proof that equivalent
  contexts give rise to isomorphic dependent functor spaces.  Note that
  the rule is defined on type families and not on the dependent type
  formulation because here we need a fibration point of view. \begin{coqdoccode}
\coqdocemptyline
\coqdocnoindent
\coqdockw{Definition} \coqdef{Groupoid.groupoid interpretation.Prod}{Prod}{\coqdocdefinition{Prod}} \{\coqdocvar{Γ}\} (\coqdocvar{A}:\coqdocdefinition{Typ} \coqdocvariable{Γ}) (\coqdocvar{F}:\coqdocdefinition{TypFam} \coqdocvariable{A}) \coqdoceol
\coqdocindent{1.00em}
: \coqdocdefinition{Typ} \coqdocvariable{Γ} := \coqdocnotation{(}\coqexternalref{::'xCExBB' x '..' x ',' x}{http://coq.inria.fr/stdlib/Coq.Unicode.Utf8\_core}{\coqdocnotation{\ensuremath{\lambda}}} \coqdocvar{s}\coqexternalref{::'xCExBB' x '..' x ',' x}{http://coq.inria.fr/stdlib/Coq.Unicode.Utf8\_core}{\coqdocnotation{,}} \coqdocdefinition{$\Pi_0$} (\coqdocvariable{F} \coqdocnotation{$\star$} \coqdocvariable{s})\coqdocnotation{;} \coqref{Groupoid.groupoid interpretation.Prod 1}{\coqdocinstance{$\mathsf{Prod_{comp}}$}} \coqdocvariable{A} \coqdocvariable{F}\coqdocnotation{)}.\coqdoceol
\coqdocemptyline
\end{coqdoccode}
  \paragraph{\textsc{App}.}


  The rule \textsc{App} is interpreted using an up-to context application 
  and a proof of dependent functoriality. We abusively note \coqdocvar{M} $\star$ \coqdocvar{N} the application 
  of \coqref{Groupoid.groupoid interpretation.App}{\coqdocdefinition{App}}.
\begin{coqdoccode}
\coqdocemptyline
\coqdocnoindent
\coqdockw{Definition} \coqdef{Groupoid.groupoid interpretation.App}{App}{\coqdocdefinition{App}} \{\coqdocvar{Γ}\} \{\coqdocvar{A}:\coqdocdefinition{Typ} \coqdocvariable{Γ}\} \{\coqdocvar{F}:\coqdocdefinition{TypFam} \coqdocvariable{A}\} (\coqdocvar{c}:\coqdocdefinition{$\mathsf{Tm}$} (\coqref{Groupoid.groupoid interpretation.Prod}{\coqdocdefinition{Prod}} \coqdocvariable{F})) (\coqdocvar{a}:\coqdocdefinition{$\mathsf{Tm}$} \coqdocvariable{A}) \coqdoceol
\coqdocindent{1.00em}
: \coqdocdefinition{$\mathsf{Tm}$} (\coqdocvariable{F} \coqref{Groupoid.groupoid interpretation.::x 'x7Bx7B' x 'x7Dx7D'}{\coqdocnotation{\{\{}}\coqdocvariable{a}\coqref{Groupoid.groupoid interpretation.::x 'x7Bx7B' x 'x7Dx7D'}{\coqdocnotation{\}\}}}) := \coqdocnotation{(}\coqexternalref{::'xCExBB' x '..' x ',' x}{http://coq.inria.fr/stdlib/Coq.Unicode.Utf8\_core}{\coqdocnotation{\ensuremath{\lambda}}} \coqdocvar{s}\coqexternalref{::'xCExBB' x '..' x ',' x}{http://coq.inria.fr/stdlib/Coq.Unicode.Utf8\_core}{\coqdocnotation{,}} \coqdocnotation{(}\coqdocvariable{c} \coqdocnotation{$\star$} \coqdocvariable{s}\coqdocnotation{)} \coqdocnotation{$\star$} \coqdocnotation{(}\coqdocvariable{a} \coqdocnotation{$\star$} \coqdocvariable{s}\coqdocnotation{)}\coqdocnotation{;} \coqref{Groupoid.groupoid interpretation.App 1}{\coqdocinstance{$\mathsf{App_{comp}}$}} \coqdocvariable{c} \coqdocvariable{a}\coqdocnotation{)}.\coqdoceol
\coqdocemptyline
\end{coqdoccode}
  \paragraph{\lrule{Lam}.}


  Term-level $\lambda$-abstraction is defined with the same
  computational meaning as type-level $\lambda$-abstraction, but it
  differs on the proof of dependent functoriality. Note that we use
  \coqref{Groupoid.groupoid interpretation.LamT}{\coqdocdefinition{$\Lambda$}} in the definition because we need both the fibration (for
  \coqref{Groupoid.groupoid interpretation.Prod}{\coqdocdefinition{Prod}}) and the morphism (for \coqdocdefinition{$\mathsf{Tm}$} \coqdocvariable{B}) point of view. 
\begin{coqdoccode}
\coqdocemptyline
\coqdocnoindent
\coqdockw{Definition} \coqdef{Groupoid.groupoid interpretation.Lam}{Lam}{\coqdocdefinition{Lam}} \{\coqdocvar{Γ}\} \{\coqdocvar{A}:\coqdocdefinition{Typ} \coqdocvariable{Γ}\} \{\coqdocvar{B}:\coqdocdefinition{TypDep} \coqdocvariable{A}\} (\coqdocvar{b}:\coqdocdefinition{$\mathsf{Tm}$} \coqdocvariable{B})\coqdoceol
\coqdocindent{1.00em}
: \coqdocdefinition{$\mathsf{Tm}$} (\coqref{Groupoid.groupoid interpretation.Prod}{\coqdocdefinition{Prod}} (\coqref{Groupoid.groupoid interpretation.LamT}{\coqdocdefinition{$\Lambda$}} \coqdocvariable{B})) := \coqdocnotation{(}\coqexternalref{::'xCExBB' x '..' x ',' x}{http://coq.inria.fr/stdlib/Coq.Unicode.Utf8\_core}{\coqdocnotation{\ensuremath{\lambda}}} \coqdocvar{$\gamma$}\coqexternalref{::'xCExBB' x '..' x ',' x}{http://coq.inria.fr/stdlib/Coq.Unicode.Utf8\_core}{\coqdocnotation{,}} \coqdocnotation{(}\coqexternalref{::'xCExBB' x '..' x ',' x}{http://coq.inria.fr/stdlib/Coq.Unicode.Utf8\_core}{\coqdocnotation{\ensuremath{\lambda}}} \coqdocvar{t}\coqexternalref{::'xCExBB' x '..' x ',' x}{http://coq.inria.fr/stdlib/Coq.Unicode.Utf8\_core}{\coqdocnotation{,}} \coqdocvariable{b} \coqdocnotation{$\star$} \coqdocnotation{(}\coqdocvariable{$\gamma$} \coqdocnotation{;} \coqdocvariable{t}\coqdocnotation{)} \coqdocnotation{;} \coqdocvar{\_}\coqdocnotation{);} \coqref{Groupoid.groupoid interpretation.Lam 2}{\coqdocinstance{$\mathsf{Lam_{comp}}$}} \coqdocvariable{b}\coqdocnotation{)}.\coqdoceol
\coqdocemptyline
\coqdocemptyline
\end{coqdoccode}
  \paragraph{\textsc{Sigma}, \textsc{Pair} and \textsc{Projs}.}
  The rules for Σ types are interpreted using the 
  dependent sum \coqdocdefinition{$\Sigma$} on setoids.  
\begin{coqdoccode}
\coqdocemptyline
\coqdocnoindent
\coqdockw{Definition} \coqdef{Groupoid.groupoid interpretation.Sigma}{Sigma}{\coqdocdefinition{Sigma}} \{\coqdocvar{Γ}\} (\coqdocvar{A}:\coqdocdefinition{Typ} \coqdocvariable{Γ}) (\coqdocvar{F}:\coqdocdefinition{TypFam} \coqdocvariable{A}) \coqdoceol
\coqdocindent{1.00em}
: \coqdocdefinition{Typ} \coqdocvariable{Γ} := \coqdocnotation{(}\coqexternalref{::'xCExBB' x '..' x ',' x}{http://coq.inria.fr/stdlib/Coq.Unicode.Utf8\_core}{\coqdocnotation{\ensuremath{\lambda}}} \coqdocvar{$\gamma$}: \coqdocnotation{[}\coqdocvariable{Γ}\coqdocnotation{]}\coqexternalref{::'xCExBB' x '..' x ',' x}{http://coq.inria.fr/stdlib/Coq.Unicode.Utf8\_core}{\coqdocnotation{,}} \coqdocdefinition{$\Sigma$} (\coqdocvariable{F} \coqdocnotation{$\star$} \coqdocvariable{$\gamma$})\coqdocnotation{;} \coqref{Groupoid.groupoid interpretation.Sigma 1}{\coqdocinstance{$\mathsf{Sigma_{comp}}$}} \coqdocvariable{A} \coqdocvariable{F}\coqdocnotation{)}.\coqdoceol
\coqdocemptyline
\end{coqdoccode}
\noindent Pairing and projections are obtained
  by a context lift of pairing and projection of the underlying dependent sum.
\begin{coqdoccode}
\coqdocemptyline
\coqdocemptyline
\end{coqdoccode}


\subsection{Identity Types}


  One of the main interests of the setoid and groupoid interpretations is that they
  allow to interpret a type directed notion of equality which validates 
  the J eliminator of identity types but also various extensional principles,
  including functional extensionality. 
  For any terms \coqdocvariable{a} and \coqdocvariable{b} of a dependent type \coqdocvariable{A}:\coqdocdefinition{Typ} \coqdocvariable{Γ}, we note \coqref{Groupoid.groupoid interpretation.Id}{\coqdocdefinition{Id}} \coqdocvariable{a} \coqdocvariable{b} the equality type
  between \coqdocvariable{a} and \coqdocvariable{b} obtained by lifting $\sim_1$ to get a type depending on \coqdocvariable{Γ}.
\begin{coqdoccode}
\coqdocemptyline
\coqdocnoindent
\coqdockw{Definition} \coqdef{Groupoid.groupoid interpretation.Id}{Id}{\coqdocdefinition{Id}} \{\coqdocvar{Γ}\} (\coqdocvar{A}: \coqdocdefinition{Typ} \coqdocvariable{Γ}) (\coqdocvar{a} \coqdocvar{b} : \coqdocdefinition{$\mathsf{Tm}$} \coqdocvariable{A}) \coqdoceol
\coqdocindent{1.00em}
: \coqdocdefinition{Typ} \coqdocvariable{Γ} := \coqdocnotation{(}\coqexternalref{::'xCExBB' x '..' x ',' x}{http://coq.inria.fr/stdlib/Coq.Unicode.Utf8\_core}{\coqdocnotation{\ensuremath{\lambda}}} \coqdocvar{$\gamma$}\coqexternalref{::'xCExBB' x '..' x ',' x}{http://coq.inria.fr/stdlib/Coq.Unicode.Utf8\_core}{\coqdocnotation{,}} \coqdocnotation{(}\coqdocvariable{a} \coqdocnotation{$\star$} \coqdocvariable{$\gamma$} \coqdocnotation{$\sim_1$} \coqdocvariable{b} \coqdocnotation{$\star$} \coqdocvariable{$\gamma$} \coqdocnotation{;} \coqdocvar{\_}\coqdocnotation{);} \coqref{Groupoid.groupoid interpretation.Id 1}{\coqdocinstance{$\mathsf{Id_{comp}}$}} \coqdocvariable{A} \coqdocvariable{a} \coqdocvariable{b}\coqdocnotation{)}.\coqdoceol
\coqdocemptyline
\coqdocemptyline
\end{coqdoccode}
The introduction rule of identity types which corresponds to reflexivity is interpreted by the (lifting of) identity of the underlying setoid. \begin{coqdoccode}
\coqdocemptyline
\coqdocnoindent
\coqdockw{Definition} \coqdef{Groupoid.groupoid interpretation.Refl}{Refl}{\coqdocdefinition{Refl}} \coqdocvar{Γ} (\coqdocvar{A}: \coqdocdefinition{Typ} \coqdocvariable{Γ}) (\coqdocvar{a} : \coqdocdefinition{$\mathsf{Tm}$} \coqdocvariable{A}) \coqdoceol
\coqdocindent{1.00em}
: \coqdocdefinition{$\mathsf{Tm}$} (\coqref{Groupoid.groupoid interpretation.Id}{\coqdocdefinition{Id}} \coqdocvariable{a} \coqdocvariable{a}) := \coqdocnotation{(}\coqexternalref{::'xCExBB' x '..' x ',' x}{http://coq.inria.fr/stdlib/Coq.Unicode.Utf8\_core}{\coqdocnotation{\ensuremath{\lambda}}} \coqdocvar{$\gamma$}\coqexternalref{::'xCExBB' x '..' x ',' x}{http://coq.inria.fr/stdlib/Coq.Unicode.Utf8\_core}{\coqdocnotation{,}} \coqdocprojection{identity} (\coqdocvariable{a} \coqdocnotation{$\star$} \coqdocvariable{$\gamma$})\coqdocnotation{;} \coqref{Groupoid.groupoid interpretation.Refl 1}{\coqdocinstance{$\mathsf{Refl_{comp}}$}} \coqdocvar{\_} \coqdocvar{\_} \coqdocvar{\_}\coqdocnotation{)}.\coqdoceol
\coqdocemptyline
\coqdocemptyline
\end{coqdoccode}
We can interpret the \coqref{Groupoid.groupoid interpretation.J}{\coqdocdefinition{J}} eliminator of MLTT on \coqref{Groupoid.groupoid interpretation.Id}{\coqdocdefinition{Id}} using
  functoriality of \coqdocvariable{P} and of product (\coqref{Groupoid.groupoid interpretation.prod comp}{\coqdocdefinition{$\mathsf{\Pi_{comp}}$}}). In the definition
  of \coqref{Groupoid.groupoid interpretation.J}{\coqdocdefinition{J}}, the predicate \coqdocvariable{P} depends on the proof of equality, which is
  interpreted using a \coqref{Groupoid.groupoid interpretation.Sigma}{\coqdocdefinition{Sigma}} type. The functoriality of \coqdocvariable{P} is used on
  the term \coqref{Groupoid.groupoid interpretation.J Pair}{\coqdocdefinition{J\_Pair}} \coqdocvariable{e} \coqdocvariable{P} \coqdocvariable{$\gamma$}, which is a proof that (\coqdocvariable{a};\coqref{Groupoid.groupoid interpretation.Refl}{\coqdocdefinition{Refl}} \coqdocvariable{a}) is equal
  to (\coqdocvariable{b};\coqdocvariable{e}). To state the rule, we need to do a rewriting at 
  the level of terms, i.e., given an equality \coqdocvariable{e} : \coqdocvariable{T} $\sim_1$ \coqdocvariable{U} between two 
  types in \coqdocvariable{A}, we use the map from \coqdocvariable{t} : \coqdocdefinition{$\mathsf{Tm}$} \coqdocvariable{T} to 
  \coqdocvariable{t} \coqdockw{with} \coqdocvariable{e} : \coqdocdefinition{$\mathsf{Tm}$} \coqdocvariable{U} that comes from the
  functoriality of \coqdocvar{$\Pi$}:\begin{coqdoccode}
\coqdocemptyline
\coqdocnoindent
\coqdockw{Definition} \coqdef{Groupoid.groupoid interpretation.J}{J}{\coqdocdefinition{J}} \coqdocvar{Γ} (\coqdocvar{A}:\coqdocdefinition{Typ} \coqdocvariable{Γ}) (\coqdocvar{a} \coqdocvar{b}:\coqdocdefinition{$\mathsf{Tm}$} \coqdocvariable{A}) (\coqdocvar{P}:\coqdocdefinition{TypFam} (\coqref{Groupoid.groupoid interpretation.Sigma}{\coqdocdefinition{Sigma}} (\coqref{Groupoid.groupoid interpretation.LamT}{\coqdocdefinition{$\Lambda$}} (\coqref{Groupoid.groupoid interpretation.Id}{\coqdocdefinition{Id}} (\coqdocvariable{a} \coqdocnotation{$\circ$} \coqref{Groupoid.groupoid interpretation.Sub}{\coqdocdefinition{Sub}}) (\coqref{Groupoid.groupoid interpretation.Var}{\coqdocdefinition{Var}} \coqdocvariable{A})))))\coqdoceol
\coqdocindent{7.50em}
(\coqdocvar{e}:\coqdocdefinition{$\mathsf{Tm}$} (\coqref{Groupoid.groupoid interpretation.Id}{\coqdocdefinition{Id}} \coqdocvariable{a} \coqdocvariable{b})) (\coqdocvar{p}:\coqdocdefinition{$\mathsf{Tm}$} (\coqdocvariable{P}\coqref{Groupoid.groupoid interpretation.::x 'x7Bx7B' x 'x7Dx7D'}{\coqdocnotation{\{\{}}\coqref{Groupoid.groupoid interpretation.Pair}{\coqdocdefinition{Pair}} (\coqref{Groupoid.groupoid interpretation.Refl}{\coqdocdefinition{Refl}} \coqdocvariable{a} \coqref{Groupoid.groupoid interpretation.::x 'with' x}{\coqdocnotation{with}} \coqref{Groupoid.groupoid interpretation.BetaT2}{\coqdocdefinition{$\mathsf{BetaT'}$}})\coqref{Groupoid.groupoid interpretation.::x 'x7Bx7B' x 'x7Dx7D'}{\coqdocnotation{\}\}}})):\coqdoceol
\coqdocindent{1.00em}
\coqdocdefinition{$\mathsf{Tm}$} (\coqdocvariable{P}\coqref{Groupoid.groupoid interpretation.::x 'x7Bx7B' x 'x7Dx7D'}{\coqdocnotation{\{\{}}\coqref{Groupoid.groupoid interpretation.Pair}{\coqdocdefinition{Pair}} (\coqdocvariable{e} \coqref{Groupoid.groupoid interpretation.::x 'with' x}{\coqdocnotation{with}} \coqref{Groupoid.groupoid interpretation.BetaT2}{\coqdocdefinition{$\mathsf{BetaT'}$}})\coqref{Groupoid.groupoid interpretation.::x 'x7Bx7B' x 'x7Dx7D'}{\coqdocnotation{\}\}}}) :=\coqdoceol
\coqdocindent{1.00em}
\coqref{Groupoid.groupoid interpretation.prod comp}{\coqdocdefinition{$\mathsf{\Pi_{comp}}$}} \coqdocnotation{(}\coqexternalref{::'xCExBB' x '..' x ',' x}{http://coq.inria.fr/stdlib/Coq.Unicode.Utf8\_core}{\coqdocnotation{\ensuremath{\lambda}}} \coqdocvar{$\gamma$}\coqexternalref{::'xCExBB' x '..' x ',' x}{http://coq.inria.fr/stdlib/Coq.Unicode.Utf8\_core}{\coqdocnotation{,}} \coqexternalref{::'xCExBB' x '..' x ',' x}{http://coq.inria.fr/stdlib/Coq.Unicode.Utf8\_core}{\coqdocnotation{(}}\coqdocabbreviation{map} (\coqdocvariable{P} \coqdocnotation{$\star$} \coqdocvariable{$\gamma$}) (\coqref{Groupoid.groupoid interpretation.J Pair}{\coqdocdefinition{J\_Pair}} \coqdocvariable{e} \coqdocvariable{P} \coqdocvariable{$\gamma$})\coqexternalref{::'xCExBB' x '..' x ',' x}{http://coq.inria.fr/stdlib/Coq.Unicode.Utf8\_core}{\coqdocnotation{)}}\coqdocnotation{;} \coqref{Groupoid.groupoid interpretation.J 1}{\coqdocinstance{$\mathsf{J_{comp}}$}} \coqdocvar{\_} \coqdocvar{\_}\coqdocnotation{)} \coqdocnotation{$\star$} \coqdocvariable{p}.\coqdoceol
\coqdocemptyline
\end{coqdoccode}
\noindent where \coqref{Groupoid.groupoid interpretation.BetaT2}{\coqdocdefinition{$\mathsf{BetaT'}$}} is another dedicated version of the $\beta$-rule for \coqref{Groupoid.groupoid interpretation.LamT}{\coqdocdefinition{$\Lambda$}}. \begin{coqdoccode}
\coqdocemptyline
\end{coqdoccode}
