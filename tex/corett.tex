\def\Elt#1{\texttt{Elt}(#1)}
\def\Univ{\ensuremath{\mathcal{U}}}
\def\Id#1#2#3{\texttt{Id}_{#1}\,#2\ #3}
\def\Equiv#1#2{\texttt{Equiv}\ \Elt{#1}\ \Elt{#2}}
\def\Eq#1#2#3{\texttt{Eq}_{#1}\ #2\ #3}
\def\refl#1#2{\texttt{refl}_{#1}\ #2}
\def\funext#1{\texttt{fun\_ext}(#1)}
\def\zeroType{\ensuremath{\mathbb{O}}\xspace}
\def\oneType{\ensuremath{\mathbb{1}}\xspace}
\def\twoType{\ensuremath{\mathbb{2}}\xspace}
\def\hzeroType{\ensuremath{\mathbb{O}}}
\def\honeType{\ensuremath{\mathbb{1}}}
\def\htwoType{\ensuremath{\mathbb{2}}}

\subsection{Martin-Löf Type Theory with Functional Extensionality}
\label{sec:definitions}


For the purpose of this paper we study a restricted source theory
resembling a cut-down version of the core language of the \Coq system,
with only one \Type{} universe (see \cite{DBLP:bibsonomy_cupart} for an
in-depth study of this system). This is basically Martin-Löf Type Theory
(without \Type{} : \Type{}). First we introduce the definitions of the
various objects at hand.  We start with the term language of a dependent
λ-calculus: we have a countable set of variables $x, y, z$, and the
usual typed λ-abstraction $λ x : τ, b$, à la Church, application $t~u$,
the dependent product and sum types $Π/Σ x : A. B$, and an identity type
$\Id{T}{t}{u}$. 
% There is a single universe $\Univ$. For $T$ and $U$ in
% $\Univ{}$, the type of (Set)-isomorphisms $T `= U$, is definable
% directly using the other type constructors (see \ref{sec:universe}).

The typing judgment for this calculus is written $\tcheck{Γ}{ψ}{t}{T}$
(Figure \ref{fig:emltt}) where $Γ$ is a context of declarations $Γ ::=
\Nil `| Γ, x : τ$, $t$ and $T$ are terms. If we have a valid derivation
of this judgment, we say $t$ has type $T$ in $Γ$.
%  and assume that $T$ is
% not $\Type{}$ unless explicilty stated (this just allows us to do
% without a separate judgment for carving out the types).

Most of the rules are standard. The definitional equality $A = B$ is
defined as the congruence on type and term formers compatible with
β-reductions for abstractions and projections.

\subsubsection{Identity type.} 
%
The identity type in the source theory is the standard
Martin-Löf identity type $\Id{T}{t}{u}$, generated from an
introduction rule for reflexivity with the usual \texttt{J} eliminator
and its \emph{propositional} reduction rule. The \texttt{J} reduction 
rule will actually be valid definitionally in the model for \emph{closed} terms.

\subsubsection{Functional Extensionality.} 
%
The proof-relevant equality of functions in the source theory for which we want to
give a computational model appears in rule \lrule{Fun-Ext},
extending the formation rules of identity types on dependent product. 
%
Equality of dependent functions $f$ and $g$ of type $\Pi \vdecl{x}{A}\mdot B$ are
introduced by giving a witness of $\Pi t : A, \Id{B \{\subs x {t}\}}
{(f \ t)}{(g \ t)}$.
%
The \texttt{J} rule for dependent functions witnesses the
\emph{naturality} of every predicate constructed in the source type
theory.
%
This rule corresponds 
  to the introduction of equality on dependent functions in %\cite{DBLP:conf/popl/LicataH12}%.

\def\hFin#1{\mathtt{\hat Fin}\ #1}
\def\Fin#1{\texttt{Fin}\ #1}
\def\fin#1#2{\underline{#1}_{#2}}

% \subsubsection{Universe.} The universe $\Univ$ is closed under Σ, Π,
% \zeroType, \oneType, \twoType and $\texttt{Id}$ in elements of $\Univ$, 
% \emph{not} type equivalences. The
% constructors are circumflexed, e.g. $\hat \Pi$ and $\hat \Sigma$ are
% introduced with the rules:
% %
% %\vspace{-1em} 
% \begin{mathpar}
% \irule{Univ-$\hat \Pi$, -$\hat \Sigma$}
% {\tcheck{\Gamma}{ψ}{A}{\Univ} \\
% \tcheck{\Gamma, x : \Elt{A}}{ψ}{B}{\Univ}}
% {\tcheck{\Gamma}{ψ}{\hat \Pi/\hat \Sigma x : A. B}{\Univ}}
% \end{mathpar}

% \noindent 
% For presentation purposes, we do not detail here the treatment of finite
% types (see \cite{altenkirch-mcbride-wierstra:ott-now} for a standard treatment).
% The extension to W-types would be straightforward.
% %
% \texttt{Elt} is a $\type{}$-forming map from $\Univ$ that acts as a
% homomorphism, e.g. its action on products is: $$\Elt{\hat \Pi x : A. B} = Π x : \Elt{A}. \Elt{B}$$
% \begin{mathpar}
% \begin{array}{lcll}
%   \texttt{Elt} : \Univ & → & \Type{} & \\
%   \Elt{\{\hat \Pi/\hat \Sigma\} x : A. B} & = & \{Π/Σ\} x :
%   \Elt{A}. \Elt{B} & \\
%   \Elt{\hat{τ}} & = & τ & {τ \in \{ \hzeroType, \honeType, \htwoType \}} \\
%   % \Elt{\hFin{n}} & = & \Fin{n} & \\
% %  \Elt{a =_{τ} b} & = & \texttt{Id}_{\Elt{τ}}\ a\ b & \\
%   \Elt{\hat{C}[X]} & = & C[\Elt{X}] & \text{(homomorphism)}
% \end{array}
% \end{mathpar}


\begin{figure}[t]
% \hspace{-0.0\textwidth}
% \begin{minipage}{1.0\textwidth}
\begin{mathpar}

\irule{Empty}{}{\WFc{\Nil}{ψ}}

\irule{Decl}
{\ttcheck{\Gamma}{ψ}{T}{\Type{}} \hspace{-1em}\\
 %T \neq \Univ, %\hspace{-1em}\\
 x \not \in \Gamma }
{\WFc{Γ, \vdecl{x}{T}}{ψ}}

\irule{Univ}
{}
{\ttcheck{\Gamma}{ψ}{\Univ}{\Type{}}}

\irule{Var}
{\WFc{\Gamma}{ψ} \hspace{-1em}\\ 
  (\vdecl{x}{T}) \in \Gamma}
{\tcheck{\Gamma}{ψ}{x}{T}}


\irule{Prod/Sigma}
{\ttcheck{\Gamma, \vdecl{x}{A}}{ψ}{B}{\Type{}} \hspace{-1em}}
{\ttcheck{\Gamma}{ψ}{\Pi/\Sigma \vdecl{x}{A}\mdot B}{\Type{}}}

\irule{Pair}
{%\ttcheck{\Gamma}{ψ}{Σ x : A. B}{\Type{}} \\
\tcheck{\Gamma}{ψ}{t}{A} \hspace{-1em} \\ \tcheck{\Gamma}{ψ}{u}{B\{\subs
    x t}\}}
{\tcheck{\Gamma}{ψ}{(t,u)_{x : A. B}}{Σ x : A. B}}

\irule{Proj1}
{\tcheck{\Gamma}{ψ}{t}{\Sigma \vdecl{x}{A}\mdot B}}
{\tcheck{\Gamma}{ψ}{\ensuremath{π_1 t}}{A}}

\irule{Proj2}
{\tcheck{\Gamma}{ψ}{t}{\Sigma \vdecl{x}{A}\mdot B}}
{\tcheck{\Gamma}{ψ}{π_2 t}{B\{\subs x {π_1 t}\}}}
% \irule{Fin}
% {n \in \{ 0, 1, 2 \}}
% {\tcheck{\Gamma}{ψ}{\Fin{n}}{\Type{}}}
% \irule{Fin-Intro}
% {k, n \in \{ 0, 1, 2 \}, k < n}
% {\tcheck{\Gamma}{ψ}{\fin{k}{n}}{\Fin{n}}}
% \irule{Fin-Elim}
% {n \in \{ 0, 1, 2 \} \\
%   \tcheck{\Gamma, x : \Fin{n}}{ψ}{P}{\Type{}} \\
%   ∀ k < n. \tcheck{\Gamma}{ψ}{p_k}{P\ \fin{k}{n}} \\
%   \tcheck{\Gamma}{ψ}{f}{\Fin{n}}}
% {\tcheck{\Gamma}{ψ}{\texttt{felim}_{P}\ \overrightarrow{p_k}\ f}{P\
% f}}

\irule{Conv}
{\tcheck{\Gamma}{ψ}{t}{A} \hspace{-1em} \\ \ttcheck{Γ}{ψ}{B}{\Type{}} \hspace{-1em} \\ \tconv{ψ}{A}{B}}
{\tcheck{\Gamma}{ψ}{t}{B}}


\irule{Lam}
{\tcheck{\Gamma, \vdecl{x}{A}}{ψ}{t}{B}}
{\tcheck{\Gamma}{ψ}{\lambda \vdecl{x}{A}\mdot t}{\Pi \vdecl{x}{A}\mdot B}}

\irule{App}
{\tcheck{\Gamma}{ψ}{t}{\Pi \vdecl{x}{A}\mdot B} \hspace{-1em} \\
 \tcheck{\Gamma}{ψ}{t'}{A}}
{\tcheck{\Gamma}{ψ}{t\ t'}{B\{\subs x {t'}\}}}

\irule{Id}
{\ttcheck{\Gamma}{ψ}{T}{\Type{}}\quad
\tcheck{\Gamma}{ψ}{A, B}{T}}
{\ttcheck{\Gamma}{ψ}{\Id{T}{A}{B}}{\Type{}}}
\qquad
\irule{Id-Intro}
{\tcheck{\Gamma}{ψ}{t}{T}}
{\tcheck{\Gamma}{ψ}{\refl{T}{t}}{\Id{T}{t}{t}} }
\qquad
\irule{Fun-Ext}
{\tcheck{\Gamma}{ψ}{e}{ \Pi t : A, \Id{B \{\subs x {t}\}} {(f \ t)}{(g \ t)}}}
{\tcheck{\Gamma}{ψ}{\funext{e}}{\Id{\Pi \vdecl{x}{A}\mdot B}{f}{g} }}


% \irule{Equiv-Intro}
% {\tcheck{\Gamma}{ψ}{i}{ \Elt{A} ` = \Elt{B}}}
% {\tcheck{\Gamma}{ψ}{\texttt{equiv}\ i}{\Id{\Univ}{A}{B}}}


\irule{Id-Elim (J)}
{\tcheck{\Gamma}{ψ}{i}{\Id{T}{t}{u}} \\
\ttcheck{\Gamma, x : T, e : \Id{T}{t}{x}}{ψ}{P}{\Type{}} \\
\tcheck{\Gamma}{ψ}{p}{P\{\subs x t\}\{\refl{T}{t}/e\}}}
{\tcheck{\Gamma}{ψ}{\texttt{J}_{λx\ e. P}~i~p}{P\{\subs x u, \subs e i\}}}
\end{mathpar}
\caption{Typing judgments for our extended MLTT}\label{fig:emltt}
\end{figure}



% We abuse notations
% and consider Π, Σ and \texttt{Id} as codes when seen as inhabitants of
% \Univ, and as regular syntax for the type constructors in the rest of
% the type theory.

% P[t,refl] -> P[t,equiv i]: P is abstracted by x : U and Eq t x, can't look
% inside the universe x, but p : P[t,refl] might use J q refl =
% q. Replace by J q (equiv i) = q[i.1 y/y]. 

%We write $\tcheck{Γ}{}{T}{s}$
%as a shorthand for $\tcheck{Γ}{}{T}{\Type{u}}$ for some universe $u$.
% \begin{figure}
% \begin{mathpar}

% % \irule{Univ-Id}
% % {\tcheck{\Gamma}{ψ}{A}{\Univ} \\
% % \tcheck{\Gamma}{ψ}{a, b}{\Elt{A}}}
% % {\tcheck{\Gamma}{ψ}{a =_A b}{\Univ}}

% \irule{Univ-Fin}
% {τ \in \{ \hzeroType, \honeType, \htwoType \}}
% {\tcheck{\Gamma}{ψ}{\hat{τ}}{\Univ}}

% \irule{Univ-$\hat \Pi$, -$\hat \Sigma$}
% {\tcheck{\Gamma}{ψ}{A}{\Univ} \\
% \tcheck{\Gamma, x : \Elt{A}}{ψ}{B}{\Univ}}
% {\tcheck{\Gamma}{ψ}{\hat \Pi/\hat \Sigma x : A. B}{\Univ}}
% \end{mathpar}
% \caption{Definition of \Univ{} (inductive-recursive with \Elt{\_})}\label{fig:univ}
% \end{figure}

% \begin{figure}[!h]
% \begin{mathpar}

% \begin{array}{lcll}
%   \texttt{Elt} : \Univ & → & \Type{} & \\
%   \Elt{\{\hat \Pi/\hat \Sigma\} x : A. B} & = & \{Π/Σ\} x :
%   \Elt{A}. \Elt{B} & \\
%   \Elt{\hat{τ}} & = & τ & {τ \in \{ \hzeroType, \honeType, \htwoType \}} \\
%   % \Elt{\hFin{n}} & = & \Fin{n} & \\
% %  \Elt{a =_{τ} b} & = & \texttt{Id}_{\Elt{τ}}\ a\ b & \\
%   \Elt{\hat{C}[X]} & = & C[\Elt{X}] & \text{(homomorphism)}
% \end{array}

% \begin{array}{lcll}
%   \texttt{Eq}_{U : \Univ} : \Elt{U} → \Elt{U} & → & \Type{} & \\
%   \Eq{\hat \Pi x : A. B}{f}{g} & = & Π x : \Elt{A}. \Elt{f\ x =_B g\ x} \\
%   \Eq{\hat \Sigma x : A. B}{t}{u} & = & %Π t u : (Σ x : \Elt{A}. \Elt{B}), 
%   Π e : \Elt{π_1\ t =_A π_1\ u}, π_2\ t =_{B\ (π_1\ t)} \texttt{J}_{λx. \Elt{B x}}\
%   e\ (π_2\ u)
%  % \Eq{a =_{τ} b}{e}{e'} & = & \mathbf{1} &
% \end{array}

% \end{mathpar}
% \caption{Universe decoding}\label{fig:univelt}
% \end{figure}

