\section{Introduction}
\label{sec:introduction}
A notorious difficulty with intensional type theories like Martin-Löf
type theory or the calculus of inductive constructions is the
lack of extensionality principles in the core theory, and notably in
its notion of propositional equality. This makes the theory both
inflexible with respect to the notion of equality one might want to
use on a given type and also departs from the traditional equality
notion from set theory. Functional extensionality (pointwise equal
functions are equal), propositional extensionality (logically
equivalent propositions are equal) and proof irrelevance (all proofs
of the same proposition are equal) are principles that are valid
extensions of type theory and can be internalized in an intensional
theory, in the manner of Altenkirch et al.~\cite{altenkirch-mcbride-wierstra:ott-now}.  Another
extensionality principle coming from homotopy theory is univalence,
whose first instance is the idea that isomorphic types are
equal~\cite{Voevodsky:2011yq,Pelayo:2012uq}. All these principles
should be imposeable on a type theory like MLTT because, intuitively,
no term can distinguish between isomorphic types, pointwise equal
functions, logically equivalent propositions or two proofs of the same
proposition. This hints at the idea that there ought to exist an
internal model of type theory where equality is defined on a type by
type basis using the aforementioned principles and a translation that
witnesses that any term of the theory is equipped with proofs that
they respect these properties. Formalizing this definitional
translation is the goal of this paper.

% However, before even starting to think about formalizing the
% translation, we need to enhance the universe system of \Coq to handle
% polymorphism. As the translation will be given as a set of definitions
% inside the assistant, we will have to instantiate these definitions at
% various levels depending on the source term. With fixed universes for
% each translated construct, we would quickly run into inconsistencies. We
% hence also present the design and implementation of a universe polymorphic
% extension of \Coq, that is conservative over the original system.

The central change in the theory is in the treatment of equality.  Much
interest has been devoted to the study of the identity type of type
theory and models thereof, starting with the groupoid model of Hofmann
and Streicher \cite{groupoid-interp}. This eventually led to the
introduction of homotopy type theory and the study of the ω-groupoid
model of type theory with identity types, which validates extensionality
principles.
% \cite{DBLP:journals/corr/abs-0812-0409}. 
This model in turn
guides work to redesign type theory itself to profit from its
richness, and develop a new theory that internalizes the new
principles. Preliminary attempts have been made, notably by Licata and
Harper \cite{DBLP:conf/popl/LicataH12} who develop a 2-dimensional
version of an hybrid intensional/extensional type theory which
integrates functional extensionality and univalence in the definition
of equality. Work is underway to make it more intensional, and this
starts by making the step to higher dimensions, whether finite (weak
n-groupoids) or infinite (weak ω-groupoids)
\cite{DBLP:conf/csl/AltenkirchR12}. %
%
Our work here concentrates on the internalization in \Coq of Hofmann and
Streicher's groupoid model where we can have a self-contained definition
of the structures involved.
%
Our first motivation to implement this translation is to explore the
interpretation of type theory in groupoids in a completely
intensional setting and in the type theoretic language, leaving no space
for imprecision on the notions of equality and coherence involved.  We
also hope to give with this translation a basic exposition of the
possible type theoretic implications of the groupoid/homotopy models,
bridging a gap in the literature. 
%
On the technical side, we nevertheless have to slightly move away from
Hofmann and Streicher's interpretation.  In particular, it is not
possible to interpret a univalent universe in our type-theoretic
formalization of their model; and this for two important reasons that we
now discuss.


\paragraph{\bf A univalent model that makes no use of univalence.}


Our long term goal is to provide an interpretation of homotopy type
theory into type theory (without extensional principles).
%
This would give a meaning to all extensionality
principles without relying on them in the target theory.

However, if we use a traditional approach and formalize groupoid laws
using the identity type, it turns out that the type of isomorphisms
between two objects $x$ and $y$ of a groupoid, noted $x \sim_1 y$, must
also be formalized using the identity type. Or, putting it in the
homotopy type theoretic language, we have to consider univalent
groupoids.
%
This is because, in the groupoid interpretation, we need isomorphisms
between groupoids to satisfy groupoids laws using the identity
type as the notion of equality between isomorphisms. Then, using
functional extensionality, this amounts to compare the two functions
pointwise, but again using the identity type, and not the internal
notion of equality in the groupoid, given by $\sim_1$.
%
This means that $\sim_1$ has to coincide with the identity type, which
is precisely the property  of being a univalent groupoid.
%
But then, this means that isomorphisms between groupoids should be
reflected in the identity type, which forces the target theory to
satisfy univalence already\ldots
%
So the first conclusion is that an internalization of the groupoid
interpretation in the style of formalization of categories presented
in~\cite{hottbook} would require the target theory to be univalent already.
 
To avoid this issue in our internalization of groupoids, the groupoid
laws are imposed using a notion of 2-dimensional equality that does
not have to be the identity type, just an equivalence relation.
%
Then, to enforce that the types of isomorphisms constitute
(homotopical) sets, we still use the identity type, but only to
express triviality of higher dimensions, not coherences themselves.
%
This interpretation of strictness is closer to the idea that a
groupoid is a weak ω-groupoid for which all equalities at dimension 2
are the same.
%
%
Note that our presentation requires less properties on identity types,
but we still need the axiom of functional extensionality to prove
triviality of higher dimensions for the groupoid of functors.
%
Also, this indicates that if we scale to ω-groupoids, the presence of
identity types in the core type theory will
not be necessary anymore and so the core type theory will be axiom free.
%
Thus, this paper can be seen as a proof of concept that it is possible
to interpret homotopy type theory into type theory without identity
types.


\paragraph{\bf Taking size issue into account.}
%

In Hofmann and Streicher's groupoid model, a type $A$ depending on
context $\Gamma$ is interpreted as a functor 
\hspace{-1em}
$$
\sem{A} \ : \ \sem{\Gamma} \longrightarrow \GPD
\hspace{-1em}
$$
%
where $\GPD$ is said to be \emph{``the (large) category of groupoids''}. 
%
But groupoids form naturally a 2-category/2-groupoid and not a
category/groupoid, because functors/isomorphisms between (large)
groupoids do not form a set but an arbitrary collection (a type in type
theory). Otherwise said, groupoids are not enriched over themselves.
%
Hofmann and Streicher have solved this issue as we usually do in
category theory, by considering small groupoids, that is groupoids for
which the type of objects is also a set, and relying on set-theoretic
extensional equality to witness natural isomorphisms between groupoid 
isomorphisms. This is fine as long as we consider set theory.
%
But when moving to type theory, small groupoids that form a
category/groupoid actually correspond to setoids. Indeed, to impose the
smallness condition internally, we have to impose either that (i)
the identity type on the type of objects is an h-prop, or that (ii) $x
\sim_1 y$ is an h-prop for all objects $x$ and $y$. But
condition (i) does not give rise to a category: it is not possible to prove that
functors form an h-set because we have an hypothesis on the identity type
where we need an hypothesis on $\sim_1$. 
%
This means that only condition (ii) is valid, so that the
correct notion of small groupoids in type theory is the notion of
setoids.
%
Note that this is also true for the notion of univalent groupoids
introduced above as in that case, $\sim_1$ coincides with the identity type.
%
In other words, using an internalization of groupoids, one can only
interpret types as functors into the groupoid of setoids. This is what we do
in this paper, which prevents us from giving a complete formulation of
univalence in the source theory. That is, the invariance by
isomorphism principle is respected in the model but has no counterpart
in the source theory. 

Another way to solve this size issue would be to formalize the notion of
2-groupoids instead but we do not think that the interpretation would
have gained much. We believe that the real challenge is to interpret
$\omega$-groupoids which is the proper notion of self-enriched 
groupoid-like structure and is the subject of on-going work.

% coherences even limited at
% groupoids, the the structures are already quite tedious to manipulate and we
% found some interesting conditions on the structure of $Π$ and $Σ$ types
% that we believe were never presented in this form before. In the
% development, we strived to be as generic as possible and use the
% abstract structures of category theory to not be essentially tied to a
% particular dimension. That level of genericity also relies on the
% extension of the system with universe polymorphism.

% This translation could also be used to complete a forcing
% translation of type theory into type theory~\cite{hal-00685150}. By
% building up forcing layers on top of a core type theory, one can
% introduce new logical operators or type constructors (e.g. modal logic,
% recursive types), defined by translation. For correctness, this
% translation relies on a type theory that integrates proof-irrelevance
% and functional extensionality. The present translation gives the
% expressive power needed to compose with the forcing translation and get
% a fully definitional extension of type theory with forcing.

% To summarize, our contributions are: 
% \begin{enumerate}
% \item A theory of groupoids compatible with the ω-groupoids approach
%   and a partially mechanized interpretation of type theory into groupoids.
% \item A type-theoretic description of the necessary conditions on the
%   interpretation of function types, including dependent product types
%   which give rise to dependent groupoid functors and dependent sums.
% \item A proof of concept that it is possible to interpret homotopy
%   type theory into type theory without identity types.
% \end{enumerate}

\paragraph{\bf Outline of the paper.}
% The paper is organized as follows: 
Section \ref{sec:setting-translation} introduces the source type
theory of the translation and some features of the proof assistant
that are used in the formalization. The formal model includes a
formalization of groupoids and associated structures
(\S\ref{sec:w2gpds}-\ref{sec:rew}) and a construction of the groupoids
interpreting the standard type constructors
(\S\ref{sec:depprod}-\ref{sec:sigma}). Section
\ref{sec:interpretation} presents a formal interpretation of the
dependent type theory in the groupoid model and gives proofs that our
interpretation validates functional extensionality~(\S\ref{sec:extprinc}).
 % and conclude (Section \ref{sec:conclusion}).
%

%%% Local Variables: 
%%% mode: latex
%%% TeX-master: "main"
%%% End: 