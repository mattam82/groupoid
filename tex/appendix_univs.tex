We show here that the algebraic universes are sufficient to typecheck
terms in the calculus, and that the kernel only needs to handle atomic
constraints. Indeed, we have the invariant that taking least upper
bounds $\lub{u}{v}$ will always give algebraic universes and that the
constraints to be checked during conversion can always be reduced to
atomic ones.

To show this, we first have to introduce a few definitions related to the form 
of universes appearing in terms:

\begin{definition}[Arity]
  A term $t$ is an arity if it is of the form $Π \overrightarrow{(x : T)}, \Type{u}$.
\end{definition}

\begin{definition}[Non-algebraic term]
  A term $t$ is \emph{non-algebraic} if all universes appearing in it are
  variables/levels.
\end{definition}

\begin{definition}[Depth-1 algebraic term]
  A term $t$ is depth-1 algebraic if:
  \begin{itemize}
  \item $t$ is an arity $Π \overrightarrow{(x : T)}, \Type{u}$, 
    the types $\overrightarrow{T}$ are non-algebraic and $u$ is an algebraic universe.
  \item Otherwise, if $t$ is non-algebraic.
  \end{itemize}
\end{definition}

We can show:
\begin{lemma}[Depth-1 algebraic types]
  If $\tcheck{Γ}{ψ}{t}{T}$ then $t$ is non-algebraic and $T$ is depth-1 algebraic.
  If $\WFc{Γ}{ψ}$ and $\vdecl{x}{T} \in Γ$ then $T$ is non-algebraic.
\end{lemma}
\begin{proof}
  By mutual induction on the derivations.
  The interesting case is \lrule{Prod}:
  We have by induction that $s$ and $s'$ are depth-1 algebraic, we have
  to show that their least upper bound is as well: 
  $\lub{\max{le}{lt}}{\max{le'}{lt'}} = \max{le \cup le'}{lt \cup lt'}$.
  Clearly, least upper bounds of algebraic universes stay in the subset
  of algebraic universes.

  For \lrule{Lambda}, we have to check that the domain $A$ is
  non-algebraic: this follows by induction on the first premise, as $A$
  appears in a term position.

  The other cases follow by induction.
\end{proof}

This invariant is useful to show that the constraints necessary to
typecheck a term can be reduced to a set of \emph{atomic} constraints
between universe levels. Looking at a typing derivation, we see that
constraints are checked only in the rule \lrule{R-Type}, in a premise
of a \lrule{Conv} rule application:
\begin{mathpar}
\irule{}
{\tgenconstr{ψ}{u}{\relR}{v}}
{\tgenconv{\relR}{ψ}{\Type{u}}{\Type{v}}}
\end{mathpar}

We know that the type on the right is non-algebraic due to
\lrule{Conv}'s second premise, hence $v$ is a variable $i$. For
\lrule{$\preceq$-Type}, $u$ could be depth-1 algebraic and the
constraint we must check could be of the form: $\max{\vec{le}}{\vec{lt}}
\leq i$.  This can be translated to an equivalent set of atomic
constraints:
\[\max{\vec{le}}{\vec{lt}} \leq i "<=>" \overrightarrow{le \leq i} ∧ \overrightarrow{lt < i}\]

For \lrule{Conv-Type}, we know that the two types are non-algebraic as
they come from domains of products or are non-algebraic in the first
place (arities are convertible with arities only).
Hence the kernel need only check that sets of atomic constraints are
consistent. 
