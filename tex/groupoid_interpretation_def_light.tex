\coqlibrary{groupoid interpretation def light}{Library }{groupoid\_interpretation\_def\_light}

\begin{coqdoccode}
\end{coqdoccode}


  We now organize our formalization of groupoids into a model of the dependent 
  type theory introduced in Section~\ref{sec:definitions}.
  The interpretation is based on the notion of categories with families 
  introduced by Dybjer~\cite{dybjer:internaltt} later used in \cite{groupoid-interp}.
  This interpretation can also be seen as an extension of the Takeuti-Gandy interpretation of simple type theory, recently generalized to dependent type theory by Coquand et al. using Kan semisimplicial sets or cubical sets~\cite{barras:_gener_takeut_gandy_inter}. 


  There are two main novelties in our interpretation. First, we
  take advantage of universe polymorphism to interpret dependent types
  directly as functors into \coqdocdefinition{$\mathsf{Type}_{0}^1$}. Second, we provide an
  interpretation in a model where structures that are definitionally
  equal for Kan semisimplicial sets are only homotopically equal, which
  requires more care in the definitions (see for instance the definition
  of \coqref{groupoid interpretation def light.Lam}{\coqdocdefinition{Lam}} in Section \ref{sec:interp} which mixes two points of view
  on fibrations).


  We only present the computational part of the interpretation, the
  proofs of functoriality and naturality are not detailed but are valid
  in the model. Some of them are available in the \Coq development but
  we have also admitted some lemmas, due to the complete absence of
  automated rewriting on \coqdockw{Type} in the current version of \Coq. This
  will be addressed by the first author in a future version, relying on
  the polymorphic universe extension.


\subsection{Dependent types}




  The judgment context $\Gamma \vdash$ of Section
  \ref{sec:definitions} is represented in \Coq as a groupoid, noted
  \coqdockw{Context} := \coqdocvar{$\mathsf{Type}_1$}. The empty context (Rule \textsc{Empty})
  is interpreted as the groupoid with exactly one element at each
  dimension.  Types in a context \coqdocvariable{Γ}, noted \coqref{groupoid interpretation def light.Typ}{\coqdocdefinition{Typ}} \coqdocvariable{Γ}, are (context)
  functors from \coqdocvariable{Γ} to the groupoid of setoids \coqdocdefinition{$\mathsf{Type}_{0}^1$}.  Thus, a
  judgment $\Gamma \vdash A : \Type{}$ is represented as a term \coqdocvariable{A} of
  type \coqref{groupoid interpretation def light.Typ}{\coqdocdefinition{Typ}} \coqdocvariable{Γ}. Context extension (Rule \textsc{Decl}) is given by
  dependent sums, i.e., the judgment $\Gamma, x:A \vdash$ is represented
  as \coqdocvar{$\Sigma$} \coqdocvariable{A}.


\begin{coqdoccode}
\end{coqdoccode}
Terms of \coqdocvariable{A} introduced by a sequent $\Gamma \vdash x:A$ are
  dependent (context) functors from \coqdocvariable{Γ} to \coqdocvariable{A} that return for each
  context valuation \coqdocvariable{$\gamma$}, an object of \coqdocvariable{A} $\star$ \coqdocvariable{$\gamma$} respecting equality of
  contexts.  The type of terms of \coqdocvariable{A} is noted \coqref{groupoid interpretation def light.Elt}{\coqdocdefinition{$\mathsf{Tm}$}} \coqdocvariable{A} := [\coqdocdefinition{$\Pi$} \coqdocvariable{A}]
  (context is implicit).  

  A dependent type $\Gamma, x:A \vdash B$ is interpreted in two
  equivalent ways: simply as a type \coqref{groupoid interpretation def light.TypDep}{\coqdocdefinition{TypDep}} \coqdocvariable{A} := \coqref{groupoid interpretation def light.Typ}{\coqdocdefinition{Typ}} (\coqdocvar{$\Sigma$} \coqdocvariable{A}) over the
  dependent sum of \coqdocvariable{Γ} and \coqdocvariable{A} or as a type family \coqref{groupoid interpretation def light.TypFam}{\coqdocdefinition{TypFam}} \coqdocvariable{A} over \coqdocvariable{A}
  (corresponding to a family of sets in constructive mathematics). A
  type family can be seen as a fibration (or bundle) from \coqdocvariable{B} to \coqdocvariable{A}.
  In what follows, the suffix $\mathsf{_{comp}}$ is given to proofs of 
  (dependent) functoriality.
\begin{coqdoccode}
\coqdocemptyline
\coqdocnoindent
\coqdockw{Definition} \coqdef{groupoid interpretation def light.TypFam}{TypFam}{\coqdocdefinition{TypFam}} \{\coqdocvar{Γ} : \coqref{groupoid interpretation def light.Context}{\coqdocdefinition{Context}}\} (\coqdocvar{A}: \coqref{groupoid interpretation def light.Typ}{\coqdocdefinition{Typ}} \coqdocvariable{Γ}) := \coqdoceol
\coqdocindent{1.00em}
\coqdocnotation{[}\coqdocdefinition{$\Pi$} \coqdocnotation{(}\coqexternalref{::'xCExBB' x '..' x ',' x}{http://coq.inria.fr/stdlib/Coq.Unicode.Utf8\_core}{\coqdocnotation{\ensuremath{\lambda}}} \coqdocvar{$\gamma$}\coqexternalref{::'xCExBB' x '..' x ',' x}{http://coq.inria.fr/stdlib/Coq.Unicode.Utf8\_core}{\coqdocnotation{,}} \coqdocnotation{ } \coqdocnotation{(}\coqdocvariable{A} \coqdocnotation{$\star$} \coqdocvariable{$\gamma$}\coqdocnotation{)} \coqdocnotation{$_{\upharpoonright s}$} \coqdocnotation{$\longrightarrow$} \coqdocdefinition{$\mathsf{Type}_{0}^1$}\coqdocnotation{;} \coqref{groupoid interpretation def light.TypFam 1}{\coqdocinstance{$\mathsf{TypFam_{comp}}$}} \coqdocvar{\_}\coqdocnotation{)}\coqdocnotation{]}.\coqdoceol
\coqdocemptyline
\end{coqdoccode}


  Terms of \coqref{groupoid interpretation def light.TypDep}{\coqdocdefinition{TypDep}} \coqdocvariable{A} and \coqref{groupoid interpretation def light.TypFam}{\coqdocdefinition{TypFam}} \coqdocvariable{A} can be related using a dependent closure
  at the level of types. In the interpretation of typing judgments, this connection 
  will be used to switch between the fibration and the morphism points of view.
\begin{coqdoccode}
\coqdocemptyline
\coqdocnoindent
\coqdockw{Definition} \coqdef{groupoid interpretation def light.LamT}{$\Lambda$}{\coqdocdefinition{$\Lambda$}} \{\coqdocvar{Γ}: \coqref{groupoid interpretation def light.Context}{\coqdocdefinition{Context}}\} \{\coqdocvar{A} : \coqref{groupoid interpretation def light.Typ}{\coqdocdefinition{Typ}} \coqdocvariable{Γ}\} (\coqdocvar{B}: \coqref{groupoid interpretation def light.TypDep}{\coqdocdefinition{TypDep}} \coqdocvariable{A})\coqdoceol
\coqdocindent{1.00em}
: \coqref{groupoid interpretation def light.TypFam}{\coqdocdefinition{TypFam}} \coqdocvariable{A} := \coqdocnotation{(}\coqexternalref{::'xCExBB' x '..' x ',' x}{http://coq.inria.fr/stdlib/Coq.Unicode.Utf8\_core}{\coqdocnotation{\ensuremath{\lambda}}} \coqdocvar{$\gamma$}\coqexternalref{::'xCExBB' x '..' x ',' x}{http://coq.inria.fr/stdlib/Coq.Unicode.Utf8\_core}{\coqdocnotation{,}} \coqdocnotation{(}\coqexternalref{::'xCExBB' x '..' x ',' x}{http://coq.inria.fr/stdlib/Coq.Unicode.Utf8\_core}{\coqdocnotation{\ensuremath{\lambda}}} \coqdocvar{t}\coqexternalref{::'xCExBB' x '..' x ',' x}{http://coq.inria.fr/stdlib/Coq.Unicode.Utf8\_core}{\coqdocnotation{,}} \coqdocvariable{B} \coqdocnotation{$\star$} \coqdocnotation{(}\coqdocvariable{$\gamma$}\coqdocnotation{;} \coqdocvariable{t}\coqdocnotation{)} \coqdocnotation{;} \coqdocvar{\_}\coqdocnotation{);} \coqref{groupoid interpretation def light.LamT 1}{\coqdocinstance{$\mathsf{\Lambda_{comp}}$}} \coqdocvariable{B}\coqdocnotation{)}.\coqdoceol
\coqdocemptyline
\end{coqdoccode}


\subsection{Substitutions}




  A substitution is represented by a context morphism [\coqdocvariable{Γ} $\longrightarrow$ \coqdocvariable{Δ}].  Note
  that although a substitution σ can be composed with a dependent type
  \coqdocvariable{A} by using composition of functors, we need to define a fresh notion
  of composition, noted \coqdocvariable{A} ⋅ σ, with the same computational content but
  with different universe constraints, to avoid universe inconsistency:
  composition otherwise forces all endpoints to be at the same level, which
  is too restrictive for \coqdockw{Type}-valued functors and context morphisms.


  A substitution σ can be extended by a term \coqdocvariable{a}: \coqref{groupoid interpretation def light.Elt}{\coqdocdefinition{$\mathsf{Tm}$}} (\coqdocvariable{A} ⋅ σ) 
  of \coqdocvariable{A} : \coqref{groupoid interpretation def light.Typ}{\coqdocdefinition{Typ}} \coqdocvariable{Δ}.


\begin{coqdoccode}
\coqdocemptyline
\coqdocnoindent
\coqdockw{Definition} \coqdef{groupoid interpretation def light.SubExt}{SubExt}{\coqdocdefinition{SubExt}} \{\coqdocvar{Γ} \coqdocvar{Δ} : \coqref{groupoid interpretation def light.Context}{\coqdocdefinition{Context}}\} \{\coqdocvar{A} : \coqref{groupoid interpretation def light.Typ}{\coqdocdefinition{Typ}} \coqdocvariable{Δ}\} (σ: \coqdocnotation{[}\coqdocvariable{Γ} \coqdocnotation{$\longrightarrow$} \coqdocvariable{Δ}\coqdocnotation{]}) (\coqdocvar{a}: \coqref{groupoid interpretation def light.Elt}{\coqdocdefinition{$\mathsf{Tm}$}} (\coqdocvariable{A} \coqref{groupoid interpretation def light.::x 'xE2x8Bx85xE2x8Bx85' x}{\coqdocnotation{$⋅$}} \coqdocvariable{σ})) \coqdoceol
\coqdocindent{1.00em}
: \coqdocnotation{[}\coqdocvariable{Γ} \coqdocnotation{$\longrightarrow$} \coqdocdefinition{$\Sigma$} \coqdocvariable{A} \coqdocnotation{]} := \coqdocnotation{(}\coqexternalref{::'xCExBB' x '..' x ',' x}{http://coq.inria.fr/stdlib/Coq.Unicode.Utf8\_core}{\coqdocnotation{\ensuremath{\lambda}}} \coqdocvar{$\gamma$}\coqexternalref{::'xCExBB' x '..' x ',' x}{http://coq.inria.fr/stdlib/Coq.Unicode.Utf8\_core}{\coqdocnotation{,}} \coqdocnotation{(}\coqdocvariable{σ} \coqdocnotation{$\star$} \coqdocvariable{$\gamma$}\coqdocnotation{;} \coqdocvariable{a} \coqdocnotation{$\star$} \coqdocvariable{$\gamma$}\coqdocnotation{)} \coqdocnotation{;} \coqref{groupoid interpretation def light.SubExt 1}{\coqdocinstance{$\mathsf{SubExt_{comp}}$}} \coqdocvar{\_} \coqdocvar{\_}\coqdocnotation{)}.\coqdoceol
\coqdocemptyline
\end{coqdoccode}
\noindent where \coqref{groupoid interpretation def light.SubExt 1}{\coqdocinstance{$\mathsf{SubExt_{comp}}$}} is a proof that it is functorial. 
\begin{coqdoccode}
\coqdocemptyline
\coqdocnoindent
\coqdockw{Definition} \coqdef{groupoid interpretation def light.substF}{substF}{\coqdocdefinition{substF}} \{\coqdocvar{T} \coqdocvar{Γ}\} \{\coqdocvar{A}:\coqref{groupoid interpretation def light.Typ}{\coqdocdefinition{Typ}} \coqdocvariable{Γ}\} (\coqdocvar{F}:\coqref{groupoid interpretation def light.TypFam}{\coqdocdefinition{TypFam}} \coqdocvariable{A}) (σ:\coqdocnotation{[}\coqdocvariable{T} \coqdocnotation{$\longrightarrow$} \coqdocvariable{Γ}\coqdocnotation{]}) : \coqref{groupoid interpretation def light.TypFam}{\coqdocdefinition{TypFam}} (\coqdocvariable{A} \coqref{groupoid interpretation def light.::x 'xE2x8Bx85xE2x8Bx85' x}{\coqdocnotation{$⋅$}} \coqdocvariable{σ}) \coqdoceol
\coqdocindent{1.00em}
:= \coqdocnotation{(}\coqdocnotation{[}\coqdocvariable{F} \coqref{groupoid interpretation def light.::x 'xC2xB0xC2xB0' x}{\coqdocnotation{$\circ$}} \coqdocvariable{σ}\coqdocnotation{]} : \coqexternalref{:type scope:'xE2x88x80' x '..' x ',' x}{http://coq.inria.fr/stdlib/Coq.Unicode.Utf8\_core}{\coqdocnotation{∀}} \coqdocvar{t} : \coqdocnotation{[}\coqdocvariable{T}\coqdocnotation{]}\coqexternalref{:type scope:'xE2x88x80' x '..' x ',' x}{http://coq.inria.fr/stdlib/Coq.Unicode.Utf8\_core}{\coqdocnotation{,}} \coqdocnotation{ }\coqdocvariable{A} \coqref{groupoid interpretation def light.::x 'xE2x8Bx85xE2x8Bx85' x}{\coqdocnotation{$⋅$}} \coqdocvariable{σ}\coqdocnotation{$_{\upharpoonright s}$} \coqdocnotation{$\star$} \coqdocvariable{t} \coqref{groupoid interpretation def light.::x '--->' x}{\coqdocnotation{$\longrightarrow$}} \coqdocdefinition{$\mathsf{Type}_{0}^1$}\coqdocnotation{;} \coqref{groupoid interpretation def light.substF 1}{\coqdocinstance{$\mathsf{substF_{comp}}$}} \coqdocvariable{F} \coqdocvariable{σ}\coqdocnotation{)}.\coqdoceol
\coqdocemptyline
\end{coqdoccode}
A substitution σ can be applied to a type family \coqdocvariable{F} using the
  composition of a functor with a dependent functor. We
  abusively note all those different compositions with $\circ$ as it is done in
  mathematics, whereas they are distinct operators in the \Coq
  development.
  The weakening substitution of $\Gamma, x:A \vdash$ is given by the first
  projection. \begin{coqdoccode}
\coqdocemptyline
\end{coqdoccode}


  A type family \coqdocvariable{F} in \coqref{groupoid interpretation def light.TypFam}{\coqdocdefinition{TypFam}} \coqdocvariable{A} can be partially substituted with an
  term \coqdocvariable{a} in \coqref{groupoid interpretation def light.Elt}{\coqdocdefinition{$\mathsf{Tm}$}} \coqdocvariable{A}, noted \coqdocvariable{F} \{\{\coqdocvariable{a}\}\}, to get its value (a type) at
  \coqdocvariable{a}. This process is defined as \coqdocvariable{F} \{\{\coqdocvariable{a}\}\} := (\coqdocvar{\ensuremath{\lambda}} \coqdocvariable{$\gamma$}, (\coqdocvariable{F} $\star$ \coqdocvariable{$\gamma$}) $\star$ (\coqdocvariable{a} $\star$ \coqdocvariable{$\gamma$}) ;
  \coqdocvar{\_}) (where \coqdocvar{\_} is a proof it is functorial). Note that this
  pattern of application \emph{up-to a context $\gamma$} will be used
  later to defined other notions of application. Although the
  computational definitions are the same, the compatibility conditions
  are always different.  This notion of partial substitution in a type
  family enables to state that \coqref{groupoid interpretation def light.LamT}{\coqdocdefinition{$\Lambda$}} defines a type level
  $\lambda$-abstraction.  \begin{coqdoccode}
\coqdocemptyline
\coqdocnoindent
\coqdockw{Definition} \coqdef{groupoid interpretation def light.BetaT}{BetaT}{\coqdocdefinition{BetaT}} \coqdocvar{Δ} \coqdocvar{Γ} (\coqdocvar{A}:\coqref{groupoid interpretation def light.Typ}{\coqdocdefinition{Typ}} \coqdocvariable{Γ}) (\coqdocvar{B}:\coqref{groupoid interpretation def light.TypDep}{\coqdocdefinition{TypDep}} \coqdocvariable{A}) (σ:\coqdocnotation{[}\coqdocvariable{Δ} \coqdocnotation{$\longrightarrow$} \coqdocvariable{Γ}\coqdocnotation{]}) (\coqdocvar{a}:\coqref{groupoid interpretation def light.Elt}{\coqdocdefinition{$\mathsf{Tm}$}} (\coqdocvariable{A} \coqref{groupoid interpretation def light.::x 'xE2x8Bx85xE2x8Bx85' x}{\coqdocnotation{$⋅$}} \coqdocvariable{σ})) \coqdoceol
\coqdocindent{1.00em}
: \coqref{groupoid interpretation def light.LamT}{\coqdocdefinition{$\Lambda$}} \coqdocvariable{B} \coqref{groupoid interpretation def light.::x 'xC2xB0xC2xB0xC2xB0' x}{\coqdocnotation{$\circ$}} \coqdocvariable{σ} \coqref{groupoid interpretation def light.::x 'x7Bx7B' x 'x7Dx7D'}{\coqdocnotation{\{\{}}\coqdocvariable{a}\coqref{groupoid interpretation def light.::x 'x7Bx7B' x 'x7Dx7D'}{\coqdocnotation{\}\}}} \coqdocnotation{$\sim_1$} \coqdocvariable{B} \coqref{groupoid interpretation def light.::x 'xE2x8Bx85xE2x8Bx85' x}{\coqdocnotation{$⋅$}} \coqref{groupoid interpretation def light.::x 'xE2x8Bx85xE2x8Bx85' x}{\coqdocnotation{(}}\coqref{groupoid interpretation def light.SubExt}{\coqdocdefinition{SubExt}} \coqdocvariable{σ} \coqdocvariable{a}\coqref{groupoid interpretation def light.::x 'xE2x8Bx85xE2x8Bx85' x}{\coqdocnotation{)}} := \coqdocnotation{(}\coqexternalref{::'xCExBB' x '..' x ',' x}{http://coq.inria.fr/stdlib/Coq.Unicode.Utf8\_core}{\coqdocnotation{\ensuremath{\lambda}}} \coqdocvar{\_}\coqexternalref{::'xCExBB' x '..' x ',' x}{http://coq.inria.fr/stdlib/Coq.Unicode.Utf8\_core}{\coqdocnotation{,}} \coqdocmethod{identity} \coqdocvar{\_} \coqdocnotation{;} \coqref{groupoid interpretation def light.BetaT 1}{\coqdocaxiom{$\mathsf{BetaT_{comp}}$}} \coqdocvar{\_} \coqdocvar{\_} \coqdocvar{\_}\coqdocnotation{)}.\coqdoceol
\coqdocemptyline
\end{coqdoccode}


\subsection{Interpretation of the typing judgment}


  \label{sec:interp}


  The typing rules of Figure \ref{fig:emltt} are
  interpreted in the groupoid model as described below.


  \paragraph{\textsc{Var}.} 


  The rule \textsc{Var} is given by the second projection plus a proof
  that the projection is dependently functorial. Note the explicit
  weakening of \coqdocvariable{A} in the returned type. This is because we need to
  make explicit that the context used to type \coqdocvariable{A} is extended with an
  term of type \coqdocvariable{A}. The rule of Figure \ref{fig:emltt} is more general 
  as it performs an implicit weakening. We do not interpret this part of 
  the rule as weakening is explicit in our model. 


\begin{coqdoccode}
\coqdocemptyline
\coqdocnoindent
\coqdockw{Definition} \coqdef{groupoid interpretation def light.Var}{Var}{\coqdocdefinition{Var}} \{\coqdocvar{Γ}\} (\coqdocvar{A}:\coqref{groupoid interpretation def light.Typ}{\coqdocdefinition{Typ}} \coqdocvariable{Γ}) : \coqref{groupoid interpretation def light.Elt}{\coqdocdefinition{$\mathsf{Tm}$}} \coqref{groupoid interpretation def light.::'xE2x87x91' x}{\coqdocnotation{$\shortuparrow$}}\coqdocvariable{A} := \coqdocnotation{(}\coqexternalref{::'xCExBB' x '..' x ',' x}{http://coq.inria.fr/stdlib/Coq.Unicode.Utf8\_core}{\coqdocnotation{\ensuremath{\lambda}}} \coqdocvar{t}\coqexternalref{::'xCExBB' x '..' x ',' x}{http://coq.inria.fr/stdlib/Coq.Unicode.Utf8\_core}{\coqdocnotation{,}} \coqdocabbreviation{$\pi_2$} \coqdocvariable{t}\coqdocnotation{;} \coqref{groupoid interpretation def light.Var 1}{\coqdocinstance{$\mathsf{Var_{comp}}$}} \coqdocvariable{A}\coqdocnotation{)}.\coqdoceol
\coqdocemptyline
\coqdocemptyline
\end{coqdoccode}
\paragraph{\textsc{Prod}.} The rule \textsc{Prod} is interpreted
  using the dependent functor space, plus a proof that equivalent
  contexts give rise to isomorphic dependent functor spaces.  Note that
  the rule is defined on type families and not on the dependent type
  formulation because here we need a fibration point of view. \begin{coqdoccode}
\coqdocemptyline
\coqdocnoindent
\coqdockw{Definition} \coqdef{groupoid interpretation def light.Prod}{Prod}{\coqdocdefinition{Prod}} \{\coqdocvar{Γ}\} (\coqdocvar{A}:\coqref{groupoid interpretation def light.Typ}{\coqdocdefinition{Typ}} \coqdocvariable{Γ}) (\coqdocvar{F}:\coqref{groupoid interpretation def light.TypFam}{\coqdocdefinition{TypFam}} \coqdocvariable{A}) \coqdoceol
\coqdocindent{1.00em}
: \coqref{groupoid interpretation def light.Typ}{\coqdocdefinition{Typ}} \coqdocvariable{Γ} := \coqdocnotation{(}\coqexternalref{::'xCExBB' x '..' x ',' x}{http://coq.inria.fr/stdlib/Coq.Unicode.Utf8\_core}{\coqdocnotation{\ensuremath{\lambda}}} \coqdocvar{s}\coqexternalref{::'xCExBB' x '..' x ',' x}{http://coq.inria.fr/stdlib/Coq.Unicode.Utf8\_core}{\coqdocnotation{,}} \coqdocdefinition{$\Pi_0$} (\coqdocvariable{F} \coqdocnotation{$\star$} \coqdocvariable{s})\coqdocnotation{;} \coqref{groupoid interpretation def light.Prod 1}{\coqdocaxiom{$\mathsf{Prod_{comp}}$}} \coqdocvariable{A} \coqdocvariable{F}\coqdocnotation{)}.\coqdoceol
\coqdocemptyline
\end{coqdoccode}
  \paragraph{\textsc{App}.}


  The rule \textsc{App} is interpreted using an up-to context application 
  and a proof of dependent functoriality. We abusively note \coqdocvar{M} $\star$ \coqdocvar{N} the application 
  of \coqref{groupoid interpretation def light.App}{\coqdocdefinition{App}}.
\begin{coqdoccode}
\coqdocemptyline
\coqdocnoindent
\coqdockw{Definition} \coqdef{groupoid interpretation def light.App}{App}{\coqdocdefinition{App}} \{\coqdocvar{Γ}\} \{\coqdocvar{A}:\coqref{groupoid interpretation def light.Typ}{\coqdocdefinition{Typ}} \coqdocvariable{Γ}\} \{\coqdocvar{F}:\coqref{groupoid interpretation def light.TypFam}{\coqdocdefinition{TypFam}} \coqdocvariable{A}\} (\coqdocvar{c}:\coqref{groupoid interpretation def light.Elt}{\coqdocdefinition{$\mathsf{Tm}$}} (\coqref{groupoid interpretation def light.Prod}{\coqdocdefinition{Prod}} \coqdocvariable{F})) (\coqdocvar{a}:\coqref{groupoid interpretation def light.Elt}{\coqdocdefinition{$\mathsf{Tm}$}} \coqdocvariable{A}) \coqdoceol
\coqdocindent{1.00em}
: \coqref{groupoid interpretation def light.Elt}{\coqdocdefinition{$\mathsf{Tm}$}} (\coqdocvariable{F} \coqref{groupoid interpretation def light.::x 'x7Bx7B' x 'x7Dx7D'}{\coqdocnotation{\{\{}}\coqdocvariable{a}\coqref{groupoid interpretation def light.::x 'x7Bx7B' x 'x7Dx7D'}{\coqdocnotation{\}\}}}) := \coqdocnotation{(}\coqexternalref{::'xCExBB' x '..' x ',' x}{http://coq.inria.fr/stdlib/Coq.Unicode.Utf8\_core}{\coqdocnotation{\ensuremath{\lambda}}} \coqdocvar{s}\coqexternalref{::'xCExBB' x '..' x ',' x}{http://coq.inria.fr/stdlib/Coq.Unicode.Utf8\_core}{\coqdocnotation{,}} \coqdocnotation{(}\coqdocvariable{c} \coqdocnotation{$\star$} \coqdocvariable{s}\coqdocnotation{)} \coqdocnotation{$\star$} \coqdocnotation{(}\coqdocvariable{a} \coqdocnotation{$\star$} \coqdocvariable{s}\coqdocnotation{)}\coqdocnotation{;} \coqref{groupoid interpretation def light.App 1}{\coqdocinstance{$\mathsf{App_{comp}}$}} \coqdocvariable{c} \coqdocvariable{a}\coqdocnotation{)}.\coqdoceol
\coqdocemptyline
\end{coqdoccode}
  \paragraph{\lrule{Lam}.}


  Term-level $\lambda$-abstraction is defined with the same
  computational meaning as type-level $\lambda$-abstraction, but it
  differs on the proof of dependent functoriality. Note that we use
  \coqref{groupoid interpretation def light.LamT}{\coqdocdefinition{$\Lambda$}} in the definition because we need both the fibration (for
  \coqref{groupoid interpretation def light.Prod}{\coqdocdefinition{Prod}}) and the morphism (for \coqref{groupoid interpretation def light.Elt}{\coqdocdefinition{$\mathsf{Tm}$}} \coqdocvariable{B}) point of view. 
\begin{coqdoccode}
\coqdocemptyline
\coqdocnoindent
\coqdockw{Definition} \coqdef{groupoid interpretation def light.Lam}{Lam}{\coqdocdefinition{Lam}} \{\coqdocvar{Γ}\} \{\coqdocvar{A}:\coqref{groupoid interpretation def light.Typ}{\coqdocdefinition{Typ}} \coqdocvariable{Γ}\} \{\coqdocvar{B}:\coqref{groupoid interpretation def light.TypDep}{\coqdocdefinition{TypDep}} \coqdocvariable{A}\} (\coqdocvar{b}:\coqref{groupoid interpretation def light.Elt}{\coqdocdefinition{$\mathsf{Tm}$}} \coqdocvariable{B})\coqdoceol
\coqdocindent{1.00em}
: \coqref{groupoid interpretation def light.Elt}{\coqdocdefinition{$\mathsf{Tm}$}} (\coqref{groupoid interpretation def light.Prod}{\coqdocdefinition{Prod}} (\coqref{groupoid interpretation def light.LamT}{\coqdocdefinition{$\Lambda$}} \coqdocvariable{B})) := \coqdocnotation{(}\coqexternalref{::'xCExBB' x '..' x ',' x}{http://coq.inria.fr/stdlib/Coq.Unicode.Utf8\_core}{\coqdocnotation{\ensuremath{\lambda}}} \coqdocvar{$\gamma$}\coqexternalref{::'xCExBB' x '..' x ',' x}{http://coq.inria.fr/stdlib/Coq.Unicode.Utf8\_core}{\coqdocnotation{,}} \coqdocnotation{(}\coqexternalref{::'xCExBB' x '..' x ',' x}{http://coq.inria.fr/stdlib/Coq.Unicode.Utf8\_core}{\coqdocnotation{\ensuremath{\lambda}}} \coqdocvar{t}\coqexternalref{::'xCExBB' x '..' x ',' x}{http://coq.inria.fr/stdlib/Coq.Unicode.Utf8\_core}{\coqdocnotation{,}} \coqdocvariable{b} \coqdocnotation{$\star$} \coqdocnotation{(}\coqdocvariable{$\gamma$} \coqdocnotation{;} \coqdocvariable{t}\coqdocnotation{)} \coqdocnotation{;} \coqdocvar{\_}\coqdocnotation{);} \coqref{groupoid interpretation def light.Lam 2}{\coqdocaxiom{$\mathsf{Lam_{comp}}$}} \coqdocvariable{b}\coqdocnotation{)}.\coqdoceol
\coqdocemptyline
\coqdocemptyline
\end{coqdoccode}
  \paragraph{\textsc{Sigma}, \textsc{Pair} and \textsc{Projs}.}
  The rules for Σ types are interpreted using the 
  dependent sum \coqdocdefinition{$\Sigma$} on setoids.  
\begin{coqdoccode}
\coqdocemptyline
\coqdocnoindent
\coqdockw{Definition} \coqdef{groupoid interpretation def light.Sigma}{Sigma}{\coqdocdefinition{Sigma}} \{\coqdocvar{Γ}\} (\coqdocvar{A}:\coqref{groupoid interpretation def light.Typ}{\coqdocdefinition{Typ}} \coqdocvariable{Γ}) (\coqdocvar{F}:\coqref{groupoid interpretation def light.TypFam}{\coqdocdefinition{TypFam}} \coqdocvariable{A}) \coqdoceol
\coqdocindent{1.00em}
: \coqref{groupoid interpretation def light.Typ}{\coqdocdefinition{Typ}} \coqdocvariable{Γ} := \coqdocnotation{(}\coqexternalref{::'xCExBB' x '..' x ',' x}{http://coq.inria.fr/stdlib/Coq.Unicode.Utf8\_core}{\coqdocnotation{\ensuremath{\lambda}}} \coqdocvar{$\gamma$}: \coqdocnotation{[}\coqdocvariable{Γ}\coqdocnotation{]}\coqexternalref{::'xCExBB' x '..' x ',' x}{http://coq.inria.fr/stdlib/Coq.Unicode.Utf8\_core}{\coqdocnotation{,}} \coqdocdefinition{$\Sigma$} (\coqdocvariable{F} \coqdocnotation{$\star$} \coqdocvariable{$\gamma$})\coqdocnotation{;} \coqref{groupoid interpretation def light.Sigma 1}{\coqdocaxiom{$\mathsf{Sigma_{comp}}$}} \coqdocvariable{A} \coqdocvariable{F}\coqdocnotation{)}.\coqdoceol
\coqdocemptyline
\end{coqdoccode}
\noindent Pairing and projections are obtained
  by a context lift of pairing and projection of the underlying dependent sum.
\begin{coqdoccode}
\coqdocemptyline
\coqdocemptyline
\end{coqdoccode}
  \paragraph{\lrule{Conv}.}
  It is not possible to prove in \Coq that the conversion rule is
  preserved because the application of this rule is implicit and
  can not be reified. Nevertheless, to witness this preservation, 
  we show that beta conversion is valid as a definitional equality
  on the first projection. As conversion is only done on 
  types and interpretation of types is always projected, this is 
  enough to guarantee that the conversion rule is also admissible.


\begin{coqdoccode}
\coqdocemptyline
\coqdocnoindent
\coqdockw{Definition} \coqdef{groupoid interpretation def light.Beta}{Beta}{\coqdocdefinition{Beta}} \{\coqdocvar{Γ}\} \{\coqdocvar{A}:\coqref{groupoid interpretation def light.Typ}{\coqdocdefinition{Typ}} \coqdocvariable{Γ}\} \{\coqdocvar{F}:\coqref{groupoid interpretation def light.TypDep}{\coqdocdefinition{TypDep}} \coqdocvariable{A}\} (\coqdocvar{b}:\coqref{groupoid interpretation def light.Elt}{\coqdocdefinition{$\mathsf{Tm}$}} \coqdocvariable{F}) (\coqdocvar{a}:\coqref{groupoid interpretation def light.Elt}{\coqdocdefinition{$\mathsf{Tm}$}} \coqdocvariable{A}) \coqdoceol
\coqdocindent{1.00em}
: \coqdocnotation{[}\coqref{groupoid interpretation def light.Lam}{\coqdocdefinition{Lam}} \coqdocvariable{b} \coqref{groupoid interpretation def light.::x '@@' x}{\coqdocnotation{$\star$}} \coqdocvariable{a}\coqdocnotation{]} \coqdocnotation{=} \coqdocnotation{[}\coqdocvariable{b} \coqref{groupoid interpretation def light.::x 'xC2xB0xC2xB0' x}{\coqdocnotation{$\circ$}} \coqref{groupoid interpretation def light.SubExtId}{\coqdocdefinition{SubExtId}} \coqdocvariable{a}\coqdocnotation{]} := \coqdocconstructor{eq\_refl} \coqdocvar{\_}.\coqdoceol
\coqdocemptyline
\end{coqdoccode}
 \noindent where \coqref{groupoid interpretation def light.SubExtId}{\coqdocdefinition{SubExtId}} is a specialization of \coqref{groupoid interpretation def light.SubExt}{\coqdocdefinition{SubExt}} with 
  the identity substitution.
\begin{coqdoccode}
\coqdocemptyline
\end{coqdoccode}


\subsection{Identity Types}


  \label{sec:extprinc}
  One of the main interest of the groupoid interpretation is that it
  allows to interpret a type directed notion of equality which validates 
  the J eliminator of identity types but also various extensional principles.
  For any terms \coqdocvariable{a} and \coqdocvariable{b} of a dependent type \coqdocvariable{A}:\coqref{groupoid interpretation def light.Typ}{\coqdocdefinition{Typ}} \coqdocvariable{Γ}, we note \coqref{groupoid interpretation def light.Id}{\coqdocdefinition{Id}} \coqdocvariable{a} \coqdocvariable{b} the equality type
  between \coqdocvariable{a} and \coqdocvariable{b} obtained by lifting $\sim_1$ to get a type depending on \coqdocvariable{Γ}.
\begin{coqdoccode}
\coqdocemptyline
\coqdocnoindent
\coqdockw{Definition} \coqdef{groupoid interpretation def light.Id}{Id}{\coqdocdefinition{Id}} \{\coqdocvar{Γ}\} (\coqdocvar{A}: \coqref{groupoid interpretation def light.Typ}{\coqdocdefinition{Typ}} \coqdocvariable{Γ}) (\coqdocvar{a} \coqdocvar{b} : \coqref{groupoid interpretation def light.Elt}{\coqdocdefinition{$\mathsf{Tm}$}} \coqdocvariable{A}) \coqdoceol
\coqdocindent{1.00em}
: \coqref{groupoid interpretation def light.Typ}{\coqdocdefinition{Typ}} \coqdocvariable{Γ} := \coqdocnotation{(}\coqexternalref{::'xCExBB' x '..' x ',' x}{http://coq.inria.fr/stdlib/Coq.Unicode.Utf8\_core}{\coqdocnotation{\ensuremath{\lambda}}} \coqdocvar{$\gamma$}\coqexternalref{::'xCExBB' x '..' x ',' x}{http://coq.inria.fr/stdlib/Coq.Unicode.Utf8\_core}{\coqdocnotation{,}} \coqdocnotation{(}\coqdocvariable{a} \coqdocnotation{$\star$} \coqdocvariable{$\gamma$} \coqdocnotation{$\sim_1$} \coqdocvariable{b} \coqdocnotation{$\star$} \coqdocvariable{$\gamma$} \coqdocnotation{;} \coqdocvar{\_}\coqdocnotation{);} \coqref{groupoid interpretation def light.Id 1}{\coqdocinstance{$\mathsf{Id_{comp}}$}} \coqdocvariable{A} \coqdocvariable{a} \coqdocvariable{b}\coqdocnotation{)}.\coqdoceol
\coqdocemptyline
\coqdocemptyline
\end{coqdoccode}
The introduction rule of identity types which corresponds to reflexivity is interpreted by the (lifting of) identity of the underlying setoid. \begin{coqdoccode}
\coqdocemptyline
\coqdocnoindent
\coqdockw{Definition} \coqdef{groupoid interpretation def light.Refl}{Refl}{\coqdocdefinition{Refl}} \coqdocvar{Γ} (\coqdocvar{A}: \coqref{groupoid interpretation def light.Typ}{\coqdocdefinition{Typ}} \coqdocvariable{Γ}) (\coqdocvar{a} : \coqref{groupoid interpretation def light.Elt}{\coqdocdefinition{$\mathsf{Tm}$}} \coqdocvariable{A}) \coqdoceol
\coqdocindent{1.00em}
: \coqref{groupoid interpretation def light.Elt}{\coqdocdefinition{$\mathsf{Tm}$}} (\coqref{groupoid interpretation def light.Id}{\coqdocdefinition{Id}} \coqdocvariable{a} \coqdocvariable{a}) := \coqdocnotation{(}\coqexternalref{::'xCExBB' x '..' x ',' x}{http://coq.inria.fr/stdlib/Coq.Unicode.Utf8\_core}{\coqdocnotation{\ensuremath{\lambda}}} \coqdocvar{$\gamma$}\coqexternalref{::'xCExBB' x '..' x ',' x}{http://coq.inria.fr/stdlib/Coq.Unicode.Utf8\_core}{\coqdocnotation{,}} \coqdocmethod{identity} (\coqdocvariable{a} \coqdocnotation{$\star$} \coqdocvariable{$\gamma$})\coqdocnotation{;} \coqref{groupoid interpretation def light.Refl 1}{\coqdocaxiom{$\mathsf{Refl_{comp}}$}} \coqdocvar{\_}\coqdocnotation{)}.\coqdoceol
\coqdocemptyline
\coqdocemptyline
\end{coqdoccode}
We can interpret the J eliminator of MLTT on \coqref{groupoid interpretation def light.Id}{\coqdocdefinition{Id}} using functoriality of \coqdocvariable{P} and of product (\coqref{groupoid interpretation def light.prod comp}{\coqdocdefinition{$\mathsf{\Pi_{comp}}$}}). In the definition of J, the predicate \coqdocvariable{P} depends on the proof of equality, which is interpreted using a \coqref{groupoid interpretation def light.Sigma}{\coqdocdefinition{Sigma}} type. The functoriality of \coqdocvariable{P} is used on the term \coqref{groupoid interpretation def light.J Pair}{\coqdocdefinition{J\_Pair}} \coqdocvariable{e} \coqdocvariable{P} \coqdocvariable{$\gamma$}, which is a proof that (\coqdocvariable{a};\coqref{groupoid interpretation def light.Refl}{\coqdocdefinition{Refl}} \coqdocvariable{a}) is equal to (\coqdocvariable{b};\coqdocvariable{e}). The notation $\shortuparrow$ is used to convert the type of terms according to equality on \coqref{groupoid interpretation def light.LamT}{\coqdocdefinition{$\Lambda$}}. \begin{coqdoccode}
\coqdocemptyline
\coqdocnoindent
\coqdockw{Definition} \coqdef{groupoid interpretation def light.J}{J}{\coqdocdefinition{J}} \coqdocvar{Γ} (\coqdocvar{A}:\coqref{groupoid interpretation def light.Typ}{\coqdocdefinition{Typ}} \coqdocvariable{Γ}) (\coqdocvar{a} \coqdocvar{b}:\coqref{groupoid interpretation def light.Elt}{\coqdocdefinition{$\mathsf{Tm}$}} \coqdocvariable{A}) (\coqdocvar{P}:\coqref{groupoid interpretation def light.TypFam}{\coqdocdefinition{TypFam}} (\coqref{groupoid interpretation def light.Sigma}{\coqdocdefinition{Sigma}} (\coqref{groupoid interpretation def light.LamT}{\coqdocdefinition{$\Lambda$}} (\coqref{groupoid interpretation def light.Id}{\coqdocdefinition{Id}} (\coqdocvariable{a} \coqref{groupoid interpretation def light.::x 'xC2xB0xC2xB0xC2xB0xC2xB0' x}{\coqdocnotation{$\circ$}} \coqref{groupoid interpretation def light.Sub}{\coqdocdefinition{Sub}}) (\coqref{groupoid interpretation def light.Var}{\coqdocdefinition{Var}} \coqdocvariable{A})))))\coqdoceol
\coqdocindent{7.50em}
(\coqdocvar{e}:\coqref{groupoid interpretation def light.Elt}{\coqdocdefinition{$\mathsf{Tm}$}} (\coqref{groupoid interpretation def light.Id}{\coqdocdefinition{Id}} \coqdocvariable{a} \coqdocvariable{b})) (\coqdocvar{p}:\coqref{groupoid interpretation def light.Elt}{\coqdocdefinition{$\mathsf{Tm}$}} (\coqdocvariable{P}\coqref{groupoid interpretation def light.::x 'x7Bx7B' x 'x7Dx7D'}{\coqdocnotation{\{\{}}\coqref{groupoid interpretation def light.Pair}{\coqdocdefinition{Pair}} \coqref{groupoid interpretation def light.::'xE2x87x91xE2x87x91' x}{\coqdocnotation{$\shortuparrow$}} \coqref{groupoid interpretation def light.::'xE2x87x91xE2x87x91' x}{\coqdocnotation{(}}\coqref{groupoid interpretation def light.Refl}{\coqdocdefinition{Refl}} \coqdocvariable{a}\coqref{groupoid interpretation def light.::'xE2x87x91xE2x87x91' x}{\coqdocnotation{)}}\coqref{groupoid interpretation def light.::x 'x7Bx7B' x 'x7Dx7D'}{\coqdocnotation{\}\}}})) \coqdoceol
\coqdocindent{1.00em}
: \coqref{groupoid interpretation def light.Elt}{\coqdocdefinition{$\mathsf{Tm}$}} (\coqdocvariable{P}\coqref{groupoid interpretation def light.::x 'x7Bx7B' x 'x7Dx7D'}{\coqdocnotation{\{\{}}\coqref{groupoid interpretation def light.Pair}{\coqdocdefinition{Pair}} \coqref{groupoid interpretation def light.::'xE2x87x91xE2x87x91' x}{\coqdocnotation{$\shortuparrow$}}\coqdocvariable{e}\coqref{groupoid interpretation def light.::x 'x7Bx7B' x 'x7Dx7D'}{\coqdocnotation{\}\}}}) := \coqref{groupoid interpretation def light.prod comp}{\coqdocdefinition{$\mathsf{\Pi_{comp}}$}} \coqdocnotation{(}\coqexternalref{::'xCExBB' x '..' x ',' x}{http://coq.inria.fr/stdlib/Coq.Unicode.Utf8\_core}{\coqdocnotation{\ensuremath{\lambda}}} \coqdocvar{$\gamma$}\coqexternalref{::'xCExBB' x '..' x ',' x}{http://coq.inria.fr/stdlib/Coq.Unicode.Utf8\_core}{\coqdocnotation{,}} \coqexternalref{::'xCExBB' x '..' x ',' x}{http://coq.inria.fr/stdlib/Coq.Unicode.Utf8\_core}{\coqdocnotation{(}}\coqdocnotation{map} \coqdocnotation{(}\coqdocvariable{P} \coqdocnotation{$\star$} \coqdocvariable{$\gamma$}\coqdocnotation{)} (\coqref{groupoid interpretation def light.J Pair}{\coqdocdefinition{J\_Pair}} \coqdocvariable{e} \coqdocvariable{P} \coqdocvariable{$\gamma$})\coqexternalref{::'xCExBB' x '..' x ',' x}{http://coq.inria.fr/stdlib/Coq.Unicode.Utf8\_core}{\coqdocnotation{)}}\coqdocnotation{;} \coqref{groupoid interpretation def light.J 1}{\coqdocaxiom{$\mathsf{J_{comp}}$}} \coqdocvar{\_} \coqdocvar{\_}\coqdocnotation{)} \coqdocnotation{$\star$} \coqdocvariable{p}.\coqdoceol
\coqdocemptyline
\coqdocemptyline
\end{coqdoccode}


\subsection{Universe}


  \label{sec:universe} 
  To interpret the universe $\Univ$, we need to define its syntax and interpretation of syntax as setoids altogether. That is, $\Univ$ requires inductive-recursive definitions to be interpreted.
  As such definition are not available in \Coq, we cannot completely interpret $\Univ$\footnote{The folklore coding trick to encode inductive-recursive definition using an indexed family as done in~\cite{altenkirch-mcbride-wierstra:ott-now} does not work here because it transforms an inductive-recursive Agda Set into a larger universe which is no longer a setoid.}. Nevertheless, we present a way to interpret the identity type on $\Univ$ and Rule \textsc{Id-Equiv-Intro} which defines equality of types in $\Univ$ as isomorphism.


   We interpret the identity type on $\Univ$ in the same way as \coqref{groupoid interpretation def light.Id}{\coqdocdefinition{Id}}, except that it 
relates two dependent types \coqdocvariable{A} and \coqdocvariable{B} instead of terms of a dependent type.
\begin{coqdoccode}
\coqdocemptyline
\coqdocnoindent
\coqdockw{Definition} \coqdef{groupoid interpretation def light.UId}{$≡$}{\coqdocdefinition{$≡$}} \{\coqdocvar{Γ}\} (\coqdocvar{A} \coqdocvar{B}: \coqref{groupoid interpretation def light.Typ}{\coqdocdefinition{Typ}} \coqdocvariable{Γ}) : \coqref{groupoid interpretation def light.Typ}{\coqdocdefinition{Typ}} \coqdocvariable{Γ} := \coqdocnotation{(}\coqexternalref{::'xCExBB' x '..' x ',' x}{http://coq.inria.fr/stdlib/Coq.Unicode.Utf8\_core}{\coqdocnotation{\ensuremath{\lambda}}} \coqdocvar{$\gamma$}\coqexternalref{::'xCExBB' x '..' x ',' x}{http://coq.inria.fr/stdlib/Coq.Unicode.Utf8\_core}{\coqdocnotation{,}} \coqdocnotation{(}\coqdocvariable{A} \coqdocnotation{$\star$} \coqdocvariable{$\gamma$} \coqref{groupoid interpretation def light.::x 'x7E11' x}{\coqdocnotation{$\sim_1$}} \coqdocvariable{B} \coqdocnotation{$\star$} \coqdocvariable{$\gamma$} \coqdocnotation{;} \coqdocvar{\_}\coqdocnotation{);} \coqref{groupoid interpretation def light.UId 1}{\coqdocaxiom{$\mathsf{≡_{comp}}$}} \coqdocvariable{A} \coqdocvariable{B}\coqdocnotation{)}.\coqdoceol
\coqdocemptyline
\end{coqdoccode}
To define the notion of isomorphism, we need to define a proper
 notion of function (noted \coqdocvariable{A} $\longrightarrow_\Univ$ \coqdocvariable{B}) that does not use the
 restriction of \coqref{groupoid interpretation def light.Prod}{\coqdocdefinition{Prod}} to constant type families. This is because the
 definition of an isomorphism involves two functions that have to be
 composed in both ways, which lead to universe inconsistency if we use
 our asymmetric dependent products to encode these functions. We define
 the notion of application (noted \coqdocvar{g} $\star_\Univ$ \coqdocvariable{f}) for this kind of functions. \begin{coqdoccode}
\coqdocemptyline
\coqdocnoindent
\coqdockw{Class} \coqdef{groupoid interpretation def light.iso struct}{iso\_struct}{\coqdocrecord{iso\_struct}} (\coqdocvar{Γ}: \coqref{groupoid interpretation def light.Context}{\coqdocdefinition{Context}}) (\coqdocvar{A} \coqdocvar{B} : \coqref{groupoid interpretation def light.Typ}{\coqdocdefinition{Typ}} \coqdocvariable{Γ}) (\coqdocvar{f} : \coqref{groupoid interpretation def light.Elt}{\coqdocdefinition{$\mathsf{Tm}$}} (\coqdocvariable{A} \coqref{groupoid interpretation def light.::x '---->' x}{\coqdocnotation{$\longrightarrow_\Univ$}} \coqdocvariable{B})) := \coqdoceol
\coqdocnoindent
\{ \coqdef{groupoid interpretation def light.iso adjoint}{iso\_adjoint}{\coqdocprojection{iso\_adjoint}} : \coqref{groupoid interpretation def light.Elt}{\coqdocdefinition{$\mathsf{Tm}$}} (\coqdocvariable{B} \coqref{groupoid interpretation def light.::x '---->' x}{\coqdocnotation{$\longrightarrow_\Univ$}} \coqdocvariable{A})  ;\coqdoceol
\coqdocindent{1.00em}
\coqdef{groupoid interpretation def light.iso section}{iso\_section}{\coqdocprojection{iso\_section}} : \coqref{groupoid interpretation def light.Elt}{\coqdocdefinition{$\mathsf{Tm}$}} (\coqref{groupoid interpretation def light.Prod}{\coqdocdefinition{Prod}} (\coqref{groupoid interpretation def light.LamT}{\coqdocdefinition{$\Lambda$}} (\coqref{groupoid interpretation def light.Id}{\coqdocdefinition{Id}} (\coqref{groupoid interpretation def light.iso adjoint}{\coqdocmethod{iso\_adjoint}} \coqref{groupoid interpretation def light.::x '@@@' x}{\coqdocnotation{$\star_\Univ$}} \coqref{groupoid interpretation def light.::x '@@@' x}{\coqdocnotation{(}}\coqdocvariable{f} \coqref{groupoid interpretation def light.::x '@@@' x}{\coqdocnotation{$\star_\Univ$}} \coqref{groupoid interpretation def light.Var}{\coqdocdefinition{Var}} \coqdocvariable{A}\coqref{groupoid interpretation def light.::x '@@@' x}{\coqdocnotation{)}}) (\coqref{groupoid interpretation def light.Var}{\coqdocdefinition{Var}} \coqdocvariable{A})))) ;\coqdoceol
\coqdocindent{1.00em}
\coqdef{groupoid interpretation def light.iso retract}{iso\_retract}{\coqdocprojection{iso\_retract}} : \coqref{groupoid interpretation def light.Elt}{\coqdocdefinition{$\mathsf{Tm}$}} (\coqref{groupoid interpretation def light.Prod}{\coqdocdefinition{Prod}} (\coqref{groupoid interpretation def light.LamT}{\coqdocdefinition{$\Lambda$}} (\coqref{groupoid interpretation def light.Id}{\coqdocdefinition{Id}} (\coqdocvariable{f} \coqref{groupoid interpretation def light.::x '@@@' x}{\coqdocnotation{$\star_\Univ$}} \coqref{groupoid interpretation def light.::x '@@@' x}{\coqdocnotation{(}}\coqref{groupoid interpretation def light.iso adjoint}{\coqdocmethod{iso\_adjoint}} \coqref{groupoid interpretation def light.::x '@@@' x}{\coqdocnotation{$\star_\Univ$}} \coqref{groupoid interpretation def light.Var}{\coqdocdefinition{Var}} \coqdocvariable{B}\coqref{groupoid interpretation def light.::x '@@@' x}{\coqdocnotation{)}}) (\coqref{groupoid interpretation def light.Var}{\coqdocdefinition{Var}} \coqdocvariable{B}))))\}.\coqdoceol
\coqdocemptyline
\coqdocnoindent
\coqdockw{Definition} \coqdef{groupoid interpretation def light.iso}{iso}{\coqdocdefinition{iso}} (\coqdocvar{Γ}: \coqref{groupoid interpretation def light.Context}{\coqdocdefinition{Context}}) (\coqdocvar{A} \coqdocvar{B} : \coqref{groupoid interpretation def light.Typ}{\coqdocdefinition{Typ}} \coqdocvariable{Γ}) := \coqdocnotation{\{}\coqdocvar{f} \coqdocnotation{:} \coqref{groupoid interpretation def light.Elt}{\coqdocdefinition{$\mathsf{Tm}$}} (\coqdocvariable{A} \coqref{groupoid interpretation def light.::x '---->' x}{\coqdocnotation{$\longrightarrow_\Univ$}} \coqdocvariable{B}) \coqdocnotation{\&} \coqref{groupoid interpretation def light.iso struct}{\coqdocclass{iso\_struct}} \coqdocvar{f}\coqdocnotation{\}}.\coqdoceol
\coqdocemptyline
\end{coqdoccode}
Then, we can show that this definition of isomorphism corresponds to
  equivalence of setoids.  Again, the only extra work is with the
  management of context lifting. This provides a computational content
  to the univalence principle restricted to setoids.  \begin{coqdoccode}
\coqdocemptyline
\coqdocnoindent
\coqdockw{Definition} \coqdef{groupoid interpretation def light.Equiv Intro}{Equiv\_Intro}{\coqdocdefinition{Equiv\_Intro}} (\coqdocvar{Γ}: \coqref{groupoid interpretation def light.Context}{\coqdocdefinition{Context}}) (\coqdocvar{A} \coqdocvar{B} : \coqref{groupoid interpretation def light.Typ}{\coqdocdefinition{Typ}} \coqdocvariable{Γ}) (\coqdocvar{e} : \coqref{groupoid interpretation def light.iso}{\coqdocdefinition{iso}} \coqdocvariable{A} \coqdocvariable{B}) : \coqref{groupoid interpretation def light.Elt}{\coqdocdefinition{$\mathsf{Tm}$}} (\coqdocvariable{A} \coqref{groupoid interpretation def light.::x 'xE2x89xA1' x}{\coqdocnotation{≡}} \coqdocvariable{B}).\coqdoceol
 \coqdocemptyline
\coqdocemptyline
\end{coqdoccode}
