\coqlibrary{groupoid interpretation def}{Library }{groupoid\_interpretation\_def}

\begin{coqdoccode}
\end{coqdoccode}


  The interpretation is based on the Takeuti-Gandy interpretation of
  simple type theory, recently generalized to dependent type theory in
  \cite{barras:_gener_takeut_gandy_inter} using Kan semisimplicial
  sets. There are two main novelties in our interpretation. First, we
  take advantage of universe polymorphism to interpret dependent types
  directly as functors into \coqdocdefinition{$\mathsf{Type}_{0}^1$}. Second, we provide an
  interpretation in a model where structures that are definitionally
  equal for Kan semisimplicial sets are only homotopically equal, which
  requires more care in the definitions (see for instance the definition
  of \coqref{groupoid interpretation def.Lam}{\coqdocdefinition{Lam}} in Section \ref{sec:interp} which mixes two points of view
  on fibrations).


  We only present the computational part of the interpretation, the
  proofs of functoriality and naturality are not detailled but most of
  them are available in the \Coq development. We have admitted some of
  these administrative compatibility lemmas.


\subsection{Dependent types}




  The judgment context $\Gamma \vdash$ of Section
  \ref{sec:definitions} is represented in \Coq as a groupoid, noted
  \coqdockw{Context} := \coqdocdefinition{$\mathsf{Type_1}$}. The empty context (Rule \textsc{Empty})
  is interpreted as the groupoid with exactly one element at each
  dimension.  Types in a context \coqdocvariable{Γ}, noted \coqref{groupoid interpretation def.Typ}{\coqdocdefinition{Typ}} \coqdocvariable{Γ}, are (context)
  functors from \coqdocvariable{Γ} to the groupoid of setoids \coqdocdefinition{$\mathsf{Type}_{0}^1$}.  Thus, a
  judgment $\Gamma \vdash A : \Type{}$ is represented as a term \coqdocvariable{A} of
  type \coqref{groupoid interpretation def.Typ}{\coqdocdefinition{Typ}} \coqdocvariable{Γ}. Context extension (Rule \textsc{Decl}) is given by
  dependent sums, i.e., the judgment $\Gamma, x:A \vdash$ is represented
  as \coqdocdefinition{$\Sigma$} \coqdocvariable{A}.


\begin{coqdoccode}
\end{coqdoccode}
Elements of \coqdocvariable{A} introduced by a sequent $\Gamma \vdash x:A$ are
  dependent (context) functors from \coqdocvariable{Γ} to \coqdocvariable{A} that return for each
  context valuation \coqdocvariable{$\gamma$}, an object of \coqdocvariable{A} $\star$ \coqdocvariable{$\gamma$} respecting equality of
  contexts.  The type of elements of \coqdocvariable{A} is noted \coqref{groupoid interpretation def.Elt}{\coqdocdefinition{Elt}} \coqdocvariable{A} := [\coqdocdefinition{$\Pi$} \coqdocvariable{A}]
  (context is implicit).  

  A dependent type $\Gamma, x:A \vdash B$ is interpreted in two
  equivalent ways: simply as a type \coqref{groupoid interpretation def.TypDep}{\coqdocdefinition{TypDep}} \coqdocvariable{A} := \coqref{groupoid interpretation def.Typ}{\coqdocdefinition{Typ}} (\coqdocdefinition{$\Sigma$} \coqdocvariable{A}) over the
  dependent sum of \coqdocvariable{Γ} and \coqdocvariable{A} or as a type family \coqref{groupoid interpretation def.TypFam}{\coqdocdefinition{TypFam}} \coqdocvariable{A} over \coqdocvariable{A}
  (corresponding to a family of sets in constructive mathematics). A
  type family can be seen as a fibration (or bundle) from \coqdocvariable{B} to \coqdocvariable{A}.
  In what follows, the indice $\mathsf{_{comp}}$ is given to proofs of 
  (dependent) functoriality.
\begin{coqdoccode}
\coqdocemptyline
\coqdocnoindent
\coqdockw{Definition} \coqdef{groupoid interpretation def.TypFam}{TypFam}{\coqdocdefinition{TypFam}} \{\coqdocvar{Γ} : \coqref{groupoid interpretation def.Context}{\coqdocdefinition{Context}}\} (\coqdocvar{A}: \coqref{groupoid interpretation def.Typ}{\coqdocdefinition{Typ}} \coqdocvariable{Γ}) := \coqdoceol
\coqdocindent{1.00em}
\coqdocnotation{[}\coqdocdefinition{$\Pi$} \coqdocnotation{(}\coqexternalref{::'xCExBB' x '..' x ',' x}{http://coq.inria.fr/stdlib/Coq.Unicode.Utf8\_core}{\coqdocnotation{\ensuremath{\lambda}}} \coqdocvar{$\gamma$}\coqexternalref{::'xCExBB' x '..' x ',' x}{http://coq.inria.fr/stdlib/Coq.Unicode.Utf8\_core}{\coqdocnotation{,}} \coqdocnotation{ } \coqdocnotation{(}\coqdocvariable{A} \coqdocnotation{$\star$} \coqdocvariable{$\gamma$}\coqdocnotation{)} \coqdocnotation{$_{\upharpoonright s}$} \coqdocnotation{$\longrightarrow$} \coqdocdefinition{$\mathsf{Type}_{0}^1$}\coqdocnotation{;} \coqref{groupoid interpretation def.TypFam 1}{\coqdocinstance{$\mathsf{TypFam_{comp}}$}} \coqdocvar{\_}\coqdocnotation{)}\coqdocnotation{]}.\coqdoceol
\coqdocemptyline
\end{coqdoccode}


  Elements of \coqref{groupoid interpretation def.TypDep}{\coqdocdefinition{TypDep}} \coqdocvariable{A} and \coqref{groupoid interpretation def.TypFam}{\coqdocdefinition{TypFam}} \coqdocvariable{A} can be related using a dependent closure
  at the level of types. In the interpretation of typing judgments, this connection 
  will be used to switch between the fibration and the morphism points of view.
\begin{coqdoccode}
\coqdocemptyline
\coqdocnoindent
\coqdockw{Definition} \coqdef{groupoid interpretation def.LamT}{LamT}{\coqdocdefinition{LamT}} \{\coqdocvar{Γ}: \coqref{groupoid interpretation def.Context}{\coqdocdefinition{Context}}\} \{\coqdocvar{A} : \coqref{groupoid interpretation def.Typ}{\coqdocdefinition{Typ}} \coqdocvariable{Γ}\} (\coqdocvar{B}: \coqref{groupoid interpretation def.UTypDep}{\coqdocdefinition{UTypDep}} \coqdocvariable{A})\coqdoceol
\coqdocindent{1.00em}
: \coqref{groupoid interpretation def.UTypFam}{\coqdocdefinition{UTypFam}} (\coqref{groupoid interpretation def.::'[[[' x ']]]'}{\coqdocnotation{ }}\coqdocvariable{A}\coqref{groupoid interpretation def.::'[[[' x ']]]'}{\coqdocnotation{$_{\upharpoonright s}$}}) := \coqdocnotation{(}\coqexternalref{::'xCExBB' x '..' x ',' x}{http://coq.inria.fr/stdlib/Coq.Unicode.Utf8\_core}{\coqdocnotation{\ensuremath{\lambda}}} \coqdocvar{$\gamma$}\coqexternalref{::'xCExBB' x '..' x ',' x}{http://coq.inria.fr/stdlib/Coq.Unicode.Utf8\_core}{\coqdocnotation{,}} \coqdocnotation{(}\coqexternalref{::'xCExBB' x '..' x ',' x}{http://coq.inria.fr/stdlib/Coq.Unicode.Utf8\_core}{\coqdocnotation{\ensuremath{\lambda}}} \coqdocvar{t}\coqexternalref{::'xCExBB' x '..' x ',' x}{http://coq.inria.fr/stdlib/Coq.Unicode.Utf8\_core}{\coqdocnotation{,}} \coqdocvariable{B} \coqdocnotation{$\star$} \coqdocnotation{(}\coqdocvariable{$\gamma$}\coqdocnotation{;} \coqdocvariable{t}\coqdocnotation{)} \coqdocnotation{;} \coqdocvar{\_}\coqdocnotation{);} \coqref{groupoid interpretation def.LamT 1}{\coqdocaxiom{$\mathsf{LamT_{comp}}$}} \coqdocvariable{B}\coqdocnotation{)}.\coqdoceol
\coqdocemptyline
\end{coqdoccode}


\subsection{Substitutions}




  A substitution is represented by a context morphism [\coqdocvariable{Γ} $\longrightarrow$ \coqdocvariable{Δ}]. 
  A substitution σ can be extended by an element \coqdocvariable{a}: \coqref{groupoid interpretation def.Elt}{\coqdocdefinition{Elt}} (\coqdocvariable{A} $\circ$ σ) 
  of \coqdocvariable{A} : \coqref{groupoid interpretation def.Typ}{\coqdocdefinition{Typ}} \coqdocvariable{Δ}.


\begin{coqdoccode}
\coqdocemptyline
\coqdocnoindent
\coqdockw{Definition} \coqdef{groupoid interpretation def.SubExt}{SubExt}{\coqdocdefinition{SubExt}} \{\coqdocvar{Γ} \coqdocvar{Δ} : \coqref{groupoid interpretation def.Context}{\coqdocdefinition{Context}}\} \{\coqdocvar{A} : \coqref{groupoid interpretation def.Typ}{\coqdocdefinition{Typ}} \coqdocvariable{Δ}\} (σ: \coqdocnotation{[}\coqdocvariable{Γ} \coqdocnotation{$\longrightarrow$} \coqdocvariable{Δ}\coqdocnotation{]}) (\coqdocvar{a}: \coqref{groupoid interpretation def.Elt}{\coqdocdefinition{Elt}} (\coqdocvariable{A} \coqref{groupoid interpretation def.::x 'xE2x8Bx85xE2x8Bx85' x}{\coqdocnotation{$⋅$}} \coqdocvariable{σ})) \coqdoceol
\coqdocindent{1.00em}
: \coqdocnotation{[}\coqdocvariable{Γ} \coqdocnotation{$\longrightarrow$} \coqdocdefinition{\_Sum1} \coqdocvariable{A} \coqdocnotation{]} := \coqdocnotation{(}\coqexternalref{::'xCExBB' x '..' x ',' x}{http://coq.inria.fr/stdlib/Coq.Unicode.Utf8\_core}{\coqdocnotation{\ensuremath{\lambda}}} \coqdocvar{$\gamma$}\coqexternalref{::'xCExBB' x '..' x ',' x}{http://coq.inria.fr/stdlib/Coq.Unicode.Utf8\_core}{\coqdocnotation{,}} \coqdocnotation{(}\coqdocvariable{σ} \coqdocnotation{$\star$} \coqdocvariable{$\gamma$}\coqdocnotation{;} \coqdocvariable{a} \coqdocnotation{$\star$} \coqdocvariable{$\gamma$}\coqdocnotation{)} \coqdocnotation{;} \coqref{groupoid interpretation def.SubExt 1}{\coqdocinstance{$\mathsf{SubExt_{comp}}$}} \coqdocvar{\_} \coqdocvar{\_}\coqdocnotation{)}.\coqdoceol
\coqdocemptyline
\end{coqdoccode}
\noindent where \coqref{groupoid interpretation def.SubExt 1}{\coqdocinstance{$\mathsf{SubExt_{comp}}$}} is a proof that it is functorial. 
\begin{coqdoccode}
\coqdocemptyline
\coqdocnoindent
\coqdockw{Definition} \coqdef{groupoid interpretation def.substF}{substF}{\coqdocdefinition{substF}} \{\coqdocvar{T} \coqdocvar{Γ}\} \{\coqdocvar{A}:\coqref{groupoid interpretation def.Typ}{\coqdocdefinition{Typ}} \coqdocvariable{Γ}\} (\coqdocvar{F}:\coqref{groupoid interpretation def.UTypFam}{\coqdocdefinition{UTypFam}} (\coqref{groupoid interpretation def.::'[[[' x ']]]'}{\coqdocnotation{ }}\coqdocvariable{A}\coqref{groupoid interpretation def.::'[[[' x ']]]'}{\coqdocnotation{$_{\upharpoonright s}$}})) (σ:\coqdocnotation{[}\coqdocvariable{T} \coqdocnotation{$\longrightarrow$} \coqdocvariable{Γ}\coqdocnotation{]}) : \coqref{groupoid interpretation def.UTypFam}{\coqdocdefinition{UTypFam}} (\coqref{groupoid interpretation def.::'[[[' x ']]]'}{\coqdocnotation{ }}\coqdocvariable{A} \coqref{groupoid interpretation def.::x 'xE2x8Bx85xE2x8Bx85' x}{\coqdocnotation{$⋅$}} \coqdocvariable{σ}\coqref{groupoid interpretation def.::'[[[' x ']]]'}{\coqdocnotation{$_{\upharpoonright s}$}}) \coqdoceol
\coqdocindent{1.00em}
:= \coqdocnotation{(}\coqdocnotation{[}\coqdocvariable{F} \coqref{groupoid interpretation def.::x 'xC2xB0xC2xB0' x}{\coqdocnotation{$\circ$}} \coqdocvariable{σ}\coqdocnotation{]} : \coqexternalref{:type scope:'xE2x88x80' x '..' x ',' x}{http://coq.inria.fr/stdlib/Coq.Unicode.Utf8\_core}{\coqdocnotation{∀}} \coqdocvar{t} : \coqdocnotation{[}\coqdocvariable{T}\coqdocnotation{]}\coqexternalref{:type scope:'xE2x88x80' x '..' x ',' x}{http://coq.inria.fr/stdlib/Coq.Unicode.Utf8\_core}{\coqdocnotation{,}} \coqdocnotation{\ensuremath{|}}\coqdocnotation{(}\coqref{groupoid interpretation def.::'[[[' x ']]]'}{\coqdocnotation{ }}\coqdocvariable{A} \coqref{groupoid interpretation def.::x 'xE2x8Bx85xE2x8Bx85' x}{\coqdocnotation{$⋅$}} \coqdocvariable{σ}\coqref{groupoid interpretation def.::'[[[' x ']]]'}{\coqdocnotation{$_{\upharpoonright s}$}}\coqdocnotation{)} \coqdocnotation{$\star$} \coqdocvariable{t}\coqdocnotation{\ensuremath{|}}\coqdocnotation{g} \coqdocnotation{$\longrightarrow$} \coqdocdefinition{Type1}\coqdocnotation{;} \coqref{groupoid interpretation def.substF 1}{\coqdocinstance{$\mathsf{substF_{comp}}$}} \coqdocvariable{F} \coqdocvariable{σ}\coqdocnotation{)}.\coqdoceol
\coqdocemptyline
\end{coqdoccode}
A substitution σ can be applied to a type family \coqdocvariable{F} using the
  composition of a functor with a dependent functor. We
  abusively note all those different compositions with $\circ$ as it is done in
  mathematics, whereas they are distinct operators in the \Coq
  development.
  The weakening substitution of $\Gamma, x:A \vdash$ is given by the first
  projection. This allows to interpret a type A in a weakened context, 
  which is noted  $\shortuparrow$ \coqdocvariable{A}. \begin{coqdoccode}
\coqdocemptyline
\end{coqdoccode}


  A type family \coqdocvariable{F} in \coqref{groupoid interpretation def.TypFam}{\coqdocdefinition{TypFam}} \coqdocvariable{A} can be partially substituted with an
  element \coqdocvariable{a} in \coqref{groupoid interpretation def.Elt}{\coqdocdefinition{Elt}} \coqdocvariable{A}, noted \coqdocvariable{F} \{\{\coqdocvariable{a}\}\}, to get its value (a type) at
  \coqdocvariable{a}. This process is defined as \coqdocvariable{F} \{\{\coqdocvariable{a}\}\} := (\coqdocvar{\ensuremath{\lambda}} \coqdocvariable{$\gamma$}, (\coqdocvariable{F} $\star$ \coqdocvariable{$\gamma$}) $\star$ (\coqdocvariable{a} $\star$ \coqdocvariable{$\gamma$}) ;
  \coqdocvar{\_}) (where \coqdocvar{\_} is a proof it is functorial). Note that this
  pattern of application \emph{up-to a context $\gamma$} will be used
  later to defined other notions of application. Although the
  computational definitions are the same, the compatibility conditions
  are always different.  This notion of partial substitution in a type
  family enables to state that \coqref{groupoid interpretation def.LamT}{\coqdocdefinition{LamT}} defines a type level
  $\lambda$-abstraction.  \begin{coqdoccode}
\coqdocemptyline
\coqdocnoindent
\coqdockw{Definition} \coqdef{groupoid interpretation def.BetaT}{BetaT}{\coqdocdefinition{BetaT}} \coqdocvar{Δ} \coqdocvar{Γ} (\coqdocvar{A}:\coqref{groupoid interpretation def.Typ}{\coqdocdefinition{Typ}} \coqdocvariable{Γ}) (\coqdocvar{B}:\coqref{groupoid interpretation def.UTypDep}{\coqdocdefinition{UTypDep}} \coqdocvariable{A}) (σ:\coqdocnotation{[}\coqdocvariable{Δ} \coqdocnotation{$\longrightarrow$} \coqdocvariable{Γ}\coqdocnotation{]}) (\coqdocvar{a}:\coqref{groupoid interpretation def.Elt}{\coqdocdefinition{Elt}} (\coqdocvariable{A} \coqref{groupoid interpretation def.::x 'xE2x8Bx85xE2x8Bx85' x}{\coqdocnotation{$⋅$}} \coqdocvariable{σ})) \coqdoceol
\coqdocindent{1.00em}
: \coqref{groupoid interpretation def.LamT}{\coqdocdefinition{LamT}} \coqdocvariable{B} \coqref{groupoid interpretation def.::x 'xC2xB0xC2xB0xC2xB0' x}{\coqdocnotation{$\circ$}} \coqdocvariable{σ} \coqref{groupoid interpretation def.::x 'x7Bx7Bx7B' x 'x7Dx7Dx7D'}{\coqdocnotation{\{\{\{}}\coqdocvariable{a}\coqref{groupoid interpretation def.::x 'x7Bx7Bx7B' x 'x7Dx7Dx7D'}{\coqdocnotation{\}\}\}}} \coqdocnotation{$\sim_1$} \coqdocvariable{B} \coqref{groupoid interpretation def.::x 'xE2x8Bx85xE2x8Bx85xE2x8Bx85' x}{\coqdocnotation{$⋅$}} \coqref{groupoid interpretation def.::x 'xE2x8Bx85xE2x8Bx85xE2x8Bx85' x}{\coqdocnotation{(}}\coqref{groupoid interpretation def.SubExt}{\coqdocdefinition{SubExt}} \coqdocvariable{σ} \coqdocvariable{a}\coqref{groupoid interpretation def.::x 'xE2x8Bx85xE2x8Bx85xE2x8Bx85' x}{\coqdocnotation{)}} := \coqdocnotation{(}\coqexternalref{::'xCExBB' x '..' x ',' x}{http://coq.inria.fr/stdlib/Coq.Unicode.Utf8\_core}{\coqdocnotation{\ensuremath{\lambda}}} \coqdocvar{\_}\coqexternalref{::'xCExBB' x '..' x ',' x}{http://coq.inria.fr/stdlib/Coq.Unicode.Utf8\_core}{\coqdocnotation{,}} \coqdocmethod{identity} \coqdocvar{\_} \coqdocnotation{;} \coqref{groupoid interpretation def.BetaT 1}{\coqdocaxiom{$\mathsf{BetaT_{comp}}$}} \coqdocvar{\_} \coqdocvar{\_} \coqdocvar{\_}\coqdocnotation{)}.\coqdoceol
\coqdocemptyline
\end{coqdoccode}


\subsection{Interpretation of the typing judgment}


  \label{sec:interp}


  The typing rules of \TTu\xspace of Figure \ref{fig:emltt} are
  interpreted in the groupoid model as described below.


  \paragraph{\textsc{Var}.} 


  The rule \textsc{Var} is given by the second projection plus a proof
  that the projection is dependently functorial. Note the explicit
  weakening of \coqdocvariable{A} in the returned type. This is because we need to
  make explicit that the context used to type \coqdocvariable{A} is extended with an
  element of type \coqdocvariable{A}. The rule of Figure \ref{fig:emltt} is more general 
  as it performs an implicit weakening. We do not interpret this part of 
  the rule as weakening is explicit in our model. 


\begin{coqdoccode}
\coqdocemptyline
\coqdocnoindent
\coqdockw{Definition} \coqdef{groupoid interpretation def.Var}{Var}{\coqdocdefinition{Var}} \{\coqdocvar{Γ}\} (\coqdocvar{A}:\coqref{groupoid interpretation def.Typ}{\coqdocdefinition{Typ}} \coqdocvariable{Γ}) : \coqref{groupoid interpretation def.Elt}{\coqdocdefinition{Elt}} \coqref{groupoid interpretation def.::'xE2x87x91' x}{\coqdocnotation{$\shortuparrow$}}\coqdocvariable{A} := \coqdocnotation{(}\coqexternalref{::'xCExBB' x '..' x ',' x}{http://coq.inria.fr/stdlib/Coq.Unicode.Utf8\_core}{\coqdocnotation{\ensuremath{\lambda}}} \coqdocvar{t}\coqexternalref{::'xCExBB' x '..' x ',' x}{http://coq.inria.fr/stdlib/Coq.Unicode.Utf8\_core}{\coqdocnotation{,}} \coqdocabbreviation{$\pi_2$} \coqdocvariable{t}\coqdocnotation{;} \coqref{groupoid interpretation def.Var 1}{\coqdocinstance{$\mathsf{Var_{comp}}$}} \coqdocvariable{A}\coqdocnotation{)}.\coqdoceol
\coqdocemptyline
\coqdocemptyline
\end{coqdoccode}
\paragraph{\textsc{Prod}.} The rule \textsc{Prod} is interpreted
  using the dependent functor space, plus a proof that equivalent
  contexts give rise to isomorphic dependent functor spaces.  Note that
  the rule is defined on type families and not on the dependent type
  formulation because here we need a fibration point of view. \begin{coqdoccode}
\coqdocemptyline
\coqdocnoindent
\coqdockw{Definition} \coqdef{groupoid interpretation def.Prod}{Prod}{\coqdocdefinition{Prod}} \{\coqdocvar{Γ}\} (\coqdocvar{A}:\coqref{groupoid interpretation def.Typ}{\coqdocdefinition{Typ}} \coqdocvariable{Γ}) (\coqdocvar{F}:\coqref{groupoid interpretation def.UTypFam}{\coqdocdefinition{UTypFam}} (\coqref{groupoid interpretation def.::'[[[' x ']]]'}{\coqdocnotation{ }}\coqdocvariable{A}\coqref{groupoid interpretation def.::'[[[' x ']]]'}{\coqdocnotation{$_{\upharpoonright s}$}})) \coqdoceol
\coqdocindent{1.00em}
: \coqref{groupoid interpretation def.UTyp}{\coqdocdefinition{UTyp}} \coqdocvariable{Γ} := \coqdocnotation{(}\coqexternalref{::'xCExBB' x '..' x ',' x}{http://coq.inria.fr/stdlib/Coq.Unicode.Utf8\_core}{\coqdocnotation{\ensuremath{\lambda}}} \coqdocvar{s}\coqexternalref{::'xCExBB' x '..' x ',' x}{http://coq.inria.fr/stdlib/Coq.Unicode.Utf8\_core}{\coqdocnotation{,}} \coqdocdefinition{Prod1} (\coqdocvariable{F} \coqdocnotation{$\star$} \coqdocvariable{s})\coqdocnotation{;} \coqref{groupoid interpretation def.Prod 1}{\coqdocaxiom{$\mathsf{Prod_{comp}}$}} \coqdocvariable{A} \coqdocvariable{F}\coqdocnotation{)}.\coqdoceol
\coqdocemptyline
\end{coqdoccode}
  \paragraph{\textsc{App}.}


  The rule \textsc{App} is interpreted using an up-to context application 
  and a proof of dependent functoriality. We abusively note \coqdocvar{M} $\star$ \coqdocvar{N} the application 
  of \coqref{groupoid interpretation def.App}{\coqdocdefinition{App}}.
\begin{coqdoccode}
\coqdocemptyline
\coqdocnoindent
\coqdockw{Definition} \coqdef{groupoid interpretation def.App}{App}{\coqdocdefinition{App}} \{\coqdocvar{Γ}\} \{\coqdocvar{A}:\coqref{groupoid interpretation def.Typ}{\coqdocdefinition{Typ}} \coqdocvariable{Γ}\} \{\coqdocvar{F}:\coqref{groupoid interpretation def.UTypFam}{\coqdocdefinition{UTypFam}} (\coqref{groupoid interpretation def.::'[[[' x ']]]'}{\coqdocnotation{ }}\coqdocvariable{A}\coqref{groupoid interpretation def.::'[[[' x ']]]'}{\coqdocnotation{$_{\upharpoonright s}$}})\} (\coqdocvar{c}:\coqref{groupoid interpretation def.UElt}{\coqdocdefinition{UElt}} (\coqref{groupoid interpretation def.Prod}{\coqdocdefinition{Prod}} \coqdocvariable{F})) (\coqdocvar{a}:\coqref{groupoid interpretation def.Elt}{\coqdocdefinition{Elt}} \coqdocvariable{A}) \coqdoceol
\coqdocindent{1.00em}
: \coqref{groupoid interpretation def.UElt}{\coqdocdefinition{UElt}} (\coqdocvariable{F} \coqref{groupoid interpretation def.::x 'x7Bx7Bx7B' x 'x7Dx7Dx7D'}{\coqdocnotation{\{\{\{}}\coqdocvariable{a}\coqref{groupoid interpretation def.::x 'x7Bx7Bx7B' x 'x7Dx7Dx7D'}{\coqdocnotation{\}\}\}}}) := \coqdocnotation{(}\coqexternalref{::'xCExBB' x '..' x ',' x}{http://coq.inria.fr/stdlib/Coq.Unicode.Utf8\_core}{\coqdocnotation{\ensuremath{\lambda}}} \coqdocvar{s}\coqexternalref{::'xCExBB' x '..' x ',' x}{http://coq.inria.fr/stdlib/Coq.Unicode.Utf8\_core}{\coqdocnotation{,}} \coqdocnotation{(}\coqdocvariable{c} \coqdocnotation{$\star$} \coqdocvariable{s}\coqdocnotation{)} \coqdocnotation{$\star$} \coqdocnotation{(}\coqdocvariable{a} \coqdocnotation{$\star$} \coqdocvariable{s}\coqdocnotation{)}\coqdocnotation{;} \coqref{groupoid interpretation def.App 1}{\coqdocinstance{$\mathsf{App_{comp}}$}} \coqdocvariable{c} \coqdocvariable{a}\coqdocnotation{)}.\coqdoceol
\coqdocemptyline
\end{coqdoccode}
  \paragraph{\lrule{Lam}.}


  Term-level $\lambda$-abstraction is defined with the same
  computational meaning as type-level $\lambda$-abstraction, but it
  differs on the proof of dependent functoriality. Note that we use
  \coqref{groupoid interpretation def.LamT}{\coqdocdefinition{LamT}} in the definition because we need both the fibration (for
  \coqref{groupoid interpretation def.Prod}{\coqdocdefinition{Prod}}) and the morphism (for \coqref{groupoid interpretation def.Elt}{\coqdocdefinition{Elt}} \coqdocvariable{B}) point of view. 
\begin{coqdoccode}
\coqdocemptyline
\coqdocnoindent
\coqdockw{Definition} \coqdef{groupoid interpretation def.Lam}{Lam}{\coqdocdefinition{Lam}} \{\coqdocvar{Γ}\} \{\coqdocvar{A}:\coqref{groupoid interpretation def.Typ}{\coqdocdefinition{Typ}} \coqdocvariable{Γ}\} \{\coqdocvar{B}:\coqref{groupoid interpretation def.UTypDep}{\coqdocdefinition{UTypDep}} \coqdocvariable{A}\} (\coqdocvar{b}:\coqref{groupoid interpretation def.UElt}{\coqdocdefinition{UElt}} \coqdocvariable{B})\coqdoceol
\coqdocindent{1.00em}
: \coqref{groupoid interpretation def.UElt}{\coqdocdefinition{UElt}} (\coqref{groupoid interpretation def.Prod}{\coqdocdefinition{Prod}} (\coqref{groupoid interpretation def.LamT}{\coqdocdefinition{LamT}} \coqdocvariable{B})) := \coqdocnotation{(}\coqexternalref{::'xCExBB' x '..' x ',' x}{http://coq.inria.fr/stdlib/Coq.Unicode.Utf8\_core}{\coqdocnotation{\ensuremath{\lambda}}} \coqdocvar{$\gamma$}\coqexternalref{::'xCExBB' x '..' x ',' x}{http://coq.inria.fr/stdlib/Coq.Unicode.Utf8\_core}{\coqdocnotation{,}} \coqdocnotation{(}\coqexternalref{::'xCExBB' x '..' x ',' x}{http://coq.inria.fr/stdlib/Coq.Unicode.Utf8\_core}{\coqdocnotation{\ensuremath{\lambda}}} \coqdocvar{t}\coqexternalref{::'xCExBB' x '..' x ',' x}{http://coq.inria.fr/stdlib/Coq.Unicode.Utf8\_core}{\coqdocnotation{,}} \coqdocvariable{b} \coqdocnotation{$\star$} \coqdocnotation{(}\coqdocvariable{$\gamma$} \coqdocnotation{;} \coqdocvariable{t}\coqdocnotation{)} \coqdocnotation{;} \coqdocvar{\_}\coqdocnotation{);} \coqref{groupoid interpretation def.Lam 2}{\coqdocaxiom{$\mathsf{Lam_{comp}}$}} \coqdocvariable{b}\coqdocnotation{)}.\coqdoceol
\coqdocemptyline
\coqdocemptyline
\end{coqdoccode}
  \paragraph{\textsc{Sigma}, \textsc{Pair} and \textsc{Projs}.}
  The rules for Σ types are interpreted using the 
  dependent sum \coqdocvar{$\Sigma$} on setoids.  
\begin{coqdoccode}
\coqdocemptyline
\coqdocnoindent
\coqdockw{Definition} \coqdef{groupoid interpretation def.Sigma}{Sigma}{\coqdocdefinition{Sigma}} \{\coqdocvar{Γ}\} (\coqdocvar{A}:\coqref{groupoid interpretation def.Typ}{\coqdocdefinition{Typ}} \coqdocvariable{Γ}) (\coqdocvar{F}:\coqref{groupoid interpretation def.TypFam}{\coqdocdefinition{TypFam}} \coqdocvariable{A}) \coqdoceol
\coqdocindent{1.00em}
: \coqref{groupoid interpretation def.UTyp}{\coqdocdefinition{UTyp}} \coqdocvariable{Γ} := \coqdocnotation{(}\coqexternalref{::'xCExBB' x '..' x ',' x}{http://coq.inria.fr/stdlib/Coq.Unicode.Utf8\_core}{\coqdocnotation{\ensuremath{\lambda}}} \coqdocvar{$\gamma$}: \coqdocnotation{[}\coqdocvariable{Γ}\coqdocnotation{]}\coqexternalref{::'xCExBB' x '..' x ',' x}{http://coq.inria.fr/stdlib/Coq.Unicode.Utf8\_core}{\coqdocnotation{,}} \coqdocdefinition{\_Sum1} (\coqdocvariable{F} \coqdocnotation{$\star$} \coqdocvariable{$\gamma$})\coqdocnotation{;} \coqref{groupoid interpretation def.Sigma 1}{\coqdocaxiom{$\mathsf{Sigma_{comp}}$}} \coqdocvariable{A} \coqdocvariable{F}\coqdocnotation{)}.\coqdoceol
\coqdocemptyline
\end{coqdoccode}
\noindent Pairing and projections are obtained
  by a context lift of pairing and projection of the underlying dependent sum.
\begin{coqdoccode}
\coqdocemptyline
\coqdocemptyline
\end{coqdoccode}
  \paragraph{\lrule{Conv}.}


  It is not possible to prove in \Coq that the conversion rule is
  preserved because the application of this rule is implicit and
  can not be reified. Nevertheless, to witness this preservation, 
  we show that beta conversion is valid as a definitional equality
  on the first projection. As conversion is only done on 
  types and interpretation of types is always projected, this is 
  enough to guarantee that the conversion rule is also admissible.


\begin{coqdoccode}
\coqdocemptyline
\coqdocnoindent
\coqdockw{Definition} \coqdef{groupoid interpretation def.Beta}{Beta}{\coqdocdefinition{Beta}} \{\coqdocvar{Γ}\} \{\coqdocvar{A}:\coqref{groupoid interpretation def.Typ}{\coqdocdefinition{Typ}} \coqdocvariable{Γ}\} \{\coqdocvar{F}:\coqref{groupoid interpretation def.UTypDep}{\coqdocdefinition{UTypDep}} \coqdocvariable{A}\} (\coqdocvar{b}:\coqref{groupoid interpretation def.UElt}{\coqdocdefinition{UElt}} \coqdocvariable{F}) (\coqdocvar{a}:\coqref{groupoid interpretation def.Elt}{\coqdocdefinition{Elt}} \coqdocvariable{A}) \coqdoceol
\coqdocindent{1.00em}
: \coqdocnotation{[}\coqref{groupoid interpretation def.Lam}{\coqdocdefinition{Lam}} \coqdocvariable{b} \coqref{groupoid interpretation def.::x '@@' x}{\coqdocnotation{$\star$}} \coqdocvariable{a}\coqdocnotation{]} \coqdocnotation{=} \coqdocnotation{[}\coqdocvariable{b} \coqref{groupoid interpretation def.::x 'xC2xB0xC2xB0' x}{\coqdocnotation{$\circ$}} \coqref{groupoid interpretation def.SubExtId}{\coqdocdefinition{SubExtId}} \coqdocvariable{a}\coqdocnotation{]} := \coqdocconstructor{eq\_refl} \coqdocvar{\_}.\coqdoceol
\coqdocemptyline
\end{coqdoccode}
 \noindent where \coqref{groupoid interpretation def.SubExtId}{\coqdocdefinition{SubExtId}} is a specialization of \coqref{groupoid interpretation def.SubExt}{\coqdocdefinition{SubExt}} with 
  the identity substitution.
\begin{coqdoccode}
\coqdocemptyline
\end{coqdoccode}


\subsection{Extensional principles}


  \label{sec:extprinc}
  One of the main interest of the groupoid interpretation is that it
  allows to interpret a type directed notion of equality which validates 
  the J eliminator of identity types but also various extensional principles.
  For any elements \coqdocvariable{a} and \coqdocvariable{b} of a type \coqdocvariable{A}, we note \coqref{groupoid interpretation def.Id}{\coqdocdefinition{Id}} \coqdocvariable{a} \coqdocvariable{b} the equality type
  between \coqdocvariable{a} and \coqdocvariable{b}.
\begin{coqdoccode}
\coqdocemptyline
\coqdocnoindent
\coqdockw{Definition} \coqdef{groupoid interpretation def.Id}{Id}{\coqdocdefinition{Id}} \{\coqdocvar{Γ}\} (\coqdocvar{A}: \coqref{groupoid interpretation def.UTyp}{\coqdocdefinition{UTyp}} \coqdocvariable{Γ}) (\coqdocvar{a} \coqdocvar{b} : \coqref{groupoid interpretation def.UElt}{\coqdocdefinition{UElt}} \coqdocvariable{A}) \coqdoceol
\coqdocindent{1.00em}
: \coqref{groupoid interpretation def.Typ}{\coqdocdefinition{Typ}} \coqdocvariable{Γ} := \coqdocnotation{(}\coqexternalref{::'xCExBB' x '..' x ',' x}{http://coq.inria.fr/stdlib/Coq.Unicode.Utf8\_core}{\coqdocnotation{\ensuremath{\lambda}}} \coqdocvar{$\gamma$}\coqexternalref{::'xCExBB' x '..' x ',' x}{http://coq.inria.fr/stdlib/Coq.Unicode.Utf8\_core}{\coqdocnotation{,}} \coqdocnotation{(}\coqdocvariable{a} \coqdocnotation{$\star$} \coqdocvariable{$\gamma$} \coqdocnotation{$\sim_1$} \coqdocvariable{b} \coqdocnotation{$\star$} \coqdocvariable{$\gamma$} \coqdocnotation{;} \coqdocvar{\_}\coqdocnotation{);} \coqref{groupoid interpretation def.Id 1}{\coqdocaxiom{$\mathsf{Id_{comp}}$}} \coqdocvariable{A} \coqdocvariable{a} \coqdocvariable{b}\coqdocnotation{)}.\coqdoceol
\coqdocemptyline
\coqdocemptyline
\end{coqdoccode}
We can interpret the J eliminator of MLTT on \coqref{groupoid interpretation def.Id}{\coqdocdefinition{Id}} using functoriality of \coqdocvariable{P} and of product 
   (\coqref{groupoid interpretation def.prod comp}{\coqdocdefinition{$\mathsf{\Pi_{comp}}$}}). \begin{coqdoccode}
\coqdocemptyline
\coqdocnoindent
\coqdockw{Definition} \coqdef{groupoid interpretation def.J}{J}{\coqdocdefinition{J}} \coqdocvar{Γ} (\coqdocvar{A}:\coqref{groupoid interpretation def.UTyp}{\coqdocdefinition{UTyp}} \coqdocvariable{Γ}) (\coqdocvar{a} \coqdocvar{b}:\coqref{groupoid interpretation def.UElt}{\coqdocdefinition{UElt}} \coqdocvariable{A}) (\coqdocvar{e}:\coqref{groupoid interpretation def.Elt}{\coqdocdefinition{Elt}} (\coqref{groupoid interpretation def.Id}{\coqdocdefinition{Id}} \coqdocvariable{a} \coqdocvariable{b})) (\coqdocvar{P}:\coqref{groupoid interpretation def.UTypFam}{\coqdocdefinition{UTypFam}} \coqdocvariable{A}) (\coqdocvar{p}:\coqref{groupoid interpretation def.UElt}{\coqdocdefinition{UElt}} (\coqdocvariable{P}\coqref{groupoid interpretation def.::x 'x7Bx7Bx7B' x 'x7Dx7Dx7D'}{\coqdocnotation{\{\{\{}}\coqdocvariable{a}\coqref{groupoid interpretation def.::x 'x7Bx7Bx7B' x 'x7Dx7Dx7D'}{\coqdocnotation{\}\}\}}}))\coqdoceol
\coqdocindent{0.50em}
: \coqref{groupoid interpretation def.UElt}{\coqdocdefinition{UElt}} (\coqdocvariable{P}\coqref{groupoid interpretation def.::x 'x7Bx7Bx7B' x 'x7Dx7Dx7D'}{\coqdocnotation{\{\{\{}}\coqdocvariable{b}\coqref{groupoid interpretation def.::x 'x7Bx7Bx7B' x 'x7Dx7Dx7D'}{\coqdocnotation{\}\}\}}}) := \coqdocnotation{(}\coqref{groupoid interpretation def.prod comp}{\coqdocdefinition{$\mathsf{\Pi_{comp}}$}} \coqdocnotation{(}\coqexternalref{::'xCExBB' x '..' x ',' x}{http://coq.inria.fr/stdlib/Coq.Unicode.Utf8\_core}{\coqdocnotation{\ensuremath{\lambda}}} \coqdocvar{$\gamma$}\coqexternalref{::'xCExBB' x '..' x ',' x}{http://coq.inria.fr/stdlib/Coq.Unicode.Utf8\_core}{\coqdocnotation{,}} \coqexternalref{::'xCExBB' x '..' x ',' x}{http://coq.inria.fr/stdlib/Coq.Unicode.Utf8\_core}{\coqdocnotation{(}}\coqdocnotation{map} \coqdocnotation{(}\coqdocvariable{P} \coqdocnotation{$\star$} \coqdocvariable{$\gamma$}\coqdocnotation{)} (\coqdocvariable{e} \coqdocnotation{$\star$} \coqdocvariable{$\gamma$})\coqexternalref{::'xCExBB' x '..' x ',' x}{http://coq.inria.fr/stdlib/Coq.Unicode.Utf8\_core}{\coqdocnotation{)}}\coqdocnotation{;} \coqref{groupoid interpretation def.J 1}{\coqdocaxiom{$\mathsf{J_{comp}}$}} \coqdocvar{\_} \coqdocvar{\_}\coqdocnotation{)}\coqdocnotation{)} \coqdocnotation{$\star$} \coqdocvariable{p}.\coqdoceol
\coqdocemptyline
\coqdocemptyline
\end{coqdoccode}
Rule \textsc{ValId-Intro} is simply given by the identity of the underlying groupoid. \begin{coqdoccode}
\coqdocemptyline
\coqdocnoindent
\coqdockw{Definition} \coqdef{groupoid interpretation def.Refl}{Refl}{\coqdocdefinition{Refl}} \coqdocvar{Γ} (\coqdocvar{A}: \coqref{groupoid interpretation def.Typ}{\coqdocdefinition{Typ}} \coqdocvariable{Γ}) (\coqdocvar{a} : \coqref{groupoid interpretation def.Elt}{\coqdocdefinition{Elt}} \coqdocvariable{A}) \coqdoceol
\coqdocindent{1.00em}
: \coqref{groupoid interpretation def.Elt}{\coqdocdefinition{Elt}} (\coqref{groupoid interpretation def.Id}{\coqdocdefinition{Id}} \coqdocvariable{a} \coqdocvariable{a}) := \coqdocnotation{(}\coqdockw{fun} \coqdocvar{$\gamma$} \ensuremath{\Rightarrow} \coqdocmethod{identity} \coqdocvar{\_}\coqdocnotation{;} \coqref{groupoid interpretation def.Refl 1}{\coqdocaxiom{$\mathsf{Refl_{comp}}$}} \coqdocvar{\_}\coqdocnotation{)}.\coqdoceol
\coqdocemptyline
\coqdocemptyline
\coqdocnoindent
\coqdockw{Definition} \coqdef{groupoid interpretation def.Equiv Intro}{Equiv\_Intro}{\coqdocdefinition{Equiv\_Intro}} (\coqdocvar{Γ}: \coqref{groupoid interpretation def.Context}{\coqdocdefinition{Context}}) (\coqdocvar{A} \coqdocvar{B} : \coqref{groupoid interpretation def.Typ}{\coqdocdefinition{Typ}} \coqdocvariable{Γ}) (\coqdocvar{e} : \coqref{groupoid interpretation def.iso}{\coqdocdefinition{iso}} \coqdocvariable{A} \coqdocvariable{B})\coqdoceol
\coqdocindent{1.00em}
: \coqref{groupoid interpretation def.Elt}{\coqdocdefinition{Elt}} (\coqdocvariable{A} \coqref{groupoid interpretation def.::x 'xE2x89xA1' x}{\coqdocnotation{≡}} \coqdocvariable{B}) := \coqdocnotation{(}\coqref{groupoid interpretation def.Equiv Intro }{\coqdocdefinition{Equiv\_Intro\_}} \coqdocvariable{e}\coqdocnotation{;} \coqref{groupoid interpretation def.Equiv Intro 1}{\coqdocaxiom{$\mathsf{Equiv\_Intro_{comp}}$}} \coqdocvariable{e}\coqdocnotation{)}.\coqdoceol
\coqdocemptyline
\coqdocemptyline
\end{coqdoccode}
