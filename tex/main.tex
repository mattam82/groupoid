\documentclass[runningheads,a4paper]{llncs}
\usepackage[T1]{fontenc}

\usepackage{utf}
\usepackage{float}
\usepackage{stfloats}
%\usepackage{midfloat}
%\usepackage{dblfloatfix}
\usepackage{subfig}
\usepackage{latexsym,amssymb,amsmath,ae,aeguill,amscd,stmaryrd}
\let\mathbb\undefined  % delete the command definition 
\usepackage{bbold}     % let bbold define its own \mathbb command
%\usepackage{mathrsfs}
\usepackage{xcolor}
\usepackage{tikz,xspace}
\usetikzlibrary{arrows,automata}
\usepackage[]{hyperref}

% My packages
\usepackage{abbrevs}
\usepackage{coq}
\usepackage{code}
\usepackage{qsymbols}

\usepackage[color]{coqdoc}
\definecolor{kwblack}{rgb}{0,0,0}
\def\coqdockwcolor{kwred}
\renewcommand{\coqdocnotation}[1]{#1}
\renewcommand{\coqexternalref}[3]{\href{#2.html\##1}{\textcolor{black}{#3}}}

\renewcommand{\coqdef}[3]{\phantomsection\hypertarget{coq:#1}{\text{#3}}}

\setlength{\coqdocbaseindent}{0.5em}

\newcommand{\aidememoire}[1]{\textcolor{red}{[[ #1 ]]}}
\def\interp#1{\ensuremath{\llbracket #1 \rrbracket}}

\newcommand{\mynote}[2]{
    \fbox{\bfseries\sffamily\scriptsize#1}
    {\small$\blacktriangleright$\textsf{\emph{#2}}$\blacktriangleleft$}~}

\newcommand\nt[1]{\mynote{\textcolor{magenta}{NT}}{#1}}
\newcommand\ms[1]{\mynote{\textcolor{blue}{MS}}{#1}}

\usepackage{url}

\usepackage{mathpartir}

\newcommand{\Cmd}[1]{{\mathsf{#1}}}
\newcommand{\Type}[1]{\ensuremath{\coqdockw{Type}_{#1}}\xspace}
\renewcommand{\Prop}{\ensuremath{\coqdockw{Prop}}\xspace}
\newcommand{\type}{\ensuremath{\coqdockw{type}}\xspace}

\newcommand{\Sort}{\mathcal{S}}
\newcommand{\Var}{\mathcal{V}}
\newcommand{\Metavar}{\mathcal{M}}
\newcommand{\Constant}{\mathcal{C}}
%\newcommand{\Inductive}{\mathcal{I}}
\newcommand{\Constr}{\mathcal{K}}
\newcommand{\Constraint}{\mathcal{O}}
\newcommand{\Nil}{\cdot}
\newcommand{\WFe}[1]{\vdash #1}
\newcommand{\WFs}[2]{#1 \vdash #2}
\newcommand{\WFc}[2]{#1 \vdash_{#2}}
\newcommand{\trule}[1]{\textsc{#1}}
%\newcommand{\irule}[3]
%{\inferrule{ #2 }{ #3 }{\ \trule{#1}}}
\newcommand{\irule}[3]
{\inferrule[#1] {#2 }{ #3 }}
\newcommand{\mdot}{.}
\newcommand{\subs}[2]{#2 / #1}
\newcommand{\msubs}[2]{\{\!\!\{#2 / #1\}\!\!\}}

\newcommand{\TTu}{\textsc{TT}$_{\mathcal{U}}$}
\newcommand{\whd}[1]{#1^*}
\newcommand{\whdbdi}[1]{#1 \downarrow_{\beta\delta\iota}}
\newcommand{\whdb}[1]{#1 \downarrow_{\beta}}
\def\order{\mathcal{O}}

\newcommand{\lub}[2]{#1\,\sqcup\,#2}

%\newcommand{\gen}{\ \triangleright\ }
%\newcommand{\gent}{\ \triangleright\ }

\newcommand{\types}[3]{#1 \vdash #2 : #3}

\def\tcoc{\vdash}
% \def\tcheck#1#2#3#4{\ensuremath{#1 \tcoc_{#2} #3 : #4}}
\def\tcheck#1#2#3#4{\ensuremath{#1 \tcoc #3 : #4}}
\def\ttcheck#1#2#3#4{\ensuremath{#1 \tcoc #3\ \type}}
\def\tcheckd#1#2#3#4{\ensuremath{#1 \tcoc_{#2}^{d} #3 : #4}}
\def\vdecl#1#2{#1 : #2}
\def\vdef#1#2#3{#1 := #2 : #3}
\def\tgenconv#1#2#3#4{#3 =^{#1}_{#2} #4}
\def\tgenconstr#1#2#3#4{#1 \models #2 #3 #4}
%\def\tconv#1#2#3{#2 =_{#1} #3}
\def\tconv#1#2#3{#2 = #3}
%\def\tcumul#1#2#3{#2 \preceq_{#1} #3}
\def\tcumul#1#2#3{#2 \preceq #3}
\renewcommand{\WFc}[2]{#1 \vdash}


\def\tconstrleq#1#2#3{#1 \models #2 \preceq #3}
\def\tconstreq#1#2#3{#1 \models #2 = #3}
\def\tconstrneq#1#2#3{#1 \not\models #2 = #3}

\def\tconsistent#1{#1\models}

\def\cstu#1#2{\underline{\cst{#1}}_{#2}}

\def\relR{\mathrel{R}}


\def\icumul#1#2#3#4{#2 \preceq_{#1} #3 "~>" #4}
\def\iunif#1#2#3#4{#2 \equiv_{#1} #3 "~>" #4}
\def\igenconv#1#2#3#4#5{#3 \equiv^{#1}_{#2} #4 "~>" #5}
\def\iconstreq#1#2#3#4{#1 \models #2 \equiv #3 "~>" #4}

\def\glb#1#2{\sqcup_#1 #2}

\newcommand \GPD {\mathbf{Gpd}}

\newcommand \sem [1] {\llbracket #1 \rrbracket}


\def\textchi{\ensuremath{\chi}}
\def\textbeta{\ensuremath{\beta}}
\DeclareUnicodeCharacter{8657}{\ensuremath{\uparrow}}

\def\coqlibrary#1#2#3{}

\renewenvironment{coqdoccode}{\begin{footnotesize}}{\end{footnotesize}}
\def\coqcode#1{#1}

% \renewcommand{\coqdocemptyline}{\vspace{0.3em}}

\def\kw#1{\mathsf{#1}}
\def\coqdoccst#1{{\color{\coqdoccstcolor}{\textsf{#1}}}}
\def\cst#1{{\color{\coqdoccstcolor}{#1}}}

\def\eqty#1#2#3{\kw{eq}~#1~#2~#3}
\def\matcheqkw{\kw{matcheq}}
\def\matcheq#1#2#3{\matcheqkw\ #1\ \kw{in}\ #2\ \kw{gives}\ #3}
\def\vec#1{\ensuremath{\overrightarrow{#1}}}
\def\nfdelta{{\downarrow^{δ}}}
\def\nfdeltab#1{#1{\downarrow^{δ}}}
\def\hnfbetadelta{\downarrow_{βδ}}
\def\lrule#1{\RefTirName{#1}}

%\renewcommand{\irule}[3]
%{\inferrule[#1] {\scriptstyle#2 }{\scriptstyle#3 }}

\begin{document}

\mainmatter  % start of an individual contribution

% \title{Internalization of the Groupoid Interpretation of Type Theory}
\title{An internalization of a model of type theory in intensional type theory}
\titlerunning {An internalization of a model of type theory in
  intensional type theory}
%A mechanized model of Type Theory based on groupoids

\author{Matthieu Sozeau\inst{1,2} \and Nicolas Tabareau\inst{1,3}}

\date{}

\institute{
$\pi r^2$ and Ascola teams, INRIA \and 
Preuves, Programmes et Systèmes (PPS) \and 
Laboratoire d'Informatique de Nantes Altantique (LINA)  
\\
\email{firstname.surname@inria.fr}
}


\def\mathrm#1{#1}

\maketitle
%  Already,formalizing such weak structures is a difficult endeavor. 

\begin{abstract}

  Homotopical interpretations of Martin-Löf type theory lead toward an
  interpretation of equality as a richer, more extensional
  notion. Extensional or axiomatic presentations of the theory with
  principles based on such models do not yet fully benefit from the
  power of dependent type theory, that is its computational
  character. Reconciling intensional type theory with this richer notion
  of equality requires to move to higher-dimensional structures where
  equality reasoning is explicit, which explains the emerging interest on
  simplicial or cubical models. 
  %
  However, such models are still based on set theory, which is somehow
  in opposition with the goal to replace set theory by homotopy type
  theory in the foundations of mathematics. 
  %
  Therefore, it is important to be able to express those models
  directly in type theory.
  
  In this paper, we follow this idea and pursue the internalization of a
  setoid model of Martin-Löf type theory based on an internalization of
  groupoids.
% and respecting the invariance by isomorphism principle.
  % 
  Our work shows that even a simple (proof-relevant) setoid model of
  type theory does not form a model (at least in the sense of
  traditional categories with families) when internalized inside type
  theory, as the distinction between definitional equality and equality
  in the model forbids some equalities on substitutions to hold without
  explicit rewriting.
  % 
  The observation is already present in the seminal work of Dybjer on
  internal categories with families. 
  %
  We also present a clean statement of the coherence problem that arises
  due to rewriting.
  %
  Our formal development relies crucially on ad-hoc polymorphism to
  overload notions of equality and on a conservative extension of
  Coq's universe mechanism with polymorphism.
  % 

  % Homotopical interpretations of Martin-Löf type theory lead toward an
  % interpretation of equality as a richer, more extensional
  % notion. Extensional or axiomatic presentations of the theory with
  % principles based on such models do not yet fully benefit from the
  % power of dependent type theory, that is its computational
  % character. Reconciling intensional type theory with this richer notion
  % of equality requires to move to higher-dimensional structures where
  % equality reasoning is explicit. In this paper, we follow this idea and
  % develop an internalization of a model of Martin-Löf type theory based
  % on groupoids and respecting the invariance by isomorphism principle.
  % Our formal development relies crucially on ad-hoc polymorphism to
  % overload notions of equality and on a conservative extension of
  % Coq's universe mechanism with polymorphism.
\end{abstract}



\section{Introduction}
\label{sec:introduction}
A notorious difficulty with intensional type theories like Martin-Löf
type theory or the calculus of inductive constructions is the
lack of extensionality principles in the core theory, and notably in
its notion of propositional equality. This makes the theory both
inflexible with respect to the notion of equality one might want to
use on a given type and also departs from the traditional equality
notion from set theory. Functional extensionality (pointwise equal
functions are equal), propositional extensionality (logically
equivalent propositions are equal) and proof irrelevance (all proofs
of the same proposition are equal) are principles that are valid
extensions of type theory and can be internalized in an intensional
theory, in the manner of Altenkirch et al.~\cite{altenkirch-mcbride-wierstra:ott-now}.  Another
extensionality principle coming from homotopy theory is univalence,
whose first instance is the idea that isomorphic types are
equal~\cite{Voevodsky:2011yq,Pelayo:2012uq}. All these principles
should be imposeable on a type theory like MLTT because, intuitively,
no term can distinguish between isomorphic types, pointwise equal
functions, logically equivalent propositions or two proofs of the same
proposition. This hints at the idea that there ought to exist an
internal model of type theory where equality is defined on a type by
type basis using the aforementioned principles and a translation that
witnesses that any term of the theory is equipped with proofs that
they respect these properties. Formalizing this definitional
translation is the goal of this paper.

% However, before even starting to think about formalizing the
% translation, we need to enhance the universe system of \Coq to handle
% polymorphism. As the translation will be given as a set of definitions
% inside the assistant, we will have to instantiate these definitions at
% various levels depending on the source term. With fixed universes for
% each translated construct, we would quickly run into inconsistencies. We
% hence also present the design and implementation of a universe polymorphic
% extension of \Coq, that is conservative over the original system.

The central change in the theory is in the treatment of equality.  Much
interest has been devoted to the study of the identity type of type
theory and models thereof, starting with the groupoid model of Hofmann
and Streicher \cite{groupoid-interp}. This eventually led to the
introduction of homotopy type theory and the study of the ω-groupoid
model of type theory with identity types, which validates extensionality
principles.
% \cite{DBLP:journals/corr/abs-0812-0409}. 
This model in turn
guides work to redesign type theory itself to profit from its
richness, and develop a new theory that internalizes the new
principles. Preliminary attempts have been made, notably by Licata and
Harper \cite{DBLP:conf/popl/LicataH12} who develop a 2-dimensional
version of an hybrid intensional/extensional type theory which
integrates functional extensionality and univalence in the definition
of equality. Work is underway to make it more intensional, and this
starts by making the step to higher dimensions, whether finite (weak
n-groupoids) or infinite (weak ω-groupoids)
\cite{DBLP:conf/csl/AltenkirchR12}. %
%
Our work here concentrates on the internalization in \Coq of Hofmann and
Streicher's groupoid model where we can have a self-contained definition
of the structures involved.
%
Our first motivation to implement this translation is to explore the
interpretation of type theory in groupoids in a completely
intensional setting and in the type theoretic language, leaving no space
for imprecision on the notions of equality and coherence involved.  We
also hope to give with this translation a basic exposition of the
possible type theoretic implications of the groupoid/homotopy models,
bridging a gap in the literature. 
%
On the technical side, we nevertheless have to slightly move away from
Hofmann and Streicher's interpretation.  In particular, it is not
possible to interpret a univalent universe in our type-theoretic
formalization of their model; and this for two important reasons that we
now discuss.


\paragraph{\bf A univalent model that makes no use of univalence.}


Our long term goal is to provide an interpretation of homotopy type
theory into type theory (without extensional principles).
%
This would give a meaning to all extensionality
principles without relying on them in the target theory.

However, if we use a traditional approach and formalize groupoid laws
using the identity type, it turns out that the type of isomorphisms
between two objects $x$ and $y$ of a groupoid, noted $x \sim_1 y$, must
also be formalized using the identity type. Or, putting it in the
homotopy type theoretic language, we have to consider univalent
groupoids.
%
This is because, in the groupoid interpretation, we need isomorphisms
between groupoids to satisfy groupoids laws using the identity
type as the notion of equality between isomorphisms. Then, using
functional extensionality, this amounts to compare the two functions
pointwise, but again using the identity type, and not the internal
notion of equality in the groupoid, given by $\sim_1$.
%
This means that $\sim_1$ has to coincide with the identity type, which
is precisely the property  of being a univalent groupoid.
%
But then, this means that isomorphisms between groupoids should be
reflected in the identity type, which forces the target theory to
satisfy univalence already\ldots
%
So the first conclusion is that an internalization of the groupoid
interpretation in the style of formalization of categories presented
in~\cite{hottbook} would require the target theory to be univalent already.
 
To avoid this issue in our internalization of groupoids, the groupoid
laws are imposed using a notion of 2-dimensional equality that does
not have to be the identity type, just an equivalence relation.
%
Then, to enforce that the types of isomorphisms constitute
(homotopical) sets, we still use the identity type, but only to
express triviality of higher dimensions, not coherences themselves.
%
This interpretation of strictness is closer to the idea that a
groupoid is a weak ω-groupoid for which all equalities at dimension 2
are the same.
%
%
Note that our presentation requires less properties on identity types,
but we still need the axiom of functional extensionality to prove
triviality of higher dimensions for the groupoid of functors.
%
Also, this indicates that if we scale to ω-groupoids, the presence of
identity types in the core type theory will
not be necessary anymore and so the core type theory will be axiom free.
%
Thus, this paper can be seen as a proof of concept that it is possible
to interpret homotopy type theory into type theory without identity
types.


\paragraph{\bf Taking size issue into account.}
%

In Hofmann and Streicher's groupoid model, a type $A$ depending on
context $\Gamma$ is interpreted as a functor 
$$
\sem{A} \ : \ \sem{\Gamma} \longrightarrow \GPD
$$
%
where $\GPD$ is said to be \emph{``the (large) category of groupoids''}. 
%
But groupoids form naturally a 2-category/2-groupoid and not a
category/groupoid, because functors/isomorphisms between (large)
groupoids do not form a set but an arbitrary collection (a type in type
theory). Otherwise said, groupoids are not enriched over themselves.
%
Hofmann and Streicher have solved this issue as we usually do in
category theory, by considering small groupoids, that is groupoids for
which the type of objects is also a set, and relying on set-theoretic
extensional equality to witness natural isomorphisms between groupoid 
isomorphisms. This is fine as long as we consider set theory.
%
But when moving to type theory, small groupoids that form a
category/groupoid actually correspond to setoids. Indeed, to impose the
smallness condition internally, we have to impose either that (i)
the identity type on the type of objects is an h-prop, or that (ii) $x
\sim_1 y$ is an h-prop for all objects $x$ and $y$. But
condition (i) does not give rise to a category: it is not possible to prove that
functors form an h-set because we have an hypothesis on the identity type
where we need an hypothesis on $\sim_1$. 
%
This means that only condition (ii) is valid, so that the
correct notion of small groupoids in type theory is the notion of
setoids.
%
In other words, using an internalization of groupoids, one can only
interpret types as functors into the groupoid of setoids. This is what we do
in this paper, which prevents us from giving a complete formulation of
univalence in the source theory. That is, the invariance by
isomorphism principle is respected in the model but has no counterpart
in the source theory. 

Another way to solve this size issue would be to formalize the notion of
2-groupoids instead but we do not think that the interpretation would
have gained much. We believe that the real challenge is to interpret
$\omega$-groupoids which is the proper notion of self-enriched 
groupoid-like structure and is the subject of on-going work.

% coherences even limited at
% groupoids, the the structures are already quite tedious to manipulate and we
% found some interesting conditions on the structure of $Π$ and $Σ$ types
% that we believe were never presented in this form before. In the
% development, we strived to be as generic as possible and use the
% abstract structures of category theory to not be essentially tied to a
% particular dimension. That level of genericity also relies on the
% extension of the system with universe polymorphism.

% This translation could also be used to complete a forcing
% translation of type theory into type theory~\cite{hal-00685150}. By
% building up forcing layers on top of a core type theory, one can
% introduce new logical operators or type constructors (e.g. modal logic,
% recursive types), defined by translation. For correctness, this
% translation relies on a type theory that integrates proof-irrelevance
% and functional extensionality. The present translation gives the
% expressive power needed to compose with the forcing translation and get
% a fully definitional extension of type theory with forcing.

% To summarize, our contributions are: 
% \begin{enumerate}
% \item A theory of groupoids compatible with the ω-groupoids approach
%   and a partially mechanized interpretation of type theory into groupoids.
% \item A type-theoretic description of the necessary conditions on the
%   interpretation of function types, including dependent product types
%   which give rise to dependent groupoid functors and dependent sums.
% \item A proof of concept that it is possible to interpret homotopy
%   type theory into type theory without identity types.
% \end{enumerate}

\paragraph{\bf Outline of the paper.}
% The paper is organized as follows: 
Section \ref{sec:setting-translation} introduces the source type
theory of the translation and some features of the proof assistant
that are used in the formalization. The formal model includes a
formalization of groupoids and associated structures
(\S\ref{sec:w2gpds}-\ref{sec:rew}) and a construction of the groupoids
interpreting the standard type constructors
(\S\ref{sec:depprod}-\ref{sec:sigma}). Section
\ref{sec:interpretation} presents the model proper.
 % and conclude (Section \ref{sec:conclusion}).
%

%%% Local Variables: 
%%% mode: latex
%%% TeX-master: "main"
%%% End: 

\section{Setting of the translation}
\label{sec:setting-translation}

\def\Elt#1{\texttt{Elt}(#1)}
\def\Univ{\ensuremath{\mathcal{U}}}
\def\Id#1#2#3{\texttt{Id}_{#1}\,#2\ #3}
\def\Equiv#1#2{\texttt{Equiv}\ \Elt{#1}\ \Elt{#2}}
\def\Eq#1#2#3{\texttt{Eq}_{#1}\ #2\ #3}
\def\refl#1#2{\texttt{refl}_{#1}\ #2}
\def\funext#1{\texttt{fun\_ext}(#1)}
\def\zeroType{\ensuremath{\mathbb{O}}\xspace}
\def\oneType{\ensuremath{\mathbb{1}}\xspace}
\def\twoType{\ensuremath{\mathbb{2}}\xspace}
\def\hzeroType{\ensuremath{\mathbb{O}}}
\def\honeType{\ensuremath{\mathbb{1}}}
\def\htwoType{\ensuremath{\mathbb{2}}}

\subsection{Martin-Löf Type Theory with Functional Extensionality}
\label{sec:definitions}


For the purpose of this paper we study a restricted source theory
resembling a cut-down version of the core language of the \Coq system,
with only one \Type{} universe (see \cite{DBLP:bibsonomy_cupart} for an
in-depth study of this system). This is basically Martin-Löf Type Theory
(without \Type{} : \Type{}). First we introduce the definitions of the
various objects at hand.  We start with the term language of a dependent
λ-calculus: we have a countable set of variables $x, y, z$, and the
usual typed λ-abstraction $λ x : τ, b$, à la Church, application $t~u$,
the dependent product and sum types $Π/Σ x : A. B$, and an identity type
$\Id{T}{t}{u}$. 
% There is a single universe $\Univ$. For $T$ and $U$ in
% $\Univ{}$, the type of (Set)-isomorphisms $T `= U$, is definable
% directly using the other type constructors (see \ref{sec:universe}).

The typing judgment for this calculus is written $\tcheck{Γ}{ψ}{t}{T}$
(Figure \ref{fig:emltt}) where $Γ$ is a context of declarations $Γ ::=
\Nil `| Γ, x : τ$, $t$ and $T$ are terms. If we have a valid derivation
of this judgment, we say $t$ has type $T$ in $Γ$.
%  and assume that $T$ is
% not $\Type{}$ unless explicilty stated (this just allows us to do
% without a separate judgment for carving out the types).

Most of the rules are standard. The definitional equality $A = B$ is
defined as the congruence on type and term formers compatible with
β-reductions for abstractions and projections.

\subsubsection{Identity type.} 
%
The identity type in the source theory is the standard
Martin-Löf identity type $\Id{T}{t}{u}$, generated from an
introduction rule for reflexivity with the usual \texttt{J} eliminator
and its \emph{propositional} reduction rule. The \texttt{J} reduction 
rule will actually be valid definitionally in the model for \emph{closed} terms.

\subsubsection{Functional Extensionality.} 
%
The proof-relevant equality of functions in the source theory for which we want to
give a computational model appears in rule \lrule{Fun-Ext},
extending the formation rules of identity types on dependent product. 
%
Equality of dependent functions $f$ and $g$ of type $\Pi \vdecl{x}{A}\mdot B$ are
introduced by giving a witness of $\Pi t : A, \Id{B \{\subs x {t}\}}
{(f \ t)}{(g \ t)}$.
%
The \texttt{J} rule for dependent functions witnesses the
\emph{naturality} of every predicate constructed in the source type
theory.
%
This rule corresponds 
  to the introduction of equality on dependent functions in %\cite{DBLP:conf/popl/LicataH12}%.

\def\hFin#1{\mathtt{\hat Fin}\ #1}
\def\Fin#1{\texttt{Fin}\ #1}
\def\fin#1#2{\underline{#1}_{#2}}

% \subsubsection{Universe.} The universe $\Univ$ is closed under Σ, Π,
% \zeroType, \oneType, \twoType and $\texttt{Id}$ in elements of $\Univ$, 
% \emph{not} type equivalences. The
% constructors are circumflexed, e.g. $\hat \Pi$ and $\hat \Sigma$ are
% introduced with the rules:
% %
% %\vspace{-1em} 
% \begin{mathpar}
% \irule{Univ-$\hat \Pi$, -$\hat \Sigma$}
% {\tcheck{\Gamma}{ψ}{A}{\Univ} \\
% \tcheck{\Gamma, x : \Elt{A}}{ψ}{B}{\Univ}}
% {\tcheck{\Gamma}{ψ}{\hat \Pi/\hat \Sigma x : A. B}{\Univ}}
% \end{mathpar}

% \noindent 
% For presentation purposes, we do not detail here the treatment of finite
% types (see \cite{altenkirch-mcbride-wierstra:ott-now} for a standard treatment).
% The extension to W-types would be straightforward.
% %
% \texttt{Elt} is a $\type{}$-forming map from $\Univ$ that acts as a
% homomorphism, e.g. its action on products is: $$\Elt{\hat \Pi x : A. B} = Π x : \Elt{A}. \Elt{B}$$
% \begin{mathpar}
% \begin{array}{lcll}
%   \texttt{Elt} : \Univ & → & \Type{} & \\
%   \Elt{\{\hat \Pi/\hat \Sigma\} x : A. B} & = & \{Π/Σ\} x :
%   \Elt{A}. \Elt{B} & \\
%   \Elt{\hat{τ}} & = & τ & {τ \in \{ \hzeroType, \honeType, \htwoType \}} \\
%   % \Elt{\hFin{n}} & = & \Fin{n} & \\
% %  \Elt{a =_{τ} b} & = & \texttt{Id}_{\Elt{τ}}\ a\ b & \\
%   \Elt{\hat{C}[X]} & = & C[\Elt{X}] & \text{(homomorphism)}
% \end{array}
% \end{mathpar}


\begin{figure}[t]
% \hspace{-0.0\textwidth}
% \begin{minipage}{1.0\textwidth}
\begin{mathpar}

\irule{Empty}{}{\WFc{\Nil}{ψ}}

\irule{Decl}
{\ttcheck{\Gamma}{ψ}{T}{\Type{}} \hspace{-1em}\\
 %T \neq \Univ, %\hspace{-1em}\\
 x \not \in \Gamma }
{\WFc{Γ, \vdecl{x}{T}}{ψ}}

\irule{Univ}
{}
{\ttcheck{\Gamma}{ψ}{\Univ}{\Type{}}}

\irule{Var}
{\WFc{\Gamma}{ψ} \hspace{-1em}\\ 
  (\vdecl{x}{T}) \in \Gamma}
{\tcheck{\Gamma}{ψ}{x}{T}}


\irule{Prod/Sigma}
{\ttcheck{\Gamma, \vdecl{x}{A}}{ψ}{B}{\Type{}} \hspace{-1em}}
{\ttcheck{\Gamma}{ψ}{\Pi/\Sigma \vdecl{x}{A}\mdot B}{\Type{}}}

\irule{Pair}
{%\ttcheck{\Gamma}{ψ}{Σ x : A. B}{\Type{}} \\
\tcheck{\Gamma}{ψ}{t}{A} \hspace{-1em} \\ \tcheck{\Gamma}{ψ}{u}{B\{\subs
    x t}\}}
{\tcheck{\Gamma}{ψ}{(t,u)_{x : A. B}}{Σ x : A. B}}

\irule{Proj1}
{\tcheck{\Gamma}{ψ}{t}{\Sigma \vdecl{x}{A}\mdot B}}
{\tcheck{\Gamma}{ψ}{\ensuremath{π_1 t}}{A}}

\irule{Proj2}
{\tcheck{\Gamma}{ψ}{t}{\Sigma \vdecl{x}{A}\mdot B}}
{\tcheck{\Gamma}{ψ}{π_2 t}{B\{\subs x {π_1 t}\}}}
% \irule{Fin}
% {n \in \{ 0, 1, 2 \}}
% {\tcheck{\Gamma}{ψ}{\Fin{n}}{\Type{}}}
% \irule{Fin-Intro}
% {k, n \in \{ 0, 1, 2 \}, k < n}
% {\tcheck{\Gamma}{ψ}{\fin{k}{n}}{\Fin{n}}}
% \irule{Fin-Elim}
% {n \in \{ 0, 1, 2 \} \\
%   \tcheck{\Gamma, x : \Fin{n}}{ψ}{P}{\Type{}} \\
%   ∀ k < n. \tcheck{\Gamma}{ψ}{p_k}{P\ \fin{k}{n}} \\
%   \tcheck{\Gamma}{ψ}{f}{\Fin{n}}}
% {\tcheck{\Gamma}{ψ}{\texttt{felim}_{P}\ \overrightarrow{p_k}\ f}{P\
% f}}

\irule{Conv}
{\tcheck{\Gamma}{ψ}{t}{A} \hspace{-1em} \\ \ttcheck{Γ}{ψ}{B}{\Type{}} \hspace{-1em} \\ \tconv{ψ}{A}{B}}
{\tcheck{\Gamma}{ψ}{t}{B}}


\irule{Lam}
{\tcheck{\Gamma, \vdecl{x}{A}}{ψ}{t}{B}}
{\tcheck{\Gamma}{ψ}{\lambda \vdecl{x}{A}\mdot t}{\Pi \vdecl{x}{A}\mdot B}}

\irule{App}
{\tcheck{\Gamma}{ψ}{t}{\Pi \vdecl{x}{A}\mdot B} \hspace{-1em} \\
 \tcheck{\Gamma}{ψ}{t'}{A}}
{\tcheck{\Gamma}{ψ}{t\ t'}{B\{\subs x {t'}\}}}

\irule{Id}
{\ttcheck{\Gamma}{ψ}{T}{\Type{}}\quad
\tcheck{\Gamma}{ψ}{A, B}{T}}
{\ttcheck{\Gamma}{ψ}{\Id{T}{A}{B}}{\Type{}}}
\qquad
\irule{Id-Intro}
{\tcheck{\Gamma}{ψ}{t}{T}}
{\tcheck{\Gamma}{ψ}{\refl{T}{t}}{\Id{T}{t}{t}} }
\qquad
\irule{Fun-Ext}
{\tcheck{\Gamma}{ψ}{e}{ \Pi t : A, \Id{B \{\subs x {t}\}} {(f \ t)}{(g \ t)}}}
{\tcheck{\Gamma}{ψ}{\funext{e}}{\Id{\Pi \vdecl{x}{A}\mdot B}{f}{g} }}


% \irule{Equiv-Intro}
% {\tcheck{\Gamma}{ψ}{i}{ \Elt{A} ` = \Elt{B}}}
% {\tcheck{\Gamma}{ψ}{\texttt{equiv}\ i}{\Id{\Univ}{A}{B}}}


\irule{Id-Elim (J)}
{\tcheck{\Gamma}{ψ}{i}{\Id{T}{t}{u}} \\
\ttcheck{\Gamma, x : T, e : \Id{T}{t}{x}}{ψ}{P}{\Type{}} \\
\tcheck{\Gamma}{ψ}{p}{P\{\subs x t\}\{\refl{T}{t}/e\}}}
{\tcheck{\Gamma}{ψ}{\texttt{J}_{λx\ e. P}~i~p}{P\{\subs x u, \subs e i\}}}
\end{mathpar}
\caption{Typing judgments for our extended MLTT}\label{fig:emltt}
\end{figure}



% We abuse notations
% and consider Π, Σ and \texttt{Id} as codes when seen as inhabitants of
% \Univ, and as regular syntax for the type constructors in the rest of
% the type theory.

% P[t,refl] -> P[t,equiv i]: P is abstracted by x : U and Eq t x, can't look
% inside the universe x, but p : P[t,refl] might use J q refl =
% q. Replace by J q (equiv i) = q[i.1 y/y]. 

%We write $\tcheck{Γ}{}{T}{s}$
%as a shorthand for $\tcheck{Γ}{}{T}{\Type{u}}$ for some universe $u$.
% \begin{figure}
% \begin{mathpar}

% % \irule{Univ-Id}
% % {\tcheck{\Gamma}{ψ}{A}{\Univ} \\
% % \tcheck{\Gamma}{ψ}{a, b}{\Elt{A}}}
% % {\tcheck{\Gamma}{ψ}{a =_A b}{\Univ}}

% \irule{Univ-Fin}
% {τ \in \{ \hzeroType, \honeType, \htwoType \}}
% {\tcheck{\Gamma}{ψ}{\hat{τ}}{\Univ}}

% \irule{Univ-$\hat \Pi$, -$\hat \Sigma$}
% {\tcheck{\Gamma}{ψ}{A}{\Univ} \\
% \tcheck{\Gamma, x : \Elt{A}}{ψ}{B}{\Univ}}
% {\tcheck{\Gamma}{ψ}{\hat \Pi/\hat \Sigma x : A. B}{\Univ}}
% \end{mathpar}
% \caption{Definition of \Univ{} (inductive-recursive with \Elt{\_})}\label{fig:univ}
% \end{figure}

% \begin{figure}[!h]
% \begin{mathpar}

% \begin{array}{lcll}
%   \texttt{Elt} : \Univ & → & \Type{} & \\
%   \Elt{\{\hat \Pi/\hat \Sigma\} x : A. B} & = & \{Π/Σ\} x :
%   \Elt{A}. \Elt{B} & \\
%   \Elt{\hat{τ}} & = & τ & {τ \in \{ \hzeroType, \honeType, \htwoType \}} \\
%   % \Elt{\hFin{n}} & = & \Fin{n} & \\
% %  \Elt{a =_{τ} b} & = & \texttt{Id}_{\Elt{τ}}\ a\ b & \\
%   \Elt{\hat{C}[X]} & = & C[\Elt{X}] & \text{(homomorphism)}
% \end{array}

% \begin{array}{lcll}
%   \texttt{Eq}_{U : \Univ} : \Elt{U} → \Elt{U} & → & \Type{} & \\
%   \Eq{\hat \Pi x : A. B}{f}{g} & = & Π x : \Elt{A}. \Elt{f\ x =_B g\ x} \\
%   \Eq{\hat \Sigma x : A. B}{t}{u} & = & %Π t u : (Σ x : \Elt{A}. \Elt{B}), 
%   Π e : \Elt{π_1\ t =_A π_1\ u}, π_2\ t =_{B\ (π_1\ t)} \texttt{J}_{λx. \Elt{B x}}\
%   e\ (π_2\ u)
%  % \Eq{a =_{τ} b}{e}{e'} & = & \mathbf{1} &
% \end{array}

% \end{mathpar}
% \caption{Universe decoding}\label{fig:univelt}
% \end{figure}




\subsection{The proof assistant}
\label{sec:proof-assistant}
We use the latest version (8.5) of the \Coq proof assistant to formally define
our translation\footnote{At the time of writing, a beta version is
  available}. Vanilla features of \Coq allow us to define overloaded
notations and hierarchies of structures through type classes
\cite{sozeau.Coq/classes/fctc}, and to separate definitions and proofs
using the \Program extension \cite{sozeau.Coq/FingerTrees/article}, they
are both documented in \Coq's reference manual \cite{coq:refman:8.5}.
We also use the recent extension to polymorphic
universes~\cite{DBLP-conf/itp/SozeauT14short}.


\subsubsection{Classes and projections.}
The formalization makes heavy use of type classes and sigma types, both
defined internally as parameterized records. We also use the new
representation of record projections, making them primitive to allow a
more economical representation, leaving out the parameters of the record
type they are applied to. This change, which is justified by
bidirectional presentations of type theory, makes typechecking
exponentially faster in the case of nested structures (see
\cite{garillot:pastel-00649586} for a detailed explanation of this phenomenon).



% not including In the usual Calculus of
% Inductive Constructions, records are encoded as non-recursive inductive
% types with a single constructor and their projections are defined by
% pattern-matching. While this representation is adequate, it is
% inefficient in terms of space and time consumption as projection
% $\coqdocvar{p}$ from a parameterized record
% $\coqdocind{R}~(\coqdocvar{A} : \Type{}) := \{ \coqdoccst{p} :
% \coqdocvar{A} \}$ have type $Π A : \Type{},
% \coqdocind{R}~\coqdocvar{A} → \coqdocvar{A}$. Every application of a
% projection hence repeats the parameters of the record, which in the
% nested case results in a exponential blowup in the size of terms,
% and typechecking likewise repeats work unnecessarily . We
% modified the representation of record projections to skip parameters,
% which is also justified by bidirectional presentations of type theory,
% to recover a workable formalization.

One peculiarity of \Coq's class system we use is the ability to nest
classes. We use the \coqdoccst{A\_of\_B} \coqcode{:>} \coqdocind{A} notation in a type class
definition \coqdockw{Class} \coqdocind{B} as an abbreviation for
defining \coqdoccst{A\_of\_B} as an instance of \coqdocind{A}.

\subsubsection{Polymorphic Universes.}
\label{sec:polym-univ}
\def\Types{\coqdockw{Type}s\xspace}

  % Type theories such as the Calculus of Inductive Constructions maintain
  % a universe hierarchy to prevent paradoxes that naturally appear if one
  % is not careful about the sizes of types that are manipulated in the
  % language. To ensure consistency while not troubling the user with 
  % this necessary information, systems using typical ambiguity were
  % designed. We present an elaboration from terms using typical ambiguity 
  % into explicit terms which also accomodates universe polymorphism, i.e.
  % the ability to write a term once and use it at different universe
  % levels. Elaboration relies on an enhanced type inference algorithm to 
  % provide the freedom of typical ambiguity while also supporting
  % polymorphism, in a fashion similar to usual Hindley-Milner polymorphic
  % type inference. This elaboration is implemented as a drop-in
  % replacement for the existing universe system of Coq and has been
  % benchmarked favorably against the previous version. We demonstrate how
  % it provides a solution to a number of formalization issues present in
  % the original system.

To typecheck our formalization, we also need an expressive universe
system. Indeed, if we are to give a uniform (shallow) translation of
type theory in type theory, we have to define a translation of the
\type{} universe (a groupoid) as a term of the calculus and equip type
constructors like $Π$ and $Σ$ with \interp{\type{}} structures as
well. As \interp{\type{}} itself contains a \Type{}, the following
situation occurs when we define the translation of, e.g. sums: we should
have \interp{Σ~U~T\ \type{}} = \interp{Σ} \interp{U} \interp{T} :
\interp{\type}. To ensure consistency of the interpretations of types
inside \interp{U}, \interp{T} and the outer one, they must be at
different levels, with the outer one at least as large as the inner
ones. The universe polymorphic extension of \Coq has been designed to
allow such highly generic developments~\cite{DBLP-conf/itp/SozeauT14short}.
The design was implemented by the first author and is already used to
check the \name{HoTT} library in \Coq \cite{HoTT/HoTT}.

% This is however not supported in the current
% version of \Coq, as the universe system does not allow a definition to
% live at different levels. Hence, there could be only one universe level
% assigned to the translation of any \Type{} and they couldn't be nested,
% as this would result in an obvious inconsistency: the usual \Type{} :
% \Type{} inconsistency would show up as \interp{\Type{}} :
% \interp{\Type{}}. One solution to this problem would be to have $n$
% different interpretations of \Type{} to handle $n$ different levels of
% universes.  This is clearly unsatisfactory, as this would mean
% duplicating every lemma and every structure depending on the translation
% of types, and the numer of duplications would depend on the use of
% universes at the source level. Instead, we can extend the system with
% universe polymorphic definitions that are parametric on universe levels
% and instantiate them at different ones, just like parametric
% polymorphism is used to instantiate a definition at different
% types. This can be interpreted as building fresh instances of the
% constant that can be handled by the core type checker without
% polymorphism.  We give a detailed presentation of this system in 
% Section \ref{sec:type-theory-with}, but as the handling of universes stays
% entirely implicit at the source level, we just present here the core
% calculus on which the universe polymorphism system and our
% interpretation rests.




\section{Formalization of groupoids}
\label{sec:formalization}

\coqlibrary{Groupoid.groupoid}{Library }{Groupoid.groupoid}

\begin{coqdoccode}
\end{coqdoccode}
  This section presents our formalization of groupoids in \Coq with
  universe polymorphism. 
  We first explain our overloaded management of
  equalities and introduce type classes for groupoids and their
  associated structures, i.e., functors, natural transformations and
  homotopy equivalences
  (\S\ref{sec:w2gpds}-\ref{sec:homequiv}).
  Natural transformations give access to a homotopic form of
  functional extensionality, while homotopy equivalences provide
  extensionality at the level of 0-types.  Polymorphic universes are
  needed to state that setoids and homotopy equivalences form a
  groupoid.  Homotopic equivalences directly provide access to a
  rewriting mechanism on types (\S\ref{sec:rew}). This rewriting is
  used to extend functors and products to dependent functors and
  dependent sums (\S\ref{sec:depprod}-\ref{sec:sigma}).
 \subsection{Notations} \coqlibrary{Groupoid.notations}{Library }{Groupoid.notations}

\begin{coqdoccode}
\end{coqdoccode}
We use the following notations throughout: Sigma type introduction
  is written (\coqdocvar{t} ; \coqdocvar{p}) when its predicate/fibration is inferrable from
  the context, and projections are denoted \coqdocabbreviation{$\pi_1$} and \coqdocabbreviation{$\pi_2$}. The bracket
  notation [\coqdocvar{\_}] is an alias for \coqdocabbreviation{$\pi_1$}. The following is directly
  extracted from Coq files using the \texttt{coqdoc} tool (source files
  are available at \url{http://mattam82.github.io/groupoid}). If
  you are reading the colored version of the paper, keywords are
  typeset in red, inductive types and classes in blue, inductive
  constructors in dark red, and defined constants and lemmas in green.
  \begin{coqdoccode}
\end{coqdoccode}



\subsection{Definition of groupoids \label{sec:w2gpds}}




We formalize groupoids using type classes.  Contrarily to what is done
in the setoid translation, the basic notion of a morphism is an
inhabitant of a relation in \coqdockw{Type} (i.e., a proof-relevant relation): \begin{coqdoccode}
\coqdocemptyline
\coqdocnoindent
\coqdockw{Definition} \coqdef{Groupoid.groupoid.HomT}{$\mathsf{HomSet}$}{\coqdocdefinition{$\mathsf{HomSet}$}} (\coqdocvar{T} : \coqdockw{Type}) := \coqdocvariable{T} \coqexternalref{:type scope:x '->' x}{http://coq.inria.fr/stdlib/Coq.Init.Logic}{\coqdocnotation{\ensuremath{\rightarrow}}} \coqdocvariable{T} \coqexternalref{:type scope:x '->' x}{http://coq.inria.fr/stdlib/Coq.Init.Logic}{\coqdocnotation{\ensuremath{\rightarrow}}} \coqdockw{Type}.\coqdoceol
\coqdocemptyline
\end{coqdoccode}
  \noindent
  Given \coqdocvariable{x} and \coqdocvariable{y} in \coqdocvariable{T}, \coqref{Groupoid.groupoid.HomT}{\coqdocdefinition{$\mathsf{HomSet}$}} \coqdocvariable{T} \coqdocvariable{x} \coqdocvariable{y} is the type of morphism from \coqdocvariable{x} to \coqdocvariable{y}. 
  To manipulate different \coqref{Groupoid.groupoid.HomT}{\coqdocdefinition{$\mathsf{HomSet}$}}'s at dimension 1 and 2 more abstractly, we use ad-hoc 
  polymorphism and introduce type classes \coqref{Groupoid.groupoid.HomT1}{\coqdocrecord{$\mathsf{HomSet}_1$}} and \coqref{Groupoid.groupoid.HomT2}{\coqdocrecord{$\mathsf{HomSet}_2$}} with according notations. 
\begin{coqdoccode}
\coqdocemptyline
\coqdocnoindent
\coqdockw{Class} \coqdef{Groupoid.groupoid.HomT1}{$\mathsf{HomSet}_1$}{\coqdocrecord{$\mathsf{HomSet}_1$}} \coqdocvar{T} := \{\coqdef{Groupoid.groupoid.eq1}{$\coqdoccst{eq}_1$}{\coqdocprojection{$\coqdoccst{eq}_1$}} : \coqref{Groupoid.groupoid.HomT}{\coqdocdefinition{$\mathsf{HomSet}$}} \coqdocvariable{T}\}.\coqdoceol
\coqdocnoindent
\coqdockw{Infix} \coqdef{Groupoid.groupoid.::x 'x7E1' x}{"}{"}$\sim_1$" := \coqref{Groupoid.groupoid.eq1}{\coqdocprojection{$\coqdoccst{eq}_1$}} (\coqdoctac{at} \coqdockw{level} 80).\coqdoceol
\coqdocemptyline
\coqdocnoindent
\coqdockw{Class} \coqdef{Groupoid.groupoid.HomT2}{$\mathsf{HomSet}_2$}{\coqdocrecord{$\mathsf{HomSet}_2$}} \{\coqdocvar{T}\} (\coqdocvar{Hom} : \coqref{Groupoid.groupoid.HomT}{\coqdocdefinition{$\mathsf{HomSet}$}} \coqdocvariable{T}) := \{\coqdef{Groupoid.groupoid.eq2}{$\coqdoccst{eq}_2$}{\coqdocprojection{$\coqdoccst{eq}_2$}} : \coqexternalref{:type scope:'xE2x88x80' x '..' x ',' x}{http://coq.inria.fr/stdlib/Coq.Unicode.Utf8\_core}{\coqdocnotation{∀}} \coqexternalref{:type scope:'xE2x88x80' x '..' x ',' x}{http://coq.inria.fr/stdlib/Coq.Unicode.Utf8\_core}{\coqdocnotation{\{}}\coqdocvar{x} \coqdocvar{y} : \coqdocvariable{T}\coqexternalref{:type scope:'xE2x88x80' x '..' x ',' x}{http://coq.inria.fr/stdlib/Coq.Unicode.Utf8\_core}{\coqdocnotation{\},}} \coqref{Groupoid.groupoid.HomT}{\coqdocdefinition{$\mathsf{HomSet}$}} (\coqdocvariable{Hom} \coqdocvariable{x} \coqdocvariable{y})\}.\coqdoceol
\coqdocnoindent
\coqdockw{Infix} \coqdef{Groupoid.groupoid.::x 'x7E2' x}{"}{"}$\sim_2$" := \coqref{Groupoid.groupoid.eq2}{\coqdocprojection{$\coqdoccst{eq}_2$}} (\coqdoctac{at} \coqdockw{level} 80).\coqdoceol
\coqdocemptyline
\coqdocemptyline
\end{coqdoccode}
Given a \coqref{Groupoid.groupoid.HomT}{\coqdocdefinition{$\mathsf{HomSet}$}}, we define type classes: \coqref{Groupoid.groupoid.Identity}{\coqdocrecord{Identity}} that gives the
  identity morphism, \coqref{Groupoid.groupoid.Inverse}{\coqdocrecord{Inverse}} which corresponds to the existence of an
  inverse morphism for every morphism (noted \coqdocvariable{f} $\hspace{-1ex}^{-1}$) and \coqref{Groupoid.groupoid.Composition}{\coqdocrecord{Composition}}
  which corresponds to morphism composition (noted \coqdocvariable{g} $\circ$ \coqdocvariable{f}). Those three
  properties are gathered by the type class \coqref{Groupoid.groupoid.Equivalence}{\coqdocrecord{Equivalence}}.\begin{coqdoccode}
\end{coqdoccode}
  A \coqref{Groupoid.groupoid.CategoryP}{\coqdocrecord{$\mathsf{PreCategory}$}} is defined as a category
  where coherences are given up-to an equivalence relation denoted by
  $\sim_2$.  Ordinary categories are derived with the additional requirement
  that higher equalities are trivial, which can be expressed using
  identity types (see the definition of \coqref{Groupoid.groupoid.Groupoid}{\coqdocrecord{$\mathsf{IsType_1}$}}).  


  We do not put this condition into the basic definition because
categories and functors form a pre-category but not a 1-category. Thus,
working with pre-categories and pre-groupoids allows to share more
structure and is closer to the ω-groupoid model which is itself enriched.
\begin{coqdoccode}
\coqdocemptyline
\coqdocnoindent
\coqdockw{Class} \coqdef{Groupoid.groupoid.CategoryP}{$\mathsf{PreCategory}$}{\coqdocrecord{$\mathsf{PreCategory}$}} \coqdocvar{T} := \{ \coqdef{Groupoid.groupoid.Hom1}{$\mathsf{Hom}_1$}{\coqdocprojection{$\mathsf{Hom}_1$}} :> \coqref{Groupoid.groupoid.HomT1}{\coqdocclass{$\mathsf{HomSet}_1$}} \coqdocvariable{T}; \coqdef{Groupoid.groupoid.Hom2}{$\mathsf{Hom}_2$}{\coqdocprojection{$\mathsf{Hom}_2$}} :> \coqref{Groupoid.groupoid.HomT2}{\coqdocclass{$\mathsf{HomSet}_2$}} \coqref{Groupoid.groupoid.eq1}{\coqdocprojection{$\coqdoccst{eq}_1$}};\coqdoceol
\coqdocindent{2.50em}
\coqdef{Groupoid.groupoid.Id}{Id}{\coqdocprojection{Id}} :> \coqref{Groupoid.groupoid.Identity}{\coqdocclass{Identity}} \coqref{Groupoid.groupoid.eq1}{\coqdocprojection{$\coqdoccst{eq}_1$}}; \coqdef{Groupoid.groupoid.Comp}{Comp}{\coqdocprojection{Comp}} :> \coqref{Groupoid.groupoid.Composition}{\coqdocclass{Composition}} \coqref{Groupoid.groupoid.eq1}{\coqdocprojection{$\coqdoccst{eq}_1$}};\coqdoceol
\coqdocindent{2.50em}
\coqdef{Groupoid.groupoid.Equivalence 2}{$\mathsf{Equivalence}_2$}{\coqdocprojection{$\mathsf{Equivalence}_2$}} :> \coqexternalref{:type scope:'xE2x88x80' x '..' x ',' x}{http://coq.inria.fr/stdlib/Coq.Unicode.Utf8\_core}{\coqdocnotation{∀}} \coqdocvar{x} \coqdocvar{y}\coqexternalref{:type scope:'xE2x88x80' x '..' x ',' x}{http://coq.inria.fr/stdlib/Coq.Unicode.Utf8\_core}{\coqdocnotation{,}} \coqexternalref{:type scope:'xE2x88x80' x '..' x ',' x}{http://coq.inria.fr/stdlib/Coq.Unicode.Utf8\_core}{\coqdocnotation{(}}\coqref{Groupoid.groupoid.Equivalence}{\coqdocclass{Equivalence}} (\coqref{Groupoid.groupoid.eq2}{\coqdocprojection{$\coqdoccst{eq}_2$}} (\coqdocvar{x}:=\coqdocvariable{x}) (\coqdocvar{y}:=\coqdocvariable{y}))\coqexternalref{:type scope:'xE2x88x80' x '..' x ',' x}{http://coq.inria.fr/stdlib/Coq.Unicode.Utf8\_core}{\coqdocnotation{)}};\coqdoceol
\coqdocindent{2.50em}
\coqdef{Groupoid.groupoid.id R}{$\coqdoccst{id}_R$}{\coqdocprojection{$\coqdoccst{id}_R$}} : \coqexternalref{:type scope:'xE2x88x80' x '..' x ',' x}{http://coq.inria.fr/stdlib/Coq.Unicode.Utf8\_core}{\coqdocnotation{∀}} \coqdocvar{x} \coqdocvar{y} (\coqdocvar{f} : \coqdocvariable{x} \coqref{Groupoid.groupoid.::x 'x7E1' x}{\coqdocnotation{$\sim_1$}} \coqdocvariable{y}\coqexternalref{:type scope:'xE2x88x80' x '..' x ',' x}{http://coq.inria.fr/stdlib/Coq.Unicode.Utf8\_core}{\coqdocnotation{),}} \coqdocvariable{f} \coqref{Groupoid.groupoid.::x 'xC2xB0' x}{\coqdocnotation{$\circ$}} \coqref{Groupoid.groupoid.identity}{\coqdocprojection{identity}} \coqdocvariable{x} \coqref{Groupoid.groupoid.::x 'x7E' x}{\coqdocnotation{$\sim_2$}} \coqdocvariable{f} ;\coqdoceol
\coqdocindent{2.50em}
\coqdef{Groupoid.groupoid.id L}{$\coqdoccst{id}_L$}{\coqdocprojection{$\coqdoccst{id}_L$}} : \coqexternalref{:type scope:'xE2x88x80' x '..' x ',' x}{http://coq.inria.fr/stdlib/Coq.Unicode.Utf8\_core}{\coqdocnotation{∀}} \coqdocvar{x} \coqdocvar{y} (\coqdocvar{f} : \coqdocvariable{x} \coqref{Groupoid.groupoid.::x 'x7E1' x}{\coqdocnotation{$\sim_1$}} \coqdocvariable{y}\coqexternalref{:type scope:'xE2x88x80' x '..' x ',' x}{http://coq.inria.fr/stdlib/Coq.Unicode.Utf8\_core}{\coqdocnotation{),}} \coqref{Groupoid.groupoid.identity}{\coqdocprojection{identity}} \coqdocvariable{y} \coqref{Groupoid.groupoid.::x 'xC2xB0' x}{\coqdocnotation{$\circ$}} \coqdocvariable{f} \coqref{Groupoid.groupoid.::x 'x7E' x}{\coqdocnotation{$\sim_2$}} \coqdocvariable{f} ;\coqdoceol
\coqdocindent{2.50em}
\coqdef{Groupoid.groupoid.assoc}{assoc}{\coqdocprojection{assoc}} : \coqexternalref{:type scope:'xE2x88x80' x '..' x ',' x}{http://coq.inria.fr/stdlib/Coq.Unicode.Utf8\_core}{\coqdocnotation{∀}} \coqdocvar{x} \coqdocvar{y} \coqdocvar{z} \coqdocvar{w} (\coqdocvar{f}: \coqdocvariable{x} \coqref{Groupoid.groupoid.::x 'x7E1' x}{\coqdocnotation{$\sim_1$}} \coqdocvariable{y}) (\coqdocvar{g}: \coqdocvariable{y} \coqref{Groupoid.groupoid.::x 'x7E1' x}{\coqdocnotation{$\sim_1$}} \coqdocvariable{z}) (\coqdocvar{h}: \coqdocvariable{z} \coqref{Groupoid.groupoid.::x 'x7E1' x}{\coqdocnotation{$\sim_1$}} \coqdocvariable{w}\coqexternalref{:type scope:'xE2x88x80' x '..' x ',' x}{http://coq.inria.fr/stdlib/Coq.Unicode.Utf8\_core}{\coqdocnotation{),}}\coqdoceol
\coqdocindent{7.00em}
\coqref{Groupoid.groupoid.::x 'xC2xB0' x}{\coqdocnotation{(}}\coqdocvariable{h} \coqref{Groupoid.groupoid.::x 'xC2xB0' x}{\coqdocnotation{$\circ$}} \coqdocvariable{g}\coqref{Groupoid.groupoid.::x 'xC2xB0' x}{\coqdocnotation{)}} \coqref{Groupoid.groupoid.::x 'xC2xB0' x}{\coqdocnotation{$\circ$}} \coqdocvariable{f} \coqref{Groupoid.groupoid.::x 'x7E' x}{\coqdocnotation{$\sim_2$}} \coqdocvariable{h} \coqref{Groupoid.groupoid.::x 'xC2xB0' x}{\coqdocnotation{$\circ$}} \coqref{Groupoid.groupoid.::x 'xC2xB0' x}{\coqdocnotation{(}}\coqdocvariable{g} \coqref{Groupoid.groupoid.::x 'xC2xB0' x}{\coqdocnotation{$\circ$}} \coqdocvariable{f}\coqref{Groupoid.groupoid.::x 'xC2xB0' x}{\coqdocnotation{)}};\coqdoceol
\coqdocindent{2.50em}
\coqdef{Groupoid.groupoid.comp}{comp}{\coqdocprojection{comp}} : \coqexternalref{:type scope:'xE2x88x80' x '..' x ',' x}{http://coq.inria.fr/stdlib/Coq.Unicode.Utf8\_core}{\coqdocnotation{∀}} \coqdocvar{x} \coqdocvar{y} \coqdocvar{z} (\coqdocvar{f} \coqdocvar{f'}: \coqdocvariable{x} \coqref{Groupoid.groupoid.::x 'x7E1' x}{\coqdocnotation{$\sim_1$}} \coqdocvariable{y}) (\coqdocvar{g} \coqdocvar{g'}: \coqdocvariable{y} \coqref{Groupoid.groupoid.::x 'x7E1' x}{\coqdocnotation{$\sim_1$}} \coqdocvariable{z}\coqexternalref{:type scope:'xE2x88x80' x '..' x ',' x}{http://coq.inria.fr/stdlib/Coq.Unicode.Utf8\_core}{\coqdocnotation{),}} \coqdoceol
\coqdocindent{7.00em}
\coqdocvariable{f} \coqref{Groupoid.groupoid.::x 'x7E' x}{\coqdocnotation{$\sim_2$}} \coqdocvariable{f'} \coqexternalref{:type scope:x '->' x}{http://coq.inria.fr/stdlib/Coq.Init.Logic}{\coqdocnotation{\ensuremath{\rightarrow}}} \coqdocvariable{g} \coqref{Groupoid.groupoid.::x 'x7E' x}{\coqdocnotation{$\sim_2$}} \coqdocvariable{g'} \coqexternalref{:type scope:x '->' x}{http://coq.inria.fr/stdlib/Coq.Init.Logic}{\coqdocnotation{\ensuremath{\rightarrow}}} \coqdocvariable{g} \coqref{Groupoid.groupoid.::x 'xC2xB0' x}{\coqdocnotation{$\circ$}} \coqdocvariable{f} \coqref{Groupoid.groupoid.::x 'x7E' x}{\coqdocnotation{$\sim_2$}} \coqdocvariable{g'} \coqref{Groupoid.groupoid.::x 'xC2xB0' x}{\coqdocnotation{$\circ$}} \coqdocvariable{f'} \}.\coqdoceol
\coqdocemptyline
\end{coqdoccode}
  In homotopy type theory, coherences are expressed using identity types, with a further requirement that the internal notion of equality induced by the category (isomorphism between two objects) coincides with its identity type.  
  We do not share this point of view because our goal is to restrict the use of identity types to the treatment of contractedness for higher cells. 
  Note that the \coqref{Groupoid.groupoid.comp}{\coqdocprojection{comp}} law is not present in traditional definition of categories 
  because it is automatically satisfied for the identity type.
\begin{coqdoccode}
\end{coqdoccode}


 A \coqref{Groupoid.groupoid.GroupoidP}{\coqdocrecord{$\mathsf{PreGroupoid}$}} is a \coqref{Groupoid.groupoid.CategoryP}{\coqdocrecord{$\mathsf{PreCategory}$}} where all 1-Homs are invertible
 and subject to additional compatibility laws for inverses.
\begin{coqdoccode}
\coqdocemptyline
\coqdocnoindent
\coqdockw{Class} \coqdef{Groupoid.groupoid.GroupoidP}{$\mathsf{PreGroupoid}$}{\coqdocrecord{$\mathsf{PreGroupoid}$}} \coqdocvar{T} := \{ \coqdef{Groupoid.groupoid.C}{C}{\coqdocprojection{C}} :> \coqref{Groupoid.groupoid.CategoryP}{\coqdocclass{$\mathsf{PreCategory}$}} \coqdocvariable{T} ;  \coqdef{Groupoid.groupoid.Inv}{Inv}{\coqdocprojection{Inv}} :> \coqref{Groupoid.groupoid.Inverse}{\coqdocclass{Inverse}} \coqref{Groupoid.groupoid.eq1}{\coqdocprojection{$\coqdoccst{eq}_1$}} ;\coqdoceol
\coqdocindent{2.50em}
\coqdef{Groupoid.groupoid.inv R}{$\coqdoccst{inv}_R$}{\coqdocprojection{$\coqdoccst{inv}_R$}} : \coqexternalref{:type scope:'xE2x88x80' x '..' x ',' x}{http://coq.inria.fr/stdlib/Coq.Unicode.Utf8\_core}{\coqdocnotation{∀}} \coqdocvar{x} \coqdocvar{y} (\coqdocvar{f}: \coqdocvariable{x} \coqref{Groupoid.groupoid.::x 'x7E1' x}{\coqdocnotation{$\sim_1$}} \coqdocvariable{y}\coqexternalref{:type scope:'xE2x88x80' x '..' x ',' x}{http://coq.inria.fr/stdlib/Coq.Unicode.Utf8\_core}{\coqdocnotation{),}} \coqdocvariable{f} \coqref{Groupoid.groupoid.::x 'xC2xB0' x}{\coqdocnotation{$\circ$}} \coqdocvariable{f} \coqref{Groupoid.groupoid.::x 'x5E-1'}{\coqdocnotation{$\hspace{-1ex}^{-1}$}} \coqref{Groupoid.groupoid.::x 'x7E' x}{\coqdocnotation{$\sim_2$}} \coqref{Groupoid.groupoid.identity}{\coqdocprojection{identity}} \coqdocvariable{y} ;\coqdoceol
\coqdocindent{2.50em}
\coqdef{Groupoid.groupoid.inv L}{$\coqdoccst{inv}_L$}{\coqdocprojection{$\coqdoccst{inv}_L$}} : \coqexternalref{:type scope:'xE2x88x80' x '..' x ',' x}{http://coq.inria.fr/stdlib/Coq.Unicode.Utf8\_core}{\coqdocnotation{∀}} \coqdocvar{x} \coqdocvar{y} (\coqdocvar{f}: \coqdocvariable{x} \coqref{Groupoid.groupoid.::x 'x7E1' x}{\coqdocnotation{$\sim_1$}} \coqdocvariable{y}\coqexternalref{:type scope:'xE2x88x80' x '..' x ',' x}{http://coq.inria.fr/stdlib/Coq.Unicode.Utf8\_core}{\coqdocnotation{),}} \coqdocvariable{f} \coqref{Groupoid.groupoid.::x 'x5E-1'}{\coqdocnotation{$\hspace{-1ex}^{-1}$}} \coqref{Groupoid.groupoid.::x 'xC2xB0' x}{\coqdocnotation{$\circ$}} \coqdocvariable{f} \coqref{Groupoid.groupoid.::x 'x7E' x}{\coqdocnotation{$\sim_2$}} \coqref{Groupoid.groupoid.identity}{\coqdocprojection{identity}} \coqdocvariable{x} ;\coqdoceol
\coqdocindent{2.50em}
\coqdef{Groupoid.groupoid.inv}{inv}{\coqdocprojection{inv}} :   \coqexternalref{:type scope:'xE2x88x80' x '..' x ',' x}{http://coq.inria.fr/stdlib/Coq.Unicode.Utf8\_core}{\coqdocnotation{∀}} \coqdocvar{x} \coqdocvar{y} (\coqdocvar{f} \coqdocvar{f'}: \coqdocvariable{x} \coqref{Groupoid.groupoid.::x 'x7E1' x}{\coqdocnotation{$\sim_1$}} \coqdocvariable{y}\coqexternalref{:type scope:'xE2x88x80' x '..' x ',' x}{http://coq.inria.fr/stdlib/Coq.Unicode.Utf8\_core}{\coqdocnotation{),}} \coqdocvariable{f} \coqref{Groupoid.groupoid.::x 'x7E' x}{\coqdocnotation{$\sim_2$}} \coqdocvariable{f'} \coqexternalref{:type scope:x '->' x}{http://coq.inria.fr/stdlib/Coq.Init.Logic}{\coqdocnotation{\ensuremath{\rightarrow}}} \coqdocvariable{f} \coqref{Groupoid.groupoid.::x 'x5E-1'}{\coqdocnotation{$\hspace{-1ex}^{-1}$}} \coqref{Groupoid.groupoid.::x 'x7E' x}{\coqdocnotation{$\sim_2$}} \coqdocvariable{f'} \coqref{Groupoid.groupoid.::x 'x5E-1'}{\coqdocnotation{$\hspace{-1ex}^{-1}$}}\}.\coqdoceol
\coqdocemptyline
\end{coqdoccode}
   Groupoids are then pre-groupoids where equality at
   dimension 2 is irrelevant. This irrelevance is defined using a
   notion of contractibility expressed with (relevant) identity types.  \begin{coqdoccode}
\coqdocemptyline
\end{coqdoccode}
This is a way to require that all higher-cells are trivial. In our setting, we do not have the possibility to say that all 2-cells are related by a 3-cell, and so on. The price to pay will be explicit reasoning on identity types when proving for instance contractedness for the function space. In that case, we need the axiom of functional extensionality.
By analogy to homotopy type theory, we note \coqref{Groupoid.groupoid.Groupoid}{\coqdocrecord{$\mathsf{IsType_1}$}} the property of being a groupoid. 
\begin{coqdoccode}
\coqdocemptyline
\coqdocnoindent
\coqdockw{Class} \coqdef{Groupoid.groupoid.Groupoid}{$\mathsf{IsType_1}$}{\coqdocrecord{$\mathsf{IsType_1}$}} \coqdocvar{T} := \{ \coqdef{Groupoid.groupoid.G}{G}{\coqdocprojection{G}} :> \coqref{Groupoid.groupoid.GroupoidP}{\coqdocclass{$\mathsf{PreGroupoid}$}} \coqdocvariable{T} ;\coqdoceol
\coqdocindent{1.00em}
\coqdef{Groupoid.groupoid.is Trunc 2}{is\_Trunc\_2}{\coqdocprojection{is\_Trunc\_2}} : \coqexternalref{:type scope:'xE2x88x80' x '..' x ',' x}{http://coq.inria.fr/stdlib/Coq.Unicode.Utf8\_core}{\coqdocnotation{∀}} \coqexternalref{:type scope:'xE2x88x80' x '..' x ',' x}{http://coq.inria.fr/stdlib/Coq.Unicode.Utf8\_core}{\coqdocnotation{(}}\coqdocvar{x} \coqdocvar{y} : \coqdocvariable{T}) (\coqdocvar{e} \coqdocvar{e'} : \coqdocvariable{x} \coqref{Groupoid.groupoid.::x 'x7E1' x}{\coqdocnotation{$\sim_1$}} \coqdocvariable{y}) (\coqdocvar{E} \coqdocvar{E'} : \coqdocvariable{e} \coqref{Groupoid.groupoid.::x 'x7E2' x}{\coqdocnotation{$\sim_2$}} \coqdocvariable{e'}\coqexternalref{:type scope:'xE2x88x80' x '..' x ',' x}{http://coq.inria.fr/stdlib/Coq.Unicode.Utf8\_core}{\coqdocnotation{),}} \coqdocclass{Contr} (\coqdocvariable{E} \coqdocnotation{=} \coqdocvariable{E'})\}.\coqdoceol
\coqdocemptyline
\end{coqdoccode}
\noindent 
    In the same way, we define \coqref{Groupoid.groupoid.Setoid}{\coqdocrecord{$\mathsf{IsType_0}$}} when equality is irrelevant at dimension 1.
\begin{coqdoccode}
\coqdocemptyline
\coqdocnoindent
\coqdockw{Class} \coqdef{Groupoid.groupoid.Setoid}{$\mathsf{IsType_0}$}{\coqdocrecord{$\mathsf{IsType_0}$}} \coqdocvar{T} := \{ \coqdef{Groupoid.groupoid.S}{S}{\coqdocprojection{S}} :> \coqref{Groupoid.groupoid.Groupoid}{\coqdocclass{$\mathsf{IsType_1}$}} \coqdocvariable{T} ; \coqdoceol
\coqdocindent{1.00em}
\coqdef{Groupoid.groupoid.is Trunc 1}{is\_Trunc\_1}{\coqdocprojection{is\_Trunc\_1}} : \coqexternalref{:type scope:'xE2x88x80' x '..' x ',' x}{http://coq.inria.fr/stdlib/Coq.Unicode.Utf8\_core}{\coqdocnotation{∀}} \coqexternalref{:type scope:'xE2x88x80' x '..' x ',' x}{http://coq.inria.fr/stdlib/Coq.Unicode.Utf8\_core}{\coqdocnotation{(}}\coqdocvar{x} \coqdocvar{y} : \coqdocvariable{T}) (\coqdocvar{e} \coqdocvar{e'} : \coqdocvariable{x} \coqref{Groupoid.groupoid.::x 'x7E1' x}{\coqdocnotation{$\sim_1$}} \coqdocvariable{y}\coqexternalref{:type scope:'xE2x88x80' x '..' x ',' x}{http://coq.inria.fr/stdlib/Coq.Unicode.Utf8\_core}{\coqdocnotation{)}} \coqexternalref{:type scope:'xE2x88x80' x '..' x ',' x}{http://coq.inria.fr/stdlib/Coq.Unicode.Utf8\_core}{\coqdocnotation{,}} \coqdocclass{Contr} (\coqdocvariable{e} \coqdocnotation{=} \coqdocvariable{e'})\}.\coqdoceol
\coqdocemptyline
\end{coqdoccode}
   We note \coqref{Groupoid.groupoid.GroupoidType}{\coqdocdefinition{$\mathsf{Type_1}$}} for the types that form a \coqref{Groupoid.groupoid.Groupoid}{\coqdocrecord{$\mathsf{IsType_1}$}}.
       The subscript $1$ comes from the fact that groupoids are 1-truncated types
       in homotopy type theory. In the same way,  
       we note \coqref{Groupoid.groupoid.SetoidType}{\coqdocdefinition{$\mathsf{Type_0}$}} 
       for the types that form a \coqref{Groupoid.groupoid.Setoid}{\coqdocrecord{$\mathsf{IsType_0}$}}.
       We define   \coqdocvariable{T}$_{\upharpoonright s}$  the lifting of setoids (inhabitants of \coqref{Groupoid.groupoid.SetoidType}{\coqdocdefinition{$\mathsf{Type_0}$}}) to groupoids.
\begin{coqdoccode}
\end{coqdoccode}
\subsection{Functors and natural transformations}


\label{sec:funextnat}
A morphism between two groupoids is a functor, i.e., a function
between objects of the groupoids that transports homs and
subject to compatibility laws. 
\begin{coqdoccode}
\coqdocemptyline
\coqdocnoindent
\coqdockw{Class} \coqdef{Groupoid.groupoid.Functor}{Functor}{\coqdocrecord{Functor}} \{\coqdocvar{T} \coqdocvar{U} : \coqref{Groupoid.groupoid.UGroupoidType}{\coqdocdefinition{$\mathsf{Type_{1}}$}}\} (\coqdocvar{f} : \coqref{Groupoid.groupoid.::'[' x ']'}{\coqdocnotation{[}}\coqdocvariable{T}\coqref{Groupoid.groupoid.::'[' x ']'}{\coqdocnotation{]}} \coqexternalref{:type scope:x '->' x}{http://coq.inria.fr/stdlib/Coq.Init.Logic}{\coqdocnotation{\ensuremath{\rightarrow}}} \coqref{Groupoid.groupoid.::'[' x ']'}{\coqdocnotation{[}}\coqdocvariable{U}\coqref{Groupoid.groupoid.::'[' x ']'}{\coqdocnotation{]}}) : \coqdockw{Type} :=\coqdoceol
\coqdocnoindent
\{ \coqdef{Groupoid.groupoid. map}{$\coqdoccst{map}$}{\coqdocprojection{$\coqdoccst{map}$}} : \coqexternalref{:type scope:'xE2x88x80' x '..' x ',' x}{http://coq.inria.fr/stdlib/Coq.Unicode.Utf8\_core}{\coqdocnotation{∀}} \{\coqdocvar{x} \coqdocvar{y}\}\coqexternalref{:type scope:'xE2x88x80' x '..' x ',' x}{http://coq.inria.fr/stdlib/Coq.Unicode.Utf8\_core}{\coqdocnotation{,}} \coqdocvariable{x} \coqref{Groupoid.groupoid.::x 'x7E1' x}{\coqdocnotation{$\sim_1$}} \coqdocvariable{y} \coqexternalref{:type scope:x '->' x}{http://coq.inria.fr/stdlib/Coq.Init.Logic}{\coqdocnotation{\ensuremath{\rightarrow}}} \coqdocvariable{f} \coqdocvariable{x} \coqref{Groupoid.groupoid.::x 'x7E1' x}{\coqdocnotation{$\sim_1$}} \coqdocvariable{f} \coqdocvariable{y} ;\coqdoceol
\coqdocindent{1.00em}
\coqdef{Groupoid.groupoid. map id}{$\coqdoccst{map}_\coqdoccst{id}$}{\coqdocprojection{$\coqdoccst{map}_\coqdoccst{id}$}} : \coqexternalref{:type scope:'xE2x88x80' x '..' x ',' x}{http://coq.inria.fr/stdlib/Coq.Unicode.Utf8\_core}{\coqdocnotation{∀}} \{\coqdocvar{x}\}\coqexternalref{:type scope:'xE2x88x80' x '..' x ',' x}{http://coq.inria.fr/stdlib/Coq.Unicode.Utf8\_core}{\coqdocnotation{,}} \coqref{Groupoid.groupoid. map}{\coqdocmethod{$\coqdoccst{map}$}} (\coqref{Groupoid.groupoid.identity}{\coqdocprojection{identity}} \coqdocvariable{x}) \coqref{Groupoid.groupoid.::x 'x7E' x}{\coqdocnotation{$\sim_2$}} \coqref{Groupoid.groupoid.identity}{\coqdocprojection{identity}} (\coqdocvariable{f} \coqdocvariable{x}) ;\coqdoceol
\coqdocindent{1.00em}
\coqdef{Groupoid.groupoid. map comp}{$\coqdoccst{map}_\coqdoccst{comp}$}{\coqdocprojection{$\coqdoccst{map}_\coqdoccst{comp}$}} : \coqexternalref{:type scope:'xE2x88x80' x '..' x ',' x}{http://coq.inria.fr/stdlib/Coq.Unicode.Utf8\_core}{\coqdocnotation{∀}} \{\coqdocvar{x} \coqdocvar{y} \coqdocvar{z}\} (\coqdocvar{e}:\coqdocvariable{x} \coqref{Groupoid.groupoid.::x 'x7E1' x}{\coqdocnotation{$\sim_1$}} \coqdocvariable{y}) (\coqdocvar{e'}:\coqdocvariable{y} \coqref{Groupoid.groupoid.::x 'x7E1' x}{\coqdocnotation{$\sim_1$}} \coqdocvariable{z}\coqexternalref{:type scope:'xE2x88x80' x '..' x ',' x}{http://coq.inria.fr/stdlib/Coq.Unicode.Utf8\_core}{\coqdocnotation{),}} \coqref{Groupoid.groupoid. map}{\coqdocmethod{$\coqdoccst{map}$}} (\coqdocvariable{e'} \coqref{Groupoid.groupoid.::x 'xC2xB0' x}{\coqdocnotation{$\circ$}} \coqdocvariable{e}) \coqref{Groupoid.groupoid.::x 'x7E2' x}{\coqdocnotation{$\sim_2$}} \coqref{Groupoid.groupoid. map}{\coqdocmethod{$\coqdoccst{map}$}} \coqdocvariable{e'} \coqref{Groupoid.groupoid.::x 'xC2xB0' x}{\coqdocnotation{$\circ$}} \coqref{Groupoid.groupoid. map}{\coqdocmethod{$\coqdoccst{map}$}} \coqdocvariable{e} ;\coqdoceol
\coqdocindent{1.00em}
\coqdef{Groupoid.groupoid. map2}{$\coqdoccst{map}_2$}{\coqdocprojection{$\coqdoccst{map}_2$}} : \coqexternalref{:type scope:'xE2x88x80' x '..' x ',' x}{http://coq.inria.fr/stdlib/Coq.Unicode.Utf8\_core}{\coqdocnotation{∀}} \coqexternalref{:type scope:'xE2x88x80' x '..' x ',' x}{http://coq.inria.fr/stdlib/Coq.Unicode.Utf8\_core}{\coqdocnotation{\{}}\coqdocvar{x} \coqdocvar{y}:\coqref{Groupoid.groupoid.::'[' x ']'}{\coqdocnotation{[}}\coqdocvariable{T}\coqref{Groupoid.groupoid.::'[' x ']'}{\coqdocnotation{]}}\} \{\coqdocvar{e} \coqdocvar{e'} : \coqdocvariable{x} \coqref{Groupoid.groupoid.::x 'x7E1' x}{\coqdocnotation{$\sim_1$}} \coqdocvariable{y}\coqexternalref{:type scope:'xE2x88x80' x '..' x ',' x}{http://coq.inria.fr/stdlib/Coq.Unicode.Utf8\_core}{\coqdocnotation{\},}} \coqexternalref{:type scope:x '->' x}{http://coq.inria.fr/stdlib/Coq.Init.Logic}{\coqdocnotation{(}}\coqdocvariable{e} \coqref{Groupoid.groupoid.::x 'x7E2' x}{\coqdocnotation{$\sim_2$}} \coqdocvariable{e'}\coqexternalref{:type scope:x '->' x}{http://coq.inria.fr/stdlib/Coq.Init.Logic}{\coqdocnotation{)}} \coqexternalref{:type scope:x '->' x}{http://coq.inria.fr/stdlib/Coq.Init.Logic}{\coqdocnotation{\ensuremath{\rightarrow}}} \coqref{Groupoid.groupoid. map}{\coqdocmethod{$\coqdoccst{map}$}}  \coqdocvariable{e} \coqref{Groupoid.groupoid.::x 'x7E2' x}{\coqdocnotation{$\sim_2$}} \coqref{Groupoid.groupoid. map}{\coqdocmethod{$\coqdoccst{map}$}} \coqdocvariable{e'} \}.\coqdoceol
\coqdocemptyline
\coqdocnoindent
\coqdockw{Definition} \coqdef{Groupoid.groupoid.Fun Type}{Fun\_Type}{\coqdocdefinition{Fun\_Type}} (\coqdocvar{T} \coqdocvar{U} : \coqref{Groupoid.groupoid.UGroupoidType}{\coqdocdefinition{$\mathsf{Type_{1}}$}}) := \coqdocnotation{\{}\coqdocvar{f} \coqdocnotation{:} \coqref{Groupoid.groupoid.::'[' x ']'}{\coqdocnotation{[}}\coqdocvariable{T}\coqref{Groupoid.groupoid.::'[' x ']'}{\coqdocnotation{]}} \coqexternalref{:type scope:x '->' x}{http://coq.inria.fr/stdlib/Coq.Init.Logic}{\coqdocnotation{\ensuremath{\rightarrow}}} \coqref{Groupoid.groupoid.::'[' x ']'}{\coqdocnotation{[}}\coqdocvariable{U}\coqref{Groupoid.groupoid.::'[' x ']'}{\coqdocnotation{]}} \coqdocnotation{\&} \coqref{Groupoid.groupoid.Functor}{\coqdocclass{Functor}} \coqdocvar{f}\coqdocnotation{\}}.\coqdoceol
\coqdocemptyline
\end{coqdoccode}
\noindent We note \coqdocvariable{T} $\longrightarrow$ \coqdocvariable{U} the type of functors from \coqdocvariable{T} to \coqdocvariable{U}.
Note that we only impose compatibility with the composition as
compatibilities with identities and inverse Homs can be deduced from
it. We note \coqdocvariable{M} $\star$ \coqdocvariable{N} the application of a function \coqdocvariable{M} in the first
component of a dependent pair. \begin{coqdoccode}
\end{coqdoccode}
Equivalence between functors is given by natural transformations.
  We insist here that this naturality condition in the definition of
  functor equality is crucial in a higher setting.  It is usually
  derivable in formalizations of homotopy theory in Coq because there they
  only consider the 1-groupoid case where the naturality comes for
  free from functional extensionality, see for instance~\cite{coq_unival_axiom}.  \begin{coqdoccode}
\coqdocemptyline
\coqdocnoindent
\coqdockw{Class} \coqdef{Groupoid.groupoid.NaturalTransformation}{$\mathsf{NaturalTrans}$}{\coqdocrecord{$\mathsf{NaturalTrans}$}} \coqdocvar{T} \coqdocvar{U} \{\coqdocvar{f} \coqdocvar{g} : \coqdocvariable{T} \coqref{Groupoid.groupoid.::x '--->' x}{\coqdocnotation{$\longrightarrow$}} \coqdocvariable{U}\} (\coqdocvar{α} : \coqexternalref{:type scope:'xE2x88x80' x '..' x ',' x}{http://coq.inria.fr/stdlib/Coq.Unicode.Utf8\_core}{\coqdocnotation{∀}} \coqdocvar{t} : \coqref{Groupoid.groupoid.::'[' x ']'}{\coqdocnotation{[}}\coqdocvariable{T}\coqref{Groupoid.groupoid.::'[' x ']'}{\coqdocnotation{]}}\coqexternalref{:type scope:'xE2x88x80' x '..' x ',' x}{http://coq.inria.fr/stdlib/Coq.Unicode.Utf8\_core}{\coqdocnotation{,}} \coqdocvariable{f} \coqref{Groupoid.groupoid.::x '@' x}{\coqdocnotation{$\star$}} \coqdocvariable{t} \coqref{Groupoid.groupoid.::x 'x7E1' x}{\coqdocnotation{$\sim_1$}} \coqdocvariable{g} \coqref{Groupoid.groupoid.::x '@' x}{\coqdocnotation{$\star$}} \coqdocvariable{t}) := \coqdoceol
\coqdocindent{1.00em}
\coqdef{Groupoid.groupoid. xCExB1 map}{$\coqdoccst{α}_\mathsf{map}$}{\coqdocprojection{$\coqdoccst{α}_\mathsf{map}$}} : \coqexternalref{:type scope:'xE2x88x80' x '..' x ',' x}{http://coq.inria.fr/stdlib/Coq.Unicode.Utf8\_core}{\coqdocnotation{∀}} \{\coqdocvar{t} \coqdocvar{t'}\} (\coqdocvar{e} : \coqdocvariable{t} \coqref{Groupoid.groupoid.::x 'x7E1' x}{\coqdocnotation{$\sim_1$}} \coqdocvariable{t'}\coqexternalref{:type scope:'xE2x88x80' x '..' x ',' x}{http://coq.inria.fr/stdlib/Coq.Unicode.Utf8\_core}{\coqdocnotation{),}} \coqdocvariable{α} \coqdocvariable{t'} \coqref{Groupoid.groupoid.::x 'xC2xB0' x}{\coqdocnotation{$\circ$}} \coqref{Groupoid.groupoid.map}{\coqdocabbreviation{map}} \coqdocvariable{f} \coqdocvariable{e} \coqref{Groupoid.groupoid.::x 'x7E' x}{\coqdocnotation{$\sim_2$}} \coqref{Groupoid.groupoid.map}{\coqdocabbreviation{map}} \coqdocvariable{g} \coqdocvariable{e} \coqref{Groupoid.groupoid.::x 'xC2xB0' x}{\coqdocnotation{$\circ$}} \coqdocvariable{α} \coqdocvariable{t}.\coqdoceol
\coqdocemptyline
\coqdocnoindent
\coqdockw{Definition} \coqdef{Groupoid.groupoid.nat trans}{nat\_trans}{\coqdocdefinition{nat\_trans}} \coqdocvar{T} \coqdocvar{U} : \coqref{Groupoid.groupoid.HomT}{\coqdocdefinition{$\mathsf{HomSet}$}} (\coqdocvariable{T} \coqref{Groupoid.groupoid.::x '--->' x}{\coqdocnotation{$\longrightarrow$}} \coqdocvariable{U}) \coqdoceol
\coqdocindent{0.50em}
:= \coqexternalref{::'xCExBB' x '..' x ',' x}{http://coq.inria.fr/stdlib/Coq.Unicode.Utf8\_core}{\coqdocnotation{\ensuremath{\lambda}}} \coqdocvar{f} \coqdocvar{g}\coqexternalref{::'xCExBB' x '..' x ',' x}{http://coq.inria.fr/stdlib/Coq.Unicode.Utf8\_core}{\coqdocnotation{,}} \coqdocnotation{\{}\coqdocvar{α} \coqdocnotation{:} \coqexternalref{:type scope:'xE2x88x80' x '..' x ',' x}{http://coq.inria.fr/stdlib/Coq.Unicode.Utf8\_core}{\coqdocnotation{∀}} \coqdocvar{t} : \coqref{Groupoid.groupoid.::'[' x ']'}{\coqdocnotation{[}}\coqdocvariable{T}\coqref{Groupoid.groupoid.::'[' x ']'}{\coqdocnotation{]}}\coqexternalref{:type scope:'xE2x88x80' x '..' x ',' x}{http://coq.inria.fr/stdlib/Coq.Unicode.Utf8\_core}{\coqdocnotation{,}} \coqdocvariable{f} \coqref{Groupoid.groupoid.::x '@' x}{\coqdocnotation{$\star$}} \coqdocvariable{t} \coqref{Groupoid.groupoid.::x 'x7E1' x}{\coqdocnotation{$\sim_1$}} \coqdocvariable{g} \coqref{Groupoid.groupoid.::x '@' x}{\coqdocnotation{$\star$}} \coqdocvariable{t} \coqdocnotation{\&} \coqref{Groupoid.groupoid.NaturalTransformation}{\coqdocclass{$\mathsf{NaturalTrans}$}} \coqdocvar{α}\coqdocnotation{\}}.\coqdoceol
\coqdocemptyline
\end{coqdoccode}
In our setting, equality between natural transformations is not expressed using identity types, but using the higher categorical notion of modification.
\begin{coqdoccode}
\coqdocemptyline
\coqdocnoindent
\coqdockw{Definition} \coqdef{Groupoid.groupoid.modification}{modification}{\coqdocdefinition{modification}} \coqdocvar{T} \coqdocvar{U} (\coqdocvar{f} \coqdocvar{g} : \coqdocvariable{T} \coqref{Groupoid.groupoid.::x '--->' x}{\coqdocnotation{$\longrightarrow$}} \coqdocvariable{U}) : \coqref{Groupoid.groupoid.HomT}{\coqdocdefinition{$\mathsf{HomSet}$}} (\coqdocvariable{f} \coqref{Groupoid.groupoid.::x 'x7E1' x}{\coqdocnotation{$\sim_1$}} \coqdocvariable{g}) \coqdoceol
\coqdocindent{1.00em}
:= \coqexternalref{::'xCExBB' x '..' x ',' x}{http://coq.inria.fr/stdlib/Coq.Unicode.Utf8\_core}{\coqdocnotation{\ensuremath{\lambda}}} \coqdocvar{α} \coqdocvar{$\beta$}\coqexternalref{::'xCExBB' x '..' x ',' x}{http://coq.inria.fr/stdlib/Coq.Unicode.Utf8\_core}{\coqdocnotation{,}} \coqexternalref{:type scope:'xE2x88x80' x '..' x ',' x}{http://coq.inria.fr/stdlib/Coq.Unicode.Utf8\_core}{\coqdocnotation{∀}} \coqdocvar{t} : \coqref{Groupoid.groupoid.::'[' x ']'}{\coqdocnotation{[}}\coqdocvariable{T}\coqref{Groupoid.groupoid.::'[' x ']'}{\coqdocnotation{]}}\coqexternalref{:type scope:'xE2x88x80' x '..' x ',' x}{http://coq.inria.fr/stdlib/Coq.Unicode.Utf8\_core}{\coqdocnotation{,}} \coqdocvariable{α} \coqref{Groupoid.groupoid.::x '@' x}{\coqdocnotation{$\star$}} \coqdocvariable{t} \coqref{Groupoid.groupoid.::x 'x7E' x}{\coqdocnotation{$\sim_2$}} \coqdocvariable{$\beta$} \coqref{Groupoid.groupoid.::x '@' x}{\coqdocnotation{$\star$}} \coqdocvariable{t}.\coqdoceol
\coqdocemptyline
\end{coqdoccode}
\noindent
    We can now equip the functor space with a groupoid structure. Note
    here that we (abusively) use the same notation for the functor type and 
    its corresponding groupoid. \begin{coqdoccode}
\coqdocemptyline
\coqdocnoindent
\coqdockw{Definition} \coqdef{Groupoid.groupoid. fun}{\_fun}{\coqdocdefinition{\_fun}} \coqdocvar{T} \coqdocvar{U} : \coqref{Groupoid.groupoid.UGroupoidType}{\coqdocdefinition{$\mathsf{Type_{1}}$}} := \coqdocnotation{(}\coqdocvariable{T} \coqref{Groupoid.groupoid.::x '--->' x}{\coqdocnotation{$\longrightarrow$}} \coqdocvariable{U} \coqdocnotation{;} \coqref{Groupoid.groupoid.nat trans grp}{\coqdocinstance{$\mathsf{fun_{grp}}$}} \coqdocvariable{T} \coqdocvariable{U}\coqdocnotation{)}.\coqdoceol
\coqdocemptyline
\end{coqdoccode}
 In the definition above, \coqref{Groupoid.groupoid.nat trans grp}{\coqdocinstance{$\mathsf{fun_{grp}}$}} is a proof that \coqref{Groupoid.groupoid.nat trans}{\coqdocdefinition{nat\_trans}} and \coqref{Groupoid.groupoid.modification}{\coqdocdefinition{modification}} form a groupoid on \coqdocvariable{T} $\longrightarrow$ \coqdocvariable{U}. In particular, it makes use of functional extensionality, which says that the canonical proof of \coqdocvariable{f} = \coqdocvariable{g} \ensuremath{\rightarrow} ∀ \coqdocvariable{x}, \coqdocvariable{f} \coqdocvariable{x} = \coqdocvariable{g} \coqdocvariable{x} is an equivalence (in the sense of homotopy type theory).   
 \begin{coqdoccode}
\coqdocemptyline
\end{coqdoccode}
\subsection{Homotopic equivalences}


 \label{sec:homequiv}   
    The standard notion of equivalence between groupoids is given by
    adjoint equivalences, that is a map with an \coqref{Groupoid.groupoid.adjoint}{\coqdocabbreviation{adjoint}} and two proofs
    that they form a \coqref{Groupoid.groupoid.section}{\coqdocabbreviation{section}} (or counit of the adjunction) and a
    \coqref{Groupoid.groupoid.retraction}{\coqdocabbreviation{retraction}} (or unit of the adjunction). \begin{coqdoccode}
\coqdocemptyline
\coqdocnoindent
\coqdockw{Class} \coqdef{Groupoid.groupoid.Iso struct}{Iso\_struct}{\coqdocrecord{Iso\_struct}} \coqdocvar{T} \coqdocvar{U} (\coqdocvar{f} : \coqref{Groupoid.groupoid.::'[' x ']'}{\coqdocnotation{[}}\coqdocvariable{T} \coqref{Groupoid.groupoid.::x '-->' x}{\coqdocnotation{$\longrightarrow$}} \coqdocvariable{U}\coqref{Groupoid.groupoid.::'[' x ']'}{\coqdocnotation{]}}) := \coqdoceol
\coqdocnoindent
\{ \coqdef{Groupoid.groupoid. adjoint}{$\coqdoccst{adjoint}$}{\coqdocprojection{$\coqdoccst{adjoint}$}} :    \coqref{Groupoid.groupoid.::'[' x ']'}{\coqdocnotation{[}}\coqdocvariable{U} \coqref{Groupoid.groupoid.::x '-->' x}{\coqdocnotation{$\longrightarrow$}} \coqdocvariable{T}\coqref{Groupoid.groupoid.::'[' x ']'}{\coqdocnotation{]}} ;\coqdoceol
\coqdocindent{1.00em}
\coqdef{Groupoid.groupoid. section}{$\coqdoccst{section}$}{\coqdocprojection{$\coqdoccst{section}$}} :    \coqdocvariable{f} \coqref{Groupoid.groupoid.::x 'xC2xB0' x}{\coqdocnotation{$\circ$}} \coqref{Groupoid.groupoid. adjoint}{\coqdocmethod{$\coqdoccst{adjoint}$}} \coqref{Groupoid.groupoid.::x 'x7E' x}{\coqdocnotation{$\sim_2$}} \coqref{Groupoid.groupoid.identity}{\coqdocprojection{identity}} \coqdocvariable{U} ;\coqdoceol
\coqdocindent{1.00em}
\coqdef{Groupoid.groupoid. retraction}{$\coqdoccst{retraction}$}{\coqdocprojection{$\coqdoccst{retraction}$}} : \coqref{Groupoid.groupoid. adjoint}{\coqdocmethod{$\coqdoccst{adjoint}$}} \coqref{Groupoid.groupoid.::x 'xC2xB0' x}{\coqdocnotation{$\circ$}} \coqdocvariable{f} \coqref{Groupoid.groupoid.::x 'x7E' x}{\coqdocnotation{$\sim_2$}} \coqref{Groupoid.groupoid.identity}{\coqdocprojection{identity}} \coqdocvariable{T}\}.\coqdoceol
\coqdocemptyline
\end{coqdoccode}
This type class defines usual equivalences. To get an adjoint
    equivalence, an additional triangle identity between sections and
    retractions is required. This allows to eliminate a section against
    a retraction in proofs. A corresponding triangle identity involving
    \coqref{Groupoid.groupoid.adjoint}{\coqdocabbreviation{adjoint}} \coqdocvariable{f} can also be expressed, but it can be shown that each
    condition implies the other.  \begin{coqdoccode}
\coqdocemptyline
\coqdocnoindent
\coqdockw{Class} \coqdef{Groupoid.groupoid.Equiv struct}{Equiv\_struct}{\coqdocrecord{Equiv\_struct}} \coqdocvar{T} \coqdocvar{U} (\coqdocvar{f} : \coqdocvariable{T} \coqref{Groupoid.groupoid.::x '--->' x}{\coqdocnotation{$\longrightarrow$}} \coqdocvariable{U}) := \coqdoceol
\coqdocnoindent
\{ \coqdef{Groupoid.groupoid.iso}{iso}{\coqdocprojection{iso}} :> \coqref{Groupoid.groupoid.Iso struct}{\coqdocclass{Iso\_struct}} \coqdocvariable{f};\coqdoceol
\coqdocindent{1.00em}
\coqdef{Groupoid.groupoid. triangle}{$\coqdoccst{triangle}$}{\coqdocprojection{$\coqdoccst{triangle}$}} : \coqexternalref{:type scope:'xE2x88x80' x '..' x ',' x}{http://coq.inria.fr/stdlib/Coq.Unicode.Utf8\_core}{\coqdocnotation{∀}} \coqdocvar{t}\coqexternalref{:type scope:'xE2x88x80' x '..' x ',' x}{http://coq.inria.fr/stdlib/Coq.Unicode.Utf8\_core}{\coqdocnotation{,}} \coqref{Groupoid.groupoid. section}{\coqdocprojection{$\coqdoccst{section}$}} \coqref{Groupoid.groupoid.::x '@' x}{\coqdocnotation{$\star$}} \coqref{Groupoid.groupoid.::x '@' x}{\coqdocnotation{(}}\coqdocvariable{f} \coqref{Groupoid.groupoid.::x '@' x}{\coqdocnotation{$\star$}} \coqdocvariable{t}\coqref{Groupoid.groupoid.::x '@' x}{\coqdocnotation{)}} \coqref{Groupoid.groupoid.::x 'x7E' x}{\coqdocnotation{$\sim_2$}} \coqref{Groupoid.groupoid.map}{\coqdocabbreviation{map}} \coqdocvariable{f} (\coqref{Groupoid.groupoid. retraction}{\coqdocprojection{$\coqdoccst{retraction}$}} \coqref{Groupoid.groupoid.::x '@' x}{\coqdocnotation{$\star$}} \coqdocvariable{t})\}.\coqdoceol
\coqdocemptyline
\coqdocnoindent
\coqdockw{Definition} \coqdef{Groupoid.groupoid.Equiv}{Equiv}{\coqdocdefinition{Equiv}} \coqdocvar{A} \coqdocvar{B} := \coqdocnotation{\{}\coqdocvar{f} \coqdocnotation{:} \coqdocvariable{A} \coqref{Groupoid.groupoid.::x '--->' x}{\coqdocnotation{$\longrightarrow$}} \coqdocvariable{B} \coqdocnotation{\&} \coqref{Groupoid.groupoid.Equiv struct}{\coqdocclass{Equiv\_struct}} \coqdocvar{f}\coqdocnotation{\}}.\coqdoceol
\coqdocemptyline
\end{coqdoccode}


   It is well known that any equivalence can be turned into an adjoint
   equivalence by slightly modifying the section. While available in
   our formalization, this result should be used with care as it
   opacifies the underlying notion of homotopy and can harden proofs.
\begin{coqdoccode}
\coqdocemptyline
\end{coqdoccode}
Equality of homotopy equivalences is given by equivalence of
  adjunctions. Two adjunctions are equivalent if their left adjoints are
  equivalent and they agree on their sections (up-to the isomorphism).
  Note that equivalence of the right adjoints and agreement on their
  retractions can be deduced so they are not part of the definition.  \begin{coqdoccode}
\coqdocemptyline
\coqdocnoindent
\coqdockw{Class} \coqdef{Groupoid.groupoid.EquivEq}{EquivEq}{\coqdocrecord{EquivEq}} \{\coqdocvar{T} \coqdocvar{U}\} \{\coqdocvar{f} \coqdocvar{g} : \coqref{Groupoid.groupoid.Equiv}{\coqdocdefinition{Equiv}} \coqdocvariable{T} \coqdocvariable{U}\} (\coqdocvar{α} : \coqref{Groupoid.groupoid.::'[' x ']'}{\coqdocnotation{[}}\coqdocvariable{f}\coqref{Groupoid.groupoid.::'[' x ']'}{\coqdocnotation{]}} \coqref{Groupoid.groupoid.::x 'x7E' x}{\coqdocnotation{$\sim_2$}} \coqref{Groupoid.groupoid.::'[' x ']'}{\coqdocnotation{[}}\coqdocvariable{g}\coqref{Groupoid.groupoid.::'[' x ']'}{\coqdocnotation{]}}) : \coqdockw{Type} :=  \coqdoceol
\coqdocindent{0.50em}
\coqdef{Groupoid.groupoid. eq section}{\_eq\_section}{\coqdocprojection{\_eq\_section}} : \coqref{Groupoid.groupoid.section}{\coqdocabbreviation{section}} \coqdocvariable{f} \coqref{Groupoid.groupoid.::x 'x7E' x}{\coqdocnotation{$\sim_2$}} \coqref{Groupoid.groupoid.section}{\coqdocabbreviation{section}} \coqdocvariable{g} \coqref{Groupoid.groupoid.::x 'xC2xB0' x}{\coqdocnotation{$\circ$}} \coqref{Groupoid.groupoid.::x 'xC2xB0' x}{\coqdocnotation{(}}\coqdocvariable{α} \coqref{Groupoid.groupoid.::x 'xC2xB0''' x}{\coqdocnotation{$\circ$}} \coqref{Groupoid.groupoid.::x 'xC2xB0''' x}{\coqdocnotation{(}}\coqref{Groupoid.groupoid.Equiv adjoint}{\coqdocdefinition{Equiv\_adjoint}} \coqdocvariable{α}\coqref{Groupoid.groupoid.::x 'xC2xB0''' x}{\coqdocnotation{)}}\coqref{Groupoid.groupoid.::x 'xC2xB0' x}{\coqdocnotation{)}}.\coqdoceol
\coqdocemptyline
\coqdocnoindent
\coqdockw{Definition} \coqdef{Groupoid.groupoid.Equiv eq}{Equiv\_eq}{\coqdocdefinition{Equiv\_eq}} \coqdocvar{T} \coqdocvar{U} (\coqdocvar{f} \coqdocvar{g} : \coqref{Groupoid.groupoid.Equiv}{\coqdocdefinition{Equiv}} \coqdocvariable{T} \coqdocvariable{U}) := \coqdocnotation{\{}\coqdocvar{α} \coqdocnotation{:} \coqref{Groupoid.groupoid.nat trans}{\coqdocdefinition{nat\_trans}} \coqref{Groupoid.groupoid.::'[' x ']'}{\coqdocnotation{[}}\coqdocvariable{f}\coqref{Groupoid.groupoid.::'[' x ']'}{\coqdocnotation{]}} \coqref{Groupoid.groupoid.::'[' x ']'}{\coqdocnotation{[}}\coqdocvariable{g}\coqref{Groupoid.groupoid.::'[' x ']'}{\coqdocnotation{]}} \coqdocnotation{\&} \coqref{Groupoid.groupoid.EquivEq}{\coqdocclass{EquivEq}} \coqdocvar{α}\coqdocnotation{\}}.\coqdoceol
\coqdocemptyline
\end{coqdoccode}
It is crucial here to be able to express the 2-dimensional equality
  between groupoids as a particular \coqdockw{Type} and not directly using the
  identity type. Indeed, whereas the functional extensionality principle
  makes the use of the identity type and modification equivalent to
  treat equality of natural transformations, the same is not possible
  for homotopy equivalences.  \begin{coqdoccode}
\coqdocemptyline
\end{coqdoccode}
We can define the pre-groupoid \coqref{Groupoid.groupoid. Type}{\coqdocabbreviation{$\mathsf{Type}_{1}^{1}$}} of groupoids and homotopy
 equivalences.  However, groupoids together with homotopy equivalences
 do not form a groupoid but rather a 2-groupoid. As we only have a
 formalization of groupoids, this can not be expressed in our
 setting. Nevertheless, we can state that setoids (inhabitants of
 \coqref{Groupoid.groupoid.SetoidType}{\coqdocdefinition{$\mathsf{Type_0}$}}) form a groupoid.  \begin{coqdoccode}
\coqdocemptyline
\coqdocnoindent
\coqdockw{Definition} \coqdef{Groupoid.groupoid.Type0}{$\mathsf{Type}_{0}^1$}{\coqdocdefinition{$\mathsf{Type}_{0}^1$}} : \coqref{Groupoid.groupoid.GroupoidType}{\coqdocdefinition{$\mathsf{Type_1}$}} := \coqdocnotation{(}\coqref{Groupoid.groupoid.SetoidType}{\coqdocdefinition{$\mathsf{Type_0}$}} \coqdocnotation{;} \coqref{Groupoid.groupoid.Equiv Groupoid}{\coqdocinstance{$\mathsf{Equiv_{Type_0}}$}}\coqdocnotation{)}.\coqdoceol
\coqdocemptyline
\end{coqdoccode}
\noindent In the definition above, \coqref{Groupoid.groupoid.Equiv Groupoid}{\coqdocinstance{$\mathsf{Equiv_{Type_0}}$}} is a proof
that \coqref{Groupoid.groupoid.Equiv}{\coqdocdefinition{Equiv}} and \coqref{Groupoid.groupoid.Equiv eq}{\coqdocdefinition{Equiv\_eq}} form a groupoid. It makes again use of
functional extensionality to prove contractibility of higher cells.  As
the type of pre-groupoids appears both in the term and the type, the use of
polymorphic universes is crucial here to avoid an inconsistency. \begin{coqdoccode}
\end{coqdoccode}
\subsection{Rewriting in homotopy type theory}


  \label{sec:rew}


  When considering a dependent family \coqdocvariable{F} of type [\coqdocvariable{A} $\longrightarrow$ \coqref{Groupoid.groupoid. Type}{\coqdocabbreviation{$\mathsf{Type}_{1}^{1}$}}], the \coqref{Groupoid.groupoid. map}{\coqdocprojection{$\coqdoccst{map}$}} function
  provides a homotopy equivalence between \coqdocvariable{F} $\star$ \coqdocvariable{x} and \coqdocvariable{F} $\star$ \coqdocvariable{y} for any \coqdocvariable{x}
  and \coqdocvariable{y} such that \coqdocvariable{x} $\sim_1$ \coqdocvariable{y}. The underlying map of homotopy equivalence
  can hence be used to cast any term of type [\coqdocvariable{F} $\star$ \coqdocvariable{x}] to [\coqdocvariable{F} $\star$ \coqdocvariable{y}].
\begin{coqdoccode}
\coqdocemptyline
\coqdocnoindent
\coqdockw{Definition} \coqdef{Groupoid.groupoid.transport}{transport}{\coqdocdefinition{transport}} \coqdocvar{A} (\coqdocvar{F}:\coqref{Groupoid.groupoid.::'[' x ']'}{\coqdocnotation{[}}\coqdocvariable{A} \coqref{Groupoid.groupoid.::x '-->' x}{\coqdocnotation{$\longrightarrow$}} \coqref{Groupoid.groupoid. Type}{\coqdocabbreviation{$\mathsf{Type}_{1}^{1}$}}\coqref{Groupoid.groupoid.::'[' x ']'}{\coqdocnotation{]}}) \{\coqdocvar{x} \coqdocvar{y}:\coqref{Groupoid.groupoid.::'[' x ']'}{\coqdocnotation{[}}\coqdocvariable{A}\coqref{Groupoid.groupoid.::'[' x ']'}{\coqdocnotation{]}}\} (\coqdocvar{e}:\coqdocvariable{x} \coqref{Groupoid.groupoid.::x 'x7E1' x}{\coqdocnotation{$\sim_1$}} \coqdocvariable{y}) \coqdoceol
\coqdocindent{1.00em}
: \coqref{Groupoid.groupoid.::x '--->' x}{\coqdocnotation{(}}\coqdocvariable{F} \coqref{Groupoid.groupoid.::x '@' x}{\coqdocnotation{$\star$}} \coqdocvariable{x}\coqref{Groupoid.groupoid.::x '--->' x}{\coqdocnotation{)}} \coqref{Groupoid.groupoid.::x '--->' x}{\coqdocnotation{$\longrightarrow$}} \coqref{Groupoid.groupoid.::x '--->' x}{\coqdocnotation{(}}\coqdocvariable{F} \coqref{Groupoid.groupoid.::x '@' x}{\coqdocnotation{$\star$}} \coqdocvariable{y}\coqref{Groupoid.groupoid.::x '--->' x}{\coqdocnotation{)}} := \coqref{Groupoid.groupoid.::'[' x ']'}{\coqdocnotation{[}}\coqref{Groupoid.groupoid.map}{\coqdocabbreviation{map}} \coqdocvariable{F} \coqdocvariable{e}\coqref{Groupoid.groupoid.::'[' x ']'}{\coqdocnotation{]}}.\coqdoceol
\coqdocemptyline
\end{coqdoccode}
Using compatibility on \coqref{Groupoid.groupoid. map}{\coqdocprojection{$\coqdoccst{map}$}}, we can reason on different transport paths.  
  Intuitively, any two transport maps with the same domain
  and codomain should be the same up to homotopy. As we only consider
  groupoids, there is only one relevant level of compatibilities,
  higher compatibilities are trivial. \coqref{Groupoid.groupoid.transport eq}{\coqdocdefinition{$\mathsf{transport_{eq}}$}} is an example of a
  derivable equality between two transport maps, when the proofs
  relating \coqdocvariable{x} and \coqdocvariable{y} are equal.  \begin{coqdoccode}
\coqdocemptyline
\coqdocnoindent
\coqdockw{Definition} \coqdef{Groupoid.groupoid.transport eq}{$\mathsf{transport_{eq}}$}{\coqdocdefinition{$\mathsf{transport_{eq}}$}} \coqdocvar{A} (\coqdocvar{F}:\coqref{Groupoid.groupoid.::'[' x ']'}{\coqdocnotation{[}}\coqdocvariable{A} \coqref{Groupoid.groupoid.::x '-->' x}{\coqdocnotation{$\longrightarrow$}} \coqref{Groupoid.groupoid. Type}{\coqdocabbreviation{$\mathsf{Type}_{1}^{1}$}}\coqref{Groupoid.groupoid.::'[' x ']'}{\coqdocnotation{]}}) \{\coqdocvar{x} \coqdocvar{y}:\coqref{Groupoid.groupoid.::'[' x ']'}{\coqdocnotation{[}}\coqdocvariable{A}\coqref{Groupoid.groupoid.::'[' x ']'}{\coqdocnotation{]}}\} \{\coqdocvar{e} \coqdocvar{e'}:\coqdocvariable{x} \coqref{Groupoid.groupoid.::x 'x7E1' x}{\coqdocnotation{$\sim_1$}} \coqdocvariable{y}\} (\coqdocvar{H}:\coqdocvariable{e} \coqref{Groupoid.groupoid.::x 'x7E' x}{\coqdocnotation{$\sim_2$}} \coqdocvariable{e'}) \coqdoceol
\coqdocindent{1.00em}
: \coqref{Groupoid.groupoid.transport}{\coqdocdefinition{transport}} \coqdocvariable{F} \coqdocvariable{e} \coqref{Groupoid.groupoid.::x 'x7E1' x}{\coqdocnotation{$\sim_1$}} \coqref{Groupoid.groupoid.transport}{\coqdocdefinition{transport}} \coqdocvariable{F} \coqdocvariable{e'} := \coqref{Groupoid.groupoid.::'[' x ']'}{\coqdocnotation{[}}\coqref{Groupoid.groupoid.map2}{\coqdocabbreviation{$\coqdoccst{map}_2$}} \coqdocvariable{F} \coqdocvariable{H}\coqref{Groupoid.groupoid.::'[' x ']'}{\coqdocnotation{]}}.\coqdoceol
\coqdocemptyline
\end{coqdoccode}
\noindent In the text, 
  we also use \coqref{Groupoid.groupoid.transport id}{\coqdocdefinition{$\mathsf{transport_{id}}$}}, \coqref{Groupoid.groupoid.transport comp}{\coqdocdefinition{$\mathsf{transport_{comp}}$}} and \coqref{Groupoid.groupoid.transport map}{\coqdocdefinition{$\mathsf{transport_{map}}$}} for compatibilities with identities, composition and for the functoriality of \coqref{Groupoid.groupoid.transport}{\coqdocdefinition{transport}}. \begin{coqdoccode}
\coqdocemptyline
\end{coqdoccode}
\subsection{Dependent Product}


  \label{sec:depprod}
  As for functions, dependent functions will be interpreted as functors. 
  But this time, the compatibilities with higher-order morphisms cannot
  be expressed as simple equalities, as some transport has to be done to 
  make those equalities typable. We call such a functor a 
  \emph{dependent functor}. Dependent functors are defined between a groupoid \coqdocvariable{T} and a functor \coqdocvariable{U} from \coqdocvariable{T} to \coqref{Groupoid.groupoid. Type}{\coqdocabbreviation{$\mathsf{Type}_{1}^{1}$}} (the pre-groupoid of groupoids). \coqdocvariable{U} must be seen as a type depending on \coqdocvariable{T}, or as a family of types indexed by \coqdocvariable{T}. 
\begin{coqdoccode}
\coqdocemptyline
\coqdocnoindent
\coqdockw{Class} \coqdef{Groupoid.groupoid.DependentFunctor}{$\mathsf{Functor}^\Pi$}{\coqdocrecord{$\mathsf{Functor}^\Pi$}} \coqdocvar{T} (\coqdocvar{U} : \coqref{Groupoid.groupoid.::'[' x ']'}{\coqdocnotation{[}}\coqdocvariable{T} \coqref{Groupoid.groupoid.::x '-->' x}{\coqdocnotation{$\longrightarrow$}} \coqref{Groupoid.groupoid. Type}{\coqdocabbreviation{$\mathsf{Type}_{1}^{1}$}}\coqref{Groupoid.groupoid.::'[' x ']'}{\coqdocnotation{]}}) (\coqdocvar{f} : \coqexternalref{:type scope:'xE2x88x80' x '..' x ',' x}{http://coq.inria.fr/stdlib/Coq.Unicode.Utf8\_core}{\coqdocnotation{∀}} \coqdocvar{t}\coqexternalref{:type scope:'xE2x88x80' x '..' x ',' x}{http://coq.inria.fr/stdlib/Coq.Unicode.Utf8\_core}{\coqdocnotation{,}} \coqref{Groupoid.groupoid.::'[' x ']'}{\coqdocnotation{[}}\coqdocvariable{U} \coqref{Groupoid.groupoid.::x '@' x}{\coqdocnotation{$\star$}} \coqdocvariable{t}\coqref{Groupoid.groupoid.::'[' x ']'}{\coqdocnotation{]}}) : \coqdockw{Type} := \{\coqdoceol
\coqdocindent{1.00em}
\coqdef{Groupoid.groupoid. Dmap}{$\coqdoccst{map}^{\cst{\Pi}}$}{\coqdocprojection{$\coqdoccst{map}^{\cst{\Pi}}$}}      : \coqexternalref{:type scope:'xE2x88x80' x '..' x ',' x}{http://coq.inria.fr/stdlib/Coq.Unicode.Utf8\_core}{\coqdocnotation{∀}} \{\coqdocvar{x} \coqdocvar{y}\} (\coqdocvar{e}: \coqdocvariable{x} \coqref{Groupoid.groupoid.::x 'x7E1' x}{\coqdocnotation{$\sim_1$}} \coqdocvariable{y}\coqexternalref{:type scope:'xE2x88x80' x '..' x ',' x}{http://coq.inria.fr/stdlib/Coq.Unicode.Utf8\_core}{\coqdocnotation{),}} \coqref{Groupoid.groupoid.transport}{\coqdocdefinition{transport}} \coqdocvariable{U} \coqdocvariable{e} \coqref{Groupoid.groupoid.::x '@' x}{\coqdocnotation{$\star$}} \coqref{Groupoid.groupoid.::x '@' x}{\coqdocnotation{(}}\coqdocvariable{f} \coqdocvariable{x}\coqref{Groupoid.groupoid.::x '@' x}{\coqdocnotation{)}} \coqref{Groupoid.groupoid.::x 'x7E1' x}{\coqdocnotation{$\sim_1$}} \coqdocvariable{f} \coqdocvariable{y} ;\coqdoceol
\coqdocindent{1.00em}
\coqdef{Groupoid.groupoid. Dmap id}{$\coqdoccst{map}^\Pi_\mathsf{id}$}{\coqdocprojection{$\coqdoccst{map}^\Pi_\mathsf{id}$}}   : \coqexternalref{:type scope:'xE2x88x80' x '..' x ',' x}{http://coq.inria.fr/stdlib/Coq.Unicode.Utf8\_core}{\coqdocnotation{∀}} \coqdocvar{x}\coqexternalref{:type scope:'xE2x88x80' x '..' x ',' x}{http://coq.inria.fr/stdlib/Coq.Unicode.Utf8\_core}{\coqdocnotation{,}} \coqref{Groupoid.groupoid. Dmap}{\coqdocmethod{$\coqdoccst{map}^{\cst{\Pi}}$}} (\coqref{Groupoid.groupoid.identity}{\coqdocprojection{identity}} \coqdocvariable{x}) \coqref{Groupoid.groupoid.::x 'x7E' x}{\coqdocnotation{$\sim_2$}} \coqref{Groupoid.groupoid.transport id}{\coqdocdefinition{$\mathsf{transport_{id}}$}} \coqdocvariable{U} \coqref{Groupoid.groupoid.::x '@' x}{\coqdocnotation{$\star$}} \coqref{Groupoid.groupoid.::x '@' x}{\coqdocnotation{(}}\coqdocvariable{f} \coqdocvariable{x}\coqref{Groupoid.groupoid.::x '@' x}{\coqdocnotation{)}};\coqdoceol
\coqdocindent{1.00em}
\coqdef{Groupoid.groupoid. Dmap comp}{$\coqdoccst{map}^\Pi_\mathsf{comp}$}{\coqdocprojection{$\coqdoccst{map}^\Pi_\mathsf{comp}$}} : \coqexternalref{:type scope:'xE2x88x80' x '..' x ',' x}{http://coq.inria.fr/stdlib/Coq.Unicode.Utf8\_core}{\coqdocnotation{∀}} \coqdocvar{x} \coqdocvar{y} \coqdocvar{z} (\coqdocvar{e} : \coqdocvariable{x} \coqref{Groupoid.groupoid.::x 'x7E1' x}{\coqdocnotation{$\sim_1$}} \coqdocvariable{y}) (\coqdocvar{e'} : \coqdocvariable{y} \coqref{Groupoid.groupoid.::x 'x7E1' x}{\coqdocnotation{$\sim_1$}} \coqdocvariable{z}\coqexternalref{:type scope:'xE2x88x80' x '..' x ',' x}{http://coq.inria.fr/stdlib/Coq.Unicode.Utf8\_core}{\coqdocnotation{),}}\coqdoceol
\coqdocindent{1.50em}
\coqref{Groupoid.groupoid. Dmap}{\coqdocmethod{$\coqdoccst{map}^{\cst{\Pi}}$}} (\coqdocvariable{e'} \coqref{Groupoid.groupoid.::x 'xC2xB0' x}{\coqdocnotation{$\circ$}} \coqdocvariable{e}) \coqref{Groupoid.groupoid.::x 'x7E2' x}{\coqdocnotation{$\sim_2$}} \coqref{Groupoid.groupoid. Dmap}{\coqdocmethod{$\coqdoccst{map}^{\cst{\Pi}}$}} \coqdocvariable{e'} \coqref{Groupoid.groupoid.::x 'xC2xB0' x}{\coqdocnotation{$\circ$}} \coqref{Groupoid.groupoid.transport map}{\coqdocdefinition{$\mathsf{transport_{map}}$}} \coqdocvariable{U} \coqdocvar{\_} (\coqref{Groupoid.groupoid. Dmap}{\coqdocmethod{$\coqdoccst{map}^{\cst{\Pi}}$}} \coqdocvariable{e}) \coqref{Groupoid.groupoid.::x 'xC2xB0' x}{\coqdocnotation{$\circ$}} \coqdoceol
\coqdocindent{10.50em}
\coqref{Groupoid.groupoid.::x 'xC2xB0' x}{\coqdocnotation{(}}\coqref{Groupoid.groupoid.transport comp}{\coqdocdefinition{$\mathsf{transport_{comp}}$}} \coqdocvariable{U} \coqdocvariable{e} \coqdocvariable{e'} \coqref{Groupoid.groupoid.::x '@' x}{\coqdocnotation{$\star$}} \coqdocvar{\_}\coqref{Groupoid.groupoid.::x 'xC2xB0' x}{\coqdocnotation{)}};\coqdoceol
\coqdocindent{1.00em}
\coqdef{Groupoid.groupoid. Dmap2}{$\coqdoccst{map}^\Pi_2$}{\coqdocprojection{$\coqdoccst{map}^\Pi_2$}}  : \coqexternalref{:type scope:'xE2x88x80' x '..' x ',' x}{http://coq.inria.fr/stdlib/Coq.Unicode.Utf8\_core}{\coqdocnotation{∀}} \coqdocvar{x} \coqdocvar{y} (\coqdocvar{e} \coqdocvar{e'}: \coqdocvariable{x} \coqref{Groupoid.groupoid.::x 'x7E1' x}{\coqdocnotation{$\sim_1$}} \coqdocvariable{y}) (\coqdocvar{H}: \coqdocvariable{e} \coqref{Groupoid.groupoid.::x 'x7E' x}{\coqdocnotation{$\sim_2$}} \coqdocvariable{e'}\coqexternalref{:type scope:'xE2x88x80' x '..' x ',' x}{http://coq.inria.fr/stdlib/Coq.Unicode.Utf8\_core}{\coqdocnotation{),}}\coqdoceol
\coqdocindent{2.00em}
\coqref{Groupoid.groupoid. Dmap}{\coqdocmethod{$\coqdoccst{map}^{\cst{\Pi}}$}} \coqdocvariable{e} \coqref{Groupoid.groupoid.::x 'x7E' x}{\coqdocnotation{$\sim_2$}} \coqref{Groupoid.groupoid. Dmap}{\coqdocmethod{$\coqdoccst{map}^{\cst{\Pi}}$}} \coqdocvariable{e'} \coqref{Groupoid.groupoid.::x 'xC2xB0' x}{\coqdocnotation{$\circ$}} \coqref{Groupoid.groupoid.::x 'xC2xB0' x}{\coqdocnotation{(}}\coqref{Groupoid.groupoid.transport eq}{\coqdocdefinition{$\mathsf{transport_{eq}}$}} \coqdocvariable{U} \coqdocvariable{H} \coqref{Groupoid.groupoid.::x '@' x}{\coqdocnotation{$\star$}} \coqref{Groupoid.groupoid.::x '@' x}{\coqdocnotation{(}}\coqdocvariable{f} \coqdocvariable{x}\coqref{Groupoid.groupoid.::x '@' x}{\coqdocnotation{)}}\coqref{Groupoid.groupoid.::x 'xC2xB0' x}{\coqdocnotation{)}}\}.\coqdoceol
\coqdocemptyline
\coqdocnoindent
\coqdockw{Definition} \coqdef{Groupoid.groupoid.Prod Type}{$\Pi_\coqdoccst{T}$}{\coqdocdefinition{$\Pi_\coqdoccst{T}$}} \coqdocvar{T} (\coqdocvar{U}:\coqref{Groupoid.groupoid.::'[' x ']'}{\coqdocnotation{[}}\coqdocvariable{T} \coqref{Groupoid.groupoid.::x '-->' x}{\coqdocnotation{$\longrightarrow$}} \coqref{Groupoid.groupoid. Type}{\coqdocabbreviation{$\mathsf{Type}_{1}^{1}$}}\coqref{Groupoid.groupoid.::'[' x ']'}{\coqdocnotation{]}}) := \coqdocnotation{\{}\coqdocvar{f} \coqdocnotation{:} \coqexternalref{:type scope:'xE2x88x80' x '..' x ',' x}{http://coq.inria.fr/stdlib/Coq.Unicode.Utf8\_core}{\coqdocnotation{∀}} \coqdocvar{t}\coqexternalref{:type scope:'xE2x88x80' x '..' x ',' x}{http://coq.inria.fr/stdlib/Coq.Unicode.Utf8\_core}{\coqdocnotation{,}} \coqref{Groupoid.groupoid.::'[' x ']'}{\coqdocnotation{[}}\coqdocvariable{U} \coqref{Groupoid.groupoid.::x '@' x}{\coqdocnotation{$\star$}} \coqdocvariable{t}\coqref{Groupoid.groupoid.::'[' x ']'}{\coqdocnotation{]}} \coqdocnotation{\&} \coqref{Groupoid.groupoid.DependentFunctor}{\coqdocclass{$\mathsf{Functor}^\Pi$}} \coqdocvariable{U} \coqdocvar{f}\coqdocnotation{\}}.\coqdoceol
\coqdocemptyline
\end{coqdoccode}


  Equality between dependent functors is given by dependent natural transformations
  and equality at level 2 is given by dependent modifications.
\begin{coqdoccode}
\coqdocemptyline
\coqdocemptyline
\end{coqdoccode}
\noindent We can now equip dependent functors with a groupoid structure
    as we have done for functors.
    We note \coqref{Groupoid.groupoid. Prod}{\coqdocdefinition{$\Pi$}} \coqdocvariable{U} the dependent product over a family of groupoids \coqdocvariable{U}.
\begin{coqdoccode}
\coqdocemptyline
\end{coqdoccode}
A family of setoids can be seen as a family of groupoids using a
lifting that we abusively note   \coqdocvariable{U}$_{\upharpoonright s}$ . We can prove that the
dependent product over a family of setoids is also a setoid. We note
\coqref{Groupoid.groupoid.Prod0}{\coqdocdefinition{$\Pi_0$}} the restriction of \coqref{Groupoid.groupoid. Prod}{\coqdocdefinition{$\Pi$}} to families of setoids.  \begin{coqdoccode}
\coqdocemptyline
\end{coqdoccode}
\subsection{Dependent sums}




\label{sec:sigma} In the interpretation of Σ types, we pay for the
fact that we are missing the 2-dimensional nature of \coqref{Groupoid.groupoid. Type}{\coqdocabbreviation{$\mathsf{Type}_{1}^{1}$}}. Indeed, as
we will need rewriting in the definition of equality on Σ types,
delivering the corresponding groupoid structure requires to reason on
compatibility between rewritings, which amount to the missing
2-dimensional laws. However, as \coqref{Groupoid.groupoid.Type0}{\coqdocdefinition{$\mathsf{Type}_{0}^1$}} is a groupoid, all 2-dimensional
equalities become trivial on a family of setoids, so we can define the
groupoid of Σ types over a groupoid \coqdocvariable{T} and a morphism of type  [\coqdocvariable{T} $\longrightarrow$ \coqref{Groupoid.groupoid.Type0}{\coqdocdefinition{$\mathsf{Type}_{0}^1$}}] .\begin{coqdoccode}
\coqdocemptyline
\coqdocemptyline
\coqdocnoindent
\coqdockw{Definition} \coqdef{Groupoid.groupoid.sum type}{$\Sigma_\coqdoccst{T}$}{\coqdocdefinition{$\Sigma_\coqdoccst{T}$}} \coqdocvar{T} (\coqdocvar{U} : \coqref{Groupoid.groupoid.::'[' x ']'}{\coqdocnotation{[}}\coqdocvariable{T} \coqref{Groupoid.groupoid.::x '-||->' x}{\coqdocnotation{$\longrightarrow$}} \coqref{Groupoid.groupoid.Type0}{\coqdocdefinition{$\mathsf{Type}_{0}^1$}}\coqref{Groupoid.groupoid.::'[' x ']'}{\coqdocnotation{]}}) := \coqdocnotation{\{}\coqdocvar{t} \coqdocnotation{:} \coqref{Groupoid.groupoid.::'[' x ']'}{\coqdocnotation{[}}\coqdocvariable{T}\coqref{Groupoid.groupoid.::'[' x ']'}{\coqdocnotation{]}} \coqdocnotation{\&} \coqref{Groupoid.groupoid.::'[' x ']'}{\coqdocnotation{[}}\coqdocvariable{U} \coqref{Groupoid.groupoid.::x '@' x}{\coqdocnotation{$\star$}} \coqdocvar{t}\coqref{Groupoid.groupoid.::'[' x ']'}{\coqdocnotation{]}}\coqdocnotation{\}}.\coqdoceol
\coqdocemptyline
\end{coqdoccode}
\noindent
  The 1-equality between dependent pairs is given by 1-equality on the
  first and second projections, with a transport on the second
  projection on the left.
\begin{coqdoccode}
\coqdocemptyline
\coqdocnoindent
\coqdockw{Definition} \coqdef{Groupoid.groupoid.sum eq}{$\Sigma_\coqdoccst{Eq}$}{\coqdocdefinition{$\Sigma_\coqdoccst{Eq}$}} \coqdocvar{T} (\coqdocvar{U} : \coqref{Groupoid.groupoid.::'[' x ']'}{\coqdocnotation{[}}\coqdocvariable{T} \coqref{Groupoid.groupoid.::x '-||->' x}{\coqdocnotation{$\longrightarrow$}} \coqref{Groupoid.groupoid.Type0}{\coqdocdefinition{$\mathsf{Type}_{0}^1$}}\coqref{Groupoid.groupoid.::'[' x ']'}{\coqdocnotation{]}}) : \coqref{Groupoid.groupoid.HomT}{\coqdocdefinition{$\mathsf{HomSet}$}} (\coqref{Groupoid.groupoid.sum type}{\coqdocdefinition{$\Sigma_\coqdoccst{T}$}} \coqdocvariable{U}) := \coqdoceol
\coqdocindent{1.00em}
\coqexternalref{::'xCExBB' x '..' x ',' x}{http://coq.inria.fr/stdlib/Coq.Unicode.Utf8\_core}{\coqdocnotation{\ensuremath{\lambda}}} \coqdocvar{m} \coqdocvar{n}\coqexternalref{::'xCExBB' x '..' x ',' x}{http://coq.inria.fr/stdlib/Coq.Unicode.Utf8\_core}{\coqdocnotation{,}} \coqdocnotation{\{}\coqdocvar{P} \coqdocnotation{:} \coqref{Groupoid.groupoid.::'[' x ']'}{\coqdocnotation{[}}\coqdocvariable{m}\coqref{Groupoid.groupoid.::'[' x ']'}{\coqdocnotation{]}} \coqref{Groupoid.groupoid.::x 'x7E1' x}{\coqdocnotation{$\sim_1$}} \coqref{Groupoid.groupoid.::'[' x ']'}{\coqdocnotation{[}}\coqdocvariable{n}\coqref{Groupoid.groupoid.::'[' x ']'}{\coqdocnotation{]}} \coqdocnotation{\&} \coqref{Groupoid.groupoid.transport}{\coqdocdefinition{transport}} (\coqref{Groupoid.groupoid.::'[[[' x ']]]'}{\coqdocnotation{ }}\coqdocvariable{U}\coqref{Groupoid.groupoid.::'[[[' x ']]]'}{\coqdocnotation{$_{\upharpoonright s}$}}) \coqdocvar{P} \coqref{Groupoid.groupoid.::x '@' x}{\coqdocnotation{$\star$}} \coqref{Groupoid.groupoid.::x '@' x}{\coqdocnotation{(}}\coqref{Groupoid.groupoid.xCExA02}{\coqdocabbreviation{$\pi_2$}} \coqdocvariable{m}\coqref{Groupoid.groupoid.::x '@' x}{\coqdocnotation{)}} \coqref{Groupoid.groupoid.::x 'x7E1' x}{\coqdocnotation{$\sim_1$}} \coqref{Groupoid.groupoid.xCExA02}{\coqdocabbreviation{$\pi_2$}} \coqdocvariable{n}\coqdocnotation{\}}.\coqdoceol
\coqdocemptyline
\end{coqdoccode}
\noindent
  In the same way, 2-equality between 1-equalities is given by projections
  and rewriting.
\begin{coqdoccode}
\coqdocemptyline
\coqdocnoindent
\coqdockw{Definition} \coqdef{Groupoid.groupoid.sum eq2}{$\Sigma_{\coqdoccst{Eq}_2}$}{\coqdocdefinition{$\Sigma_{\coqdoccst{Eq}_2}$}} \coqdocvar{T} (\coqdocvar{U} : \coqref{Groupoid.groupoid.::'[' x ']'}{\coqdocnotation{[}}\coqdocvariable{T} \coqref{Groupoid.groupoid.::x '-||->' x}{\coqdocnotation{$\longrightarrow$}} \coqref{Groupoid.groupoid.Type0}{\coqdocdefinition{$\mathsf{Type}_{0}^1$}}\coqref{Groupoid.groupoid.::'[' x ']'}{\coqdocnotation{]}}) (\coqdocvar{M} \coqdocvar{N} : \coqref{Groupoid.groupoid.sum type}{\coqdocdefinition{$\Sigma_\coqdoccst{T}$}} \coqdocvariable{U}) : \coqref{Groupoid.groupoid.HomT}{\coqdocdefinition{$\mathsf{HomSet}$}} (\coqdocvariable{M} \coqref{Groupoid.groupoid.::x 'x7E1' x}{\coqdocnotation{$\sim_1$}} \coqdocvariable{N}) \coqdoceol
\coqdocindent{1.00em}
:= \coqexternalref{::'xCExBB' x '..' x ',' x}{http://coq.inria.fr/stdlib/Coq.Unicode.Utf8\_core}{\coqdocnotation{\ensuremath{\lambda}}} \coqdocvar{e} \coqdocvar{e'} \coqexternalref{::'xCExBB' x '..' x ',' x}{http://coq.inria.fr/stdlib/Coq.Unicode.Utf8\_core}{\coqdocnotation{,}} \coqdocnotation{\{}\coqdocvar{P} \coqdocnotation{:} \coqref{Groupoid.groupoid.::'[' x ']'}{\coqdocnotation{[}}\coqdocvariable{e}\coqref{Groupoid.groupoid.::'[' x ']'}{\coqdocnotation{]}} \coqref{Groupoid.groupoid.::x 'x7E' x}{\coqdocnotation{$\sim_2$}} \coqref{Groupoid.groupoid.::'[' x ']'}{\coqdocnotation{[}}\coqdocvariable{e'}\coqref{Groupoid.groupoid.::'[' x ']'}{\coqdocnotation{]}} \coqdocnotation{\&} \coqref{Groupoid.groupoid.xCExA02}{\coqdocabbreviation{$\pi_2$}} \coqdocvariable{e} \coqref{Groupoid.groupoid.::x 'x7E' x}{\coqdocnotation{$\sim_2$}} \coqref{Groupoid.groupoid.xCExA02}{\coqdocabbreviation{$\pi_2$}} \coqdocvariable{e'} \coqref{Groupoid.groupoid.::x 'xC2xB0' x}{\coqdocnotation{$\circ$}} \coqref{Groupoid.groupoid.::x 'xC2xB0' x}{\coqdocnotation{(}}\coqref{Groupoid.groupoid.transport eq}{\coqdocdefinition{$\mathsf{transport_{eq}}$}} (\coqref{Groupoid.groupoid.::'[[[' x ']]]'}{\coqdocnotation{ }}\coqdocvariable{U}\coqref{Groupoid.groupoid.::'[[[' x ']]]'}{\coqdocnotation{$_{\upharpoonright s}$}}) \coqdocvar{P} \coqref{Groupoid.groupoid.::x '@' x}{\coqdocnotation{$\star$}} \coqref{Groupoid.groupoid.::x '@' x}{\coqdocnotation{(}}\coqref{Groupoid.groupoid.xCExA02}{\coqdocabbreviation{$\pi_2$}} \coqdocvariable{M}\coqref{Groupoid.groupoid.::x '@' x}{\coqdocnotation{)}}\coqref{Groupoid.groupoid.::x 'xC2xB0' x}{\coqdocnotation{)}}\coqdocnotation{\}}.\coqdoceol
\coqdocemptyline
\coqdocemptyline
\end{coqdoccode}
\noindent This way, we can define the groupoid \coqref{Groupoid.groupoid. Sum0}{\coqdocdefinition{$\Sigma$}} \coqdocvariable{U} of dependent sums for any family of setoids. When \coqdocvariable{T} is a setoid, \coqref{Groupoid.groupoid. Sum0}{\coqdocdefinition{$\Sigma$}} \coqdocvariable{U} is also a setoid.
\begin{coqdoccode}
\end{coqdoccode}


\section{The setoid interpretation}
\label{sec:interpretation}


\coqlibrary{groupoid interpretation def}{Library }{groupoid\_interpretation\_def}

\begin{coqdoccode}
\end{coqdoccode}


  The interpretation is based on the Takeuti-Gandy interpretation of
  simple type theory, recently generalized to dependent type theory in
  \cite{barras:_gener_takeut_gandy_inter} using Kan semisimplicial
  sets. There are two main novelties in our interpretation. First, we
  take advantage of universe polymorphism to interpret dependent types
  directly as functors into \coqdocdefinition{$\mathsf{Type}_{0}^1$}. Second, we provide an
  interpretation in a model where structures that are definitionally
  equal for Kan semisimplicial sets are only homotopically equal, which
  requires more care in the definitions (see for instance the definition
  of \coqref{groupoid interpretation def.Lam}{\coqdocdefinition{Lam}} in Section \ref{sec:interp} which mixes two points of view
  on fibrations).


  We only present the computational part of the interpretation, the
  proofs of functoriality and naturality are not detailled but most of
  them are available in the \Coq development. We have admitted some of
  these administrative compatibility lemmas.


\subsection{Dependent types}




  The judgment context $\Gamma \vdash$ of Section
  \ref{sec:definitions} is represented in \Coq as a groupoid, noted
  \coqdockw{Context} := \coqdocdefinition{$\mathsf{Type_1}$}. The empty context (Rule \textsc{Empty})
  is interpreted as the groupoid with exactly one element at each
  dimension.  Types in a context \coqdocvariable{Γ}, noted \coqref{groupoid interpretation def.Typ}{\coqdocdefinition{Typ}} \coqdocvariable{Γ}, are (context)
  functors from \coqdocvariable{Γ} to the groupoid of setoids \coqdocdefinition{$\mathsf{Type}_{0}^1$}.  Thus, a
  judgment $\Gamma \vdash A : \Type{}$ is represented as a term \coqdocvariable{A} of
  type \coqref{groupoid interpretation def.Typ}{\coqdocdefinition{Typ}} \coqdocvariable{Γ}. Context extension (Rule \textsc{Decl}) is given by
  dependent sums, i.e., the judgment $\Gamma, x:A \vdash$ is represented
  as \coqdocdefinition{$\Sigma$} \coqdocvariable{A}.


\begin{coqdoccode}
\end{coqdoccode}
Elements of \coqdocvariable{A} introduced by a sequent $\Gamma \vdash x:A$ are
  dependent (context) functors from \coqdocvariable{Γ} to \coqdocvariable{A} that return for each
  context valuation \coqdocvariable{$\gamma$}, an object of \coqdocvariable{A} $\star$ \coqdocvariable{$\gamma$} respecting equality of
  contexts.  The type of elements of \coqdocvariable{A} is noted \coqref{groupoid interpretation def.Elt}{\coqdocdefinition{Elt}} \coqdocvariable{A} := [\coqdocdefinition{$\Pi$} \coqdocvariable{A}]
  (context is implicit).  

  A dependent type $\Gamma, x:A \vdash B$ is interpreted in two
  equivalent ways: simply as a type \coqref{groupoid interpretation def.TypDep}{\coqdocdefinition{TypDep}} \coqdocvariable{A} := \coqref{groupoid interpretation def.Typ}{\coqdocdefinition{Typ}} (\coqdocdefinition{$\Sigma$} \coqdocvariable{A}) over the
  dependent sum of \coqdocvariable{Γ} and \coqdocvariable{A} or as a type family \coqref{groupoid interpretation def.TypFam}{\coqdocdefinition{TypFam}} \coqdocvariable{A} over \coqdocvariable{A}
  (corresponding to a family of sets in constructive mathematics). A
  type family can be seen as a fibration (or bundle) from \coqdocvariable{B} to \coqdocvariable{A}.
  In what follows, the indice $\mathsf{_{comp}}$ is given to proofs of 
  (dependent) functoriality.
\begin{coqdoccode}
\coqdocemptyline
\coqdocnoindent
\coqdockw{Definition} \coqdef{groupoid interpretation def.TypFam}{TypFam}{\coqdocdefinition{TypFam}} \{\coqdocvar{Γ} : \coqref{groupoid interpretation def.Context}{\coqdocdefinition{Context}}\} (\coqdocvar{A}: \coqref{groupoid interpretation def.Typ}{\coqdocdefinition{Typ}} \coqdocvariable{Γ}) := \coqdoceol
\coqdocindent{1.00em}
\coqdocnotation{[}\coqdocdefinition{$\Pi$} \coqdocnotation{(}\coqexternalref{::'xCExBB' x '..' x ',' x}{http://coq.inria.fr/stdlib/Coq.Unicode.Utf8\_core}{\coqdocnotation{\ensuremath{\lambda}}} \coqdocvar{$\gamma$}\coqexternalref{::'xCExBB' x '..' x ',' x}{http://coq.inria.fr/stdlib/Coq.Unicode.Utf8\_core}{\coqdocnotation{,}} \coqdocnotation{ } \coqdocnotation{(}\coqdocvariable{A} \coqdocnotation{$\star$} \coqdocvariable{$\gamma$}\coqdocnotation{)} \coqdocnotation{$_{\upharpoonright s}$} \coqdocnotation{$\longrightarrow$} \coqdocdefinition{$\mathsf{Type}_{0}^1$}\coqdocnotation{;} \coqref{groupoid interpretation def.TypFam 1}{\coqdocinstance{$\mathsf{TypFam_{comp}}$}} \coqdocvar{\_}\coqdocnotation{)}\coqdocnotation{]}.\coqdoceol
\coqdocemptyline
\end{coqdoccode}


  Elements of \coqref{groupoid interpretation def.TypDep}{\coqdocdefinition{TypDep}} \coqdocvariable{A} and \coqref{groupoid interpretation def.TypFam}{\coqdocdefinition{TypFam}} \coqdocvariable{A} can be related using a dependent closure
  at the level of types. In the interpretation of typing judgments, this connection 
  will be used to switch between the fibration and the morphism points of view.
\begin{coqdoccode}
\coqdocemptyline
\coqdocnoindent
\coqdockw{Definition} \coqdef{groupoid interpretation def.LamT}{LamT}{\coqdocdefinition{LamT}} \{\coqdocvar{Γ}: \coqref{groupoid interpretation def.Context}{\coqdocdefinition{Context}}\} \{\coqdocvar{A} : \coqref{groupoid interpretation def.Typ}{\coqdocdefinition{Typ}} \coqdocvariable{Γ}\} (\coqdocvar{B}: \coqref{groupoid interpretation def.UTypDep}{\coqdocdefinition{UTypDep}} \coqdocvariable{A})\coqdoceol
\coqdocindent{1.00em}
: \coqref{groupoid interpretation def.UTypFam}{\coqdocdefinition{UTypFam}} (\coqref{groupoid interpretation def.::'[[[' x ']]]'}{\coqdocnotation{ }}\coqdocvariable{A}\coqref{groupoid interpretation def.::'[[[' x ']]]'}{\coqdocnotation{$_{\upharpoonright s}$}}) := \coqdocnotation{(}\coqexternalref{::'xCExBB' x '..' x ',' x}{http://coq.inria.fr/stdlib/Coq.Unicode.Utf8\_core}{\coqdocnotation{\ensuremath{\lambda}}} \coqdocvar{$\gamma$}\coqexternalref{::'xCExBB' x '..' x ',' x}{http://coq.inria.fr/stdlib/Coq.Unicode.Utf8\_core}{\coqdocnotation{,}} \coqdocnotation{(}\coqexternalref{::'xCExBB' x '..' x ',' x}{http://coq.inria.fr/stdlib/Coq.Unicode.Utf8\_core}{\coqdocnotation{\ensuremath{\lambda}}} \coqdocvar{t}\coqexternalref{::'xCExBB' x '..' x ',' x}{http://coq.inria.fr/stdlib/Coq.Unicode.Utf8\_core}{\coqdocnotation{,}} \coqdocvariable{B} \coqdocnotation{$\star$} \coqdocnotation{(}\coqdocvariable{$\gamma$}\coqdocnotation{;} \coqdocvariable{t}\coqdocnotation{)} \coqdocnotation{;} \coqdocvar{\_}\coqdocnotation{);} \coqref{groupoid interpretation def.LamT 1}{\coqdocaxiom{$\mathsf{LamT_{comp}}$}} \coqdocvariable{B}\coqdocnotation{)}.\coqdoceol
\coqdocemptyline
\end{coqdoccode}


\subsection{Substitutions}




  A substitution is represented by a context morphism [\coqdocvariable{Γ} $\longrightarrow$ \coqdocvariable{Δ}]. 
  A substitution σ can be extended by an element \coqdocvariable{a}: \coqref{groupoid interpretation def.Elt}{\coqdocdefinition{Elt}} (\coqdocvariable{A} $\circ$ σ) 
  of \coqdocvariable{A} : \coqref{groupoid interpretation def.Typ}{\coqdocdefinition{Typ}} \coqdocvariable{Δ}.


\begin{coqdoccode}
\coqdocemptyline
\coqdocnoindent
\coqdockw{Definition} \coqdef{groupoid interpretation def.SubExt}{SubExt}{\coqdocdefinition{SubExt}} \{\coqdocvar{Γ} \coqdocvar{Δ} : \coqref{groupoid interpretation def.Context}{\coqdocdefinition{Context}}\} \{\coqdocvar{A} : \coqref{groupoid interpretation def.Typ}{\coqdocdefinition{Typ}} \coqdocvariable{Δ}\} (σ: \coqdocnotation{[}\coqdocvariable{Γ} \coqdocnotation{$\longrightarrow$} \coqdocvariable{Δ}\coqdocnotation{]}) (\coqdocvar{a}: \coqref{groupoid interpretation def.Elt}{\coqdocdefinition{Elt}} (\coqdocvariable{A} \coqref{groupoid interpretation def.::x 'xE2x8Bx85xE2x8Bx85' x}{\coqdocnotation{$⋅$}} \coqdocvariable{σ})) \coqdoceol
\coqdocindent{1.00em}
: \coqdocnotation{[}\coqdocvariable{Γ} \coqdocnotation{$\longrightarrow$} \coqdocdefinition{\_Sum1} \coqdocvariable{A} \coqdocnotation{]} := \coqdocnotation{(}\coqexternalref{::'xCExBB' x '..' x ',' x}{http://coq.inria.fr/stdlib/Coq.Unicode.Utf8\_core}{\coqdocnotation{\ensuremath{\lambda}}} \coqdocvar{$\gamma$}\coqexternalref{::'xCExBB' x '..' x ',' x}{http://coq.inria.fr/stdlib/Coq.Unicode.Utf8\_core}{\coqdocnotation{,}} \coqdocnotation{(}\coqdocvariable{σ} \coqdocnotation{$\star$} \coqdocvariable{$\gamma$}\coqdocnotation{;} \coqdocvariable{a} \coqdocnotation{$\star$} \coqdocvariable{$\gamma$}\coqdocnotation{)} \coqdocnotation{;} \coqref{groupoid interpretation def.SubExt 1}{\coqdocinstance{$\mathsf{SubExt_{comp}}$}} \coqdocvar{\_} \coqdocvar{\_}\coqdocnotation{)}.\coqdoceol
\coqdocemptyline
\end{coqdoccode}
\noindent where \coqref{groupoid interpretation def.SubExt 1}{\coqdocinstance{$\mathsf{SubExt_{comp}}$}} is a proof that it is functorial. 
\begin{coqdoccode}
\coqdocemptyline
\coqdocnoindent
\coqdockw{Definition} \coqdef{groupoid interpretation def.substF}{substF}{\coqdocdefinition{substF}} \{\coqdocvar{T} \coqdocvar{Γ}\} \{\coqdocvar{A}:\coqref{groupoid interpretation def.Typ}{\coqdocdefinition{Typ}} \coqdocvariable{Γ}\} (\coqdocvar{F}:\coqref{groupoid interpretation def.UTypFam}{\coqdocdefinition{UTypFam}} (\coqref{groupoid interpretation def.::'[[[' x ']]]'}{\coqdocnotation{ }}\coqdocvariable{A}\coqref{groupoid interpretation def.::'[[[' x ']]]'}{\coqdocnotation{$_{\upharpoonright s}$}})) (σ:\coqdocnotation{[}\coqdocvariable{T} \coqdocnotation{$\longrightarrow$} \coqdocvariable{Γ}\coqdocnotation{]}) : \coqref{groupoid interpretation def.UTypFam}{\coqdocdefinition{UTypFam}} (\coqref{groupoid interpretation def.::'[[[' x ']]]'}{\coqdocnotation{ }}\coqdocvariable{A} \coqref{groupoid interpretation def.::x 'xE2x8Bx85xE2x8Bx85' x}{\coqdocnotation{$⋅$}} \coqdocvariable{σ}\coqref{groupoid interpretation def.::'[[[' x ']]]'}{\coqdocnotation{$_{\upharpoonright s}$}}) \coqdoceol
\coqdocindent{1.00em}
:= \coqdocnotation{(}\coqdocnotation{[}\coqdocvariable{F} \coqref{groupoid interpretation def.::x 'xC2xB0xC2xB0' x}{\coqdocnotation{$\circ$}} \coqdocvariable{σ}\coqdocnotation{]} : \coqexternalref{:type scope:'xE2x88x80' x '..' x ',' x}{http://coq.inria.fr/stdlib/Coq.Unicode.Utf8\_core}{\coqdocnotation{∀}} \coqdocvar{t} : \coqdocnotation{[}\coqdocvariable{T}\coqdocnotation{]}\coqexternalref{:type scope:'xE2x88x80' x '..' x ',' x}{http://coq.inria.fr/stdlib/Coq.Unicode.Utf8\_core}{\coqdocnotation{,}} \coqdocnotation{\ensuremath{|}}\coqdocnotation{(}\coqref{groupoid interpretation def.::'[[[' x ']]]'}{\coqdocnotation{ }}\coqdocvariable{A} \coqref{groupoid interpretation def.::x 'xE2x8Bx85xE2x8Bx85' x}{\coqdocnotation{$⋅$}} \coqdocvariable{σ}\coqref{groupoid interpretation def.::'[[[' x ']]]'}{\coqdocnotation{$_{\upharpoonright s}$}}\coqdocnotation{)} \coqdocnotation{$\star$} \coqdocvariable{t}\coqdocnotation{\ensuremath{|}}\coqdocnotation{g} \coqdocnotation{$\longrightarrow$} \coqdocdefinition{Type1}\coqdocnotation{;} \coqref{groupoid interpretation def.substF 1}{\coqdocinstance{$\mathsf{substF_{comp}}$}} \coqdocvariable{F} \coqdocvariable{σ}\coqdocnotation{)}.\coqdoceol
\coqdocemptyline
\end{coqdoccode}
A substitution σ can be applied to a type family \coqdocvariable{F} using the
  composition of a functor with a dependent functor. We
  abusively note all those different compositions with $\circ$ as it is done in
  mathematics, whereas they are distinct operators in the \Coq
  development.
  The weakening substitution of $\Gamma, x:A \vdash$ is given by the first
  projection. This allows to interpret a type A in a weakened context, 
  which is noted  $\shortuparrow$ \coqdocvariable{A}. \begin{coqdoccode}
\coqdocemptyline
\end{coqdoccode}


  A type family \coqdocvariable{F} in \coqref{groupoid interpretation def.TypFam}{\coqdocdefinition{TypFam}} \coqdocvariable{A} can be partially substituted with an
  element \coqdocvariable{a} in \coqref{groupoid interpretation def.Elt}{\coqdocdefinition{Elt}} \coqdocvariable{A}, noted \coqdocvariable{F} \{\{\coqdocvariable{a}\}\}, to get its value (a type) at
  \coqdocvariable{a}. This process is defined as \coqdocvariable{F} \{\{\coqdocvariable{a}\}\} := (\coqdocvar{\ensuremath{\lambda}} \coqdocvariable{$\gamma$}, (\coqdocvariable{F} $\star$ \coqdocvariable{$\gamma$}) $\star$ (\coqdocvariable{a} $\star$ \coqdocvariable{$\gamma$}) ;
  \coqdocvar{\_}) (where \coqdocvar{\_} is a proof it is functorial). Note that this
  pattern of application \emph{up-to a context $\gamma$} will be used
  later to defined other notions of application. Although the
  computational definitions are the same, the compatibility conditions
  are always different.  This notion of partial substitution in a type
  family enables to state that \coqref{groupoid interpretation def.LamT}{\coqdocdefinition{LamT}} defines a type level
  $\lambda$-abstraction.  \begin{coqdoccode}
\coqdocemptyline
\coqdocnoindent
\coqdockw{Definition} \coqdef{groupoid interpretation def.BetaT}{BetaT}{\coqdocdefinition{BetaT}} \coqdocvar{Δ} \coqdocvar{Γ} (\coqdocvar{A}:\coqref{groupoid interpretation def.Typ}{\coqdocdefinition{Typ}} \coqdocvariable{Γ}) (\coqdocvar{B}:\coqref{groupoid interpretation def.UTypDep}{\coqdocdefinition{UTypDep}} \coqdocvariable{A}) (σ:\coqdocnotation{[}\coqdocvariable{Δ} \coqdocnotation{$\longrightarrow$} \coqdocvariable{Γ}\coqdocnotation{]}) (\coqdocvar{a}:\coqref{groupoid interpretation def.Elt}{\coqdocdefinition{Elt}} (\coqdocvariable{A} \coqref{groupoid interpretation def.::x 'xE2x8Bx85xE2x8Bx85' x}{\coqdocnotation{$⋅$}} \coqdocvariable{σ})) \coqdoceol
\coqdocindent{1.00em}
: \coqref{groupoid interpretation def.LamT}{\coqdocdefinition{LamT}} \coqdocvariable{B} \coqref{groupoid interpretation def.::x 'xC2xB0xC2xB0xC2xB0' x}{\coqdocnotation{$\circ$}} \coqdocvariable{σ} \coqref{groupoid interpretation def.::x 'x7Bx7Bx7B' x 'x7Dx7Dx7D'}{\coqdocnotation{\{\{\{}}\coqdocvariable{a}\coqref{groupoid interpretation def.::x 'x7Bx7Bx7B' x 'x7Dx7Dx7D'}{\coqdocnotation{\}\}\}}} \coqdocnotation{$\sim_1$} \coqdocvariable{B} \coqref{groupoid interpretation def.::x 'xE2x8Bx85xE2x8Bx85xE2x8Bx85' x}{\coqdocnotation{$⋅$}} \coqref{groupoid interpretation def.::x 'xE2x8Bx85xE2x8Bx85xE2x8Bx85' x}{\coqdocnotation{(}}\coqref{groupoid interpretation def.SubExt}{\coqdocdefinition{SubExt}} \coqdocvariable{σ} \coqdocvariable{a}\coqref{groupoid interpretation def.::x 'xE2x8Bx85xE2x8Bx85xE2x8Bx85' x}{\coqdocnotation{)}} := \coqdocnotation{(}\coqexternalref{::'xCExBB' x '..' x ',' x}{http://coq.inria.fr/stdlib/Coq.Unicode.Utf8\_core}{\coqdocnotation{\ensuremath{\lambda}}} \coqdocvar{\_}\coqexternalref{::'xCExBB' x '..' x ',' x}{http://coq.inria.fr/stdlib/Coq.Unicode.Utf8\_core}{\coqdocnotation{,}} \coqdocmethod{identity} \coqdocvar{\_} \coqdocnotation{;} \coqref{groupoid interpretation def.BetaT 1}{\coqdocaxiom{$\mathsf{BetaT_{comp}}$}} \coqdocvar{\_} \coqdocvar{\_} \coqdocvar{\_}\coqdocnotation{)}.\coqdoceol
\coqdocemptyline
\end{coqdoccode}


\subsection{Interpretation of the typing judgment}


  \label{sec:interp}


  The typing rules of \TTu\xspace of Figure \ref{fig:emltt} are
  interpreted in the groupoid model as described below.


  \paragraph{\textsc{Var}.} 


  The rule \textsc{Var} is given by the second projection plus a proof
  that the projection is dependently functorial. Note the explicit
  weakening of \coqdocvariable{A} in the returned type. This is because we need to
  make explicit that the context used to type \coqdocvariable{A} is extended with an
  element of type \coqdocvariable{A}. The rule of Figure \ref{fig:emltt} is more general 
  as it performs an implicit weakening. We do not interpret this part of 
  the rule as weakening is explicit in our model. 


\begin{coqdoccode}
\coqdocemptyline
\coqdocnoindent
\coqdockw{Definition} \coqdef{groupoid interpretation def.Var}{Var}{\coqdocdefinition{Var}} \{\coqdocvar{Γ}\} (\coqdocvar{A}:\coqref{groupoid interpretation def.Typ}{\coqdocdefinition{Typ}} \coqdocvariable{Γ}) : \coqref{groupoid interpretation def.Elt}{\coqdocdefinition{Elt}} \coqref{groupoid interpretation def.::'xE2x87x91' x}{\coqdocnotation{$\shortuparrow$}}\coqdocvariable{A} := \coqdocnotation{(}\coqexternalref{::'xCExBB' x '..' x ',' x}{http://coq.inria.fr/stdlib/Coq.Unicode.Utf8\_core}{\coqdocnotation{\ensuremath{\lambda}}} \coqdocvar{t}\coqexternalref{::'xCExBB' x '..' x ',' x}{http://coq.inria.fr/stdlib/Coq.Unicode.Utf8\_core}{\coqdocnotation{,}} \coqdocabbreviation{$\pi_2$} \coqdocvariable{t}\coqdocnotation{;} \coqref{groupoid interpretation def.Var 1}{\coqdocinstance{$\mathsf{Var_{comp}}$}} \coqdocvariable{A}\coqdocnotation{)}.\coqdoceol
\coqdocemptyline
\coqdocemptyline
\end{coqdoccode}
\paragraph{\textsc{Prod}.} The rule \textsc{Prod} is interpreted
  using the dependent functor space, plus a proof that equivalent
  contexts give rise to isomorphic dependent functor spaces.  Note that
  the rule is defined on type families and not on the dependent type
  formulation because here we need a fibration point of view. \begin{coqdoccode}
\coqdocemptyline
\coqdocnoindent
\coqdockw{Definition} \coqdef{groupoid interpretation def.Prod}{Prod}{\coqdocdefinition{Prod}} \{\coqdocvar{Γ}\} (\coqdocvar{A}:\coqref{groupoid interpretation def.Typ}{\coqdocdefinition{Typ}} \coqdocvariable{Γ}) (\coqdocvar{F}:\coqref{groupoid interpretation def.UTypFam}{\coqdocdefinition{UTypFam}} (\coqref{groupoid interpretation def.::'[[[' x ']]]'}{\coqdocnotation{ }}\coqdocvariable{A}\coqref{groupoid interpretation def.::'[[[' x ']]]'}{\coqdocnotation{$_{\upharpoonright s}$}})) \coqdoceol
\coqdocindent{1.00em}
: \coqref{groupoid interpretation def.UTyp}{\coqdocdefinition{UTyp}} \coqdocvariable{Γ} := \coqdocnotation{(}\coqexternalref{::'xCExBB' x '..' x ',' x}{http://coq.inria.fr/stdlib/Coq.Unicode.Utf8\_core}{\coqdocnotation{\ensuremath{\lambda}}} \coqdocvar{s}\coqexternalref{::'xCExBB' x '..' x ',' x}{http://coq.inria.fr/stdlib/Coq.Unicode.Utf8\_core}{\coqdocnotation{,}} \coqdocdefinition{Prod1} (\coqdocvariable{F} \coqdocnotation{$\star$} \coqdocvariable{s})\coqdocnotation{;} \coqref{groupoid interpretation def.Prod 1}{\coqdocaxiom{$\mathsf{Prod_{comp}}$}} \coqdocvariable{A} \coqdocvariable{F}\coqdocnotation{)}.\coqdoceol
\coqdocemptyline
\end{coqdoccode}
  \paragraph{\textsc{App}.}


  The rule \textsc{App} is interpreted using an up-to context application 
  and a proof of dependent functoriality. We abusively note \coqdocvar{M} $\star$ \coqdocvar{N} the application 
  of \coqref{groupoid interpretation def.App}{\coqdocdefinition{App}}.
\begin{coqdoccode}
\coqdocemptyline
\coqdocnoindent
\coqdockw{Definition} \coqdef{groupoid interpretation def.App}{App}{\coqdocdefinition{App}} \{\coqdocvar{Γ}\} \{\coqdocvar{A}:\coqref{groupoid interpretation def.Typ}{\coqdocdefinition{Typ}} \coqdocvariable{Γ}\} \{\coqdocvar{F}:\coqref{groupoid interpretation def.UTypFam}{\coqdocdefinition{UTypFam}} (\coqref{groupoid interpretation def.::'[[[' x ']]]'}{\coqdocnotation{ }}\coqdocvariable{A}\coqref{groupoid interpretation def.::'[[[' x ']]]'}{\coqdocnotation{$_{\upharpoonright s}$}})\} (\coqdocvar{c}:\coqref{groupoid interpretation def.UElt}{\coqdocdefinition{UElt}} (\coqref{groupoid interpretation def.Prod}{\coqdocdefinition{Prod}} \coqdocvariable{F})) (\coqdocvar{a}:\coqref{groupoid interpretation def.Elt}{\coqdocdefinition{Elt}} \coqdocvariable{A}) \coqdoceol
\coqdocindent{1.00em}
: \coqref{groupoid interpretation def.UElt}{\coqdocdefinition{UElt}} (\coqdocvariable{F} \coqref{groupoid interpretation def.::x 'x7Bx7Bx7B' x 'x7Dx7Dx7D'}{\coqdocnotation{\{\{\{}}\coqdocvariable{a}\coqref{groupoid interpretation def.::x 'x7Bx7Bx7B' x 'x7Dx7Dx7D'}{\coqdocnotation{\}\}\}}}) := \coqdocnotation{(}\coqexternalref{::'xCExBB' x '..' x ',' x}{http://coq.inria.fr/stdlib/Coq.Unicode.Utf8\_core}{\coqdocnotation{\ensuremath{\lambda}}} \coqdocvar{s}\coqexternalref{::'xCExBB' x '..' x ',' x}{http://coq.inria.fr/stdlib/Coq.Unicode.Utf8\_core}{\coqdocnotation{,}} \coqdocnotation{(}\coqdocvariable{c} \coqdocnotation{$\star$} \coqdocvariable{s}\coqdocnotation{)} \coqdocnotation{$\star$} \coqdocnotation{(}\coqdocvariable{a} \coqdocnotation{$\star$} \coqdocvariable{s}\coqdocnotation{)}\coqdocnotation{;} \coqref{groupoid interpretation def.App 1}{\coqdocinstance{$\mathsf{App_{comp}}$}} \coqdocvariable{c} \coqdocvariable{a}\coqdocnotation{)}.\coqdoceol
\coqdocemptyline
\end{coqdoccode}
  \paragraph{\lrule{Lam}.}


  Term-level $\lambda$-abstraction is defined with the same
  computational meaning as type-level $\lambda$-abstraction, but it
  differs on the proof of dependent functoriality. Note that we use
  \coqref{groupoid interpretation def.LamT}{\coqdocdefinition{LamT}} in the definition because we need both the fibration (for
  \coqref{groupoid interpretation def.Prod}{\coqdocdefinition{Prod}}) and the morphism (for \coqref{groupoid interpretation def.Elt}{\coqdocdefinition{Elt}} \coqdocvariable{B}) point of view. 
\begin{coqdoccode}
\coqdocemptyline
\coqdocnoindent
\coqdockw{Definition} \coqdef{groupoid interpretation def.Lam}{Lam}{\coqdocdefinition{Lam}} \{\coqdocvar{Γ}\} \{\coqdocvar{A}:\coqref{groupoid interpretation def.Typ}{\coqdocdefinition{Typ}} \coqdocvariable{Γ}\} \{\coqdocvar{B}:\coqref{groupoid interpretation def.UTypDep}{\coqdocdefinition{UTypDep}} \coqdocvariable{A}\} (\coqdocvar{b}:\coqref{groupoid interpretation def.UElt}{\coqdocdefinition{UElt}} \coqdocvariable{B})\coqdoceol
\coqdocindent{1.00em}
: \coqref{groupoid interpretation def.UElt}{\coqdocdefinition{UElt}} (\coqref{groupoid interpretation def.Prod}{\coqdocdefinition{Prod}} (\coqref{groupoid interpretation def.LamT}{\coqdocdefinition{LamT}} \coqdocvariable{B})) := \coqdocnotation{(}\coqexternalref{::'xCExBB' x '..' x ',' x}{http://coq.inria.fr/stdlib/Coq.Unicode.Utf8\_core}{\coqdocnotation{\ensuremath{\lambda}}} \coqdocvar{$\gamma$}\coqexternalref{::'xCExBB' x '..' x ',' x}{http://coq.inria.fr/stdlib/Coq.Unicode.Utf8\_core}{\coqdocnotation{,}} \coqdocnotation{(}\coqexternalref{::'xCExBB' x '..' x ',' x}{http://coq.inria.fr/stdlib/Coq.Unicode.Utf8\_core}{\coqdocnotation{\ensuremath{\lambda}}} \coqdocvar{t}\coqexternalref{::'xCExBB' x '..' x ',' x}{http://coq.inria.fr/stdlib/Coq.Unicode.Utf8\_core}{\coqdocnotation{,}} \coqdocvariable{b} \coqdocnotation{$\star$} \coqdocnotation{(}\coqdocvariable{$\gamma$} \coqdocnotation{;} \coqdocvariable{t}\coqdocnotation{)} \coqdocnotation{;} \coqdocvar{\_}\coqdocnotation{);} \coqref{groupoid interpretation def.Lam 2}{\coqdocaxiom{$\mathsf{Lam_{comp}}$}} \coqdocvariable{b}\coqdocnotation{)}.\coqdoceol
\coqdocemptyline
\coqdocemptyline
\end{coqdoccode}
  \paragraph{\textsc{Sigma}, \textsc{Pair} and \textsc{Projs}.}
  The rules for Σ types are interpreted using the 
  dependent sum \coqdocvar{$\Sigma$} on setoids.  
\begin{coqdoccode}
\coqdocemptyline
\coqdocnoindent
\coqdockw{Definition} \coqdef{groupoid interpretation def.Sigma}{Sigma}{\coqdocdefinition{Sigma}} \{\coqdocvar{Γ}\} (\coqdocvar{A}:\coqref{groupoid interpretation def.Typ}{\coqdocdefinition{Typ}} \coqdocvariable{Γ}) (\coqdocvar{F}:\coqref{groupoid interpretation def.TypFam}{\coqdocdefinition{TypFam}} \coqdocvariable{A}) \coqdoceol
\coqdocindent{1.00em}
: \coqref{groupoid interpretation def.UTyp}{\coqdocdefinition{UTyp}} \coqdocvariable{Γ} := \coqdocnotation{(}\coqexternalref{::'xCExBB' x '..' x ',' x}{http://coq.inria.fr/stdlib/Coq.Unicode.Utf8\_core}{\coqdocnotation{\ensuremath{\lambda}}} \coqdocvar{$\gamma$}: \coqdocnotation{[}\coqdocvariable{Γ}\coqdocnotation{]}\coqexternalref{::'xCExBB' x '..' x ',' x}{http://coq.inria.fr/stdlib/Coq.Unicode.Utf8\_core}{\coqdocnotation{,}} \coqdocdefinition{\_Sum1} (\coqdocvariable{F} \coqdocnotation{$\star$} \coqdocvariable{$\gamma$})\coqdocnotation{;} \coqref{groupoid interpretation def.Sigma 1}{\coqdocaxiom{$\mathsf{Sigma_{comp}}$}} \coqdocvariable{A} \coqdocvariable{F}\coqdocnotation{)}.\coqdoceol
\coqdocemptyline
\end{coqdoccode}
\noindent Pairing and projections are obtained
  by a context lift of pairing and projection of the underlying dependent sum.
\begin{coqdoccode}
\coqdocemptyline
\coqdocemptyline
\end{coqdoccode}
  \paragraph{\lrule{Conv}.}


  It is not possible to prove in \Coq that the conversion rule is
  preserved because the application of this rule is implicit and
  can not be reified. Nevertheless, to witness this preservation, 
  we show that beta conversion is valid as a definitional equality
  on the first projection. As conversion is only done on 
  types and interpretation of types is always projected, this is 
  enough to guarantee that the conversion rule is also admissible.


\begin{coqdoccode}
\coqdocemptyline
\coqdocnoindent
\coqdockw{Definition} \coqdef{groupoid interpretation def.Beta}{Beta}{\coqdocdefinition{Beta}} \{\coqdocvar{Γ}\} \{\coqdocvar{A}:\coqref{groupoid interpretation def.Typ}{\coqdocdefinition{Typ}} \coqdocvariable{Γ}\} \{\coqdocvar{F}:\coqref{groupoid interpretation def.UTypDep}{\coqdocdefinition{UTypDep}} \coqdocvariable{A}\} (\coqdocvar{b}:\coqref{groupoid interpretation def.UElt}{\coqdocdefinition{UElt}} \coqdocvariable{F}) (\coqdocvar{a}:\coqref{groupoid interpretation def.Elt}{\coqdocdefinition{Elt}} \coqdocvariable{A}) \coqdoceol
\coqdocindent{1.00em}
: \coqdocnotation{[}\coqref{groupoid interpretation def.Lam}{\coqdocdefinition{Lam}} \coqdocvariable{b} \coqref{groupoid interpretation def.::x '@@' x}{\coqdocnotation{$\star$}} \coqdocvariable{a}\coqdocnotation{]} \coqdocnotation{=} \coqdocnotation{[}\coqdocvariable{b} \coqref{groupoid interpretation def.::x 'xC2xB0xC2xB0' x}{\coqdocnotation{$\circ$}} \coqref{groupoid interpretation def.SubExtId}{\coqdocdefinition{SubExtId}} \coqdocvariable{a}\coqdocnotation{]} := \coqdocconstructor{eq\_refl} \coqdocvar{\_}.\coqdoceol
\coqdocemptyline
\end{coqdoccode}
 \noindent where \coqref{groupoid interpretation def.SubExtId}{\coqdocdefinition{SubExtId}} is a specialization of \coqref{groupoid interpretation def.SubExt}{\coqdocdefinition{SubExt}} with 
  the identity substitution.
\begin{coqdoccode}
\coqdocemptyline
\end{coqdoccode}


\subsection{Extensional principles}


  \label{sec:extprinc}
  One of the main interest of the groupoid interpretation is that it
  allows to interpret a type directed notion of equality which validates 
  the J eliminator of identity types but also various extensional principles.
  For any elements \coqdocvariable{a} and \coqdocvariable{b} of a type \coqdocvariable{A}, we note \coqref{groupoid interpretation def.Id}{\coqdocdefinition{Id}} \coqdocvariable{a} \coqdocvariable{b} the equality type
  between \coqdocvariable{a} and \coqdocvariable{b}.
\begin{coqdoccode}
\coqdocemptyline
\coqdocnoindent
\coqdockw{Definition} \coqdef{groupoid interpretation def.Id}{Id}{\coqdocdefinition{Id}} \{\coqdocvar{Γ}\} (\coqdocvar{A}: \coqref{groupoid interpretation def.UTyp}{\coqdocdefinition{UTyp}} \coqdocvariable{Γ}) (\coqdocvar{a} \coqdocvar{b} : \coqref{groupoid interpretation def.UElt}{\coqdocdefinition{UElt}} \coqdocvariable{A}) \coqdoceol
\coqdocindent{1.00em}
: \coqref{groupoid interpretation def.Typ}{\coqdocdefinition{Typ}} \coqdocvariable{Γ} := \coqdocnotation{(}\coqexternalref{::'xCExBB' x '..' x ',' x}{http://coq.inria.fr/stdlib/Coq.Unicode.Utf8\_core}{\coqdocnotation{\ensuremath{\lambda}}} \coqdocvar{$\gamma$}\coqexternalref{::'xCExBB' x '..' x ',' x}{http://coq.inria.fr/stdlib/Coq.Unicode.Utf8\_core}{\coqdocnotation{,}} \coqdocnotation{(}\coqdocvariable{a} \coqdocnotation{$\star$} \coqdocvariable{$\gamma$} \coqdocnotation{$\sim_1$} \coqdocvariable{b} \coqdocnotation{$\star$} \coqdocvariable{$\gamma$} \coqdocnotation{;} \coqdocvar{\_}\coqdocnotation{);} \coqref{groupoid interpretation def.Id 1}{\coqdocaxiom{$\mathsf{Id_{comp}}$}} \coqdocvariable{A} \coqdocvariable{a} \coqdocvariable{b}\coqdocnotation{)}.\coqdoceol
\coqdocemptyline
\coqdocemptyline
\end{coqdoccode}
We can interpret the J eliminator of MLTT on \coqref{groupoid interpretation def.Id}{\coqdocdefinition{Id}} using functoriality of \coqdocvariable{P} and of product 
   (\coqref{groupoid interpretation def.prod comp}{\coqdocdefinition{$\mathsf{\Pi_{comp}}$}}). \begin{coqdoccode}
\coqdocemptyline
\coqdocnoindent
\coqdockw{Definition} \coqdef{groupoid interpretation def.J}{J}{\coqdocdefinition{J}} \coqdocvar{Γ} (\coqdocvar{A}:\coqref{groupoid interpretation def.UTyp}{\coqdocdefinition{UTyp}} \coqdocvariable{Γ}) (\coqdocvar{a} \coqdocvar{b}:\coqref{groupoid interpretation def.UElt}{\coqdocdefinition{UElt}} \coqdocvariable{A}) (\coqdocvar{e}:\coqref{groupoid interpretation def.Elt}{\coqdocdefinition{Elt}} (\coqref{groupoid interpretation def.Id}{\coqdocdefinition{Id}} \coqdocvariable{a} \coqdocvariable{b})) (\coqdocvar{P}:\coqref{groupoid interpretation def.UTypFam}{\coqdocdefinition{UTypFam}} \coqdocvariable{A}) (\coqdocvar{p}:\coqref{groupoid interpretation def.UElt}{\coqdocdefinition{UElt}} (\coqdocvariable{P}\coqref{groupoid interpretation def.::x 'x7Bx7Bx7B' x 'x7Dx7Dx7D'}{\coqdocnotation{\{\{\{}}\coqdocvariable{a}\coqref{groupoid interpretation def.::x 'x7Bx7Bx7B' x 'x7Dx7Dx7D'}{\coqdocnotation{\}\}\}}}))\coqdoceol
\coqdocindent{0.50em}
: \coqref{groupoid interpretation def.UElt}{\coqdocdefinition{UElt}} (\coqdocvariable{P}\coqref{groupoid interpretation def.::x 'x7Bx7Bx7B' x 'x7Dx7Dx7D'}{\coqdocnotation{\{\{\{}}\coqdocvariable{b}\coqref{groupoid interpretation def.::x 'x7Bx7Bx7B' x 'x7Dx7Dx7D'}{\coqdocnotation{\}\}\}}}) := \coqdocnotation{(}\coqref{groupoid interpretation def.prod comp}{\coqdocdefinition{$\mathsf{\Pi_{comp}}$}} \coqdocnotation{(}\coqexternalref{::'xCExBB' x '..' x ',' x}{http://coq.inria.fr/stdlib/Coq.Unicode.Utf8\_core}{\coqdocnotation{\ensuremath{\lambda}}} \coqdocvar{$\gamma$}\coqexternalref{::'xCExBB' x '..' x ',' x}{http://coq.inria.fr/stdlib/Coq.Unicode.Utf8\_core}{\coqdocnotation{,}} \coqexternalref{::'xCExBB' x '..' x ',' x}{http://coq.inria.fr/stdlib/Coq.Unicode.Utf8\_core}{\coqdocnotation{(}}\coqdocnotation{map} \coqdocnotation{(}\coqdocvariable{P} \coqdocnotation{$\star$} \coqdocvariable{$\gamma$}\coqdocnotation{)} (\coqdocvariable{e} \coqdocnotation{$\star$} \coqdocvariable{$\gamma$})\coqexternalref{::'xCExBB' x '..' x ',' x}{http://coq.inria.fr/stdlib/Coq.Unicode.Utf8\_core}{\coqdocnotation{)}}\coqdocnotation{;} \coqref{groupoid interpretation def.J 1}{\coqdocaxiom{$\mathsf{J_{comp}}$}} \coqdocvar{\_} \coqdocvar{\_}\coqdocnotation{)}\coqdocnotation{)} \coqdocnotation{$\star$} \coqdocvariable{p}.\coqdoceol
\coqdocemptyline
\coqdocemptyline
\end{coqdoccode}
Rule \textsc{ValId-Intro} is simply given by the identity of the underlying groupoid. \begin{coqdoccode}
\coqdocemptyline
\coqdocnoindent
\coqdockw{Definition} \coqdef{groupoid interpretation def.Refl}{Refl}{\coqdocdefinition{Refl}} \coqdocvar{Γ} (\coqdocvar{A}: \coqref{groupoid interpretation def.Typ}{\coqdocdefinition{Typ}} \coqdocvariable{Γ}) (\coqdocvar{a} : \coqref{groupoid interpretation def.Elt}{\coqdocdefinition{Elt}} \coqdocvariable{A}) \coqdoceol
\coqdocindent{1.00em}
: \coqref{groupoid interpretation def.Elt}{\coqdocdefinition{Elt}} (\coqref{groupoid interpretation def.Id}{\coqdocdefinition{Id}} \coqdocvariable{a} \coqdocvariable{a}) := \coqdocnotation{(}\coqdockw{fun} \coqdocvar{$\gamma$} \ensuremath{\Rightarrow} \coqdocmethod{identity} \coqdocvar{\_}\coqdocnotation{;} \coqref{groupoid interpretation def.Refl 1}{\coqdocaxiom{$\mathsf{Refl_{comp}}$}} \coqdocvar{\_}\coqdocnotation{)}.\coqdoceol
\coqdocemptyline
\coqdocemptyline
\coqdocnoindent
\coqdockw{Definition} \coqdef{groupoid interpretation def.Equiv Intro}{Equiv\_Intro}{\coqdocdefinition{Equiv\_Intro}} (\coqdocvar{Γ}: \coqref{groupoid interpretation def.Context}{\coqdocdefinition{Context}}) (\coqdocvar{A} \coqdocvar{B} : \coqref{groupoid interpretation def.Typ}{\coqdocdefinition{Typ}} \coqdocvariable{Γ}) (\coqdocvar{e} : \coqref{groupoid interpretation def.iso}{\coqdocdefinition{iso}} \coqdocvariable{A} \coqdocvariable{B})\coqdoceol
\coqdocindent{1.00em}
: \coqref{groupoid interpretation def.Elt}{\coqdocdefinition{Elt}} (\coqdocvariable{A} \coqref{groupoid interpretation def.::x 'xE2x89xA1' x}{\coqdocnotation{≡}} \coqdocvariable{B}) := \coqdocnotation{(}\coqref{groupoid interpretation def.Equiv Intro }{\coqdocdefinition{Equiv\_Intro\_}} \coqdocvariable{e}\coqdocnotation{;} \coqref{groupoid interpretation def.Equiv Intro 1}{\coqdocaxiom{$\mathsf{Equiv\_Intro_{comp}}$}} \coqdocvariable{e}\coqdocnotation{)}.\coqdoceol
\coqdocemptyline
\coqdocemptyline
\end{coqdoccode}


\coqlibrary{Groupoid.groupoid interpretation}{Library }{Groupoid.groupoid\_interpretation}

\begin{coqdoccode}
\end{coqdoccode}


  Terms of \coqdocdefinition{TypDep} \coqdocvariable{A} and \coqdocdefinition{TypFam} \coqdocvariable{A} can be related using a dependent closure
  at the level of types. In the interpretation of typing judgments, this connection 
  will be used to switch between the fibration and the morphism points of view.
\begin{coqdoccode}
\coqdocemptyline
\coqdocnoindent
\coqdockw{Definition} \coqdef{Groupoid.groupoid interpretation.LamT}{$\Lambda$}{\coqdocdefinition{$\Lambda$}} \{\coqdocvar{Γ}: \coqdocdefinition{Context}\} \{\coqdocvar{A} : \coqdocdefinition{Typ} \coqdocvariable{Γ}\} (\coqdocvar{B}: \coqdocdefinition{TypDep} \coqdocvariable{A})\coqdoceol
\coqdocindent{1.00em}
: \coqdocdefinition{TypFam} \coqdocvariable{A} := \coqdocnotation{(}\coqexternalref{::'xCExBB' x '..' x ',' x}{http://coq.inria.fr/stdlib/Coq.Unicode.Utf8\_core}{\coqdocnotation{\ensuremath{\lambda}}} \coqdocvar{$\gamma$}\coqexternalref{::'xCExBB' x '..' x ',' x}{http://coq.inria.fr/stdlib/Coq.Unicode.Utf8\_core}{\coqdocnotation{,}} \coqdocnotation{(}\coqexternalref{::'xCExBB' x '..' x ',' x}{http://coq.inria.fr/stdlib/Coq.Unicode.Utf8\_core}{\coqdocnotation{\ensuremath{\lambda}}} \coqdocvar{t}\coqexternalref{::'xCExBB' x '..' x ',' x}{http://coq.inria.fr/stdlib/Coq.Unicode.Utf8\_core}{\coqdocnotation{,}} \coqdocvariable{B} \coqdocnotation{$\star$} \coqdocnotation{(}\coqdocvariable{$\gamma$}\coqdocnotation{;} \coqdocvariable{t}\coqdocnotation{)} \coqdocnotation{;} \coqdocvar{\_}\coqdocnotation{);} \coqref{Groupoid.groupoid interpretation.LamT 1}{\coqdocinstance{$\mathsf{\Lambda_{comp}}$}} \coqdocvariable{B}\coqdocnotation{)}.\coqdoceol
\coqdocemptyline
\end{coqdoccode}


\subsection{Substitutions}




  A substitution is represented by a context morphism [\coqdocvariable{Γ} $\longrightarrow$ \coqdocvariable{Δ}].  Note
  that although a substitution σ can be composed with a dependent type
  \coqdocvariable{A} by using composition of functors, we define a relaxed notion of
  composition, noted \coqdocvariable{A} ⋅ σ. It has the same computational content but
  a different relation on the universe indices: homogeneous functor
  composition otherwise forces the three categories and two functors to
  live at exactly the same levels, which is not necessary.


  A substitution σ can be extended by a term \coqdocvariable{a}: \coqdocdefinition{$\mathsf{Tm}$} (\coqdocvariable{A} ⋅ σ) 
  of \coqdocvariable{A} : \coqdocdefinition{Typ} \coqdocvariable{Δ}.


\begin{coqdoccode}
\coqdocemptyline
\coqdocnoindent
\coqdockw{Definition} \coqdef{Groupoid.groupoid interpretation.SubExt}{SubExt}{\coqdocdefinition{SubExt}} \{\coqdocvar{Γ} \coqdocvar{Δ} : \coqdocdefinition{Context}\} \{\coqdocvar{A} : \coqdocdefinition{Typ} \coqdocvariable{Δ}\} (σ: \coqdocnotation{[}\coqdocvariable{Γ} \coqdocnotation{$\longrightarrow$} \coqdocvariable{Δ}\coqdocnotation{]}) (\coqdocvar{a}: \coqdocdefinition{$\mathsf{Tm}$} (\coqdocvariable{A} \coqdocnotation{$⋅$} \coqdocvariable{σ})) \coqdoceol
\coqdocindent{1.00em}
: \coqdocnotation{[}\coqdocvariable{Γ} \coqdocnotation{$\longrightarrow$} \coqdocdefinition{$\Sigma$} \coqdocvariable{A} \coqdocnotation{]} := \coqdocnotation{(}\coqexternalref{::'xCExBB' x '..' x ',' x}{http://coq.inria.fr/stdlib/Coq.Unicode.Utf8\_core}{\coqdocnotation{\ensuremath{\lambda}}} \coqdocvar{$\gamma$}\coqexternalref{::'xCExBB' x '..' x ',' x}{http://coq.inria.fr/stdlib/Coq.Unicode.Utf8\_core}{\coqdocnotation{,}} \coqdocnotation{(}\coqdocvariable{σ} \coqdocnotation{$\star$} \coqdocvariable{$\gamma$}\coqdocnotation{;} \coqdocvariable{a} \coqdocnotation{$\star$} \coqdocvariable{$\gamma$}\coqdocnotation{)} \coqdocnotation{;} \coqref{Groupoid.groupoid interpretation.SubExt 1}{\coqdocinstance{$\mathsf{SubExt_{comp}}$}} \coqdocvar{\_} \coqdocvar{\_}\coqdocnotation{)}.\coqdoceol
\coqdocemptyline
\end{coqdoccode}
\noindent where \coqref{Groupoid.groupoid interpretation.SubExt 1}{\coqdocinstance{$\mathsf{SubExt_{comp}}$}} is a proof that it is functorial. 
  A substitution σ can be applied to a type family \coqdocvariable{F} using the
  composition of a functor with a dependent functor. 
\begin{coqdoccode}
\coqdocemptyline
\coqdocnoindent
\coqdockw{Definition} \coqdef{Groupoid.groupoid interpretation.substF}{substF}{\coqdocdefinition{substF}} \{\coqdocvar{T} \coqdocvar{Γ}\} \{\coqdocvar{A}:\coqdocdefinition{Typ} \coqdocvariable{Γ}\} (\coqdocvar{F}:\coqdocdefinition{TypFam} \coqdocvariable{A}) (σ:\coqdocnotation{[}\coqdocvariable{T} \coqdocnotation{$\longrightarrow$} \coqdocvariable{Γ}\coqdocnotation{]}) : \coqdocdefinition{TypFam} (\coqdocvariable{A} \coqdocnotation{$⋅$} \coqdocvariable{σ}) \coqdoceol
\coqdocindent{1.00em}
:= \coqdocnotation{(}\coqdocnotation{[}\coqdocvariable{F} \coqdocnotation{$\circ$} \coqdocvariable{σ}\coqdocnotation{]} : \coqexternalref{:type scope:'xE2x88x80' x '..' x ',' x}{http://coq.inria.fr/stdlib/Coq.Unicode.Utf8\_core}{\coqdocnotation{∀}} \coqdocvar{t} : \coqdocnotation{[}\coqdocvariable{T}\coqdocnotation{]}\coqexternalref{:type scope:'xE2x88x80' x '..' x ',' x}{http://coq.inria.fr/stdlib/Coq.Unicode.Utf8\_core}{\coqdocnotation{,}} \coqdocnotation{ }\coqdocvariable{A} \coqdocnotation{$⋅$} \coqdocvariable{σ}\coqdocnotation{$_{\upharpoonright s}$} \coqdocnotation{$\star$} \coqdocvariable{t} \coqref{Groupoid.groupoid interpretation.::x '--->' x}{\coqdocnotation{$\longrightarrow$}} \coqdocdefinition{$\mathsf{Type}_{0}^1$}\coqdocnotation{;} \coqref{Groupoid.groupoid interpretation.substF 1}{\coqdocinstance{$\mathsf{substF_{comp}}$}} \coqdocvariable{F} \coqdocvariable{σ}\coqdocnotation{)}.\coqdoceol
\coqdocemptyline
\end{coqdoccode}
  We abusively note all those different compositions with $\circ$ as it is done in
  mathematics, whereas they are distinct operators in the \Coq
  development.
  The weakening substitution of $\Gamma, x:A \vdash$ is given by the first
  projection. \begin{coqdoccode}
\coqdocemptyline
\end{coqdoccode}


  A type family \coqdocvariable{F} in \coqdocdefinition{TypFam} \coqdocvariable{A} can be partially substituted with an
  term \coqdocvariable{a} in \coqdocdefinition{$\mathsf{Tm}$} \coqdocvariable{A}, noted \coqdocvariable{F} \{\{\coqdocvariable{a}\}\}, to get its value (a type) at
  \coqdocvariable{a}. This process is defined as \coqdocvariable{F} \{\{\coqdocvariable{a}\}\} := (\coqdocvar{\ensuremath{\lambda}} \coqdocvariable{$\gamma$}, (\coqdocvariable{F} $\star$ \coqdocvariable{$\gamma$}) $\star$ (\coqdocvariable{a} $\star$ \coqdocvariable{$\gamma$}) ;
  \coqdocvar{\_}) (where \coqdocvar{\_} is a proof it is functorial). Note that this
  pattern of application \emph{up-to a context $\gamma$} will be used
  later to defined other notions of application. Although the
  computational definitions are the same, the compatibility conditions
  are always different.  This notion of partial substitution in a type
  family enables to state that \coqref{Groupoid.groupoid interpretation.LamT}{\coqdocdefinition{$\Lambda$}} defines a type level
  $\lambda$-abstraction.  \begin{coqdoccode}
\coqdocemptyline
\coqdocnoindent
\coqdockw{Definition} \coqdef{Groupoid.groupoid interpretation.BetaT}{BetaT}{\coqdocdefinition{BetaT}} \coqdocvar{Δ} \coqdocvar{Γ} (\coqdocvar{A}:\coqdocdefinition{Typ} \coqdocvariable{Γ}) (\coqdocvar{B}:\coqdocdefinition{TypDep} \coqdocvariable{A}) (σ:\coqdocnotation{[}\coqdocvariable{Δ} \coqdocnotation{$\longrightarrow$} \coqdocvariable{Γ}\coqdocnotation{]}) (\coqdocvar{a}:\coqdocdefinition{$\mathsf{Tm}$} (\coqdocvariable{A} \coqdocnotation{$⋅$} \coqdocvariable{σ})) \coqdoceol
\coqdocnoindent
: \coqref{Groupoid.groupoid interpretation.LamT}{\coqdocdefinition{$\Lambda$}} \coqdocvariable{B} \coqref{Groupoid.groupoid interpretation.::x 'xC2xB0xC2xB0xC2xB0' x}{\coqdocnotation{$\circ$}} \coqdocvariable{σ} \coqref{Groupoid.groupoid interpretation.::x 'x7Bx7B' x 'x7Dx7D'}{\coqdocnotation{\{\{}}\coqdocvariable{a}\coqref{Groupoid.groupoid interpretation.::x 'x7Bx7B' x 'x7Dx7D'}{\coqdocnotation{\}\}}} \coqdocnotation{$\sim_1$} \coqdocvariable{B} \coqdocnotation{$⋅$} \coqdocnotation{(}\coqref{Groupoid.groupoid interpretation.SubExt}{\coqdocdefinition{SubExt}} \coqdocvariable{σ} \coqdocvariable{a}\coqdocnotation{)} := \coqdocnotation{(}\coqexternalref{::'xCExBB' x '..' x ',' x}{http://coq.inria.fr/stdlib/Coq.Unicode.Utf8\_core}{\coqdocnotation{\ensuremath{\lambda}}} \coqdocvar{$\gamma$}\coqexternalref{::'xCExBB' x '..' x ',' x}{http://coq.inria.fr/stdlib/Coq.Unicode.Utf8\_core}{\coqdocnotation{,}} \coqdocprojection{identity} \coqdocvar{\_} \coqdocnotation{;} \coqref{Groupoid.groupoid interpretation.BetaT 1}{\coqdocinstance{$\mathsf{BetaT_{comp}}$}} \coqdocvariable{B} \coqdocvariable{σ} \coqdocvariable{a}\coqdocnotation{)}.\coqdoceol
\coqdocemptyline
\end{coqdoccode}


\subsection{Interpretation of the typing judgment}


  \label{sec:interp}


  The explicit substitution versions of the typing rules of Figure
  \ref{fig:emltt} are modelled as described below.


  \paragraph{\textsc{Var}.} 


  The rule \textsc{Var} is given by the second projection plus a proof
  that the projection is dependently functorial. Note the explicit
  weakening of \coqdocvariable{A} in the returned type. This is because we need to
  make explicit that the context used to type \coqdocvariable{A} is extended with an
  term of type \coqdocvariable{A}.


\begin{coqdoccode}
\coqdocemptyline
\coqdocnoindent
\coqdockw{Definition} \coqdef{Groupoid.groupoid interpretation.Var}{Var}{\coqdocdefinition{Var}} \{\coqdocvar{Γ}\} (\coqdocvar{A}:\coqdocdefinition{Typ} \coqdocvariable{Γ}) : \coqdocdefinition{$\mathsf{Tm}$} \coqref{Groupoid.groupoid interpretation.::'xE2x87x91' x}{\coqdocnotation{$\shortuparrow$}}\coqdocvariable{A} := \coqdocnotation{(}\coqexternalref{::'xCExBB' x '..' x ',' x}{http://coq.inria.fr/stdlib/Coq.Unicode.Utf8\_core}{\coqdocnotation{\ensuremath{\lambda}}} \coqdocvar{t}\coqexternalref{::'xCExBB' x '..' x ',' x}{http://coq.inria.fr/stdlib/Coq.Unicode.Utf8\_core}{\coqdocnotation{,}} \coqdocabbreviation{$\pi_2$} \coqdocvariable{t}\coqdocnotation{;} \coqref{Groupoid.groupoid interpretation.Var 1}{\coqdocinstance{$\mathsf{Var_{comp}}$}} \coqdocvariable{A}\coqdocnotation{)}.\coqdoceol
\coqdocemptyline
\coqdocemptyline
\end{coqdoccode}
\paragraph{\textsc{Prod}.} The rule \textsc{Prod} is interpreted
  using the dependent functor space, plus a proof that equivalent
  contexts give rise to isomorphic dependent functor spaces.  Note that
  the rule is defined on type families and not on the dependent type
  formulation because here we need a fibration point of view. \begin{coqdoccode}
\coqdocemptyline
\coqdocnoindent
\coqdockw{Definition} \coqdef{Groupoid.groupoid interpretation.Prod}{Prod}{\coqdocdefinition{Prod}} \{\coqdocvar{Γ}\} (\coqdocvar{A}:\coqdocdefinition{Typ} \coqdocvariable{Γ}) (\coqdocvar{F}:\coqdocdefinition{TypFam} \coqdocvariable{A}) \coqdoceol
\coqdocindent{1.00em}
: \coqdocdefinition{Typ} \coqdocvariable{Γ} := \coqdocnotation{(}\coqexternalref{::'xCExBB' x '..' x ',' x}{http://coq.inria.fr/stdlib/Coq.Unicode.Utf8\_core}{\coqdocnotation{\ensuremath{\lambda}}} \coqdocvar{s}\coqexternalref{::'xCExBB' x '..' x ',' x}{http://coq.inria.fr/stdlib/Coq.Unicode.Utf8\_core}{\coqdocnotation{,}} \coqdocdefinition{$\Pi_0$} (\coqdocvariable{F} \coqdocnotation{$\star$} \coqdocvariable{s})\coqdocnotation{;} \coqref{Groupoid.groupoid interpretation.Prod 1}{\coqdocinstance{$\mathsf{Prod_{comp}}$}} \coqdocvariable{A} \coqdocvariable{F}\coqdocnotation{)}.\coqdoceol
\coqdocemptyline
\end{coqdoccode}
  \paragraph{\textsc{App}.}


  The rule \textsc{App} is interpreted using an up-to context application 
  and a proof of dependent functoriality. We abusively note \coqdocvar{M} $\star$ \coqdocvar{N} the application 
  of \coqref{Groupoid.groupoid interpretation.App}{\coqdocdefinition{App}}.
\begin{coqdoccode}
\coqdocemptyline
\coqdocnoindent
\coqdockw{Definition} \coqdef{Groupoid.groupoid interpretation.App}{App}{\coqdocdefinition{App}} \{\coqdocvar{Γ}\} \{\coqdocvar{A}:\coqdocdefinition{Typ} \coqdocvariable{Γ}\} \{\coqdocvar{F}:\coqdocdefinition{TypFam} \coqdocvariable{A}\} (\coqdocvar{c}:\coqdocdefinition{$\mathsf{Tm}$} (\coqref{Groupoid.groupoid interpretation.Prod}{\coqdocdefinition{Prod}} \coqdocvariable{F})) (\coqdocvar{a}:\coqdocdefinition{$\mathsf{Tm}$} \coqdocvariable{A}) \coqdoceol
\coqdocindent{1.00em}
: \coqdocdefinition{$\mathsf{Tm}$} (\coqdocvariable{F} \coqref{Groupoid.groupoid interpretation.::x 'x7Bx7B' x 'x7Dx7D'}{\coqdocnotation{\{\{}}\coqdocvariable{a}\coqref{Groupoid.groupoid interpretation.::x 'x7Bx7B' x 'x7Dx7D'}{\coqdocnotation{\}\}}}) := \coqdocnotation{(}\coqexternalref{::'xCExBB' x '..' x ',' x}{http://coq.inria.fr/stdlib/Coq.Unicode.Utf8\_core}{\coqdocnotation{\ensuremath{\lambda}}} \coqdocvar{s}\coqexternalref{::'xCExBB' x '..' x ',' x}{http://coq.inria.fr/stdlib/Coq.Unicode.Utf8\_core}{\coqdocnotation{,}} \coqdocnotation{(}\coqdocvariable{c} \coqdocnotation{$\star$} \coqdocvariable{s}\coqdocnotation{)} \coqdocnotation{$\star$} \coqdocnotation{(}\coqdocvariable{a} \coqdocnotation{$\star$} \coqdocvariable{s}\coqdocnotation{)}\coqdocnotation{;} \coqref{Groupoid.groupoid interpretation.App 1}{\coqdocinstance{$\mathsf{App_{comp}}$}} \coqdocvariable{c} \coqdocvariable{a}\coqdocnotation{)}.\coqdoceol
\coqdocemptyline
\end{coqdoccode}
  \paragraph{\lrule{Lam}.}


  Term-level $\lambda$-abstraction is defined with the same
  computational meaning as type-level $\lambda$-abstraction, but it
  differs on the proof of dependent functoriality. Note that we use
  \coqref{Groupoid.groupoid interpretation.LamT}{\coqdocdefinition{$\Lambda$}} in the definition because we need both the fibration (for
  \coqref{Groupoid.groupoid interpretation.Prod}{\coqdocdefinition{Prod}}) and the morphism (for \coqdocdefinition{$\mathsf{Tm}$} \coqdocvariable{B}) point of view. 
\begin{coqdoccode}
\coqdocemptyline
\coqdocnoindent
\coqdockw{Definition} \coqdef{Groupoid.groupoid interpretation.Lam}{Lam}{\coqdocdefinition{Lam}} \{\coqdocvar{Γ}\} \{\coqdocvar{A}:\coqdocdefinition{Typ} \coqdocvariable{Γ}\} \{\coqdocvar{B}:\coqdocdefinition{TypDep} \coqdocvariable{A}\} (\coqdocvar{b}:\coqdocdefinition{$\mathsf{Tm}$} \coqdocvariable{B})\coqdoceol
\coqdocindent{1.00em}
: \coqdocdefinition{$\mathsf{Tm}$} (\coqref{Groupoid.groupoid interpretation.Prod}{\coqdocdefinition{Prod}} (\coqref{Groupoid.groupoid interpretation.LamT}{\coqdocdefinition{$\Lambda$}} \coqdocvariable{B})) := \coqdocnotation{(}\coqexternalref{::'xCExBB' x '..' x ',' x}{http://coq.inria.fr/stdlib/Coq.Unicode.Utf8\_core}{\coqdocnotation{\ensuremath{\lambda}}} \coqdocvar{$\gamma$}\coqexternalref{::'xCExBB' x '..' x ',' x}{http://coq.inria.fr/stdlib/Coq.Unicode.Utf8\_core}{\coqdocnotation{,}} \coqdocnotation{(}\coqexternalref{::'xCExBB' x '..' x ',' x}{http://coq.inria.fr/stdlib/Coq.Unicode.Utf8\_core}{\coqdocnotation{\ensuremath{\lambda}}} \coqdocvar{t}\coqexternalref{::'xCExBB' x '..' x ',' x}{http://coq.inria.fr/stdlib/Coq.Unicode.Utf8\_core}{\coqdocnotation{,}} \coqdocvariable{b} \coqdocnotation{$\star$} \coqdocnotation{(}\coqdocvariable{$\gamma$} \coqdocnotation{;} \coqdocvariable{t}\coqdocnotation{)} \coqdocnotation{;} \coqdocvar{\_}\coqdocnotation{);} \coqref{Groupoid.groupoid interpretation.Lam 2}{\coqdocinstance{$\mathsf{Lam_{comp}}$}} \coqdocvariable{b}\coqdocnotation{)}.\coqdoceol
\coqdocemptyline
\coqdocemptyline
\end{coqdoccode}
  \paragraph{\textsc{Sigma}, \textsc{Pair} and \textsc{Projs}.}
  The rules for Σ types are interpreted using the 
  dependent sum \coqdocdefinition{$\Sigma$} on setoids.  
\begin{coqdoccode}
\coqdocemptyline
\coqdocnoindent
\coqdockw{Definition} \coqdef{Groupoid.groupoid interpretation.Sigma}{Sigma}{\coqdocdefinition{Sigma}} \{\coqdocvar{Γ}\} (\coqdocvar{A}:\coqdocdefinition{Typ} \coqdocvariable{Γ}) (\coqdocvar{F}:\coqdocdefinition{TypFam} \coqdocvariable{A}) \coqdoceol
\coqdocindent{1.00em}
: \coqdocdefinition{Typ} \coqdocvariable{Γ} := \coqdocnotation{(}\coqexternalref{::'xCExBB' x '..' x ',' x}{http://coq.inria.fr/stdlib/Coq.Unicode.Utf8\_core}{\coqdocnotation{\ensuremath{\lambda}}} \coqdocvar{$\gamma$}: \coqdocnotation{[}\coqdocvariable{Γ}\coqdocnotation{]}\coqexternalref{::'xCExBB' x '..' x ',' x}{http://coq.inria.fr/stdlib/Coq.Unicode.Utf8\_core}{\coqdocnotation{,}} \coqdocdefinition{$\Sigma$} (\coqdocvariable{F} \coqdocnotation{$\star$} \coqdocvariable{$\gamma$})\coqdocnotation{;} \coqref{Groupoid.groupoid interpretation.Sigma 1}{\coqdocinstance{$\mathsf{Sigma_{comp}}$}} \coqdocvariable{A} \coqdocvariable{F}\coqdocnotation{)}.\coqdoceol
\coqdocemptyline
\end{coqdoccode}
\noindent Pairing and projections are obtained
  by a context lift of pairing and projection of the underlying dependent sum.
\begin{coqdoccode}
\coqdocemptyline
\coqdocemptyline
\end{coqdoccode}


\subsection{Identity Types}


  One of the main interests of the setoid and groupoid interpretations is that they
  allow to interpret a type directed notion of equality which validates 
  the J eliminator of identity types but also various extensional principles,
  including functional extensionality. 
  For any terms \coqdocvariable{a} and \coqdocvariable{b} of a dependent type \coqdocvariable{A}:\coqdocdefinition{Typ} \coqdocvariable{Γ}, we note \coqref{Groupoid.groupoid interpretation.Id}{\coqdocdefinition{Id}} \coqdocvariable{a} \coqdocvariable{b} the equality type
  between \coqdocvariable{a} and \coqdocvariable{b} obtained by lifting $\sim_1$ to get a type depending on \coqdocvariable{Γ}.
\begin{coqdoccode}
\coqdocemptyline
\coqdocnoindent
\coqdockw{Definition} \coqdef{Groupoid.groupoid interpretation.Id}{Id}{\coqdocdefinition{Id}} \{\coqdocvar{Γ}\} (\coqdocvar{A}: \coqdocdefinition{Typ} \coqdocvariable{Γ}) (\coqdocvar{a} \coqdocvar{b} : \coqdocdefinition{$\mathsf{Tm}$} \coqdocvariable{A}) \coqdoceol
\coqdocindent{1.00em}
: \coqdocdefinition{Typ} \coqdocvariable{Γ} := \coqdocnotation{(}\coqexternalref{::'xCExBB' x '..' x ',' x}{http://coq.inria.fr/stdlib/Coq.Unicode.Utf8\_core}{\coqdocnotation{\ensuremath{\lambda}}} \coqdocvar{$\gamma$}\coqexternalref{::'xCExBB' x '..' x ',' x}{http://coq.inria.fr/stdlib/Coq.Unicode.Utf8\_core}{\coqdocnotation{,}} \coqdocnotation{(}\coqdocvariable{a} \coqdocnotation{$\star$} \coqdocvariable{$\gamma$} \coqdocnotation{$\sim_1$} \coqdocvariable{b} \coqdocnotation{$\star$} \coqdocvariable{$\gamma$} \coqdocnotation{;} \coqdocvar{\_}\coqdocnotation{);} \coqref{Groupoid.groupoid interpretation.Id 1}{\coqdocinstance{$\mathsf{Id_{comp}}$}} \coqdocvariable{A} \coqdocvariable{a} \coqdocvariable{b}\coqdocnotation{)}.\coqdoceol
\coqdocemptyline
\coqdocemptyline
\end{coqdoccode}
The introduction rule of identity types which corresponds to reflexivity is interpreted by the (lifting of) identity of the underlying setoid. \begin{coqdoccode}
\coqdocemptyline
\coqdocnoindent
\coqdockw{Definition} \coqdef{Groupoid.groupoid interpretation.Refl}{Refl}{\coqdocdefinition{Refl}} \coqdocvar{Γ} (\coqdocvar{A}: \coqdocdefinition{Typ} \coqdocvariable{Γ}) (\coqdocvar{a} : \coqdocdefinition{$\mathsf{Tm}$} \coqdocvariable{A}) \coqdoceol
\coqdocindent{1.00em}
: \coqdocdefinition{$\mathsf{Tm}$} (\coqref{Groupoid.groupoid interpretation.Id}{\coqdocdefinition{Id}} \coqdocvariable{a} \coqdocvariable{a}) := \coqdocnotation{(}\coqexternalref{::'xCExBB' x '..' x ',' x}{http://coq.inria.fr/stdlib/Coq.Unicode.Utf8\_core}{\coqdocnotation{\ensuremath{\lambda}}} \coqdocvar{$\gamma$}\coqexternalref{::'xCExBB' x '..' x ',' x}{http://coq.inria.fr/stdlib/Coq.Unicode.Utf8\_core}{\coqdocnotation{,}} \coqdocprojection{identity} (\coqdocvariable{a} \coqdocnotation{$\star$} \coqdocvariable{$\gamma$})\coqdocnotation{;} \coqref{Groupoid.groupoid interpretation.Refl 1}{\coqdocinstance{$\mathsf{Refl_{comp}}$}} \coqdocvar{\_} \coqdocvar{\_} \coqdocvar{\_}\coqdocnotation{)}.\coqdoceol
\coqdocemptyline
\coqdocemptyline
\end{coqdoccode}
We can interpret the \coqref{Groupoid.groupoid interpretation.J}{\coqdocdefinition{J}} eliminator of MLTT on \coqref{Groupoid.groupoid interpretation.Id}{\coqdocdefinition{Id}} using
  functoriality of \coqdocvariable{P} and of product (\coqref{Groupoid.groupoid interpretation.prod comp}{\coqdocdefinition{$\mathsf{\Pi_{comp}}$}}). In the definition
  of \coqref{Groupoid.groupoid interpretation.J}{\coqdocdefinition{J}}, the predicate \coqdocvariable{P} depends on the proof of equality, which is
  interpreted using a \coqref{Groupoid.groupoid interpretation.Sigma}{\coqdocdefinition{Sigma}} type. The functoriality of \coqdocvariable{P} is used on
  the term \coqref{Groupoid.groupoid interpretation.J Pair}{\coqdocdefinition{J\_Pair}} \coqdocvariable{e} \coqdocvariable{P} \coqdocvariable{$\gamma$}, which is a proof that (\coqdocvariable{a};\coqref{Groupoid.groupoid interpretation.Refl}{\coqdocdefinition{Refl}} \coqdocvariable{a}) is equal
  to (\coqdocvariable{b};\coqdocvariable{e}). To state the rule, we need to do a rewriting at 
  the level of terms, i.e., given an equality \coqdocvariable{e} : \coqdocvariable{T} $\sim_1$ \coqdocvariable{U} between two 
  types in \coqdocvariable{A}, we use the map from \coqdocvariable{t} : \coqdocdefinition{$\mathsf{Tm}$} \coqdocvariable{T} to 
  \coqdocvariable{t} \coqdockw{with} \coqdocvariable{e} : \coqdocdefinition{$\mathsf{Tm}$} \coqdocvariable{U} that comes from the
  functoriality of \coqdocvar{$\Pi$}:\begin{coqdoccode}
\coqdocemptyline
\coqdocnoindent
\coqdockw{Definition} \coqdef{Groupoid.groupoid interpretation.J}{J}{\coqdocdefinition{J}} \coqdocvar{Γ} (\coqdocvar{A}:\coqdocdefinition{Typ} \coqdocvariable{Γ}) (\coqdocvar{a} \coqdocvar{b}:\coqdocdefinition{$\mathsf{Tm}$} \coqdocvariable{A}) (\coqdocvar{P}:\coqdocdefinition{TypFam} (\coqref{Groupoid.groupoid interpretation.Sigma}{\coqdocdefinition{Sigma}} (\coqref{Groupoid.groupoid interpretation.LamT}{\coqdocdefinition{$\Lambda$}} (\coqref{Groupoid.groupoid interpretation.Id}{\coqdocdefinition{Id}} (\coqdocvariable{a} \coqdocnotation{$\circ$} \coqref{Groupoid.groupoid interpretation.Sub}{\coqdocdefinition{Sub}}) (\coqref{Groupoid.groupoid interpretation.Var}{\coqdocdefinition{Var}} \coqdocvariable{A})))))\coqdoceol
\coqdocindent{7.50em}
(\coqdocvar{e}:\coqdocdefinition{$\mathsf{Tm}$} (\coqref{Groupoid.groupoid interpretation.Id}{\coqdocdefinition{Id}} \coqdocvariable{a} \coqdocvariable{b})) (\coqdocvar{p}:\coqdocdefinition{$\mathsf{Tm}$} (\coqdocvariable{P}\coqref{Groupoid.groupoid interpretation.::x 'x7Bx7B' x 'x7Dx7D'}{\coqdocnotation{\{\{}}\coqref{Groupoid.groupoid interpretation.Pair}{\coqdocdefinition{Pair}} (\coqref{Groupoid.groupoid interpretation.Refl}{\coqdocdefinition{Refl}} \coqdocvariable{a} \coqref{Groupoid.groupoid interpretation.::x 'with' x}{\coqdocnotation{with}} \coqref{Groupoid.groupoid interpretation.BetaT2}{\coqdocdefinition{$\mathsf{BetaT'}$}})\coqref{Groupoid.groupoid interpretation.::x 'x7Bx7B' x 'x7Dx7D'}{\coqdocnotation{\}\}}})):\coqdoceol
\coqdocindent{1.00em}
\coqdocdefinition{$\mathsf{Tm}$} (\coqdocvariable{P}\coqref{Groupoid.groupoid interpretation.::x 'x7Bx7B' x 'x7Dx7D'}{\coqdocnotation{\{\{}}\coqref{Groupoid.groupoid interpretation.Pair}{\coqdocdefinition{Pair}} (\coqdocvariable{e} \coqref{Groupoid.groupoid interpretation.::x 'with' x}{\coqdocnotation{with}} \coqref{Groupoid.groupoid interpretation.BetaT2}{\coqdocdefinition{$\mathsf{BetaT'}$}})\coqref{Groupoid.groupoid interpretation.::x 'x7Bx7B' x 'x7Dx7D'}{\coqdocnotation{\}\}}}) :=\coqdoceol
\coqdocindent{1.00em}
\coqref{Groupoid.groupoid interpretation.prod comp}{\coqdocdefinition{$\mathsf{\Pi_{comp}}$}} \coqdocnotation{(}\coqexternalref{::'xCExBB' x '..' x ',' x}{http://coq.inria.fr/stdlib/Coq.Unicode.Utf8\_core}{\coqdocnotation{\ensuremath{\lambda}}} \coqdocvar{$\gamma$}\coqexternalref{::'xCExBB' x '..' x ',' x}{http://coq.inria.fr/stdlib/Coq.Unicode.Utf8\_core}{\coqdocnotation{,}} \coqexternalref{::'xCExBB' x '..' x ',' x}{http://coq.inria.fr/stdlib/Coq.Unicode.Utf8\_core}{\coqdocnotation{(}}\coqdocabbreviation{map} (\coqdocvariable{P} \coqdocnotation{$\star$} \coqdocvariable{$\gamma$}) (\coqref{Groupoid.groupoid interpretation.J Pair}{\coqdocdefinition{J\_Pair}} \coqdocvariable{e} \coqdocvariable{P} \coqdocvariable{$\gamma$})\coqexternalref{::'xCExBB' x '..' x ',' x}{http://coq.inria.fr/stdlib/Coq.Unicode.Utf8\_core}{\coqdocnotation{)}}\coqdocnotation{;} \coqref{Groupoid.groupoid interpretation.J 1}{\coqdocinstance{$\mathsf{J_{comp}}$}} \coqdocvar{\_} \coqdocvar{\_}\coqdocnotation{)} \coqdocnotation{$\star$} \coqdocvariable{p}.\coqdoceol
\coqdocemptyline
\end{coqdoccode}
\noindent where \coqref{Groupoid.groupoid interpretation.BetaT2}{\coqdocdefinition{$\mathsf{BetaT'}$}} is another dedicated version of the $\beta$-rule for \coqref{Groupoid.groupoid interpretation.LamT}{\coqdocdefinition{$\Lambda$}}. \begin{coqdoccode}
\coqdocemptyline
\end{coqdoccode}


\coqlibrary{Groupoid.fun ext}{Library }{Groupoid.fun\_ext}

\begin{coqdoccode}
\coqdocemptyline
\end{coqdoccode}


  Propositional functional extensionality is a direct consequence of the definition
  of equality at product types. It is simply witnessed by a natural transformation
  between (dependent) functions, that is a pointwise equality. This corresponds 
  to the introduction of equality on dependent functions in~\cite{DBLP:conf/popl/LicataH12}.
  Using $\shortuparrow$\coqdocvariable{M}, the weakening for terms, it can be stated as:
\begin{coqdoccode}
\coqdocemptyline
\coqdocnoindent
\coqdockw{Definition} \coqdef{Groupoid.fun ext.FunExt}{FunExt}{\coqdocdefinition{FunExt}} \coqdocvar{Γ} (\coqdocvar{A}:\coqdocdefinition{Typ} \coqdocvariable{Γ}) (\coqdocvar{F}:\coqdocdefinition{TypDep} \coqdocvariable{A}) (\coqdocvar{M} \coqdocvar{N}:\coqdocdefinition{$\mathsf{Tm}$} (\coqdocdefinition{Prod} (\coqdocdefinition{$\Lambda$} \coqdocvariable{F})))\coqdoceol
\coqdocindent{1.00em}
(\coqdocvar{α} : \coqdocdefinition{$\mathsf{Tm}$} (\coqdocdefinition{Prod} (\coqdocdefinition{$\Lambda$} (\coqdocdefinition{Id} (\coqref{Groupoid.fun ext.::'xE2x86x91' x}{\coqdocnotation{$\shortuparrow$}} \coqdocvariable{M} \coqdocnotation{$\star$} \coqdocdefinition{Var} \coqdocvariable{A}) (\coqref{Groupoid.fun ext.::'xE2x86x91' x}{\coqdocnotation{$\shortuparrow$}} \coqdocvariable{N} \coqdocnotation{$\star$} \coqdocdefinition{Var} \coqdocvariable{A}))))): \coqdocdefinition{$\mathsf{Tm}$} (\coqdocdefinition{Id} \coqdocvariable{M} \coqdocvariable{N}).\coqdoceol
\end{coqdoccode}


\coqlibrary{Groupoid.cwf equations}{Library }{Groupoid.cwf\_equations}

\begin{coqdoccode}
\coqdocemptyline
\coqdocemptyline
\end{coqdoccode}
\section{Connection to internal categories with families}




  We now turn to show that have actually a model in the sense of internal categories with families~\cite{dybjer:internaltt}. More precisely, our work can be seen as a formalization of setoid-indexed families of setoids. 


  First, given an equality \coqdocvariable{e} : \coqdocvariable{T} $\sim_1$ \coqdocvar{U} between two types in \coqdocdefinition{Typ} \coqdocvariable{A}, there is a function 
  from \coqdocvariable{t} : \coqdocdefinition{$\mathsf{Tm}$} \coqdocvariable{T} to \coqdocvariable{t} \coqdockw{with} \coqdocvariable{e} : \coqdocdefinition{$\mathsf{Tm}$} \coqdocvar{U} that comes from the
  functoriality of \coqdocvar{$\Pi$}. This corresponds to the \emph{reindexing
  map} of families of setoids.
\begin{coqdoccode}
\coqdocemptyline
\coqdocemptyline
\end{coqdoccode}
  \paragraph{\lrule{Substitution Laws}.}
 \begin{coqdoccode}
\coqdocemptyline
\coqdocnoindent
\coqdockw{Definition} \coqdef{Groupoid.cwf equations.Prod sigma law}{$\mathsf{Prod_{\sigma law}}$}{\coqdocdefinition{$\mathsf{Prod_{\sigma law}}$}} \{\coqdocvar{Δ} \coqdocvar{Γ}\} (σ:\coqdocnotation{[}\coqdocvariable{Δ} \coqdocnotation{$\longrightarrow$} \coqdocvariable{Γ}\coqdocnotation{]}) (\coqdocvar{A}:\coqdocdefinition{Typ} \coqdocvariable{Γ}) (\coqdocvar{F}:\coqdocdefinition{TypFam} \coqdocvariable{A}):\coqdoceol
\coqdocindent{1.00em}
\coqdocdefinition{Prod} \coqdocvariable{F} \coqdocnotation{$⋅$} \coqdocvariable{σ} \coqdocnotation{$\sim_1$} \coqdocdefinition{Prod} (\coqdocvariable{F} \coqdocnotation{$\circ$} \coqdocvariable{σ}).\coqdoceol
\coqdocemptyline
\coqdocnoindent
\coqdockw{Program Definition} \coqdef{Groupoid.cwf equations.LamT sigma law}{$\mathsf{\Lambda_{\sigma law}}$}{\coqdocdefinition{$\mathsf{\Lambda_{\sigma law}}$}} \coqdocvar{Δ} \coqdocvar{Γ} (\coqdocvar{A}:\coqdocdefinition{Typ} \coqdocvariable{Γ}) (\coqdocvar{B}:\coqdocdefinition{TypDep} \coqdocvariable{A}) (σ:\coqdocnotation{[}\coqdocvariable{Δ} \coqdocnotation{$\longrightarrow$} \coqdocvariable{Γ}\coqdocnotation{]}):\coqdoceol
\coqdocindent{1.00em}
\coqdocdefinition{$\Lambda$} \coqdocvariable{B} \coqdocnotation{$\circ$} \coqdocvariable{σ} \coqdocnotation{$\sim_1$} \coqdocdefinition{$\Lambda$} (\coqdocvariable{B} \coqdocnotation{$⋅$} \coqref{Groupoid.cwf equations.Subm}{\coqdocdefinition{Subm}} \coqdocvariable{σ}).\coqdoceol
\coqdocemptyline
\coqdocnoindent
\coqdockw{Definition} \coqdef{Groupoid.cwf equations.Lam sigma law}{Lam\_sigma\_law}{\coqdocdefinition{Lam\_sigma\_law}} \{\coqdocvar{Δ} \coqdocvar{Γ}\} (σ:\coqdocnotation{[}\coqdocvariable{Δ} \coqdocnotation{$\longrightarrow$} \coqdocvariable{Γ}\coqdocnotation{]}) \{\coqdocvar{A}:\coqdocdefinition{Typ} \coqdocvariable{Γ}\} \{\coqdocvar{B}:\coqdocdefinition{TypDep} \coqdocvariable{A}\} (\coqdocvar{b}:\coqdocdefinition{$\mathsf{Tm}$} \coqdocvariable{B}) :\coqdoceol
\coqdocindent{1.00em}
\coqdocnotation{(}\coqdocdefinition{Lam} \coqdocvariable{b}\coqdocnotation{)} \coqdocnotation{$\circ$} \coqdocvariable{σ} \coqdocnotation{with} \coqref{Groupoid.cwf equations.Prod sigma law}{\coqdocdefinition{$\mathsf{Prod_{\sigma law}}$}} \coqdocvar{\_} \coqdocvar{\_} \coqdocnotation{with} \coqref{Groupoid.cwf equations.Prod eq}{\coqdocdefinition{Prod\_eq}} (\coqref{Groupoid.cwf equations.LamT sigma law}{\coqdocdefinition{$\mathsf{\Lambda_{\sigma law}}$}} \coqdocvar{\_} \coqdocvar{\_}) \coqdocnotation{$\sim_1$}\coqdoceol
\coqdocindent{1.00em}
\coqdocdefinition{Lam} (\coqdocvariable{b} \coqdocnotation{$\circ$} \coqdocnotation{(}\coqref{Groupoid.cwf equations.Subm}{\coqdocdefinition{Subm}} \coqdocvariable{σ}\coqdocnotation{)}).\coqdoceol
\coqdocnoindent
\coqdockw{Defined}.\coqdoceol
\coqdocemptyline
\coqdocemptyline
\coqdocnoindent
\coqdockw{Definition} \coqdef{Groupoid.cwf equations.App sigma law}{App\_sigma\_law}{\coqdocdefinition{App\_sigma\_law}} \coqdocvar{Δ} \coqdocvar{Γ} (\coqdocvar{A}:\coqdocdefinition{Typ} \coqdocvariable{Γ}) (\coqdocvar{F}:\coqdocdefinition{TypFam} \coqdocvariable{A}) (σ:\coqdocnotation{[}\coqdocvariable{Δ} \coqdocnotation{$\longrightarrow$} \coqdocvariable{Γ}\coqdocnotation{]})\coqdoceol
\coqdocindent{5.50em}
(\coqdocvar{c}:\coqdocdefinition{$\mathsf{Tm}$} (\coqdocdefinition{Prod} \coqdocvariable{F})) (\coqdocvar{a}:\coqdocdefinition{$\mathsf{Tm}$} \coqdocvariable{A}):\coqdoceol
\coqdocindent{1.00em}
\coqdocnotation{[}\coqdocnotation{(}\coqdocvariable{c} \coqdocnotation{$\star$} \coqdocvariable{a}\coqdocnotation{)} \coqdocnotation{$\circ$} \coqdocvariable{σ}\coqdocnotation{]} \coqdocnotation{=} \coqdocnotation{[}\coqdocnotation{(}\coqdocvariable{c} \coqdocnotation{$\circ$} \coqdocvariable{σ} \coqdocnotation{with} \coqref{Groupoid.cwf equations.Prod sigma law}{\coqdocdefinition{$\mathsf{Prod_{\sigma law}}$}} \coqdocvar{\_} \coqdocvar{\_}\coqdocnotation{)} \coqdocnotation{$\star$} \coqdocnotation{(}\coqdocvariable{a} \coqdocnotation{$\circ$} \coqdocvariable{σ}\coqdocnotation{)}\coqdocnotation{]} :=\coqdoceol
\coqdocindent{1.00em}
\coqdocconstructor{eq\_refl} \coqdocnotation{[}\coqdocnotation{(}\coqdocnotation{(}\coqdocvariable{c} \coqdocnotation{$\circ$} \coqdocvariable{σ}\coqdocnotation{)} \coqdocnotation{with} \coqref{Groupoid.cwf equations.Prod sigma law}{\coqdocdefinition{$\mathsf{Prod_{\sigma law}}$}} \coqdocvar{\_} \coqdocvar{\_}\coqdocnotation{)} \coqdocnotation{$\star$} \coqdocnotation{(}\coqdocvariable{a} \coqdocnotation{$\circ$} \coqdocvariable{σ}\coqdocnotation{)}\coqdocnotation{]}.\coqdoceol
\coqdocemptyline
\coqdocemptyline
\end{coqdoccode}
  \paragraph{\lrule{Conv}.}
  $\beta$-reduction for abstraction is valid as a definitional equality,
  where \coqdocdefinition{SubExtId} is a specialization of \coqdocvar{SubExt} with the identity substitution.
\begin{coqdoccode}
\coqdocemptyline
\coqdocnoindent
\coqdockw{Definition} \coqdef{Groupoid.cwf equations.Beta}{Beta}{\coqdocdefinition{Beta}} \{\coqdocvar{Γ}\} \{\coqdocvar{A}:\coqdocdefinition{Typ} \coqdocvariable{Γ}\} \{\coqdocvar{F}:\coqdocdefinition{TypDep} \coqdocvariable{A}\} (\coqdocvar{b}:\coqdocdefinition{$\mathsf{Tm}$} \coqdocvariable{F}) (\coqdocvar{a}:\coqdocdefinition{$\mathsf{Tm}$} \coqdocvariable{A}) \coqdoceol
\coqdocindent{1.00em}
: \coqdocnotation{[}\coqdocdefinition{Lam} \coqdocvariable{b} \coqdocnotation{$\star$} \coqdocvariable{a}\coqdocnotation{]} \coqdocnotation{=} \coqdocnotation{[}\coqdocvariable{b} \coqdocnotation{$\circ$} \coqdocdefinition{SubExtId} \coqdocvariable{a}\coqdocnotation{]} := \coqdocconstructor{eq\_refl} \coqdocvar{\_}.\coqdoceol
\coqdocemptyline
\end{coqdoccode}
 \noindent 
  The other beta rules and the equational theory of 
  explicit substitutions can be validated in the same way, showing that
  this forms a CwF.
\begin{coqdoccode}
\coqdocemptyline
\coqdocemptyline
\coqdocnoindent
\coqdockw{Definition} \coqdef{Groupoid.cwf equations.EtaT}{EtaT}{\coqdocdefinition{EtaT}} \coqdocvar{Γ} (\coqdocvar{A}:\coqdocdefinition{Typ} \coqdocvariable{Γ}) (\coqdocvar{F}:\coqdocdefinition{TypFam} \coqdocvariable{A})\coqdoceol
\coqdocnoindent
: \coqdocdefinition{$\Lambda$} (\coqdocnotation{(}\coqdocvariable{F} \coqdocnotation{$\circ$} \coqdocdefinition{Sub}\coqdocnotation{)} \coqdocnotation{\{\{}\coqdocdefinition{Var} \coqdocvariable{A}\coqdocnotation{\}\}}) \coqdocnotation{$\sim_1$} \coqdocvariable{F}.\coqdoceol
\coqdocnoindent
\coqdockw{Defined}.\coqdoceol
\coqdocemptyline
\coqdocnoindent
\coqdockw{Definition} \coqdef{Groupoid.cwf equations.Eta}{Eta}{\coqdocdefinition{Eta}} \{\coqdocvar{Γ}\} \{\coqdocvar{A}:\coqdocdefinition{Typ} \coqdocvariable{Γ}\} \{\coqdocvar{F}:\coqdocdefinition{TypFam} \coqdocvariable{A}\} (\coqdocvar{c}:\coqdocdefinition{$\mathsf{Tm}$} (\coqdocdefinition{Prod} \coqdocvariable{F}))\coqdoceol
\coqdocnoindent
: \coqdocdefinition{Lam} (\coqref{Groupoid.cwf equations.::'xE2x86x91' x}{\coqdocnotation{$\shortuparrow$}} \coqdocvariable{c} \coqdocnotation{$\star$} \coqdocdefinition{Var} \coqdocvar{\_}) \coqdocnotation{with} \coqref{Groupoid.cwf equations.Prod eq}{\coqdocdefinition{Prod\_eq}} (\coqref{Groupoid.cwf equations.EtaT}{\coqdocdefinition{EtaT}} \coqdocvariable{F}) \coqdocnotation{$\sim_1$} \coqdocvariable{c}.\coqdoceol
\coqdocnoindent
\coqdockw{Defined}.\coqdoceol
\coqdocemptyline
\end{coqdoccode}
We can interpret the J eliminator of MLTT on \coqdocdefinition{Id} using functoriality of \coqdocvariable{P} and of product (\coqdocdefinition{$\mathsf{\Pi_{comp}}$}). In the definition of J, the predicate \coqdocvariable{P} depends on the proof of equality, which is interpreted using a \coqdocdefinition{Sigma} type. The functoriality of \coqdocvariable{P} is used on the term \coqdocdefinition{J\_Pair} \coqdocvariable{e} \coqdocvariable{P} \coqdocvariable{$\gamma$}, which is a proof that (\coqdocvariable{a};\coqdocdefinition{Refl} \coqdocvariable{a}) is equal to (\coqdocvariable{b};\coqdocvariable{e}). The notation $\shortuparrow$ \coqdocvariable{a} is used to convert the type of terms according to equality on \coqdocdefinition{$\Lambda$}. \begin{coqdoccode}
\coqdocemptyline
\coqdocnoindent
\coqdockw{Definition} \coqdef{Groupoid.cwf equations.J}{J}{\coqdocdefinition{J}} \coqdocvar{Γ} (\coqdocvar{A}:\coqdocdefinition{Typ} \coqdocvariable{Γ}) (\coqdocvar{a} \coqdocvar{b}:\coqdocdefinition{$\mathsf{Tm}$} \coqdocvariable{A}) (\coqdocvar{P}:\coqdocdefinition{TypFam} (\coqdocdefinition{Sigma} (\coqdocdefinition{$\Lambda$} (\coqdocdefinition{Id} (\coqdocvariable{a} \coqdocnotation{$\circ$} \coqdocdefinition{Sub}) (\coqdocdefinition{Var} \coqdocvariable{A})))))\coqdoceol
\coqdocindent{7.50em}
(\coqdocvar{e}:\coqdocdefinition{$\mathsf{Tm}$} (\coqdocdefinition{Id} \coqdocvariable{a} \coqdocvariable{b})) (\coqdocvar{p}:\coqdocdefinition{$\mathsf{Tm}$} (\coqdocvariable{P}\coqdocnotation{\{\{}\coqdocdefinition{Pair} \coqdocnotation{$\shortuparrow$} \coqdocnotation{(}\coqdocdefinition{Refl} \coqdocvariable{a}\coqdocnotation{)}\coqdocnotation{\}\}})) \coqdoceol
\coqdocindent{1.00em}
: \coqdocdefinition{$\mathsf{Tm}$} (\coqdocvariable{P}\coqdocnotation{\{\{}\coqdocdefinition{Pair} \coqdocnotation{$\shortuparrow$}\coqdocvariable{e}\coqdocnotation{\}\}}) := \coqdocdefinition{$\mathsf{\Pi_{comp}}$} \coqdocnotation{(}\coqexternalref{::'xCExBB' x '..' x ',' x}{http://coq.inria.fr/stdlib/Coq.Unicode.Utf8\_core}{\coqdocnotation{\ensuremath{\lambda}}} \coqdocvar{$\gamma$}\coqexternalref{::'xCExBB' x '..' x ',' x}{http://coq.inria.fr/stdlib/Coq.Unicode.Utf8\_core}{\coqdocnotation{,}} \coqexternalref{::'xCExBB' x '..' x ',' x}{http://coq.inria.fr/stdlib/Coq.Unicode.Utf8\_core}{\coqdocnotation{(}}\coqdocabbreviation{map} (\coqdocvariable{P} \coqdocnotation{$\star$} \coqdocvariable{$\gamma$}) (\coqdocdefinition{J\_Pair} \coqdocvariable{e} \coqdocvariable{P} \coqdocvariable{$\gamma$})\coqexternalref{::'xCExBB' x '..' x ',' x}{http://coq.inria.fr/stdlib/Coq.Unicode.Utf8\_core}{\coqdocnotation{)}}\coqdocnotation{;} \coqdocinstance{$\mathsf{J_{comp}}$} \coqdocvar{\_} \coqdocvar{\_}\coqdocnotation{)} \coqdocnotation{$\star$} \coqdocvariable{p}.\coqdoceol
\coqdocemptyline
\end{coqdoccode}



% \section{Elaborating Polymorphic Universes}
\label{sec:type-theory-with}

This section presents our elaboration from a source level language
with typical ambiguity and universe polymorphism to a conservative
extension of the core calculus presented in
Section~\ref{sec:definitions}.
%
Typical ambiguity lets users write only anonymous levels (as
$\Type{}$) in the source language, leaving the relationship between
different universes to be implicitly inferred by the system.  The job
of the elaborator is to give names to these levels and compute the
constraints associated to them that make the term type-check, if
possible.  Roughly speaking, typical ambiguity associated with
universe polymorphism gives something akin to Hindley-Milner type
inference: the levels of universes are entirely inferred by the
system, as well as the proper generalization by universes to give a
most general type. A striking example of this additional generality is
the following.  Suppose we define two universes:
%
\input{defU}
% \begin{verbatim}
%   Definition U2 := Type.
%   Definition U1 := Type : U2.
% \end{verbatim}
% In the non-polymorphic case but with typical ambiguity, these two
% definitions are elaborated as $\cstu{\texttt{U2}}{} := \Type{u} :
% \Type{u+1}$ and $\cstu{\texttt{U1}}{} := \Type{v} :
% \cstu{\texttt{U2}}{}$ with a single, global constraint $v < u$.

% In a polymorphic setting, \texttt{U2} is elaborated as a polymorphic
% constant $\cstu{\texttt{U2}}{u} := \Type{u} : \Type{u+1}$ where $u$ is
% a bound universe variable. The monomorphic definition of \texttt{U1}
% is elaborated as $\cstu{\texttt{U1}}{} := \Type{v} :
% \cstu{\texttt{U2}}{u'} \equiv \Type{u'}$ with a single global
% constraint $v < u'$ for a fresh $u'$. In other words, \texttt{U2}'s
% universe is no longer fixed and a fresh level is generated at every
% occurence of the constant.

We can hence reuse a polymorphic constant at different, incompatible
levels. Another example is given by the polymorphic identity function,
defined as: \[\cstu{id}{u} := λ (A : \Type{u}) (a : A), a : Π (A :
\Type{u}), A → A\]

If we apply $\cstu{id}{}$ to itself, we elaborate an application:
\[(\cstu{id}{v}\ (Π (A : \Type{u}), A → A)~\cstu{id}{u} : (Π (A :
\Type{u}), A → A)\]

Type-checking generates a constraint in this case, to ensure that
the universe of $Π (A : \Type{u}), A → A$, that is $\lub{u}{u+1} = u+1$,
is smaller or equal to (the fresh) $v$. It adds indeed a constraint
$u < v$.  With a monomorphic $\cst{id}$, the generated constraint $u < u$
raises a universe inconsistency.

\subsection{Universe Polymorphic Definitions}
\label{sec:univ-polym-defin}

The elaboration actually needs a notion of definitions in its target
language, so we first formalize an extension of the core theory defined
in Section~\ref{sec:definitions}.

We build on the design of \citet{DBLP:journals/tcs/HarperP91} for the
\textsc{LEGO} proof assistant, allowing arbitrary nesting of polymorphic
constants. We simply add a new term former to the calculus
$\cstu{c}{\vec{u}}$ for referring to a constant $c$ defined in a global
environment $Σ$, instantiating its universes at $\vec{u}$. The typing judgment
(denoted $\vdash^{d}$) is made relative to this environment and there is
a new introduction rule for constants:
\begin{mathpar}
\irule{Constant}
{(c : \vec{i} \models ψ_c \vdash t : τ) \in Σ \\
  ψ \models ψ_c[\vec{u/i}]}
{\tcheckd{Σ; Γ}{ψ}{\cstu{c}{\vec{u}}}{τ[\vec{u/i}]}}
\end{mathpar}

Universe instances $\vec{u}$ are simply lists of universe \emph{levels} that
instantiate the universes abstracted in definition $c$. A single constant
can hence be instantiated at multiple different levels, giving a form of
parametric polymorphism. The constraints associated to these variables
are checked against the given constraints for consistency, just as if we 
were checking the constraints of the instantiated definitions directly.
The general principle guiding us is that the use of
constants should be \emph{transparent}, in the sense that the
system should behave exactly the same when using a constant or its body.
Substitution of universe levels for universe levels is defined in a
completely standard way over universes, terms and constraints.

Of course, well-formedness of the new global context of constants $Σ$
has to be checked (Figure~\ref{fig:constglob}). As we are adding a
global context and want to handle both polymorphic and monomorphic
definitions (mentioning global universes), both a global set of
constraints $Ψ$ and local universe constraints $ψ_c$ for
each constant must be handled.

\begin{figure}
\begin{mathpar}
\irule{Constant-Mono}
{Σ \vdash^{d}_{Ψ} \\
Ψ \cup ψ_c \models \\
\tcheckd{Σ; \Nil}{Ψ \cup ψ_c}{t}{τ} \\ 
c \notin Σ}
{Σ, (c : ε \vdash t : τ) \vdash^{d}_{Ψ \cup ψ_c}}

\irule{Constant-Poly}
{Σ \vdash^{d}_Ψ \\
Ψ \cup ψ_c \models \\
\tcheckd{Σ; \Nil}{Ψ \cup ψ_c}{t}{τ} \\ 
c \notin Σ}
{Σ, (\cst{c} : \vec{i} \vdash^{d}_{ψ_c} t : τ) \vdash_{Ψ}^{d}}
\end{mathpar}
\caption{Well-formed global environments}\label{fig:constglob}
\end{figure}

When introducing a constant in the global environment, we are given a
set of constraints necessary to typecheck the term and its type. In
the case of a monomorphic definition (Rule \textsc{Constant-Mono}), we
simply check that the local constraints are consistent with the global
ones and add them to the global environment.  In Rule
\textsc{Constant-Poly}, the abstraction of local universes is
performed. In the polymorphic case, an additional set of
universes $\vec{i}$ is given, for which the constant is meant to be
polymorphic. To support this, the global constraints are not augmented
with those of $ψ_c$ but instead are kept locally to the constant
definition $c$.

We add a new reduction rule for unfolding of constants:
\[\begin{array}{lclr}
  \cstu{c}{\vec{u}} & "->"_\delta & t[\vec{u/i}] & (\cst{c} :
  \vec{i} \models \_ \vdash t : \_) \in Σ
\end{array}\]
Again, conversion should be a congruence modulo $δ$. The actual strategy
employed in the kernel to check conversion/cumulativity of $T$ and $U$
is to always take the $β$ head normal form of $T$ and $U$ and to do head
$δ$ reductions step-by-step (choosing which side to unfold according to
an oracle if necessary), as described by the following rules:
\begin{mathpar}
\irule{R-δ-l}
{\cstu{c}{\vec{i}} "->"_\delta t \\
  \tgenconv{R}{ψ}{t~\vec{a}}{u}}
{\tgenconv{R}{ψ}{\cstu{c}{\vec{i}}~\vec{a}}{u}}

\irule{R-δ-r}
{\cstu{c}{\vec{i}} "->"_\delta u \\
  \tgenconv{R}{ψ}{t}{u~\vec{a}}}
{\tgenconv{R}{ψ}{t}{\cstu{c}{\vec{i}}~\vec{a}}}
\end{mathpar}

This allows to introduce an additional rule for \emph{first-order}
unification of constant applications, which poses a number of problems
when looking at conversion/unification with universes. The rules for
conversion include the following short-cut rule \lrule{R-FO} that
avoids unfolding definitions in case both terms start with the same
head constant. 
%
\begin{mathpar}
\irule{R-FO}
{\tgenconv{R}{ψ}{\vec{as}}{\vec{bs}}}
{\tgenconv{R}{ψ}{\cstu{c}{\vec{u}}~\vec{as}}{\cstu{c}{\vec{v}}~\vec{bs}}}
\end{mathpar}
%
This rule not only has priority over the \lrule{R-δ}
rules, but \emph{backtrack} on its application can also be done if the
premise cannot be derived.

The question is then, what can be expected on universes? A natural
choice is to allow identification if the universe instances are
pointwise equal: $\tconstreq{ψ}{\vec{u}}{\vec{v}}$. This is certainly a
sound choice, if we can show that it does not break the principle of
\emph{transparency} of constants. Indeed, due to the cumulativity
relation on universes, we might get in a situation where the $δ$-normal
forms of $\cstu{c}{\vec{u}}~\vec{as}$ and $\cstu{c}{\vec{v}}~\vec{bs}$
are convertible while $\tconstrneq{ψ}{\vec{u}}{\vec{v}}$. This is where
backtracking is useful: if the constraints are not derivable, we
backtrack and unfold one of the two sides, ultimately doing
conversion on the $βδ$-normal forms if necessary. Note that
equality of universe instances is forced even if in cumulativity mode.

\begin{mathpar}
\irule{R-FO}
{\tgenconv{R}{ψ}{\vec{as}}{\vec{bs}} \\
  \tconstreq{ψ}{\vec{u}}{\vec{v}}}
{\tgenconv{R}{ψ}{\cstu{c}{\vec{u}}~\vec{as}}{\cstu{c}{\vec{v}}~\vec{bs}}}
\end{mathpar}

There is a straightforward conservativity result of the calculus with
polymorphic definitions over the original one. Below, $\nfdeltab{T}$
denotes the δ-normalization of $T$, which is terminating as there is
no recursive constants. It leaves us with a term with no constants,
i.e., a term of \textsc{CC}.

\begin{theorem}[Conservative extension]
  If $\tcheckd{Σ; Γ}{Ψ}{t}{T}$ then $\nfdeltab{Γ}\vdash_{Ψ} t\nfdelta :
  T\nfdelta$.
  If $\tgenconv{R}{Ψ}{T}{U}$ then $\tgenconv{R}{Ψ}{T\nfdelta}{U\nfdelta}$.
  If $(c : \vec{i} \vdash^{d}_{ψ_c} t : τ) \in Σ$ then 
  for all fresh \vec{u}, $\tcheck{Σ; ε}{ψ_c[\vec{u/i}]}{(t[\vec{u/i}])\nfdelta}{(τ[\vec{u/i}])\nfdelta}$.
\end{theorem}
\begin{proof}
  By mutual induction on the typing, conversion and well-formedness
  derivations. 

  For typing and well-formedness, all cases are by induction except for
  \textsc{Constant}:
  \begin{mathpar}
    \irule{}
    {(c : \vec{i} \vdash^{d}_{ψ_c} t : τ) \in Σ \\
      Ψ \models ψ_c[\vec{u/i}]}
    {\tcheckd{Σ; Γ}{Ψ}{\cstu{c}{\vec{u}}}{τ[\vec{u/i}]}}
  \end{mathpar}  

  To show: $\tcheck{Γ\nfdelta}{Ψ}{(t[\vec{u/i}])\nfdelta}{(τ[\vec{u/i}])\nfdelta}$.
  By induction and weakening, it becomes:
  $\tcheck{Γ\nfdelta}{ψ_c[\vec{u/i}]}{(t[\vec{u/i}])\nfdelta}{(τ[\vec{u/i}])\nfdelta}$.
  We conclude using the second premise of \textsc{Constant} and
  monotonicity of typing with respect to constraint entailment.

  For conversion, we must check that it is invariant under
  $δ$-normalization. Most rules follow easily. For example, for
  \lrule{R-δ-left}, we have by induction
  $\tconv{ψ}{(t~\vec{a})\nfdelta}{\nfdelta{u}}$, and by definition
  $(\cstu{c}{\vec{i}}~\vec{a})\nfdelta = (t~\vec{a})\nfdelta$.

  For \lrule{R-FO}, we get
  $\tgenconv{R}{\phi}{\vec{as}\nfdelta}{\vec{bs}\nfdelta}$
  by induction. We have $(c : \vec{i} \vdash^{d}_{ψ_c} t : τ) \in Σ$.
  By induction $\tcheck{}{ψ_c[\vec{u/i}]}{(t[\vec{u/i}])\nfdelta}{(τ[\vec{u/i}])\nfdelta}$.
  By substitutivity of universes and the premise
  $\tconstreq{\phi}{\vec{u}}{\vec{v}}$,
  we deduce $\tgenconv{R}{\phi}{t[\vec{u/i}]\nfdelta}{t[\vec{v/i}]\nfdelta}$. We conclude by
  substitutivity of conversion.  
\end{proof}

The converse implication is straightforward as only one rule has been
added to the original calculus. 

% \subsection{Shapes of universes and constraints}
% \label{sec:shapes}
% There are important subtleties regarding the form of universes and
% constraints appearing in derivations. Not all universes appearing in a
% derivation are levels: indeed when typechecking a product $Π x : A. B$,
% we get as type the sort $\lub{u}{v}$ where $u$ and $v$ are the sorts of
% $A$ and $B$ respectively. However, the form of universes considered in
% this system is not the free algebra on $i + n$ and $\sqcup$, instead we
% consider only universes which are universes levels $i$ or formal least
% upper bounds of levels and successors of levels: $\max(\vec{i},
% \vec{j+1})$. This subset is called the algebraic universes, and they
% were introduced by \citet{HerbelinTypes} as a compact representation for
% universes. We recall in appendix \ref{sec:algunivs} the definitions
% needed to show that this subset of the universes is enough to typecheck
% terms and that the kernel need only handle atomic constraints.
% We have setup our basic framework for
% polymorphic definitions. A separate elaboration phase, described in \S
% \ref{sec:type-theory-with} produces terms annotated with universes to
% be checked by the kernel. It takes a source level expression written
% using typical ambiguity (anonymous \Type{} occurences) and produces an
% annotated term with the set of constraints its typing entails. This
% can be finally be fed to the kernel as a monomorphic or polymorphic
% definition.

% \subsection{Shapes of universes and constraints}
% \label{sec:shapes}
% There are important subtleties regarding the form of universes and
% constraints appearing in derivations. Not all universes appearing in a
% derivation are levels: indeed when typechecking a product $Π x : A. B$,
% we get as type the sort $\lub{u}{v}$ where $u$ and $v$ are the sorts of
% $A$ and $B$ respectively. However, the form of universes considered in
% this system is not the free algebra on $i + n$ and $\sqcup$, instead we
% consider only universes which are universes levels $i$ or formal least
% upper bounds of levels and successors of levels: $\max(\vec{i},
% \vec{j+1})$. This subset is called the algebraic universes, and they
% were introduced by \citet{HerbelinTypes} as a compact representation for
% universes. We recall in appendix \ref{sec:algunivs} the definitions
% needed to show that this subset of the universes is enough to typecheck
% terms and that the kernel need only handle atomic constraints.

\subsection{Elaboration}
\label{sec:elaboration}

Elaboration takes a source level expression and produces a corresponding
core term together with its inferred type. In doing so, it might use
arbitrary heuristics to fill in the missing parts of the
source expression and produce a complete core term. A canonical example
of this is the inference of implicit arguments in dependently-typed
languages: for example, applications of the $\cst{id}$ constant defined
above do not necessarily need to be annotated with their first argument
(the type $A$ at which we want the identity $A → A$), as it can be
inferred from the type of the second argument, or the typing constraint
at the point this application occurs. Other examples include the
insertion of coercions and the inference of dictionaries.

Most elaborations do not go from the source level to the core terms
directly, instead they go through an intermediate language that 
extends the core language with \emph{existential variables},
representing holes to be filled in the term. Existential variables are
declared in a context: 
\[Σ_e ::= ε ~`|~ Σ_e \cup (?_n : Γ \vdash body : τ)\]
where $body$ is empty or a term $t$ which is then called the \emph{value} of the
existential.

In the term, they appear applied to an instance $σ$ of their local
context $Γ$, which is written $?_n[σ]$. The corresponding typing rule
for the intermediate language is:
\begin{mathpar}
\irule{Evar}
{(?_n : Γ \vdash \_ : τ) \in Σ_e\\
Σ_e; Γ' \vdash σ : Γ}
{Σ_e; Γ' \vdash\,?_n[σ] : τ[σ]}
\end{mathpar}

For polymorphic universes, elaboration keeps track of the new variables,
that may be subject to unification, in a \emph{universe context}:
\[Σ_u, Φ ::= \vec{u_s} \models \mathcal{C}\] Universe levels are
annotated by a flag $s ::= \mathsf{r} `| \mathsf{f}$ during elaboration,
to indicate their rigid or flexible status.  Elaboration expands any
occurrence of the anonymous $\Type{}$ into a $\Type{i}$ for a fresh,
rigid $i$ and every occurrence of the constant $\cst{c}$ into a fresh
instance $\cstu{c}{u}$ ($\vec{u}$ being all fresh flexible levels). The
idea behind this terminology is that rigid universes may not be tampered
with during elaboration, they correspond to universes that must appear
and possibly be quantified over in the resulting term. The flexible
variables, on the other hand, do not appear in the source term and might
be instantiated during unification, like existential variables. We will
come back to this distinction when we apply minimization to universe
contexts. The $Σ_u$ context subsumes the context of constraints $Ψ$ we
used during typechecking.

The elaboration judgment is written: \[Σ; Σ_e; Σ_u ; Γ \vdash_e t "<=" τ
"~>" Σ_{e'}; Σ_{u'}; Γ \vdash t' : τ\]
%
It takes the global environment $Σ$, a set of existentials $Σ_e$, a
universe context $Σ_u$, a variable context $Γ$, a source-level term $t$
and a typing constraint $τ$ (in the intermediate language) and produces
new existentials and universes along with an (intermediate-level) term
whose type is guaranteed to be $τ$.

Most of the contexts of this judgment are folded around in the obvious
way, so we won't mention them anymore to recover lightweight notations.
The important thing to note here is that we work at the intermediate
level only, with existential variables, so instead of doing pure
conversion we are actually using a unification algorithm when applying
the conversion/cumulativity rules.

Typing constraints come from the type annotation (after the $:$) of a
definition, or are inferred from the type of a constant, variable or
existential variable declared in the context. If no typing constraint is
given, it is generated as a fresh existential variable of type
$\Type{i}$ for a fresh $i$ ($i$ is flexible in that case).

For example, when elaborating an application $\cst{f}~t$, under a typing
constraint $τ$, we first elaborate the constant $\cst{f}$ to a term of
functional type $\cstu{f}{i} : Π A : \Type{i}. B$, then we elaborate $t
"<=" \Type{i} "~>" t', Σ_{u'}$. We check cumulativity
$\icumul{Σ_{u'}}{B[t'/A]}{τ}{Σ_{u''}}$, generating constraints and
returning $Σ_{u''} \vdash \cstu{f}{i}~t' : τ$.

At the end of elaboration, we might apply some more inference to resolve
unsolved existential variables. When there are no remaining unsolved
existentials, we can simply unfold all existentials to their values in 
the term and type to produce a well-formed typing derivation of the core
calculus, together with its set of universe constraints.

\subsubsection{Unification}
\label{sec:unification}

Most of the interesting work performed by the elaboration actually
happens in the unification algorithm that is used in place of conversion
during refinement. The refinement rule firing cumulativity is:
\begin{mathpar}
\irule{Sub}
{Σ; Σ_e; Σ_u ; Γ \vdash_e t "~>" Σ_{e'}; Σ_{u'}; Γ \vdash t' : τ' \\
  Σ_{e'}; Σ_{u'} := (\vec{u_s} \models ψ); Γ \vdash
  \icumul{}{τ'}{τ}{Σ_{e''}, ψ'}}
{Σ; Σ_e; Σ_u ; Γ \vdash_e t "<=" τ "~>" Σ_{e''}; (\vec{u_s} \models ψ'); Γ \vdash t' : τ}
\end{mathpar}
%
If checking a term $t$ against a typing constraint $τ$ and $t$ is 
a neutral term (variables, constants and casts), then we infer its 
type $τ'$ and unify it with the assigned type $τ$.

Contrary to the conversion judgment $\tcumul{ψ}{T}{U}$ which only checks
that constraints are implied by $ψ$, unification and conversion during
elaboration (Figure \ref{fig:icumul}) can additionally \emph{produce} a substitution of
existentials and universe constraints, hence we have the judgment $Σ_e';
Σ_{u'} := (\vec{u_s} \models ψ); Γ \vdash \icumul{}{T}{U}{Σ_{e''}, ψ'}$
which unifies $T$ and $U$ with subtyping, refining the set of
existential variables and universe constraints to $Σ_{e''}$ and $ψ'$,
 so that $\tcumul{ψ'}{T[Σ_{e''}]}{U[Σ_{e''}]}$
is derivable.  We abbreviate this judgment
$\icumul{ψ}{T}{U}{ψ'}$, the environment of existentials $Σ_e$, the set
of universe variables $\vec{u_s}$ and the local environment $Γ$ being
inessential for our presentation.

\begin{figure}
\begin{mathpar}
\irule{Elab-R-Type}
{\tconsistent{ψ \cup u \relR v}}
{\igenconv{\relR}{ψ}{\Type{u}}{\Type{v}}{ψ \cup u \relR v}}

\irule{Elab-R-Prod}
{\igenconv{=}{ψ}{A}{A'}{ψ'}  \\
 \igenconv{\relR}{ψ'}{B}{B'}{ψ''}}
{\igenconv{\relR}{ψ}{\Pi \vdecl{x}{A}\mdot B}{\Pi \vdecl{x}{A'}\mdot B'}{ψ''}}

\irule{Elab-R-Red}
{\igenconv{\relR}{ψ}{\whdb{A}}{\whdb{B}}{ψ'}} %\\ \text{$A$ or $B$ not in whnf}}
{\igenconv{\relR}{ψ}{A}{B}{ψ'}}
\end{mathpar}
\caption{Conversion/cumulativity inference $\igenconv{\relR}{\_}{\_}{\_}{\_}$}\label{fig:icumul}
\end{figure}

The rules concerning universes follow the conversion judgment, building
up a most general, consistent set of constraints according to the
conversion problem. For the definition/existential fragment of the
intermediate language, things get a bit more involved. Indeed, in
general, higher-order unification of terms in the calculus of
constructions is undecidable, so we cannot hope for a complete
unification algorithm. Barring completeness, we might want to ensure
correctness in the sense that a unification problem $t \equiv u$ is
solved only if there is a most general unifier $σ$ (a substitution of
existentials by terms) such that $t[σ] \equiv u[σ]$, like the one
defined in \cite{Abel:2011fk}. This is however not the case in \Coq's
unification algorithm, because of the use of a first-order unification
heuristic. The next section presents a generalization of this
algorithm to polymorphic universes.

\subsubsection{First-Order unification}

Consider unification of polymorphic constants. Suppose we are unifying
the same polymorphic constant applied to different universe instances:
$\cstu{c}{\vec{u}} \equiv \cstu{c}{\vec{v}}$. We would like to avoid
having to unfold the constant each time such a unification
occurs. What should be the relation on the universe levels then? A
simple solution is to force $u$ and $v$ to be equal, as in the
following example:
\[\cstu{id}{j}~\Type{i} \equiv \cstu{id}{m}~((λ A : \Type{l}, A)~\Type{i})\]

The at-typing generated constraints give $i < j, l ≤ m, i < l$. If we
add the constraint $j = m$, then the constraints reduce to $i < m, i <
l, l ≤ m "<=>" i < l, l ≤ m$. The unification didn't add any
constraint, so it looks most general. However, if a constant hides an
arity, we might be too strict here, for example consider the definition
$\cstu{fib}{i,j} := λ A : \Type{i}, A → \Type{j}$ with no constraints and the
unification problem:
\[\icumul{}{\cstu{fib}{i,\Prop}}{\cstu{fib}{i',j}}{i = i' \cup \Prop =
  j}\]
Identifying $j$ and $\Prop$ is too restrictive, as unfolding would only
add a (trivial) constraint $\Prop ≤ j$. The issue also comes up with
universes that appear equivariantly. Unifying $\cstu{id}{i}~t \equiv
\cstu{id}{i'}~t'$ should succeed as soon as $t \equiv t'$, as the normal
forms $\cstu{id}{i}~t →_{\beta\delta}^* t$ and $\cstu{id}{i'}~t'
→_{\beta\delta}^* t'$ are convertible, but $i$ does not have to be
equated with $i'$, again due to cumulativity.

To ensure that we make the least commitment and generate the most
general constraints, there are two options. Either we find a static
analysis that tells us for each constant which constraints are to be
generated for a self-unification with different instances, or we do
without that information and restrict ourselves to unifications that add
no constraints.

The first option amounts to decide for each universe variable
appearing in a term, if it appears only in covariant position (the term
is an arity and the universe appears only in its conclusion), in which
case adding an inequality between the two instances would reflect
exactly the result of unification on the expansions. In general this is
expensive as it involves computing (head)-normal forms. Indeed consider
the definition $\cstu{idtype}{i,j} := \cstu{id}{j}~\Type{j}~\Type{i},$
with associated constraint $i < j$. Deciding that $i$ is used
covariantly here requires to take the head normal form of the
application, which reduces to $\Type{i}$ itself. Recursively, this
$\Type{i}$ might come from another substitution, and deciding covariance
would amount to do βδ-normalization, which defeats the purpose of having
definitions in the first place!

The second option---the one that has been implemented---is to
restrict first-order unification to avoid arbitrary choices as much as
possible. To do so, unification of constant applications is allowed only
when their universe instances are themselves unifiable in a restricted
sense. The inference rules related to constants are:

\begin{mathpar}
\irule{Elab-R-FO}
{\igenconv{=}{ψ}{\vec{as}}{\vec{bs}}{ψ'} \\
  \iconstreq{ψ'}{\vec{u}}{\vec{v}}{ψ''}}
{\igenconv{R}{ψ}{\cstu{c}{\vec{u}}~\vec{as}}{\cstu{c}{\vec{v}}~\vec{bs}}{ψ'}}

\irule{Elab-R-δ-left}
{\cstu{c}{\vec{i}} "->"_\delta t \hspace {-0.7em} \\
  \igenconv{R}{ψ}{t~\vec{a}}{u}{ψ'}}
{\igenconv{R}{ψ}{\cstu{c}{\vec{i}}~\vec{a}}{u}{ψ'}}

\irule{Elab-R-δ-right}
{\cstu{c}{\vec{i}} "->"_\delta u \hspace {-0.7em} \\
  \igenconv{R}{ψ}{t}{u~\vec{a}}{ψ'}}
{\igenconv{R}{ψ}{t}{\cstu{c}{\vec{i}}~\vec{a}}{ψ'}}
\end{mathpar}

The judgment $\iconstreq{ψ}{i}{j}{ψ'}$ formalizes the unification of
universe instances:
\begin{mathpar}
\irule{Elab-Univ-Eq}
{\tconstreq{ψ}{i}{j}}
{\iconstreq{ψ}{i}{j}{ψ}}

\irule{Elab-Univ-Flexible}
{{i_{\mathsf{f}} `V j_{\mathsf{f}}} \in \vec{u_s} \\
{\tconsistent{ψ ∧ i = j}}}
{\iconstreq{ψ}{i}{j}{ψ ∧ i = j}}
\end{mathpar}
If the universe levels are already equal according to the constraints,
unification succeeds (\lrule{Elab-Univ-Eq}). Otherwise, we allow
identifying universes if at least one of them is flexible. This might
lead to overly restrictive constraints on fresh universes, but this is
the price to pay for automatic inference of universe instances. 

% \subsubsection{Explicit Universes}
% \def\uflex#1{#1_{\mathsf{f}}}
% \def\urig#1{#1_{\mathsf{r}}}

% To allow explicit instantiation of universe-polymorphic definitions, we
% can introduce new syntax in our source language. Declaring universes can be
% done using syntax \verb|Type l| where $l$ is an identifier, which will
% be considered rigid in the universe context. To explicitely instantiate
% a universe-polymorphic definition, we overload the usual syntax of \Coq
% for explicitely giving implicit arguments. For example, the user can
% specify a fibration at level $l$ using syntax: \[\cst{fib}~\{j = l\}~: Π
% A : \Type{k} "->" \Type{\lub{k}{l+1}}\] According to the rules of
% unification for universe instances:
% \[\icumul{\uflex{k},\urig{l}
%   \models}{\cstu{fib}{k,\Prop}~T}{\cstu{fib}{k,l}~T}{\uflex{k},\urig{l}
%   \models \Prop \leq l}\]
% Indeed as \lrule{Elab-Univ-Flexible}'s first premise isn't satisfied, We will
% fall back to \lrule{Elab-R-δ-left}, \lrule{Elab-R-δ-right} and \lrule{Elab-R-Red}
% in the end using \lrule{Elab-R-Prod} and \lrule{Elab-R-Type} to
% derive: \[\icumul{\uflex{k},\urig{l}\models}{T "->" \Prop}{T "->"
%   \Type{l}}{\Prop \leq l}\]


\paragraph{Local Type Inference}
This way of separating the rigid and flexible universe variables
allows to do a kind of local type inference \cite{pierce-turner-00},
restricted to the flexible universes. Elaboration does not generate
the most general constraints, but heuristically tries to instantiate
the flexible universe variables to sensible values that make the term
type-check. Resorting to explicit universes would alleviate this
problem by letting the user be completely explicit, if necessary. As
explicitly manipulated universes are rigid, the heuristic part of
inference does not apply to them. In all practical cases we
encountered, no explicitation was needed though.

\subsubsection{Abstraction and simplification of constraints}

After computing the set of constraints resulting from type-checking a
term, we get a set of universe constraints referring to
\emph{undefined}, flexible universe variables as well as global, rigid
universe variables. The set of flexible variables can grow very quickly
and keeping them along with their constraints would result in overly
general and unmanageable terms. Hence we heuristically simplify the
constraints by instantiating undefined variables to their most precise
levels. Again, this might only endanger generality, not consistency. In
particular, for level variables that appear only in types of parameters
of a definition (a very common case), this does not change
anything. Consider for example: $\cstu{id}{u}~\Prop~\coqdocind{True} :
\Prop$ with constraint $\Prop \leq u$. Clearly, identifying $u$ with
$\Prop$ does not change the type of the application, nor the normal form
of the term, hence it is harmless.

We work under the restriction that some undefined variables can
be substituted by algebraic universes while others cannot, as they
appear in the term as explained in Section~\ref{sec:type-theory-with}. We also
categorize variables according to their global or local status. Global
variables are the ones declared through monomorphic definitions in the
global universe context $Ψ$.

Simplification of constraints works in two steps. We first normalize the
constraints and then minimize them.

\paragraph{Normalization}
Variables are partitioned according to equality constraints. This
is a simple application of the Union-Find algorithm. We canonicalize
the constraints to be left with only inequality ($<, \le$) constraints
between distinct universes.  There is a subtlety here, due to the
global/local and rigid/flexible distinctions of variables. We choose
the canonical element $k$ in each equivalence class $C$ to be global
if possible, if not rigid, and build a canonizing substitution of the
form $\vec{u/k}, u \in C \setminus k$ that is applied to the remaining
constraints. We also remove the substituted variables from the
flexible set $\theta$.

\def\Le{\mathsf{Le}}
\def\Lt{\mathsf{Lt}} 
\def\Left#1{\mathsf{L}_{#1}}
\def\Right#1{\mathsf{R}_{#1}} 
\def\max#1#2{\mathsf{max}(#1, #2)}
\def\lubalgo#1{\mathsf{lub}~#1}
\def\LUB#1{\sqcup_#1}
\def\order{\mathrel{R}}

\paragraph{Minimization}
For each flexible variable $u$, we compute its instance as a
combination of the least upper bound (l.u.b.) of the universes below it
and the constraints above it. This is done using a recursive, memoized
algorithm, denoted $\lubalgo{u}$, that incrementally builds a substitution σ
from levels to universes and a new set of constraints. As we start with
a consistent set of constraints, it contains no cycle, we rely on this
for termination. We can hence start the computation with an arbitrary
undefined variable.

We first compute the set of direct lower constraints involving the
variable, recursively:
\[\begin{array}{lcl}
  \Left{u} & \eqdef & \{ (\lubalgo{l}, \order, u) `| (l, \order, u) \in Ψ \}
\end{array}\]
  
If $\Left{u}$ is empty, we directly return $u$. Otherwise, the l.u.b. of
the lower universes is computed as:
\[\LUB{u} \eqdef
\lub{\{ x `| (x, \Le, \_) \in \Left{u} \}}{\{ x+1 `| (x, \Lt, \_) \in \Left{u}\}}\]

The l.u.b. represents the minimal level of $u$, and we can lower $u$ to
it. It does not affect the satisfiability of constraints, but it can
make them more restrictive. If $\LUB{u}$ is a level $j$, we update the
constraints by setting $u = j$ in $Ψ$ and $σ$. Otherwise, we check if
$\LUB{u}$ has been recorded as the l.u.b. of another flexible universe
$j$ in $σ$, in which case we also set $u = j$ in $Ψ$ and $σ$. This
might seem dangerous if $j$ had different upper constraints than
$u$. However, if $j$ has been set equal to its l.u.b. then by definition
$j = \LUB{u} \leq u$ is valid. Otherwise we only remember the equality
$u = \LUB{u}$ in $σ$, leaving $Ψ$ unchanged.  The computation
continues until we have computed the lower bounds of all variables.

This procedure gives us a substitution $σ$ of the undefined universe
variables by (potentially algebraic) universes and a new set of
constraints. We then turn the substitution into a well-formed one
according to the algebraic status of each undefined variable. If a
substituted variable is not algebraic and the substitutend is algebraic
or an algebraic level, we remove the pair from the substitution and
instead add a constraint of the form $\max{\ldots}{\ldots} \le u$ to
$Ψ$. This ensures that only algebraic universe variables are
instantiated with algebraic universes. In the end we get a substitution
$σ$ from levels to universes to be applied to the term under
consideration and a universe context $\vec{us'} \models Ψ[σ]$
containing the variables that have not been substituted and an
associated set of constraints $Ψ[σ]$ that are sufficient to typecheck
the substituted term. We directly give that information to the
kernel, which checks that the constraints are consistent with the
global ones and that the term is well-typed.

\paragraph{Polymorphic abstraction}
The polymorphism considered here is prenex in essence and very similar
to the usual ML-style polymorphism.  The kernel itself is in charge of
making the new definition polymorphic or not according to a user given
flag. Polymorphic definitions are just stored with their universe
context, with the set of local variables serialized as a list of
universes. Note that the set of constraints can still refer to global
variables, but only the local ones are subject to
abstraction.

\subsection{Inductive types}
\label{sec:inductive-types}

Polymorphic inductive types and their constructors are treated in much
the same way as constants. Each occurrence of an inductive or
constructor comes with a universe instance used to typecheck them.
Conversion and unification for them forces equality of the instances, as
there is no unfolding behavior to account for. This behavior implies
that a polymorphic inductive type instantiated at the same type but in
two different universes will force their identification, e.g.:
$\coqdocind{list}_{i}~\coqdocind{True} =
\coqdocind{list}_{\Prop}~\coqdocind{True}$ will force $i = \Prop$, even
though $i$ might be strictly higher (in which case it would be
inconsistent). These conversions mainly happen when mixing polymorphic
and monomorphic code though, and can always be avoided with explicit uses
of $\Type{}$. Conservativity over a calculus with monomorphic
inductives carries over straightforwardly by making copies of the
inductive type for each particular instantiation.

\subsection{Implementation and future work}
\label{sec:poly-impl}

This extension of \Coq
%\footnote{Available at \url{http://www.pps.univ-paris-diderot.fr/~sozeau/repos/CoqUnivs}}
supports the formalization of the Homotopy Type Theory library from the
Univalent Foundations project and is able to check for example
Voevodsky's proof that Univalence implies Functional Extensionality.
The system simply adds a polymorphic flag for switching on the implicit
generalization. The change from universe inference to checking and the
addition of universe instances on constants required important changes
in the tactic and elaboration subsystems to properly keep track of
universes. As minimization happens as part of elaboration, it sits
outside the kernel and does not have to be trusted. There is a
performance penalty to the use of polymorphism, which is at least linear
in the number of fresh universe variables produced during a proof. On
the standard library of \Coq, with all primitive types made polymorphic,
we can see a mean 10\% increase in time and some pathological cases
taking twice as much time. The main issues come from the redundant
annotations on the constructors of polymorphic inductive types
(e.g. $\coqdocind{list}$) which could be solved by representing
type-checked terms using bidirectional judgments, and the choice of the
concrete representation of universe constraints during elaboration,
which could be improved by working directly on the graph.

At the time of writing, we can only check the basic model structures
in reasonable time, i.e., the definitions in
Section~\ref{sec:formalization} take about 10 minutes to check while
it takes 10 minutes already to check the definition of
\coqdocdefinition{LamT} in Section~\ref{sec:interpretation}. We had to
  deactivate universes completely to develop that part. The model
  indeed stresses the universe polymorphism due to the large number of
  variables involved. For example, the \coqdocdefinition{\_Type} definition
  alone is abstracted on 20 variables and twice as many
  constraints. This is due notably to the fact that we only allow
  levels as universe instances so any potential instanciation by an
  algebraic universe (even simply $l + 1$) is turned into a variable
  and a constraint. To tame this problem, we will have to extend the
  graph structure and constraint checking algorithm.

Once we resolve this issue we hope to make the translation more robust
and easy to use so that we can effectively apply it in developments that
require extensional principles, i.e., formalizations of monads or the
forcing translation discussed earlier.

\subsection{Related work}
\label{sec:universe-systems}

Other designs for working with universes have been developed in systems
based on Martin-Löf type theory. The Agda programming language provides
fully explicit universe polymorphism, making level quantification
first-class. This requires explicit quantification and lifting of
universes (no cumulativity), but instantiation can often be handled
solely by unification. The main difficulty in this setting is that
explicit levels can obfuscate definitions and make development and
debugging arduous.

The Matita proof assistant based on CIC lets users declare universes and
constraints explicitly\cite{AspertiCompact} and its kernel only checks
that user-given constraints are sufficient to typecheck terms. It has a
notion of polymorphism at the library level only: one can explicitly
make copies of a module with fresh universes. In
\cite{DBLP:conf/tphol/Courant02}, a similar extension of the \Coq system
is proposed, with user declarations and a notion of polymorphism at the
module level. Our elaboration system could be adapted to handle these
modes of use, by restricting inference.

%%% Local Variables: 
%%% mode: latex
%%% TeX-master: "main"
%%% End: 

%\input{bench-apps}


\section{Related Work and Conclusion}
\label{sec:conclusion}

We have presented an internalization of the setoid interpretation of
(weak) Martin-Löf type theory respecting the invariance under
isomorphism principle in \Coq's type theory with universe polymorphism.
%
The setoid interpretation is due to Hofmann~\cite{hofmann95These} and the groupoid
interpretation to Hofmann and Streicher~\cite{groupoid-interp}. This
interpretation is based on the notion of categories with families
introduced by Dybjer~\cite{dybjer:internaltt}.
%
This framework has recently been used by Coquand et al. to give an
interpretation in semi-simplicial sets and cubical
sets~\cite{barras:_gener_takeut_gandy_inter,coquand:cubical}.
%
Although very promising, the interpretation based on cubical sets has not
yet been mechanically checked and only an evaluation procedure based on
it has been implemented in Haskell. Besides, it is still based on set theory.

Altenkirch et al. have introduced Observational Type Theory (OTT)
\cite{altenkirch-mcbride-wierstra:ott-now}, an intentional type theory where
functional extensionality is native, but equality in the universe is structural.
%
To prove expected properties on OTT such as strong normalization,
decidable typechecking and canonicity, they use embeddings into Agda and
extensional type theory. A setoid interpretation clearly guides their
design, and our model could be adapted to interpret this theory as well.

We have strived for generality in our definitions and while our
interpretation is done for setoids, it illustrates the main structures
of the model and should adapt to higher dimensional models, specifically
$\omega$-groupoids.
%
The next step of our work is to generalize the construction to higher
dimensions. We already have a formalization of weak 2-groupoids based
on inductive definitions (\emph{à la} enriched categories) on the
computational structure.
%
The only mechanism that is not inductive is the generation of
higher-order compatibilities between coherence maps. 
%
This is because category enrichment provides a co-inductive definition
for \emph{strict} $\omega$-groupoids only. Formalizing in \Coq the recent work
of Cheng and Leinster~\cite{cheng2012weak} on weak enrichment should
provide a way to define \emph{weak} $\omega$-groupoids co-inductively using
operads to parameterize the compatibility required on coherence maps at
higher levels.
  
%\cite{DBLP:journals/aml/Palmgren12}

%%% Local Variables: 
%%% mode: latex
%%% TeX-master: "main"
%%% End: 

% This is the text of the appendix, if you need one.

% \acks

% Acknowledgments, if needed.

% We recommend abbrvnat bibliography style.

\bibliographystyle{splncs}
\bibliography{biblio}

% \newpage

% \appendix
% \section{Algebraic universes}
% \label{sec:algunivs}
% We show here that the algebraic universes are sufficient to typecheck
terms in the calculus, and that the kernel only needs to handle atomic
constraints. Indeed, we have the invariant that taking least upper
bounds $\lub{u}{v}$ will always give algebraic universes and that the
constraints to be checked during conversion can always be reduced to
atomic ones.

To show this, we first have to introduce a few definitions related to the form 
of universes appearing in terms:

\begin{definition}[Arity]
  A term $t$ is an arity if it is of the form $Π \overrightarrow{(x : T)}, \Type{u}$.
\end{definition}

\begin{definition}[Non-algebraic term]
  A term $t$ is \emph{non-algebraic} if all universes appearing in it are
  variables/levels.
\end{definition}

\begin{definition}[Depth-1 algebraic term]
  A term $t$ is depth-1 algebraic if:
  \begin{itemize}
  \item $t$ is an arity $Π \overrightarrow{(x : T)}, \Type{u}$, 
    the types $\overrightarrow{T}$ are non-algebraic and $u$ is an algebraic universe.
  \item Otherwise, if $t$ is non-algebraic.
  \end{itemize}
\end{definition}

We can show:
\begin{lemma}[Depth-1 algebraic types]
  If $\tcheck{Γ}{ψ}{t}{T}$ then $t$ is non-algebraic and $T$ is depth-1 algebraic.
  If $\WFc{Γ}{ψ}$ and $\vdecl{x}{T} \in Γ$ then $T$ is non-algebraic.
\end{lemma}
\begin{proof}
  By mutual induction on the derivations.
  The interesting case is \lrule{Prod}:
  We have by induction that $s$ and $s'$ are depth-1 algebraic, we have
  to show that their least upper bound is as well: 
  $\lub{\max{le}{lt}}{\max{le'}{lt'}} = \max{le \cup le'}{lt \cup lt'}$.
  Clearly, least upper bounds of algebraic universes stay in the subset
  of algebraic universes.

  For \lrule{Lambda}, we have to check that the domain $A$ is
  non-algebraic: this follows by induction on the first premise, as $A$
  appears in a term position.

  The other cases follow by induction.
\end{proof}

This invariant is useful to show that the constraints necessary to
typecheck a term can be reduced to a set of \emph{atomic} constraints
between universe levels. Looking at a typing derivation, we see that
constraints are checked only in the rule \lrule{R-Type}, in a premise
of a \lrule{Conv} rule application:
\begin{mathpar}
\irule{}
{\tgenconstr{ψ}{u}{\relR}{v}}
{\tgenconv{\relR}{ψ}{\Type{u}}{\Type{v}}}
\end{mathpar}

We know that the type on the right is non-algebraic due to
\lrule{Conv}'s second premise, hence $v$ is a variable $i$. For
\lrule{$\preceq$-Type}, $u$ could be depth-1 algebraic and the
constraint we must check could be of the form: $\max{\vec{le}}{\vec{lt}}
\leq i$.  This can be translated to an equivalent set of atomic
constraints:
\[\max{\vec{le}}{\vec{lt}} \leq i "<=>" \overrightarrow{le \leq i} ∧ \overrightarrow{lt < i}\]

For \lrule{Conv-Type}, we know that the two types are non-algebraic as
they come from domains of products or are non-algebraic in the first
place (arities are convertible with arities only).
Hence the kernel need only check that sets of atomic constraints are
consistent. 



% \newpage
% \appendix
% \section{Algebraic universes}
% \label{sec:algunivs}
% We show here that the algebraic universes are sufficient to typecheck
terms in the calculus, and that the kernel only needs to handle atomic
constraints. Indeed, we have the invariant that taking least upper
bounds $\lub{u}{v}$ will always give algebraic universes and that the
constraints to be checked during conversion can always be reduced to
atomic ones.

To show this, we first have to introduce a few definitions related to the form 
of universes appearing in terms:

\begin{definition}[Arity]
  A term $t$ is an arity if it is of the form $Π \overrightarrow{(x : T)}, \Type{u}$.
\end{definition}

\begin{definition}[Non-algebraic term]
  A term $t$ is \emph{non-algebraic} if all universes appearing in it are
  variables/levels.
\end{definition}

\begin{definition}[Depth-1 algebraic term]
  A term $t$ is depth-1 algebraic if:
  \begin{itemize}
  \item $t$ is an arity $Π \overrightarrow{(x : T)}, \Type{u}$, 
    the types $\overrightarrow{T}$ are non-algebraic and $u$ is an algebraic universe.
  \item Otherwise, if $t$ is non-algebraic.
  \end{itemize}
\end{definition}

We can show:
\begin{lemma}[Depth-1 algebraic types]
  If $\tcheck{Γ}{ψ}{t}{T}$ then $t$ is non-algebraic and $T$ is depth-1 algebraic.
  If $\WFc{Γ}{ψ}$ and $\vdecl{x}{T} \in Γ$ then $T$ is non-algebraic.
\end{lemma}
\begin{proof}
  By mutual induction on the derivations.
  The interesting case is \lrule{Prod}:
  We have by induction that $s$ and $s'$ are depth-1 algebraic, we have
  to show that their least upper bound is as well: 
  $\lub{\max{le}{lt}}{\max{le'}{lt'}} = \max{le \cup le'}{lt \cup lt'}$.
  Clearly, least upper bounds of algebraic universes stay in the subset
  of algebraic universes.

  For \lrule{Lambda}, we have to check that the domain $A$ is
  non-algebraic: this follows by induction on the first premise, as $A$
  appears in a term position.

  The other cases follow by induction.
\end{proof}

This invariant is useful to show that the constraints necessary to
typecheck a term can be reduced to a set of \emph{atomic} constraints
between universe levels. Looking at a typing derivation, we see that
constraints are checked only in the rule \lrule{R-Type}, in a premise
of a \lrule{Conv} rule application:
\begin{mathpar}
\irule{}
{\tgenconstr{ψ}{u}{\relR}{v}}
{\tgenconv{\relR}{ψ}{\Type{u}}{\Type{v}}}
\end{mathpar}

We know that the type on the right is non-algebraic due to
\lrule{Conv}'s second premise, hence $v$ is a variable $i$. For
\lrule{$\preceq$-Type}, $u$ could be depth-1 algebraic and the
constraint we must check could be of the form: $\max{\vec{le}}{\vec{lt}}
\leq i$.  This can be translated to an equivalent set of atomic
constraints:
\[\max{\vec{le}}{\vec{lt}} \leq i "<=>" \overrightarrow{le \leq i} ∧ \overrightarrow{lt < i}\]

For \lrule{Conv-Type}, we know that the two types are non-algebraic as
they come from domains of products or are non-algebraic in the first
place (arities are convertible with arities only).
Hence the kernel need only check that sets of atomic constraints are
consistent. 


% The bibliography should be embedded for final submission.

% \begin{thebibliography}{}
% \softraggedright

% \bibitem[Smith et~al.(2009)Smith, Jones]{smith02}
% P. Q. Smith, and X. Y. Jones. ...reference text...

% \end{thebibliography}

\end{document}
