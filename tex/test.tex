\documentclass[12pt]{report}
\usepackage[]{inputenc}
\usepackage[T1]{fontenc}
\usepackage{fullpage}
\usepackage{coqdoc}
\usepackage{amsmath,amssymb}
\begin{document}
%%%%%%%%%%%%%%%%%%%%%%%%%%%%%%%%%%%%%%%%%%%%%%%%%%%%%%%%%%%%%%%%%
%% This file has been automatically generated with the command
%% /Applications/CoqIdE_8.4pl2.app/Contents/Resources/bin/coqdoc --latex -o test.tex test.v 
%%%%%%%%%%%%%%%%%%%%%%%%%%%%%%%%%%%%%%%%%%%%%%%%%%%%%%%%%%%%%%%%%
\coqlibrary{test}{Library }{test}

\begin{coqdoccode}
\coqdocemptyline
\coqdocemptyline
\end{coqdoccode}


\subsection{Definition of weak groupoids}


 \label{sec:w2gpds}
 We formalize weak groupoids using type classes. 
 
 Contrarily to what is done in the usual Setoid translation,
 the basic notion of morphisms is given as inhabitants of a relation in [Type]:

*)

Definition HomT (A : Type) := A -> A -> Type.

(**
  Homs are relation, that is inhabitants of type [HomT T] for a particular [T], 
  and morphisms are inhabitants of a hom.

  Given a hom, we define type classes that represents that the
  [Hom]-set of morphisms on a [Type A] is reflexive (which corresponds
  to the identity morphism), symmetric (which corresponds to the
  existence of an inverse morphism for every morphism) and transitive
  (which corresponds to morphisms composition).

*)
  
Class Identity {A} (Hom : HomT A) :=
  identity : ∀ x, Hom x x.

Class Inverse {A} (Hom : HomT A) :=
  inverse : ∀ x y:A, Hom x y -> Hom y x.

Class Composition {A} (Hom : HomT A) :=
  composition : ∀ {x y z:A}, Hom x y -> Hom y z -> Hom x z.

Notation  "g ° f" := (composition f g) (at level 50). 

(* begin hide *)
Class Equivalence {T} (Eq : HomT T):= {
  Equivalence_Identity :> Identity Eq ;
  Equivalence_Inverse :> Inverse Eq ;
  Equivalence_Composition :> Composition Eq 
}.
(* end hide *)

(**

  In a weak groupoid, all morphisms up-to level 2 are invertible and
higher equalities are trivial. Thus the set of 2-Homs denoted by [~2]
corresponds to an equivalence relation.

*)
\begin{coqdoccode}
\end{coqdoccode}
\end{document}
