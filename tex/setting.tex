\section{Setting of the translation}
\label{sec:setting-translation}

\def\Elt#1{\texttt{Elt}(#1)}
\def\Univ{\ensuremath{\mathcal{U}}}
\def\Id#1#2#3{\texttt{Id}_{#1}\,#2\ #3}
\def\Equiv#1#2{\texttt{Equiv}\ \Elt{#1}\ \Elt{#2}}
\def\Eq#1#2#3{\texttt{Eq}_{#1}\ #2\ #3}
\def\refl#1#2{\texttt{refl}_{#1}\ #2}
\def\funext#1{\texttt{fun\_ext}(#1)}
\def\zeroType{\ensuremath{\mathbb{O}}\xspace}
\def\oneType{\ensuremath{\mathbb{1}}\xspace}
\def\twoType{\ensuremath{\mathbb{2}}\xspace}
\def\hzeroType{\ensuremath{\mathbb{O}}}
\def\honeType{\ensuremath{\mathbb{1}}}
\def\htwoType{\ensuremath{\mathbb{2}}}

\subsection{Martin-Löf Type Theory with Functional Extensionality}
\label{sec:definitions}


For the purpose of this paper we study a restricted source theory
resembling a cut-down version of the core language of the \Coq system,
with only one \Type{} universe (see \cite{DBLP:bibsonomy_cupart} for an
in-depth study of this system). This is basically Martin-Löf Type Theory
(without \Type{} : \Type{}). First we introduce the definitions of the
various objects at hand.  We start with the term language of a dependent
λ-calculus: we have a countable set of variables $x, y, z$, and the
usual typed λ-abstraction $λ x : τ, b$, à la Church, application $t~u$,
the dependent product and sum types $Π/Σ x : A. B$, and an identity type
$\Id{T}{t}{u}$. 
% There is a single universe $\Univ$. For $T$ and $U$ in
% $\Univ{}$, the type of (Set)-isomorphisms $T `= U$, is definable
% directly using the other type constructors (see \ref{sec:universe}).

The typing judgment for this calculus is written $\tcheck{Γ}{ψ}{t}{T}$
(Figure \ref{fig:emltt}) where $Γ$ is a context of declarations $Γ ::=
\Nil `| Γ, x : τ$, $t$ and $T$ are terms. If we have a valid derivation
of this judgment, we say $t$ has type $T$ in $Γ$.
%  and assume that $T$ is
% not $\Type{}$ unless explicilty stated (this just allows us to do
% without a separate judgment for carving out the types).

Most of the rules are standard. The definitional equality $A = B$ is
defined as the congruence on type and term formers compatible with
β-reductions for abstractions and projections.

\subsubsection{Identity type.} 
%
The identity type in the source theory is the standard
Martin-Löf identity type $\Id{T}{t}{u}$, generated from an
introduction rule for reflexivity with the usual \texttt{J} eliminator
and its \emph{propositional} reduction rule. The \texttt{J} reduction 
rule will actually be valid definitionally in the model for \emph{closed} terms.

\subsubsection{Functional Extensionality.} 
%
The proof-relevant equality of functions in the source theory for which we want to
give a computational model appears in rule \lrule{Fun-Ext},
extending the formation rules of identity types on dependent product. 
%
Equality of dependent functions $f$ and $g$ of type $\Pi \vdecl{x}{A}\mdot B$ are
introduced by giving a witness of $\Pi t : A, \Id{B \{\subs x {t}\}}
{(f \ t)}{(g \ t)}$.
%
The \texttt{J} rule for dependent functions witnesses the
\emph{naturality} of every predicate constructed in the source type
theory.
%
This rule corresponds 
  to the introduction of equality on dependent functions in %\cite{DBLP:conf/popl/LicataH12}%.

\def\hFin#1{\mathtt{\hat Fin}\ #1}
\def\Fin#1{\texttt{Fin}\ #1}
\def\fin#1#2{\underline{#1}_{#2}}

% \subsubsection{Universe.} The universe $\Univ$ is closed under Σ, Π,
% \zeroType, \oneType, \twoType and $\texttt{Id}$ in elements of $\Univ$, 
% \emph{not} type equivalences. The
% constructors are circumflexed, e.g. $\hat \Pi$ and $\hat \Sigma$ are
% introduced with the rules:
% %
% %\vspace{-1em} 
% \begin{mathpar}
% \irule{Univ-$\hat \Pi$, -$\hat \Sigma$}
% {\tcheck{\Gamma}{ψ}{A}{\Univ} \\
% \tcheck{\Gamma, x : \Elt{A}}{ψ}{B}{\Univ}}
% {\tcheck{\Gamma}{ψ}{\hat \Pi/\hat \Sigma x : A. B}{\Univ}}
% \end{mathpar}

% \noindent 
% For presentation purposes, we do not detail here the treatment of finite
% types (see \cite{altenkirch-mcbride-wierstra:ott-now} for a standard treatment).
% The extension to W-types would be straightforward.
% %
% \texttt{Elt} is a $\type{}$-forming map from $\Univ$ that acts as a
% homomorphism, e.g. its action on products is: $$\Elt{\hat \Pi x : A. B} = Π x : \Elt{A}. \Elt{B}$$
% \begin{mathpar}
% \begin{array}{lcll}
%   \texttt{Elt} : \Univ & → & \Type{} & \\
%   \Elt{\{\hat \Pi/\hat \Sigma\} x : A. B} & = & \{Π/Σ\} x :
%   \Elt{A}. \Elt{B} & \\
%   \Elt{\hat{τ}} & = & τ & {τ \in \{ \hzeroType, \honeType, \htwoType \}} \\
%   % \Elt{\hFin{n}} & = & \Fin{n} & \\
% %  \Elt{a =_{τ} b} & = & \texttt{Id}_{\Elt{τ}}\ a\ b & \\
%   \Elt{\hat{C}[X]} & = & C[\Elt{X}] & \text{(homomorphism)}
% \end{array}
% \end{mathpar}


\begin{figure}[t]
% \hspace{-0.0\textwidth}
% \begin{minipage}{1.0\textwidth}
\begin{mathpar}

\irule{Empty}{}{\WFc{\Nil}{ψ}}

\irule{Decl}
{\ttcheck{\Gamma}{ψ}{T}{\Type{}} \hspace{-1em}\\
 %T \neq \Univ, %\hspace{-1em}\\
 x \not \in \Gamma }
{\WFc{Γ, \vdecl{x}{T}}{ψ}}

\irule{Univ}
{}
{\ttcheck{\Gamma}{ψ}{\Univ}{\Type{}}}

\irule{Var}
{\WFc{\Gamma}{ψ} \hspace{-1em}\\ 
  (\vdecl{x}{T}) \in \Gamma}
{\tcheck{\Gamma}{ψ}{x}{T}}


\irule{Prod/Sigma}
{\ttcheck{\Gamma, \vdecl{x}{A}}{ψ}{B}{\Type{}} \hspace{-1em}}
{\ttcheck{\Gamma}{ψ}{\Pi/\Sigma \vdecl{x}{A}\mdot B}{\Type{}}}

\irule{Pair}
{%\ttcheck{\Gamma}{ψ}{Σ x : A. B}{\Type{}} \\
\tcheck{\Gamma}{ψ}{t}{A} \hspace{-1em} \\ \tcheck{\Gamma}{ψ}{u}{B\{\subs
    x t}\}}
{\tcheck{\Gamma}{ψ}{(t,u)_{x : A. B}}{Σ x : A. B}}

\irule{Proj1}
{\tcheck{\Gamma}{ψ}{t}{\Sigma \vdecl{x}{A}\mdot B}}
{\tcheck{\Gamma}{ψ}{\ensuremath{π_1 t}}{A}}

\irule{Proj2}
{\tcheck{\Gamma}{ψ}{t}{\Sigma \vdecl{x}{A}\mdot B}}
{\tcheck{\Gamma}{ψ}{π_2 t}{B\{\subs x {π_1 t}\}}}
% \irule{Fin}
% {n \in \{ 0, 1, 2 \}}
% {\tcheck{\Gamma}{ψ}{\Fin{n}}{\Type{}}}
% \irule{Fin-Intro}
% {k, n \in \{ 0, 1, 2 \}, k < n}
% {\tcheck{\Gamma}{ψ}{\fin{k}{n}}{\Fin{n}}}
% \irule{Fin-Elim}
% {n \in \{ 0, 1, 2 \} \\
%   \tcheck{\Gamma, x : \Fin{n}}{ψ}{P}{\Type{}} \\
%   ∀ k < n. \tcheck{\Gamma}{ψ}{p_k}{P\ \fin{k}{n}} \\
%   \tcheck{\Gamma}{ψ}{f}{\Fin{n}}}
% {\tcheck{\Gamma}{ψ}{\texttt{felim}_{P}\ \overrightarrow{p_k}\ f}{P\
% f}}

\irule{Conv}
{\tcheck{\Gamma}{ψ}{t}{A} \hspace{-1em} \\ \ttcheck{Γ}{ψ}{B}{\Type{}} \hspace{-1em} \\ \tconv{ψ}{A}{B}}
{\tcheck{\Gamma}{ψ}{t}{B}}


\irule{Lam}
{\tcheck{\Gamma, \vdecl{x}{A}}{ψ}{t}{B}}
{\tcheck{\Gamma}{ψ}{\lambda \vdecl{x}{A}\mdot t}{\Pi \vdecl{x}{A}\mdot B}}

\irule{App}
{\tcheck{\Gamma}{ψ}{t}{\Pi \vdecl{x}{A}\mdot B} \hspace{-1em} \\
 \tcheck{\Gamma}{ψ}{t'}{A}}
{\tcheck{\Gamma}{ψ}{t\ t'}{B\{\subs x {t'}\}}}

\irule{Id}
{\ttcheck{\Gamma}{ψ}{T}{\Type{}}\quad
\tcheck{\Gamma}{ψ}{A, B}{T}}
{\ttcheck{\Gamma}{ψ}{\Id{T}{A}{B}}{\Type{}}}
\qquad
\irule{Id-Intro}
{\tcheck{\Gamma}{ψ}{t}{T}}
{\tcheck{\Gamma}{ψ}{\refl{T}{t}}{\Id{T}{t}{t}} }
\qquad
\irule{Fun-Ext}
{\tcheck{\Gamma}{ψ}{e}{ \Pi t : A, \Id{B \{\subs x {t}\}} {(f \ t)}{(g \ t)}}}
{\tcheck{\Gamma}{ψ}{\funext{e}}{\Id{\Pi \vdecl{x}{A}\mdot B}{f}{g} }}


% \irule{Equiv-Intro}
% {\tcheck{\Gamma}{ψ}{i}{ \Elt{A} ` = \Elt{B}}}
% {\tcheck{\Gamma}{ψ}{\texttt{equiv}\ i}{\Id{\Univ}{A}{B}}}


\irule{Id-Elim (J)}
{\tcheck{\Gamma}{ψ}{i}{\Id{T}{t}{u}} \\
\ttcheck{\Gamma, x : T, e : \Id{T}{t}{x}}{ψ}{P}{\Type{}} \\
\tcheck{\Gamma}{ψ}{p}{P\{\subs x t\}\{\refl{T}{t}/e\}}}
{\tcheck{\Gamma}{ψ}{\texttt{J}_{λx\ e. P}~i~p}{P\{\subs x u, \subs e i\}}}
\end{mathpar}
\caption{Typing judgments for our extended MLTT}\label{fig:emltt}
\end{figure}



% We abuse notations
% and consider Π, Σ and \texttt{Id} as codes when seen as inhabitants of
% \Univ, and as regular syntax for the type constructors in the rest of
% the type theory.

% P[t,refl] -> P[t,equiv i]: P is abstracted by x : U and Eq t x, can't look
% inside the universe x, but p : P[t,refl] might use J q refl =
% q. Replace by J q (equiv i) = q[i.1 y/y]. 

%We write $\tcheck{Γ}{}{T}{s}$
%as a shorthand for $\tcheck{Γ}{}{T}{\Type{u}}$ for some universe $u$.
% \begin{figure}
% \begin{mathpar}

% % \irule{Univ-Id}
% % {\tcheck{\Gamma}{ψ}{A}{\Univ} \\
% % \tcheck{\Gamma}{ψ}{a, b}{\Elt{A}}}
% % {\tcheck{\Gamma}{ψ}{a =_A b}{\Univ}}

% \irule{Univ-Fin}
% {τ \in \{ \hzeroType, \honeType, \htwoType \}}
% {\tcheck{\Gamma}{ψ}{\hat{τ}}{\Univ}}

% \irule{Univ-$\hat \Pi$, -$\hat \Sigma$}
% {\tcheck{\Gamma}{ψ}{A}{\Univ} \\
% \tcheck{\Gamma, x : \Elt{A}}{ψ}{B}{\Univ}}
% {\tcheck{\Gamma}{ψ}{\hat \Pi/\hat \Sigma x : A. B}{\Univ}}
% \end{mathpar}
% \caption{Definition of \Univ{} (inductive-recursive with \Elt{\_})}\label{fig:univ}
% \end{figure}

% \begin{figure}[!h]
% \begin{mathpar}

% \begin{array}{lcll}
%   \texttt{Elt} : \Univ & → & \Type{} & \\
%   \Elt{\{\hat \Pi/\hat \Sigma\} x : A. B} & = & \{Π/Σ\} x :
%   \Elt{A}. \Elt{B} & \\
%   \Elt{\hat{τ}} & = & τ & {τ \in \{ \hzeroType, \honeType, \htwoType \}} \\
%   % \Elt{\hFin{n}} & = & \Fin{n} & \\
% %  \Elt{a =_{τ} b} & = & \texttt{Id}_{\Elt{τ}}\ a\ b & \\
%   \Elt{\hat{C}[X]} & = & C[\Elt{X}] & \text{(homomorphism)}
% \end{array}

% \begin{array}{lcll}
%   \texttt{Eq}_{U : \Univ} : \Elt{U} → \Elt{U} & → & \Type{} & \\
%   \Eq{\hat \Pi x : A. B}{f}{g} & = & Π x : \Elt{A}. \Elt{f\ x =_B g\ x} \\
%   \Eq{\hat \Sigma x : A. B}{t}{u} & = & %Π t u : (Σ x : \Elt{A}. \Elt{B}), 
%   Π e : \Elt{π_1\ t =_A π_1\ u}, π_2\ t =_{B\ (π_1\ t)} \texttt{J}_{λx. \Elt{B x}}\
%   e\ (π_2\ u)
%  % \Eq{a =_{τ} b}{e}{e'} & = & \mathbf{1} &
% \end{array}

% \end{mathpar}
% \caption{Universe decoding}\label{fig:univelt}
% \end{figure}




\subsection{The proof assistant}
\label{sec:proof-assistant}
We use an extension of the \Coq proof assistant to formally define
our translation. Vanilla features of \Coq allow us to define overloaded
notations and hierarchies of structures through type classes
\cite{sozeau.Coq/classes/fctc}, and to separate definitions and proofs
using the \Program extension \cite{sozeau.Coq/FingerTrees/article}, they
are both documented in \Coq's reference manual \cite{coq:refman:8.4}.
We also use the recent extension to polymorphic universes~\cite{sozeau.Coq/univs/ITP14}.

\subsubsection{Classes and projections.}
The formalization makes heavy use of type classes and sigma types, both
defined internally as parameterized records. We also have modified the
representation of record projections, making them primitive to allow a
more economical representation, leaving out the parameters of the record
type they are applied to. This change, which is justified by
bidirectional presentations of type theory, makes typechecking
exponentially faster in the case of nested structures (see
\cite{garillot:pastel-00649586} for a detailed explanation of this phenomenon).



% not including In the usual Calculus of
% Inductive Constructions, records are encoded as non-recursive inductive
% types with a single constructor and their projections are defined by
% pattern-matching. While this representation is adequate, it is
% inefficient in terms of space and time consumption as projection
% $\coqdocvar{p}$ from a parameterized record
% $\coqdocind{R}~(\coqdocvar{A} : \Type{}) := \{ \coqdoccst{p} :
% \coqdocvar{A} \}$ have type $Π A : \Type{},
% \coqdocind{R}~\coqdocvar{A} → \coqdocvar{A}$. Every application of a
% projection hence repeats the parameters of the record, which in the
% nested case results in a exponential blowup in the size of terms,
% and typechecking likewise repeats work unnecessarily . We
% modified the representation of record projections to skip parameters,
% which is also justified by bidirectional presentations of type theory,
% to recover a workable formalization.

One peculiarity of \Coq's class system we use is the ability to nest
classes. We use the \coqdoccst{A\_of\_B} \coqcode{:>} \coqdocind{A} notation in a type class
definition \coqdockw{Class} \coqdocind{B} as an abbreviation for
defining \coqdoccst{A\_of\_B} as an instance of \coqdocind{A}.

\subsubsection{Polymorphic Universes.}
\label{sec:polym-univ}
\def\Types{\coqdockw{Type}s\xspace}

  % Type theories such as the Calculus of Inductive Constructions maintain
  % a universe hierarchy to prevent paradoxes that naturally appear if one
  % is not careful about the sizes of types that are manipulated in the
  % language. To ensure consistency while not troubling the user with 
  % this necessary information, systems using typical ambiguity were
  % designed. We present an elaboration from terms using typical ambiguity 
  % into explicit terms which also accomodates universe polymorphism, i.e.
  % the ability to write a term once and use it at different universe
  % levels. Elaboration relies on an enhanced type inference algorithm to 
  % provide the freedom of typical ambiguity while also supporting
  % polymorphism, in a fashion similar to usual Hindley-Milner polymorphic
  % type inference. This elaboration is implemented as a drop-in
  % replacement for the existing universe system of Coq and has been
  % benchmarked favorably against the previous version. We demonstrate how
  % it provides a solution to a number of formalization issues present in
  % the original system.

To typecheck our formalization, we also need a stronger universe system
than what vanilla \Coq offers. Indeed, if we are to give a uniform
(shallow) translation of type theory in type theory, we have to define a
translation of the \type{} universe (a groupoid) as a term of
the calculus and equip type constructors like $Π$ and $Σ$ with
\interp{\type{}} structures as well. As \interp{\type{}} itself contains
a \Type{}, the following situation occurs when we define the translation
of, e.g. sums: we should have \interp{Σ~U~T\ \type{}} = \interp{Σ} \interp{U}
\interp{T} : \interp{\type}. To ensure consistency of the
interpretations of types inside \interp{U}, \interp{T} and the outer
one, they must be at different levels, with the outer one at least as
large as the inner ones. A universe polymorphic extension of \Coq has
been designed to allow such highly generic
developments~\cite{sozeau.Coq/univs/ITP14}.  The design was implemented
by the first author and is already in use to check the \name{HoTT}
library in \Coq \cite{HoTT/HoTT}.

% This is however not supported in the current
% version of \Coq, as the universe system does not allow a definition to
% live at different levels. Hence, there could be only one universe level
% assigned to the translation of any \Type{} and they couldn't be nested,
% as this would result in an obvious inconsistency: the usual \Type{} :
% \Type{} inconsistency would show up as \interp{\Type{}} :
% \interp{\Type{}}. One solution to this problem would be to have $n$
% different interpretations of \Type{} to handle $n$ different levels of
% universes.  This is clearly unsatisfactory, as this would mean
% duplicating every lemma and every structure depending on the translation
% of types, and the numer of duplications would depend on the use of
% universes at the source level. Instead, we can extend the system with
% universe polymorphic definitions that are parametric on universe levels
% and instantiate them at different ones, just like parametric
% polymorphism is used to instantiate a definition at different
% types. This can be interpreted as building fresh instances of the
% constant that can be handled by the core type checker without
% polymorphism.  We give a detailed presentation of this system in 
% Section \ref{sec:type-theory-with}, but as the handling of universes stays
% entirely implicit at the source level, we just present here the core
% calculus on which the universe polymorphism system and our
% interpretation rests.


