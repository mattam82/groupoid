
\section{Related Work and Conclusion}
\label{sec:conclusion}

We have presented an internalization of the groupoid interpretation of
Martin-Löf type theory with one universe respecting the invariance under
isomorphism principle in \Coq's type theory with universe polymorphism.

The groupoid interpretation is due to Hofmann
and Streicher~\cite{groupoid-interp}. This interpretation is based on
the notion of categories with families introduced by
Dybjer~\cite{dybjer:internaltt}. 
%
This framework has recently been used by Coquand et al. to give an
interpretation in semi-simplicial sets and cubical
sets~\cite{barras:_gener_takeut_gandy_inter,coquand:cubical}.
%
Although very promising, the interpretation based on cubical sets has not
yet been mechanically checked and only an extraction procedure based on
it has been implemented in Haskell.

Altenkirch et al. have introduced Observational Type Theory (OTT)
\cite{altenkirch-mcbride-wierstra:ott-now}, an intentional type theory where
functional extensionality is native, but equality in the universe is structural.
%
To prove expected properties on OTT such as strong normalization,
decidable typechecking and canonicity, they use embeddings into Agda and
extensional type theory. A setoid interpretation clearly guides their
design, and our model could be adapted to interpret this theory as well.

We have strived for generality in our definitions and while our
interpretation is done for groupoids, it illustrates the main structures of the
model and should adapt to higher dimensional models, specifically
$\omega$-groupoids.
%
The next step of our work is to generalize the construction to higher
dimensions. We already have a formalization of weak 2-groupoids based
on inductive definitions (\emph{à la} enriched categories) on the
computational structure.
%
The only mechanism that is not inductive is the generation of
higher-order compatibilities between coherence maps. 
%
This is because category enrichment provides a co-inductive definition
for \emph{strict} $\omega$-groupoids only. Formalizing in \Coq the recent work
of Cheng and Leinster~\cite{cheng2012weak} on weak enrichment should
provide a way to define \emph{weak} $\omega$-groupoids co-inductively using
operads to parameterize the compatibility required on coherence maps at
higher levels.
  
%\cite{DBLP:journals/aml/Palmgren12}

%%% Local Variables: 
%%% mode: latex
%%% TeX-master: "main"
%%% End: 